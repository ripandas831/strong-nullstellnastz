\documentclass[answers, 12pt]{exam}

\usepackage{graphicx}
\usepackage{tikz}
\usetikzlibrary{cd,matrix,arrows,decorations.pathmorphing}
\usepackage{nicematrix}

\setcounter{MaxMatrixCols}{20}
\usepackage{
  %amsmath,
  %amsthm,
  amssymb,
  euscript,
  %enumerate,% better enumitem
  url,
  verbatim,
  calc,
}

\textwidth6.5in
\textheight9in
\oddsidemargin.2in
\evensidemargin.2in
\topmargin-1cm
\renewcommand{\baselinestretch}{1.2}
\usepackage{amsmath}
\newtheorem{theorem}{Theorem}[section]
\newtheorem{definition}[theorem]{Definition}%[theorem]
\newtheorem{defn}[theorem]{Definition}
\newtheorem{qns}[theorem]{Question}
\newtheorem{qns*}{Question}
\newtheorem{note}{Note}[theorem]
\newtheorem{problem}[theorem]{Problem}
\newtheorem{exercise}[theorem]{Exercise}
\newtheorem{ex}{Exercise}
\newtheorem{example}[theorem]{Example}%[section]
\newtheorem{eg}[theorem]{Example}
\newtheorem{obs}[theorem]{Observation}
%\newtheorem*{claim*}{Claim}
\newtheorem{observation}[theorem]{Observation}
\newtheorem{proposition}[theorem]{Proposition}%[section]
\newtheorem{prop}[theorem]{Proposition}
%\newtheorem{theorem}{Theorem}[section]
\newtheorem{remark}[theorem]{Remark}%[section]
\newtheorem{remark*}[theorem]{Remark}
\newtheorem{corollary}[theorem]{Corollary}%[section]
\newtheorem{lemma}[theorem]{Lemma}%[section]
\newcommand{\QQ}{\mathbb Q}
\newcommand{\ZZ}{\mathbb Z}
\newcommand{\CC}{\mathbb C}
\newcommand{\FF}{\mathbb F}
\newcommand{\RR}{\mathbb R}
\newcommand{\NN}{\mathbb N}
%\operatorname{deg}{{\deg}}
\newcommand\sbullet[1][.5]{\mathbin{\vcenter{\hbox{\scalebox{#1}{$\bullet$}}}}}
\newcommand{\norm}[1]{\left\lVert#1\right\rVert}
\newcommand{\gen}[1]{\langle#1\rangle}
\newcommand{\Hom}[1]{\text{Hom}_K(#1)}
\newcommand{\lcm}[1]{\text{lcm}(#1)}
\newcommand{\rank}[1]{\text{rank}~(#1)}
%\DeclareMathOperator{\Homm}{\text{Hom}_R( )}
\DeclareMathOperator{\im}{\text{Im}}
\DeclareMathOperator{\Ker}{\text{Ker}}
\DeclareMathOperator{\ch}{\text{char}}
\DeclareMathOperator{\spec}{\text{spec}}
\DeclareMathOperator{\mspec}{\text{maxspec}}
\DeclareMathOperator{\Rank}{\text{Rank~}}
\DeclareMathOperator{\Null}{\text{Nullity~}}
\DeclareMathOperator{\Span}{\text{Span~}}
\DeclareMathOperator{\proj}{proj}
\newcommand{\tr}[1]{\text{trace}(#1)}
\DeclareMathOperator{\ann}{\text{Ann}}
\DeclareMathOperator{\ass}{\text{Ass}}
\DeclareMathOperator{\supp}{\text{Supp~}}
\unframedsolutions
\renewcommand{\solutiontitle}{\noindent Ans. \par\noindent}
\begin{document}
\begin{center}\underline{Solution manual of `Introduction to Commutative algebra, Atiyah-Macdonald'}\end{center}
\newpage
\tableofcontents
\newpage

\section{Rings and Ideals}



\section{Modules}



\section{Rings and Modules of Fractions}


\begin{enumerate}

\item Let $A$ be a ring, $M$ an $A-$module. The support of $M$ is defined to be the set  $\supp(M)$ of prime ideals $p$ of $A$ such that $M_p\neq 0.$ Prove the following results:

		\begin{enumerate}
	
			\item $M\neq 0\Leftrightarrow \supp(M)\neq \emptyset.$
		
			\item $V(a)=\supp (A/a).$
		
			\item If $0\to M'\to M\to M''\to 0$ is an exact sequence, then $\supp(M)=\supp(M')\cup \supp(M'').$
		
			\item If $M=\sum M_i,$ then $\supp(M)=\cup\supp(M_i).$
		
			\item If $M$ is finitely generated, then $\supp(M)=V(\ann_A(M))$(and is therefore a closed subset of $\spec A$).
		
			\item If $M$ and $N$ are finitely generated then $\supp (M\otimes_A N)=\supp(M)\cap \supp(N).$ [Use Chapter 2, Exercise 3.]
		
			\item If $M$ is finitely generated and $a$ is an ideal of $M$ then $\supp(M/aM)=V(a+\ann_A(M)).$
		
			\item If $f:A\to B$ is a ring homomorphism and $M$ is a finitely generated $A-$module, then $\supp(B\otimes_A M)=f^{*-1}(\supp(M)).$
	
		\end{enumerate}

	\textit{Answer.}	\begin{enumerate}
	
		\item dfd	
	
	
	
	
	\end{enumerate}






















\end{enumerate}





\section{Primary Decomposition}



\section{Integral Dependence and Valuation}



\section{Chain Condition}



\section{Noetherian Rings}



\section{Artin Rings}



\section{Discrete Valuation rings and Dedekind Domain}



\section{Completions}



\section{Dimension Theory}

































\end{document}