\documentclass[11pt]{amsart}
\usepackage{amssymb}
\usepackage{tikz-cd}

\usepackage{graphicx}
\usepackage{tikz}
\usetikzlibrary{cd,matrix,arrows,decorations.pathmorphing}
\usepackage{mathtools}
\usepackage{
  %amsmath,
  %amsthm,
  amssymb,
  euscript,
  %enumerate,% better enumitem
  url,
  verbatim,
  calc,
}
\usepackage{cancel}
%\usepackage{mathtools}
%\usepackage{extarrows}
\textwidth6.5in
\textheight9in
\oddsidemargin.2in
\evensidemargin.2in
\topmargin-1cm
\renewcommand{\baselinestretch}{1.2}
\usepackage{amsmath}
\newtheorem{theorem}{Theorem}[section]
\newtheorem{definition}[theorem]{Definition}%[theorem]
\newtheorem{defn}[theorem]{Definition}
\newtheorem{qns}[theorem]{Question}
\newtheorem*{qns*}{Question}
\newtheorem{problem}[theorem]{Problem}
\newtheorem{exercise}[theorem]{Exercise}
\newtheorem{ex}[theorem]{Exercise}
\newtheorem{example}[theorem]{Example}%[section]
\newtheorem{eg}[theorem]{Example}
\newtheorem*{eg*}{Example}
\newtheorem{obs}[theorem]{Observation}
\newtheorem{obs*}{Observation}
\newtheorem*{Obs*}{Observation}
\newtheorem{note}[theorem]{Note}
\newtheorem{proposition}[theorem]{Proposition}%[section]
\newtheorem{prop}[theorem]{Proposition}
%\newtheorem{theorem}{Theorem}[section]
\newtheorem{remark}[theorem]{Remark}%[section]
\newtheorem*{remark*}{Remark}
\newtheorem{corollary}[theorem]{Corollary}%[section]
\newtheorem{lemma}[theorem]{Lemma}%[section]
\newcommand{\QQ}{\mathbb Q}
\newcommand{\ZZ}{\mathbb Z}
\newcommand{\CC}{\mathbb C}
\newcommand{\FF}{\mathbb F}
\newcommand{\RR}{\mathbb R}
\newcommand{\NN}{\mathbb N}
%\operatorname{deg}{{\deg}}
\usepackage{wasysym, stackengine, makebox, graphicx}

\newcommand\isom{\mathrel{\stackon[-0.1ex]{\makebox*{\scalebox{1.08}{\AC}}{=\hfill\llap{=}}}{{\AC}}}}
\newcommand\nvisom{\rotatebox[origin=cc] {-90}{$ \isom $}}
\newcommand\visom{\rotatebox[origin=cc] {90} {$ \isom $}}


\newcommand\sbullet[1][.5]{\mathbin{\vcenter{\hbox{\scalebox{#1}{$\bullet$}}}}}
\newcommand{\norm}[1]{\left\lVert#1\right\rVert}
\newcommand{\gen}[1]{\langle#1\rangle}
\newcommand{\rk}[1]{\text{rank}~#1}
\newcommand{\Homa}[1]{\text{Hom}_A\left(#1\right)}
\newcommand{\Homb}[1]{\text{Hom}_B\left(#1\right)}
\newcommand{\Hom}[1]{\text{Hom}_R\left(#1\right)}
\newcommand{\Homs}[1]{\text{Hom}_S\left(#1\right)}
\newcommand{\lcm}[1]{\text{lcm}(#1)}
\newcommand{\cl}[1]{\text{cl}(#1)}
\newcommand{\Span}[1]{\text{span~}\{#1\}}
%\DeclareMathOperator{\Homm}{\text{Hom}_R( )}
\DeclareMathOperator{\im}{\text{Im}}
\DeclareMathOperator{\Ker}{\text{Ker}}
\DeclareMathOperator{\ch}{\text{char}}
\DeclareMathOperator{\spec}{\text{spec}}
\DeclareMathOperator{\mspec}{\text{maxspec}}
\DeclareMathOperator{\aut}{\text{Aut}}
\DeclareMathOperator{\id}{\text{id}}
\title[]{Algebra}
\author[]{Ripan Das}
\begin{document}
\maketitle
\newpage 
\tableofcontents
\newpage
%\chapter{Prerequisites \& preliminaries}
\section{Relation}
\subsection{Introduction}
\begin{defn}
Let $A,B$ be two set, we define Cartesian product of two set as $$A\times B:=\{(a,b):a\in A,b\in B\}.$$
\end{defn}
Let $S$ be an non-empty set, a binary relation $R$ is a subset of $S \times S$. Let $\rho \subseteq S \times S.$ $\rho$ is said to be $reflexive$ if $(a,a) \in \rho$ $\forall a \in S.$ $\rho$ is said to be $symmetric$ if $(a,b) \in \rho \Rightarrow (b,a) \in \rho $ and $\rho$ is said to be $transitive$ if $(a,b),(b,c) \in \rho \Rightarrow (a,c) \in \rho.$ A relation $\rho$ is said to be equivalence relation if $\rho$ is $reflexive,$ $symmetric$ and $transitive.$ $\rho$ is said to be $antisymmetric$ if $(a,b),(b,a) \in \rho \Rightarrow a=b.$ A relation $\rho$ is said to be partial ordered relation if $\rho$ is $reflexive,$ $antisymmetric$ and $transitive.$

 \begin{definition}
Let $S$ be an non-empty set. Let $\{P_{\alpha} \}_{\alpha \in \Lambda}$ be a collection of subsets of $S$, i.e. $P_{\alpha} \subseteq S$ $\forall \alpha \in \Lambda$ such that \\
(a) $P_{\alpha} \cap P_{\beta} = \emptyset$ $\forall \alpha \neq \beta$ in $\Lambda$, \\
(b) $\cup_{\alpha \in \Lambda} P_{\alpha} = S.$\\
Then the collection  $\{P_{\alpha} \}_{\alpha \in \Lambda}$ is called the partition of $S.$
\end{definition}
Let $S$ be an non-empty set, and  $\{P_{\alpha} \}_{\alpha \in \Lambda}$ be a partition of $S.$ $a,b \in S$ define a relation $\sim$ on $S,$ $a\sim b$ iff $ \exists$ $\alpha \in \Lambda$ such that $a,b \in P_{\alpha}$ (check $\sim$ is an equivalence relation).
\newline Suppose $S$ be an non-empty set and $"\sim" \subseteq S \times S $ be an equivalence relation on $S.$ For $x \in S$ $cl(x)=\{y \in S | x \sim y \}$ is a subset of $S.$ If $x \sim x'$ then $cl(x)=cl(x').$ Let $y \in cl(x)\Rightarrow x \sim y.$ Now $x \sim x'$ is given $\Rightarrow x' \sim x$ (as $\sim$ is symmetric). $x' \sim x$ and $x \sim y$ $\Rightarrow x' \sim y$ $\Rightarrow y \in cl(x').$ Therefore $cl(x) \subseteq cl(x').$ Now $y \in cl(x)$ then $x' \sim y$, $x \sim x'$ is given $\Rightarrow x \sim y$ (as $\sim$ is transitive), $\Rightarrow y \in cl(x)$ $\Rightarrow cl(x') \subseteq cl(x)$ $\Rightarrow cl(x) = cl(x').$ Note that $cl(x)=cl(x')$, now $x \sim x' \Rightarrow$ $x \in cl(x)=cl(x')$ $\Rightarrow x \in cl(x')$ $\Rightarrow x \sim x'.$
\newline \textbf{Observation:} $cl(x) \cap cl(x') = \emptyset$ iff $x \nsim x'$
suppose $x \nsim x';$ if $y \in cl(x)\cap cl(x')$ then $x \sim y$ and $x' \sim y$ $\Rightarrow x \sim x'$, contradiction.
\newline Next we want to show $cl(x) \cap cl(x')$ then $x \nsim x'.$ If $x \sim x'$ then we have proved $cl(x) = cl(x').$ Now $x \in cl(x) \cap cl(x')$ so $cl(x) \cap cl(x') \neq \emptyset.$  Therefore $\mathfrak{F}=\{cl(x)| x \in S$ and distinct class$ \}$ forms a partition of $S.$ E.g. Take the set $S= {\ZZ},$ fix $n \in {\ZZ},$ define $\sim$  $\subseteq {\ZZ} \times {\ZZ} $ as $a \sim b$ iff $n|b-a.$ Check that $\sim$ is an equivalence relation.


\section{Mappings}
\begin{definition} 
Let $A,B$ be two non empty sets. $f: A \rightarrow B$ is said to be mapping if $f \subseteq A \times B$ and for each $x \in A$ $\exists!$ $y \in B$ such that $(a,b) \in f.$
\end{definition}
\begin{example}
$A=\{1,2,3\}$ and $B=\{x,y,z,w\},$ consider $\rho= \{(1,x),(2,y),(3,w)\}$, $\sigma =\{(1,x),(2,x),(3,w)\}$, $\tau=\{(1,x),(1,y)(2,x),(3,z)\}$ $\subseteq A \times B$ then $\rho$, $\sigma$ is a mapping where $\tau$ is not.
\end{example}
\textbf{Notation:} Let $f: A \rightarrow B$ be a mapping and $(x,y) \in f \subseteq A \times B$ then we write $y=f(x)$ called the image of $x.$ $A$ is called the domain and $B$ is codomain. Let $f:A \rightarrow B$ be a mapping, $X\subseteq A$ then $g:=f|_{X}: X \rightarrow B$ be also a mapping. $f|_{X}$ is called the restriction of $f$ on $X.$
\begin{definition} 
Let $f: A \rightarrow B$ be a mapping, $f$ is said to be injective if $f(x_{1})= f(x_{2})$ $\Rightarrow x_{1} = x_{2},$ contra-positively stated if $x_{1} \neq x_{2}$ $\Rightarrow f(x_{1})\neq f(x_{2}).$ 
\end{definition}
\begin{definition}
A mapping  $f: A \rightarrow B$ is said to be surjective if for each $y \in B$ $\exists$ $x \in A$ such that $f(x)=y.$ $f$ is said to be bijective iff $f$ is injective and surjective.
\end{definition}
\textbf{Observation:} Let $A,B$ be two sets and $ \mathcal{F}=\{f: A \rightarrow B\}$ be the collection of all mapping from $A$ to $B,$ then $|\mathcal{F}|=|B|^{|A|},$ i.e., $B^{A}$ denote the functions from $A$ to $B.$ If, $|A|=n,$ $|B|=m$ and $n\leq m,$ number of injective map from $A$ to $B$ is $m(m-1)\cdots(m-(n-1))=\dfrac{m!}{(m-n)!}$
\newline\textbf{Composition of two mapping:} Let $f: A \rightarrow B$ and $g: B \rightarrow C$ be two mapping, define $g\circ f: A \rightarrow C$ as $(g\circ f)(x)=g(f(x)).$
\begin{theorem}
Let $f: A \rightarrow B$ and $g: B \rightarrow C$ be two mapping. If $f,g$ are injective then $g \circ f$ is also injective. Conversely, if $g\circ f:A \rightarrow C$ is injective then $f$ is injective.
\end{theorem}
\proof Suppose $f,g$ are injective map, then $x_{1},x_{2} \in A,$ now $g(f(x_{1}))=g(f(x_{2}))$ since $f$ is injective, $g(x_{1})=g(x_{2}),$ now injectivity of $g$ implies $x_{1}=x_{2}$ hence $g \circ f$ is injective. Now, $g\circ f$ is injective, pick $x_{1}, x_{2} \in A$, $f(x_{1})=f(x_{2})$ $\Rightarrow g(f(x_{1}))=g(f(x_{2}))$ since $g\circ f$ is injective $\Rightarrow x_{1}=x_{2}.$ \qed
\begin{theorem}
Let,  $f: A \rightarrow B$ and $g: B \rightarrow C$ be two mapping. If $f,g$ are \underline{surjective} then $g \circ f$ is also \underline{surjective}. Conversely, if $g\circ f:A \rightarrow C$ is \underline{surjective} then $g$ is \underline{surjective}.
\end{theorem}
\proof Suppose $f,g$ are surjective. Let $z \in C,$ $\exists$ $ y \in B$ such that $g(y)=z$, for this $y\in B$ $\exists$ $ x \in A$ such that $f(x)=y$ $\Rightarrow (g\circ f)(x)=g(f(x))=g(y)=z$ $\Rightarrow x$ is a preimage of $z,$ hence $g\circ f$ is surjective. Suppose $g\circ f$ is surjective. Let $z \in C$ $\exists$ $x\in A$ such that $(g\circ f)(x)=z.$ Let $y=f(x)$ $\Rightarrow g(y)=z,$ hence $g$ is surjective. \qed
\begin{theorem}
Let,  $f: A \rightarrow B$ and $g: B \rightarrow C$ be two mapping. If $f,g$ are \underline{ bijective} then $g \circ f$ is also \underline{bijective}. Conversely, if $g\circ f:A \rightarrow C$ is \underline{bijective} then we can only say that $f$ is \underline{injective} and $g$ is \underline{surjective}.
\end{theorem}
\proof Suppose $f,g$ are bijective then $g\circ f$ is injective and surjective so $g\circ f$ is bijective. Again, $g\circ f$ is bijective, then $g\circ f$ is injective so $f$ is injective, $g\circ f$ is surjective so $g$ is surjective. \qed
\begin{definition}
Let $f: A \rightarrow B$ be a mapping. If there exists $g: B \rightarrow A$ such that $g\circ f: A \rightarrow A$, $g\circ f=id_{A}$ and $f\circ g: B \rightarrow B$, $f\circ g=id_{B},$ then we say that f is invertible and $g$ is an inverse of $f.$
\end{definition}
\textbf{Notation:} $g=f^{-1}$
\begin{theorem}
Suppose $f: A \rightarrow B$ be an invertible map. Then $f$ has \underline{unique inverse}.
\end{theorem}
\proof Suppose $g_{1}:B \rightarrow A$ and $g_{2}:B \rightarrow A$ be two inverses of $f.$ We have $g_{1}\circ (f\circ g_{2})=(g_{1}\circ f)\circ g_{2}$ $\Rightarrow (g_{1}\circ id_{B})=(id_{A} \circ g_{2})$ $\Rightarrow g_{1}=g_{2}.$ \qed
\newline \textbf{Observation:} Let $f: A \rightarrow B,$ $g: B \rightarrow C$ and $h: C \rightarrow D$ be three mappings, then $(h\circ g)\circ f: A \rightarrow D$ and $h\circ (g\circ f): A \rightarrow D$ are the same mapping. Pick $x \in A$. $(h\circ g)\circ f(x)
=(h\circ g)(f(x))=h(g(f(x))).$ On the other hand, $(h\circ (g\circ f))(x)=h(g\circ f(x))=h(g(f(x))).$ $\therefore (h\circ g)\circ f= h\circ (g \circ f),$ i.e., mapping composition is associative. 
\begin{theorem}
Let $f:A \rightarrow B$ be a mapping. $f$ is invertible iff $f$ is a bijection. In this case $f^{-1}$ is also a bijection.
\end{theorem}
\proof $f:A \rightarrow B$ be a mapping. If there exists $g: B \rightarrow A$ such that $g\circ f=id_{A}$ and $f\circ g=id_{B}$ then we say that $f$ has an inverse. Because identity mapping is a bijective mapping, so in case of $g\circ f,$ $f$ is an injective mapping and in case of $f\circ g,$ $f$ is surjective. Hence $f$ is bijective. Similar argument shows that $g$ is also bijective.
\begin{definition}
A set $A$ is said to be finite if it is empty or there exists a bijection from $A$ to the set $J_n=\{1,\cdots ,n\}$ for some natural number $n.$
\end{definition}
\textbf{Observation:} Suppose $S=\{x_{1},\cdots, x_{n} \}$ be a finite set, if $f: S\rightarrow S$ is injective then $f$ is surjective. $f(S)=\{f(x_{1}),\cdots,f(x_{n}) \}.$ Suppose $f$ is injective then $f(x_{i})=f(x_{j})$ $\Rightarrow x_{i}=x_{j}.$ Therefore, $\mid f(S) \mid =n.$ Now $f(S) \subseteq S$ and $\mid S \mid =n$ $\Rightarrow f(S)=S$ (as $S$ is finite) $\Rightarrow f $ is surjective. If $f$ is surjective then for each $y \in S,$ $f^{-1}(y) \neq \emptyset.$ If $|f^{-1}(y)| > 1$ (for some $y$) we arrive a contradiction that $|S| > |S|$ (since $S$ is finite). $\therefore |f^{-1}(y)|=1$ $\forall$ $y \in S$ and $f^{-1}(y_{1}) \cap f^{-1}(y_{2}) = \emptyset$ $\forall$ $ y_{1},y_{2} \in S$ otherwise $|S| < |S|$ $\Rightarrow f$ is injective.
\begin{remark}
A map $f$ on a finite set $S$ is \underline{injective iff it is surjective.}
\end{remark}
\begin{lemma}
Let $a_0\in A.$ Then there exists a bijection $f:A\to J_{n+1}$ iff there exists a bijection $g:A\setminus \{a_0\}\to J_n.$
\end{lemma}
\proof Suppose there exists a bijection $g:A\setminus \{a_0\}\to J_n$. We define $f:A\to J_{n+1}$ by \begin{align*}
f(x)=
    \begin{cases}
        g(x), &   \text{if}~x\in A\setminus\{a_0\}\\
        n+1, &   \text{if}~x=a_0
    \end{cases}
\end{align*}
Then clearly $f$ is a bijection. Conversely, Suppose there exists a bijection $f:A\to J_{n+1}.$ If $f(a_0)=n+1$ then the required function $g$ is $f\bigg|_{A\setminus \{a_0\}}.$ If $f(a_0)\neq n+1$, let $f(a_o)=m$ where $1\leq m\leq n.$ As $f$ is surjective so there exists $a_1\in A\setminus \{a_0\}$ such that $f(a_1)=n+1.$ Define, $h:A\to J_{n+1}$ by \begin{align*}
&h(a_0)=n+1\\
&h(a_1)=m\\
&h(x)~~~~~=f(x)\quad\text{if}~x\in A\setminus \{a_0,a_1\}
\end{align*}
Then clearly $h$ is a bijection and so the required $g$ is $h\bigg|_{A\setminus \{a_0\}}.$ \qed
\begin{lemma}
Let $f:A\to J_n$ be a bijection. Let $B\subsetneq A.$ Then there doesn't exists any bijection $g:B\to J_n$ but there exists a bijection $g:B\to J_m$ for some $m<n$ provided $B\neq \emptyset.$
\end{lemma}
\proof If $B=\emptyset$ then there is nothing to prove. So we assume $B\neq \emptyset$ and we prove it by induction. If $n=1$ then $A$ is singleton and so $B=\emptyset.$ Thus there exists no bijection $g:B\to J_1.$ Next we assume that this statement holds for $n=k$ and prove it for $n=k+1.$ Suppose, $f:A\to J_{k+1}$ be a bijection and $B$ is a non-empty proper subset of $A$. Let $b_0\in B$ and $a_0\in A\setminus B.$ Then by previous lemma there exists a bijection $g:A\setminus \{b_0\}\to J_k.$ Now, $B\setminus \{b_0\}$ is a proper subset of $A\setminus \{b_0\}$ and so by induction hypothesis, \begin{enumerate}
\item there exists no bijection $h:B\setminus \{b_0\}\to J_k$,
\item either $B\setminus \{b_0\}=\emptyset$ or there is a bijection $h':B\setminus \{b_0\}\to J_m$ for some $m<k.$
\end{enumerate}
By previous lemma and (1) there is no bijection from $B$ to $J_{k+1}.$ If $B\setminus \{b_0\}=\emptyset$ then there exists a bijection from $B$ to $J_1.$ If $B\setminus \{b_0\}\neq \emptyset$ then by lemma 2.13 and (2) there is a bijection from $B$ to $J_{m+1}.$ Thus there is a bijection from $B$ to $J_p$ with $p<k+1.$ Hence, by principle of induction the result follows. \qed
\begin{corollary}
If $A$ is a finite then there is no bijection of $A$ with a proper subset of itself.
\end{corollary}
\proof If possible there is a bijection $f:A\to B$ where $B\subsetneq A.$ By definition, there is a bijection $g:A\to J_n$ for some $n\in {\NN}.$ Then the map $g\circ f^{-1}:B\to J_n$ is a bijection. This contradicts the previous lemma. \qed
\begin{corollary}
The numbers of elements in a finite set $A$ is uniquely determined by $A.$
\end{corollary}
\proof If possible let there are two different bijection $f:A\to J_n$ and $g:A\to J_m$ where $m<n$ then $J_m\subsetneq J_n$ and $f\circ g^{-1}:J_m\to J_n$ is a bijection. This contradicts lemma 2.14. \qed\\
Thus we conclude that a subset of a finite set is finite.
\begin{theorem}
Let $A\neq \emptyset$ and $n\in {\NN}.$ Then the following are equivalent:
\begin{enumerate}
\item There is a surjective function $f:J_n\to A$,
\item There is a injective function $g:A\to J_n,$
\item $A$ is finite and has at most elements.
\end{enumerate}
\end{theorem}
\proof $(1)\Rightarrow (2)$ Given $f:J_n\to A$ which is surjective. We define $g:A\to J_n$ as follows: For $a\in A,\exists m\in J_n$ such that $f(m)=a.$ Out of all such $m$'s choose the smallest one sat $m_a.$ We write $g(a)=m_a.$ So $g$ is well defined by Well ordering principle of ${\NN}.$ Let $a,b\in A$ and $a\neq b$ then $g(a)\neq g(b)$ otherwise $m_a=m_b$ but $f(m_a)\neq f(m_b)$ which contradicts the fact that $f$ is mapping. Therefore, $g$ is surjective.\\
$(2)\Rightarrow (3)$ Give $g:A\to J_n$ which is injective. Then $g:A\to g(A)\subsetneq J_n$ is a bijection. Now, there exists a bijection $h:g(A)\to J_m$ for some $m\leq n.$ So, $h\circ g:A\to J_m$ is a bijection. Thus $A$ is finite and has at most $m$ elements.\\
$(3)\Rightarrow (1)$ Let $A$ has $m$ elements where $1\leq m\leq m.$ Then there is a bijection $g:A\to J_m.$ If $m=n$ then $g^{-1}:J_m\to A$ is a surjection. If $m<n$ then define $f:J_n\to A$ by \begin{align*}
f(k)=\begin{cases}
g^{-1}(k),\quad &\text{if}~1\leq k\leq m\\
g^{-1}(1),\quad &\text{if}~m<k\leq n
\end{cases}
\end{align*}
Then $f$ is a surjection. \qed
\begin{definition}
A set $A$ is said to be infinite if it is not finite. $A$ is said to be countably infinite if there is a bijection between $A$ and ${\NN}.$
\end{definition}
\begin{definition}
A set $A$ is said to be countable if it is either finite or countably infinite. A set that is not countable is said to be uncountable.
\end{definition}
\begin{theorem}
There is a bijection from $J_m$ onto $J_n$ iff $m=n.$
\end{theorem}
\proof Suppose, $f:J_m\to J_n$ is a bijection. Then $m\leq n.$ Again, $f^{-1}:J_n\to J_m$ is a bijection, then $n\leq m.$ So, $m=n.$ Converse part is trivial. \qed
\begin{theorem}
Every subset of ${\NN}$ is countable.
\end{theorem}
\proof Let $A\subseteq {\NN}$. If $A=\emptyset$ or contains finite number of elements then $A$ is finite and so countable. Suppose, $A$ be not finite then we want to show that there is a bijection between $A$ and ${\NN}.$ We define $f$ by induction as follows: $$f(1)=\text{smallest element of}~A$$ This choice is possible for well ordering principle of ${\NN}.$ Suppose, $f(1),f(2),\cdots f(k)$ are all defined then define $f(k+1)=\text{smallest element of}~A\setminus \{f(1),\cdots ,f(k)\}.$ Note that $A\setminus \{f(1),\cdots ,f(k)\}\neq \emptyset$ otherwise $A$ is finite as $f:\{1,\cdots ,k\}\to A$ is bijective. Thus $f(n)$ is defined for all $n\in {\NN}.$ By definition $f:{\NN}\to A$ is injective. We next show that $f$ is surjective. Let $a\in A$ be any element. By the injectivity of $f$, we have $f({\NN})\nsubseteq \{1,\cdots,a\}$ then $\exists~m_0\in {\NN}$ such that $f(m_0)>a$. Let $S=\{n\in {\NN}:f(n)\geq a\}$ then $m_0\in S$ and so $S\neq \emptyset.$ Hence $S$ has a least element say $m_a,f(m_a)\geq a$ then $\forall~n<m_a$ we have $f(n)<a \Rightarrow a\notin \{f(1),\cdots,f(m_a-1)\}.$ By definition $f(m_a)$ is the smallest element of $A\setminus \{f(1),\cdots,f(m_a-1)\} \Rightarrow f(m_a)\leq a,$ so $f(m_a)=a$ and $f$ is surjective. Thus $f$ is bijective so $A$ is countably infinite. \qed

\begin{theorem}
Let $A\neq \emptyset.$ Then the followings are equivalent: \begin{enumerate}
\item There exists a surjection $f:{\NN}\to A$,
\item There exists an injection $g:A\to {\NN}$,
\item $A$ is countable.
\end{enumerate}
\end{theorem}
\proof 























\begin{tikzcd}
                                                &  & {\{1,2,3\}}                                                                   &  &                                                       \\
                                                &  &                                                                               &  &                                                       \\
{\{1,2\}} \arrow[rruu, no head]                 &  & {\{1,3\}} \arrow[uu, no head]                                                 &  & {\{2,3\}} \arrow[lluu, phantom] \arrow[lluu, no head] \\
                                                &  &                                                                               &  &                                                       \\
\{1\} \arrow[rruu, no head] \arrow[uu, no head] &  & \{2\} \arrow[lluu, no head] \arrow[rruu, no head]                             &  & \{3\} \arrow[lluu, no head] \arrow[uu, no head]       \\
                                                &  &                                                                               &  &                                                       \\
                                                &  & \{\emptyset\} \arrow[uu, no head] \arrow[rruu, no head] \arrow[lluu, no head] &  &                                                      
\end{tikzcd}



































\newpage
%\chapter{Cardinal and Ordinal}
\section{Cardinal and Ordinal}




























\newpage
%\chapter{Group theory}
\section{Group theory}

\begin{definition}
Let G be a non-empty set. $\circ :G\times G \rightarrow G$ be a binary operation on $G$ such that 
\newline $(i)$ $a\circ b \in G$ $\forall a,b \in G$
\newline $(ii)$ $a\circ (b\circ c)=(a\circ b)\circ c$ $\forall a,b,c \in G$ i.e., $\circ$ is associative
\newline $(iii)$ $\exists$ $e_{G} \in G$ such that $a\circ e_{G}=e_{G}\circ a=a$ $\forall$ $a \in G,$ $e_{G}$ called identity element of $G$
\newline $(iv)$ for each $a\in G$ $\exists$ $a^{-1} \in G$ such that $a\circ a^{-1}=a^{-1}\circ a=e_{G}$ $a^{-1})$ is called inverse element of $a$ 
\newline Then $(G, \circ)$ is called a group.
\end{definition}
\begin{example} 
$(1)$ Show that $({\ZZ}, +)$ is a group where $+: {\ZZ}\times{\ZZ} \rightarrow {\ZZ}$ defined by $(a,b)\mapsto a+b$
\newline $(2)$ Show that $({\ZZ}, .)$ is not a group where $.: {\ZZ}\times{\ZZ} \rightarrow {\ZZ}$ defined by $(a,b)\mapsto ab$
\newline $(3)$  Show that $({\NN}\cup\{0\}, +)$ is not a group where $+: {\NN}\cup\{0\}\times{\NN}\cup\{0\} \rightarrow {\NN}\cup\{0\}$ defined by $(a,b)\mapsto a+b.$ Example $2,3$ are called semi-group.
\newline $(4)$ A group is said to be finite group if the underlying set $G$ is finite. Consider the set ${\ZZ}/n{\ZZ}=\{ \overline{0},\cdots,\overline{n-1}\},$ set of all reduced residue class of $n$ i.e., set of remainders upon dividing by $n.$ Then ${\ZZ}/n{\ZZ}$ form a group under addition defined as $+: {\ZZ}/n{\ZZ} \times {\ZZ}/n{\ZZ} \rightarrow {\ZZ}/n{\ZZ}$ by $(\overline{a},\overline{b})\mapsto \overline{a+b}.$ Check that $({\ZZ}/n{\ZZ}, +)$ is a group. Try to find more finite group.
\newline \textbf{Define,} $M_{n}({\RR})=\{A_{n\times n} |$ $A=(a_{ij})_{n\times n},$ $(a_{ij}) \in {\RR} \}$
\newline $GL_{n}({\RR})=\{A_{n\times n}\in M_{n}({\RR} |$ $det(A) \neq 0$\}
\newline Check that whether the followings are group or not:
\newline $(5)$ $(M_{n}({\RR}),+),$ $(6)$ $(M_{n}({\RR}),.),$ $(7)$ $(GL_{n}({\RR}),+),$ $(8)$ $(GL_{n}({\RR}),.)$
\end{example} 
\begin{theorem}
Let, $(G, \circ)$ be a group, then identity element is \underline{unique}.
\end{theorem}
\proof Let $e_{G}$ and $e'_{G}$ be two identity element. $e'_{G} \circ e_{G}=e'_{G}$ (by property of $e_{G}$) and $e'_{G} \circ e_{G}=e_{G}$ (by property of $e'_{G}$). Therefore, $e_{G}=e'_{G}.$ \qed
\begin{theorem}
Let, $(G, \circ)$ be a group, then inverse of an element is \underline{unique}.
\end{theorem}
\proof Let $a'$ and $a''$ be two inverses of $a$. $a'\circ(a\circ a'')=(a'\circ a)\circ a''$ $\Rightarrow a'\circ e_{G}=e_{G}\circ a''$ $\Rightarrow a'=a''$ \qed

\subsection{Sub-Group}
\begin{definition}
Let $G$ be a group with respect to $\circ,$ Let $H\subseteq G$ such that $(H, \circ|_{H\times H}: H\times H \rightarrow H)$ is a group, then H is called the subgroup of $G.$
\end{definition}
\textbf{Notation:} $H<G$
\begin{example}
Consider the group $({\QQ},+)$. Now ${\ZZ}\subseteq {\QQ}.$ Consider $+|_{ {\ZZ}\times{\ZZ}}: {\ZZ}\times{\ZZ}\rightarrow {\ZZ}.$ $({\ZZ}, +|_{ {\ZZ}\times{\ZZ}})$ forms a group then ${\ZZ}<{\QQ}.$
\end{example}
\begin{theorem}
Let $(G, \circ)$ be a group and $H<G$ iff \underline{ $x,y \in H$ $\Rightarrow y^{-1}x \in H.$ }
\end{theorem}
\proof $y,y\in H$ $\Rightarrow y^{-1}y \in H$ $\Rightarrow e \in H.$ Pick $y \in H$ and $e \in H$ $\Rightarrow y^{-1}e \in H$ $\Rightarrow y^{-1} \in H.$ $y,x \in H$ $\Rightarrow y,x^{-1} \in H$ $\Rightarrow (x^{-1})^{-1}y \in H$ $\Rightarrow xy \in H.$ $H$ is closed under multiplication. $\circ$ is associative on $G$ then $\circ$ is associative on $H.$ So, $H$ is a subgroup of $G.$ Conversely, If $H<G$ then $x,y \in H$ $\Rightarrow x,y^{-1} \in H$ $\Rightarrow y^{-1}x \in H.$ \qed
\begin{theorem}
Let $(G,\circ)$ be a group. $H \subseteq G,$ $H$ is a subgroup of $G$ iff \underline{$(a)$ $x,y \in H$ $\Rightarrow xy \in H$ } and \underline{ $(b)$ $x\in H$ $\Rightarrow x^{-1} \in H.$ }
\end{theorem}
\proof To show that $H$ is a subgroup of $G$, it is enough to show that $e \in H.$ Given, $x\in H$ $\Rightarrow x^{-1} \in H$ so that $x\cdot x^{-1} \in H$ $\Rightarrow e \in H.$ Conversely, $H$ is a subgroup of $G$ then $x,y \in H$ $\Rightarrow xy \in H,$ $H$ is closed under multiplication. Since $H$ is a subgroup of $G$ an element $x \in H$ $\Rightarrow x^{-1} \in H.$ \qed
\subsection{Symmetric Group}
Let $S$ be a set and $S\neq \emptyset$, define \begin{align*}
A(S)=\{f:S\to S:f~\text{is bijection}\}
\end{align*}
Now, \begin{align*}
\circ: A(S)\times A(S)&\to A(S)\\
(f,g)&\mapsto f\circ g
\end{align*}
Then it is easy to check that $A(S)$ is a group under mapping composition. Let $S=\{1,\cdots,n\}$ then we denote  $A(S)$ by $S_n.$ Therefore, \begin{align*}
S_n=\{f:\{1,\cdots,n\}\to \{1,\cdots,n\}:f~\text{is bijection}\}
\end{align*}
If $f\in S_n$ then we denote $f$ as follow \begin{align*}
f:=\begin{pmatrix}
1\quad&2\quad&\cdots \quad&n\\
f(1)&f(2)&\cdots &f(n)
\end{pmatrix}
\end{align*}
Let \begin{align*}
f=\begin{pmatrix}
1\quad2\quad3\quad4\\
2\quad3\quad1\quad4
\end{pmatrix},\quad g=\begin{pmatrix}
1\quad2\quad3\quad4\\
4\quad2\quad1\quad3
\end{pmatrix}
\end{align*}
Then \begin{align*}
f\circ g&=\begin{pmatrix}
1\quad2\quad3\quad4\\
2\quad3\quad1\quad4
\end{pmatrix}\begin{pmatrix}
1\quad2\quad3\quad4\\
4\quad2\quad1\quad3
\end{pmatrix}\\
&=\begin{pmatrix}
1\quad2\quad3\quad4\\
4\quad3\quad2\quad1
\end{pmatrix}
\end{align*}
And \begin{align*}
g\circ f&=\begin{pmatrix}
1\quad2\quad3\quad4\\
4\quad2\quad1\quad3
\end{pmatrix}\begin{pmatrix}
1\quad2\quad3\quad4\\
2\quad3\quad1\quad4
\end{pmatrix}\\
&=\begin{pmatrix}
1\quad2\quad3\quad4\\
2\quad1\quad4\quad3
\end{pmatrix}
\end{align*}
Let $1<r\leq n$ and pick $x_1,\cdots,x_r\in \{1,\cdots,n\}$ then an $r$ cycle $(x_1\cdots x_n)=x_1\mapsto x_2\mapsto x_3\mapsto\cdots\mapsto x_{r-1}\mapsto x_r\mapsto x_1$ and $y\mapsto y$ where $y\in \{1,\cdots,n\}\setminus \{x_1,\cdots,x_n\}.$ An 3 cycle $(2,3,5)\in S_5$ is look like $\begin{pmatrix}
1\quad2\quad3\quad4\quad5\\
1\quad3\quad5\quad4\quad2
\end{pmatrix}$. Let $a\in S_n$ be a permutation then $a^i$


























\subsection{Group Homomorphism}
\begin{defn}
Let $G,G'$ be two groups. A map $\phi:G\to G'$ is said to be a group morphism if $\phi(xy)=\phi(x)\phi(y),\forall~x,y\in G.$
\end{defn}
\begin{defn}
Let $\phi:G\to G'$ be a group morphsim if there exists another group morphism $\psi:G'\to G$ such that $\phi\circ \psi=id_{G'}$ and $\psi\circ \phi=id_G$ then $\phi$ is said to be an isomorphism (or invertible group homomorphism) and $\psi$ is called inverse of $\phi.$
\end{defn}
\begin{theorem}
A group homomorphism $\phi:G\to G'$ is an isomorphsim iff it is bijective.
\end{theorem}
\proof If $|phi:G\to G'$ is an isomorphism then $\exists~\psi:G'\to G$ such that $\phi\circ \psi=id_{G'}$ and $\psi\circ \phi=id_G \Rightarrow \phi\circ \psi$ and $\psi\circ \phi$ are bijective map hence $\phi$ is bijective. Conversely, if $\phi$ is a bijective group homomorphism then as a set map there exists a mapping $\psi:G'\to G$. Our claim is that $\psi$ is a morphism. Let $x',y'\in G'$ then $\psi(x'y')=t$ (say) then $\phi\circ \psi(x'y')=\phi(t).$ Let $\phi(x)=x',\phi(y)=y' \Rightarrow \psi(x')=x,\psi(y')=y$ then $\phi(x)\phi(y)=x'y'=\phi(t) \Rightarrow \phi(xy)=\phi(t) \Rightarrow xy=t \Rightarrow \psi(x')\psi(y')=t=\psi(x'y').$ Therefore, $\psi$ is a group morphism. \qed
\begin{defn}
Let $\phi:G\to G'$ be a group morphism. We define kernel of $\phi$ is $$\ker\phi:=\{x\in G:\phi(x)=e_{G'}\}.$$
\end{defn}

\begin{defn}
Let $G$ be a group and $N<G$. $N$ is said to be normal subgroup of $G$ is it satisfies following equivalent conditions:\begin{enumerate}
\item $xyx^{-1}\in N,\forall~y\in N,\forall~x\in G,$
\item $xNx^{-1}=N,\forall~x\in G,$
\item $xN=Nx,\forall~x\in G.$
\end{enumerate}
\end{defn}
\textbf{Notation.} If $N$ is a normal subgroup of $G$ then we write $N\unlhd G.$
\begin{obs}
Let $N<G$, $N$ is normal iff $N$ is kernel of some group homomorphism.
\end{obs}
\proof At first we show that if $\phi:G\to G'$ is a group morphism then $\ker\phi$ is a normal subgroup of $G.$ Let $a,b\in \ker\phi$ then $\phi(a^{-1}b)=\phi(a)^{-1}\phi(b)=e_{G'} \Rightarrow a^{-1}b\in\ker\phi.$ Thus $\ker\phi$ is a subgroup of $G.$ Let $x\in \ker\phi$ and $g\in G$ then $\phi(gxg^{-1})=\phi(g)\phi(x)\phi(g^{-1})=\phi(g)\phi(g)^{-1}=e_{G'} \Rightarrow gxg^{-1} \in \ker\phi \Rightarrow \ker\phi \unlhd G.$ Conversely, if $N\unlhd G$ then we consider the following group morphism \begin{align*}
\phi:G&\to G/N\\
x&\mapsto xN
\end{align*}
Then $\ker\phi=\{x\in G:\pi(x)=eN\}=\{x\in G: xN=eN\}=\{x\in G:x\in N\}=N\cap G=N.$ \qed
\begin{obs}
Let $\phi:G\to G'$ is a injective group homomrophism iff $\ker\phi=\{e_G\}.$
\end{obs}
\proof Suppose, $\phi$ is injective. Note that $\phi(e_G)=e_{G'}$ ($\phi(e_G\cdot e_G)=\phi(e_G) \Rightarrow \phi(e_G)\phi(e_G)=\phi(e_G) \Rightarrow \phi(e_G)=e_{G'}$) then $\ker\phi=\{e_G\}$ (as $\phi$ is injective). Conversely, let $\ker\phi=\{e_G\}$ and $\phi(x)=\phi(y) \Rightarrow \phi(x)\phi(y)^{-1}=e_{G'} \Rightarrow \phi(xy^{-1})=e_{G'} \Rightarrow xy^{-1}\in \ker\phi=\{e_G\} \Rightarrow xy^{-1}=e_G \Rightarrow x=y \Rightarrow \phi$ is injective. \qed

\begin{theorem}[First isomorphism theorem]
Let $\phi:G\to G'$ be a group homomorphism and $H\unlhd G$. If $H\leq \ker\phi$ then there exists an unique group homomorphism $\tilde{\phi}:G/H\to G'$ such that the diagram is commutative i.e, $\phi=\tilde{\phi}\circ \pi.$
\begin{center}
\begin{tikzcd}
G \arrow[dd, "\pi"'] \arrow[rr, "\phi"] &  & G' \\
                                        &  &    \\
G/H \arrow[rruu, "\tilde{\phi}"']       &  &   
\end{tikzcd}
\end{center}
Moreover, \begin{enumerate}
\item $\phi$ is surjective iff $\tilde{\phi}:G/H\to G'$ is surjective,
\item If $H=\ker\phi$ then $\tilde{\phi}:G/\ker\phi\to G'$ is injective,
\item If $\phi$ is surjective and $H=\ker\phi$ then $G/\ker\phi\cong G'.$
\end{enumerate}
\end{theorem}
\proof Given $H\unlhd G$ and $H\leq \ker\phi$ then we define $\tilde{\phi}:G/H\to G'$ by $\tilde{\phi}(xH)=\phi(x).$ We claim that $\tilde{\phi}$ is well defined. Let $xH=yH \Rightarrow y^{-1}x\in H\subseteq \ker\phi \Rightarrow \phi(y^{-1}x)=e_{G'} \Rightarrow \phi{y}=\phi(x) \Rightarrow \tilde{\phi}(xH)=\tilde{\phi}(yH) \Rightarrow \tilde{\phi}$ is well defined. If $\theta:G/H\to G'$ be another morphism such that $\theta\circ \pi=\phi$ then $\theta\circ \pi(x)=\phi(x) \Rightarrow \theta(xH)=\tilde{\phi}(xH),\forall~x\in G \Rightarrow \theta=\tilde{\phi}. $ Therefore, $\tilde{\phi}$ is unique.\\
(1) We note that $\im \phi=\im\tilde{\phi}.$ If $x'\in \im \phi$ then $\exists~x\in G$ such that $\phi(x)=x' \Rightarrow \tilde{\phi}(xH)=x' \Rightarrow x'\in \im\tilde{\phi} \Rightarrow \im \phi\subseteq\im \tilde{\phi}.$ Pick $y'\in \im\tilde{\phi}$ then $\exists~yH\in G/H$ such that $\tilde{\phi}(yH)=y' \Rightarrow\tilde{\phi}\circ \pi(y)=y' \Rightarrow\phi(y)=y' \Rightarrow y\in \im\phi \Rightarrow \im\tilde{\phi}\subseteq \im\phi.$ Hence, $\im\phi=\im\tilde{\phi}.$ Therefore $\phi$ is surjective if and only if $\tilde{\phi}$ is surjective.\\
(2) Let $H=\ker\phi$ and $\tilde{\phi}(x\ker\phi)=\tilde{\phi}(y\ker\phi) \Rightarrow \phi(x)=\phi(y) \Rightarrow \phi(y^{-1}x)=e_{G'} \Rightarrow y^{-1}x\in \ker\phi \Rightarrow y\ker\phi=x\ker\phi \Rightarrow \tilde{\phi}$ is injective. Therefore, $G/\ker\phi\cong \im\phi.$\\
(3) If $\phi$ is surjective then $\im\phi=G' \Rightarrow G/\ker\phi\cong G'.$ \qed
\begin{theorem}[Third isomorphism theorem]
If $N\unlhd G$, $H\unlhd G$ and $N\leq H$ then $H/N\unlhd G/N$ and $\dfrac{G/N}{H/N}\cong G/H.$
\end{theorem}
\begin{proof} Let us consider the following commutative diagram.
\begin{center}
\begin{tikzcd}
G \arrow[rr, "\pi_H"] \arrow[dd, "\pi_N"'] &  & G/H \\
                                           &  &     \\
G/N \arrow[rruu, "\phi"']                  &  &    
\end{tikzcd}
\end{center}
From above diagram $\ker\pi_H=H$ and $\ker\pi_N=N$. By first isomorphism theorem $\phi:G/N\to G/H$ is well defined group homomorphism. Since $\pi_H$ is surjective, $\phi$ is surjective. $\ker\phi=\{xN\in G/N:\phi(xN)=eH\}=\{xN\in G/H:x\in H\}=H/N \Rightarrow H/N\unlhd G/N$ (as $H/N$ is kernel of $\phi$) and by first isomorphism theorem $\dfrac{G/N}{H/N}\cong G/H.$ 
\end{proof}

\begin{qns}
Classify all the subgroups of $G/N$ where $N\unlhd G.$
\end{qns}
Ans. Let $Q\leq G/N$ and we consider the natural projection map $\pi:G\to G/N$ defined by $\pi(x)=xN.$ Clearly $\pi$ is surjective. We claim that $\pi^{-1}(Q)$ is a subgroup of $G.$ Let $x,y\in \pi^{-1}(Q) \Rightarrow xN,yN\in Q.$ Since, $Q\leq G/N \Rightarrow (y^{-1}N)(xN)\in Q \Rightarrow (y^{-1}x)N \in Q \Rightarrow y^{-1}x\in \pi^{-1}(Q) \Rightarrow \pi^{-1}(Q)\leq G.$ Let $H=\pi^{-1}(Q).$ Note that $e_GN\in Q$ (as $Q\leq G/N$) then $\pi^{-1}(\{e_GN\})\subseteq \pi^{-1}(Q) \Rightarrow N\subseteq H \Rightarrow N\leq H.$ Now we will show that $H/N=Q.$ We have $\pi(H)=H/N \Rightarrow \pi(\pi^{-1}(Q))=H/N.$ Since $\pi$ is surjective $H/N=\pi(\pi^{-1}(Q))=Q.$ Conversely, if $N\leq H$ then $H/N\leq G/N$ so all the subgroups of $G/N$ are of the form $H/N$ where $N\leq H\leq G.$
\begin{qns}
What are the normal subgroups of $G/N$ where $N\unlhd G?$
\end{qns}
Ans. From previous problem we see that every subgroups of $G/N$ is of the form $H/N$ where $N\leq H\leq G.$ Suppose, $H/N\unlhd G/N$ then $\dfrac{G/N}{H/N}\cong G/H.$ If $H\ntrianglelefteq G$ then $G/H$ will not form a group but $\dfrac{G/N}{H/N}$ is a group which is a contradiction. Therefore, $H$ must be a normal subgroup of $G.$ Conversely, if $N\unlhd H\unlhd G$ then we will show that $H/N\unlhd G/N$. Let $xN\in H/N$ and $gN\in G/N \Rightarrow (gN)(xN)(gN)^{-1}=(gxg^{-1})N\in H/N$ (as $H\unlhd G$ and $x\in H,g\in G \Rightarrow gxg^{-1}\in H$) hence $H/N\unlhd G/N.$

\subsubsection{Correspondence theorem}
Let $\phi:G\to G'$ be a group morphism and $H\leq G$. Define $\phi_H:=\phi\big|_H:H\to G'$ be a group morphism then $\ker\phi_H=\ker\phi\cap H.$
\begin{prop}
Let $\phi:G\to G'$ be a group homomprhism and $H'\leq G$. Define $\phi^{-1}(H')=H:=\{g\in G:\phi(g)\in H'\}$ then $H\leq G$ containing $\ker\phi$. If $H'\unlhd G'$ then $H\unlhd G.$ If $\phi$ is surjective then $H\unlhd G \Rightarrow H'\unlhd G'.$
\end{prop}
\proof We know that $\ker\phi\leq G.$ Let $g\in \ker\phi \Rightarrow \phi(g)=e_{G'}\in H' \Rightarrow g\in \phi^{-1}(H')=H \Rightarrow \ker\phi\subseteq H.$ Let $x,y\in H \Rightarrow \phi(x),\phi(y)\in H'.$ Since $H'\leq G' \Rightarrow \phi(x)^{-1}\phi(y)=\phi(x^{-1}y)\in H' \Rightarrow x^{-1}y\in H \Rightarrow H\leq G.$ Suppose, $H'\unlhd G'$. Let $x\in H$ and $g\in G$ then $\phi(gxg^{-1})=\phi(g)\phi(x)\phi(g)^{-1}\in H'$ (as $\phi(x)\in H'$ and $\phi(g)\in G'$ and $H'\unlhd G'$) therefore, $gxg^{-1}\in H$ hence $H\unlhd G.$ Let $\phi$ be a surjective map and $H$ is  normal subgroup of $G$. Let $x'\in H'$ and $g'\in G'$. Since $\phi$ is surjective there exists $x\in H$ and $g\in G$ such that $\phi(x)=x'$ and $\phi(g)=g'.$ Now, $g'x'g'^{-1}=\phi(g)\phi(x)\phi(g)^{-1}=\phi(gxg^{-1}).$

\begin{theorem}[Correspondence theorem]
Let $\phi:G\to G'$ be a surjective group morphism with kernel $K$. Then there exists a bijective correspondence between subgroups of $G'$ and subgroups of $G$ that contain $K.$ Moreover, If $H$ and $H'$ are corresponding subgroups then $H\unlhd G$ iff $H'\unlhd G'$ and $|H|=|H'||K|.$
\end{theorem}
\proof Let $K\leq H\leq G$ and $H'\leq G$. We need to show \begin{enumerate}
\item $\phi(H)\leq G'$,
\item $K\leq \phi^{-1}H\leq G$,
\item $H'\unlhd G'$ iff $H\unlhd G,$
\item $\phi(\phi^{-1}(H'))=H'$ and $\phi^{-1}(\phi(H))=H,$
\item $|\phi^{-1}(H')|=|H'||K|.$
\end{enumerate} 
(1) Pick $x,y\in H$, as $H\leq G \Rightarrow x^{-1}y\in H.$ Now $\phi(x^{-1}y)=\phi(x)^{-1}\phi(y)\in \phi(H) \Rightarrow \phi(H)\leq G'.$\\
(2) and (3) follows from previous proposition.\\
(4) $\phi^{-1}(H')=\{x\in G:\phi(x)\in H'\}=H\leq G.$ Clearly, $x\in H \Rightarrow \phi(x)\in H' \Rightarrow \phi(H)=H'$ (as $\phi$ is surjective, every $y\in H'$ has preimage in $G$) then $\phi(\phi^{-1}(H'))=H'.$ For any map $\phi:G\to G'$ with $H\subseteq G \Rightarrow H\subseteq \phi^{-1}(\phi(H)).$ We only need to prove the reverse inclusion. Let $x\in \phi^{-1}(\phi(H)) \Rightarrow \phi(x)\in\phi(H)=H'.$ Since $\phi$ is surjective and $\phi(x)\in H' \Rightarrow x\in H \Rightarrow \phi^{-1}(\phi(H))\subseteq H\Rightarrow H=\phi^{-1}(\phi(H)).$\\
(5) Since $\phi$ is surjective, by first isomorphism theorem $G/K\cong G'.$ Pick $H'\leq G'$ then $H'$ is of the form $H/K$ for some $K\leq H\leq G$ where $H=\phi^{-1}(H')=\{x\in G:\phi(x)\in H'\} \Rightarrow |H'|=\dfrac{|H|}{|K|} \Rightarrow |H|=|H'||K| \Rightarrow |\phi^{-1}(H')|=|H'||K|.$ \qed



\subsubsection{Chinese Remainder theorem}
\begin{theorem}
If $\gcd(p,q)=1$ then $${\ZZ}/p{\ZZ}\times {\ZZ}/q{\ZZ}\cong {\ZZ}/pq{\ZZ}.$$
\end{theorem}
\proof Since $\gcd(p,q)=1$ there exists $u,v\in {\ZZ}$ such that $pu+qv=1.$ Define, \begin{align*}
{\ZZ}&\stackrel{\theta}{\longrightarrow} {\ZZ}/p{\ZZ}\times {\ZZ}/q{\ZZ}\\
x&\mapsto (x+p{\ZZ},x+q{\ZZ})
\end{align*}
Then clearly $\theta$ is a group morphism. Now, $\ker\theta=\{x\in {\ZZ}:(x+p{\ZZ},x+q{\ZZ})=(0+p{\ZZ},0+q{\ZZ})\}.$ Since $x+p{\ZZ}=0+p{\ZZ}\Rightarrow x\in p{\ZZ} \Rightarrow p\big|x.$ Similarly, $q\big| x$ therefore, $\lcm {p,q}\big|x \Rightarrow pq\big| x \Rightarrow x\in pq{\ZZ}.$ Conversely, if $x\in pq{\ZZ} \Rightarrow pq\big|x \Rightarrow p|x$ and $q|x \Rightarrow x+p{\ZZ}=0+p{\ZZ}$ and $x+q{\ZZ}=0+q{\ZZ} \Rightarrow x\in \ker\theta.$ Therefore, $\ker\theta=pq{\ZZ}.$ Now we will show that $\theta$ is surjective. Let $(a+p{\ZZ},b+q{\ZZ})\in {\ZZ}/p{\ZZ}\times {\ZZ}/q{\ZZ}.$ Now consider $y=aqv+bpu$ then $y-a=aqv+bpu-a=a(qv-1)+bpu=-apu+bpu\in p{\ZZ} \Rightarrow y+ p{\ZZ}=a+p{\ZZ}.$ Similarly, $y-b=aqv+bpu-b=aqv+b(pu-1)=aqv-bqv\in q{\ZZ} \Rightarrow y+q{\ZZ}=b+q{\ZZ}.$ Therefore, $\theta(y)=(y+p{\ZZ},y+q{\ZZ})=(a+p{\ZZ},b+q{\ZZ}) \Rightarrow \theta$ is surjective. By first isomorphsim theorem ${\ZZ}/\ker\theta\cong {\ZZ}/p{\ZZ}\times {\ZZ}/q{\ZZ} \Rightarrow {\ZZ}/pq{\ZZ}\cong {\ZZ}/p{\ZZ}\times {\ZZ}/q{\ZZ}.$ \qed




\subsubsection{Product group}
\begin{prop}
Let $H,K\leq G$ and $f:H\times K\to G$ be the multiplication map defined by $(h,k)\mapsto hk$. Image set $\im f(=HK)=\{hk:h\in H,k\in K\}.$ \begin{enumerate}
\item $f$ is injective iff $H\cap K=\{e\},$
\item $f$ is homomorphism iff $HK=KH,$
\item $H$ is normal subgroup of $G$ then $HK<G,$
\item $f$ is an isomorphism iff $H\cap K=\{e\}; HK=G; H,K\unlhd G.$
\end{enumerate}
\end{prop}
\proof (1) Let $f$ is injective and $x\in H\cap K \Rightarrow x^{-1}\in H$ and $f(x^{-1},x)=e=f(e,e)$ which is a contradiction therefore, $H\cap K=\{e\}.$ Conversely, let $H\cap K=\{e\}$. Let $f(h_1,k_1)=f(h_2,k_2) \Rightarrow h_1k_1=h_2k_2 \Rightarrow h_2^{-1}h_1=k_2k_1^{-1} \Rightarrow h_2^{-1}h_1=e=k_2k_1{-1} \Rightarrow h_1=h_2$ and $k_1=k_2$ which implies $f$ is injective.\\
(2) Let $A=(h_1,k_1),B=(h_2,k_2) \Rightarrow AB=(h_1h_2,k_1k_2)$ then $f(AB)=h_1h_2k_1k_2=f(A)f(B)=h_1k_1h_2k_2$ thus $f$ is a group morphism iff $HK=KH.$\\
(3) Let $H\unlhd G$ then for all $g\in G, gH=Hg.$ In particularly $kH=Hk,\forall~k\in K.$ Now, $(h_1k_1)^{-1}h_2k_2=k_1^{-1}h_1^{-1}h_2k_2=h_1^{-1}k_1^{-1}h_2k_2=h_1^{-1}h_2k^{-1}_1k_2\in HK \Rightarrow HK\leq G.$\\
(4) Suppose all the given condition holds then $f$ is both injective and surjective so $f$ is bijective. From (2) we have $f$ is a morphsim hence $f$ is an isomorphism. Converse is also holds. \qed




























\newpage
\subsection{Group Action}
\begin{defn}
An action of a group $G$ on a set $S(\neq \emptyset)$ is a function \begin{align*}
\cdot:G\times S&\to S\\
(g,s)&\mapsto gs
\end{align*}
such that \begin{enumerate}
\item $e\cdot x=x,\forall~x\in S,$
\item $g_1(g_2s)=(g_1g_2)x,\forall~g_1,g_2\in G$ and $\forall~x\in S.$
\end{enumerate}
\end{defn}
\begin{eg}
\begin{enumerate}
\item Let $G$ be a group and $H\leq G$ then  \begin{align*}
\cdot:H\times G&\to G\\
(h,g)&\mapsto hg
\end{align*}
Clearly this is a group action.
\item Let $G$ and $H$ are same as (1) and cosider  \begin{align*}
\cdot:H\times G&\to G\\
(h,g)&\mapsto hgh^{-1}
\end{align*}
Check that this is a group action.
\item Let $I_n=\{1,\cdots,n\}$ and let us consider the map \begin{align*}
\cdot:S_n\times I_n&\to I_n\\
(\sigma,n)&\mapsto \sigma(n)
\end{align*}
Then `$\cdot$' is a group action.
\item Let $H$ and $K$ be a subgroup of $G$ and let $\sum$ be the set of all left coset of $K$ in G then \begin{align*}
\cdot:H\times \sum&\to \sum\\
(h,xK)&\mapsto hxK
\end{align*}
Then check that `$\cdot$' is a group action.
\item Let $H$ and $G$ are same as (1) and $S$ be the set of all subgroups of $G$. Consider the mapping \begin{align*}
\cdot:H\times S&\to S\\
(h,K)&\mapsto hKh^{-1}
\end{align*}
`$\cdot$' is a group action on $S$.
\end{enumerate}
\end{eg}


\begin{theorem}
Let $G$ be a group that acts on a set $S$ \begin{enumerate}
\item The relation on $S$ defined by $x\sim x'$ iff there exists $g\in G$ such that $x'=gx.$ Then `$\sim$' is an equivalence relation.
\item For each $x\in S$, $G_x=\{g\in G:gx=g\}$ is a subgroup of $G$.
\end{enumerate}
\end{theorem}
\proof (1) $x\sim x$ as $x=ex,\forall~x\in S.$ If $x\sim x'$ then by definition we have $x'=gx \Rightarrow g^{-1}x'=x \Rightarrow x'\sim x$ and if $x\sim x'$ and $x'\sim x''$ then we have $x'=g_1x,x''=g_2x'$ for some $g_1,g_2\in G$ then $x''=(g_2g_1)x$ hence $x\sim x''$. Therefore, `$\sim$' is an equivalence relation on $S$.\\
(2) Clearly $G_x$ is non empty as $e\in G_x$ for all $x\in S.$ Let $g_1,g_2\in G_x$ then by definition $g_1x=x$ and $g_2x=x \Rightarrow x=g_2^{-1}x.$ So, $(g_2^{-1}g_1)x=g_2^{-1}(g_1x)=g_2^{-1}x=x \Rightarrow g_2^{-1}g_1\in G_x$ hence $G_x$ is subgroup of $G$. \qed\\
\textbf{Notation:} \begin{enumerate}
\item For each $x\in S$ we denote $\cl{x}$ by $\mathcal{O}_x$ is called orbit of $x$, thus $$\mathcal{O}_x=\{gx:g\in G\}$$
\item $G_x$ is called stabilizer or isotropy group of $x.$
\end{enumerate}
\begin{proposition}
Let $G$ be a group act on a set $S$ then $|\mathcal{O}_x|=[G:G_x].$
\end{proposition}
\proof Let, $\sum=\{gG_x:g\in G\}$ and define the map \begin{align*}
\phi:\sum&\to \mathcal{O}_x\\
gG_x&\mapsto gx
\end{align*}
We claim that $\phi$ is well defined. Let $g_1G_x=g_2G_x \Rightarrow g_2^{-1}g_1\in G_x \Rightarrow (g_2^{-1}g_1)x=x \Rightarrow g_1x=g_2x.$ Hence $\phi$ is well defined and injective map. Let $gx\in \mathcal{O}_x \Rightarrow gx=\phi(gG_x)$ which imply $\phi$ is surjective. Therefore, $\phi$ is bijective and $|\mathcal{O}_x|=|\sum|=[G:G_x].$ \qed
\begin{corollary}
Let $G$ be a finite group. Then number of elements in conjugacy class of $x\in G$ is $[G:C_G(x)]$ which divides $|G|$ where $C_G(x)$ is centralizer of $x.$
\end{corollary}
\proof Recall $C_G(x)=\{g\in G: gx=xg\}$. Now we consider the conjugacy group action \begin{align*}
\cdot:G\times G&\to G\\
(g,h)&\mapsto ghg^{-1}
\end{align*}
Then $\mathcal{O}_x=\{gxg^{-1}:g\in G\}$ and $G_x=\{g\in G:gxg^{-1}=x\}=C_G(x).$ By the previous proposition $|\mathcal{O}_x|=[G:C_G(x)].$ Since $[G:C_G(x)]\big||G| \Rightarrow |\mathcal{O}_x|\big||G|.$ \qed
\begin{corollary}
Let $G$ be a finite group and $K\leq G$. Then the number of subgroups conjugates to $K$ is $[G:N_G(K)],$ which divides $|G|$ where $N_G(K)=\{g\in G:gK=Kg\}.$
\end{corollary}
\proof Let, $S$ be the set of all subgroup of $G$ and we define an action on $S$ as \begin{align*}
\cdot:G\times S&\to S\\
(g,K)&\mapsto gKg^{-1}
\end{align*}
Then $G_K=\{g\in G:gKg^{-1}=K\}=N_G(K)$ and $\mathcal{O}_K=\{gKg^{-1}:g\in G\}$ then $|\mathcal{O}_K|$ is the number of conjugates of $K$. So, $|\mathcal{O}_K|=[G:N_G(K)]$ therefore, number of conjugates of $K$ is $[N:N_G(K)]$ which divides $|G|$. \qed

\begin{proposition}
Let $G$ be a finite group then \begin{align*}
|G|=|Z(G)|+\displaystyle\sum_{[G:C_G(x_i)]\neq 1} [G:C_G(x_i)] 
\end{align*}
This equation is called the class equation of $G$.
\end{proposition}
\proof Consider the conjugacy group action \begin{align*}
\cdot:G\times G&\to G\\
(g,h)&\mapsto ghg^{-1}
\end{align*}
Let $\mathcal{O}_{x_1},\mathcal{O}_{x_2},\cdots,\mathcal{O}_{x_n}$ be distinct classes of $G$ then $\mathcal{O}_{x_1},\mathcal{O}_{x_2},\cdots,\mathcal{O}_{x_n}$ create a partition on $G$ hence $$|G|=\displaystyle\sum_{i=1}^n |\mathcal{O}_{x_i}|$$ Now, $\mathcal{O}_{x_i}=[G:C_G(x_i)]$ so $|G|=\displaystyle\sum_{i=1}^n [G:C_G(x_i)].$ Next we observe that $|\mathcal{O}_{x_i}|=1 \Leftrightarrow gxg^{-1}=x_i,\forall~g\in G \Leftrightarrow x_ig=gx_i,\forall~ g\in G \Leftrightarrow x_i\in Z(G)$ then we can write \begin{align*}
|G|&=\displaystyle\sum_{i=1}^n |\mathcal{O}_{x_i}|\\
 &=\displaystyle\sum_{|\mathcal{O}_{x_i}|=1} |\mathcal{O}_{x_i}|+\displaystyle\sum_{|\mathcal{O}_{x_i}|\neq 1}|\mathcal{O}_{x_i}|\\
 &=|Z(G)|+\displaystyle\sum_{|\mathcal{O}_{x_i}|\neq 1} |\mathcal{O}_{x_i}|\\
 &=|Z(G)|+\displaystyle\sum_{[G:C_G(x_i)]\neq 1} [G:C_G(x_i)]
\end{align*}
This completes the proof.  \qed\\
\subsubsection{Cayley's theorem}
 Let $S$ be a set and $S\neq \emptyset$, define \begin{align*}
A(S)=\{f:S\to S:f~\text{is bijection}\}
\end{align*}
Now, \begin{align*}
\circ: A(S)\times A(S)&\to A(S)\\
(f,g)&\mapsto f\circ g
\end{align*}
Then it is easy to check that $A(S)$ is a group under mapping composition.

\begin{theorem}
If a group $G$ acts on a set $S$ then this action induce a group homomorphism from $G$ to $A(S).$
\end{theorem}
\proof Let \begin{align*}
\cdot:G\times S&\to S\\
(g,s)&\to gs
\end{align*} be the group action.
Define \begin{align*}
\theta:G&\to A(S)\\
&g\mapsto \tau_g
\end{align*}
where \begin{align*}
\tau_g:S&\to S\\
s&\mapsto gs
\end{align*}
We will show that $\tau_g$ is a bijection. Let $\tau_g(s_1)=\tau_g(s_2) \Rightarrow gs_1=gs_2 \Rightarrow s_1=s_2.$ Let $s\in S$ then $s=\tau_g(g^{-1}s)$. Therefore, $\tau_g$ is a bijection and $\tau_g\in A(S).$ Next we will show that $\tau_{g_1}\circ \tau_{g_2}=\tau_{g_1g_2}.$ Let $\tau_{g_1}\circ \tau_{g_2}(s)=\tau_{g_1}(g_2s)=(g_1g_2)(s)=\tau_{g_1g_2}(s),\forall~s\in S.$ Hence, $\tau_{g_1}\circ \tau_{g_2}=\tau_{g_1g_2}.$ Now $\theta(g_1g_2)=\tau_{g_1g_2}=\tau_{g_1}\circ \tau_{g_2}=\theta(g_1)\theta(g_2).$ Therefore, $\theta$ is a group homomorphism. \qed
\begin{corollary}
If we take $S=G$ and the action is usual group action then $\theta:G\to A(G)$ is injection hence $G\leq A(G).$ In particular if $|G|=n$ then $G\hookrightarrow S_n.$ This is known as Cayley's theorem.
\end{corollary}
\proof $\ker \theta=\{g\in G:\tau_g=id\}.$ Now, $\tau_g=id \Leftrightarrow \tau_g(x)=id(x),\forall~x\in G.$ In particularly, if $x=e$ then $\tau_g(e)=e \Rightarrow ge=e \Rightarrow g=e \Rightarrow \ker \theta =\{id\}.$ Hence $\theta$ is injective and $G\leq A(G).$ In particularly if $|G|=n$ then $A(G)=S_n$ and $G\hookrightarrow S_n.$ \qed
\begin{corollary}
Let $G$ be a group, \begin{enumerate}
\item For each $g\in G$, conjugaction by $g$ induces a automorphism of $G,$ 
\item There is a homomorphism whose kernel is $Z(G).$
\end{enumerate}
\end{corollary}
\proof (1) Consider, \begin{align*}
\tau_g:G&\to G\\
x&\mapsto gxg^{-1}
\end{align*}
Clearly $\tau_g$ is a homomorphism. We will show that $\tau_g$ is an automorphism. Let $\ker \tau_g=\{x\in G:gxg^{-1}=e\}=\{e\}$ and let $x\in G$ then $x=\tau_g(g^{-1}xg)$ then $\tau_g$ is a bijection so it is an automorphism.\\
(2) Consider the map \begin{align*}
\theta:G&\to Aut(G)\\
g&\to \tau_g
\end{align*}
where $\tau_g:G\to G$ is defined by $\tau_g(x)=gxg^{-1}.$
Then $\theta$ is a group morphism and $\ker \theta=\{x\in G:\tau_g=id_G\}=\{x\in G:gxg^{-1}=x,\forall~x\in G\}=\{x\in G:xg=gx,\forall~x\in G\}=Z(G).$ \qed
\begin{obs}
By first isomorphism theorem $G/Z(G)\hookrightarrow Aut(G).$ Now, $\im \theta=\{\tau_g:g\in G$ where $\tau_g$ is defined above. $\tau_g$ is called inner automorphism and $\im \theta$ will be denoted by $Inn(G).$ So, $G/Z(G)\cong Inn(G)\leq Aut(G).$
\end{obs}
\begin{prop}
Let $H$ be a subgroup of $G$ and $G$ acts on the set of all left coset of $H$ by left translation i.e, \begin{align*}
\cdot:G\times S&\to S\\
(g,xH)&\mapsto gxH
\end{align*}
This action will induce a group homomorphism and $\ker \theta<H.$
\end{prop}
\proof Let $g\in \ker \theta$ then $\theta(g)=\tau_g=id \Rightarrow gxH=xH,\forall~g\in G$ [where $\tau_g:S\to S$ defined by $\tau_g(xH)=gxH$]. In particular, if we take $x=e$ then $gH=H \Rightarrow g\in H \Rightarrow \ker \theta<H.$ \qed
 
\begin{corollary}
Let $G$ be a finite group and $H<G$ with $[G:H]=n,$ and no non-trivial normal subgroup of $G$ is contained in $H$, then $G$ is isomorphic to a subgroup of  $S_n.$
\end{corollary}
\proof Let $S$ be the set of all left coset of $H$ then $|H|=[G:H]=n$. We consider \begin{align*}
\cdot:G\times S&\to S\\
(g,xH)&\mapsto gxH
\end{align*}
then this action will induce a group homomrphism $\theta:G\to A(S)$. By previous corollary, $\ker \theta<H$ but $\ker \theta$ is a normal subgroup of $G$ so by our hypothesis  $\ker \theta=\{e\}.$ Therefore, $\theta:G\to A(S)=S_n$ is injective and $G<S_n.$ \qed
\begin{corollary}
Let $G$ is a finite group and $H<G$ with $[G:H]=p,$ where $p$ is the least prime dividing $|G|$, then $H$ is normal in $G.$
\end{corollary}
\proof 
Let $S$ be the set of all left coset of $H$ in $G$ and $G$ acts on $S$ by left translation then this action will induce a group morphism $$\theta:G\to A(S)=S_p$$ and $\ker\theta <H<G.$ Therefore, $$[G:\ker\theta]=[G:H][H:\ker \theta].$$ By first isomorphism theorem $G/\ker\theta\hookrightarrow S_p$ and $$|G/\ker\theta|\big||S_p|=p! \Rightarrow [G:\ker\theta]\big|p! \Rightarrow [G:H][H:\ker\theta]\big|p! \Rightarrow [H:\ker\theta]\big|(p-1)!.$$
If $q$ is another prime such that $q|(p-1)!$ then $q<p$. Now, $[H:\ker\theta]\big|[G:\ker\theta]$ and $[G:\ker\theta]\big||G| \Rightarrow [H:\ker\theta]\big||G|.$ If $[H:\ker\theta]\neq 1$ then there exists a prime $q$ such that $q\big|[H:\ker\theta]\Rightarrow q\big||G|.$ But $q\big| [H:\ker\theta] \Rightarrow q<p$ and $p$ is the least prime dividing group's order which is a contradiction. Therefore, $[H:\ker\theta]=1 \Rightarrow H=\ker\theta$ and $H$ is a normal subgroup of $G.$ \qed

\begin{defn}
Let $G$ be a group acting on a set $X$. The action is said to be faithful or effective if $gx=x \Rightarrow g=e,\forall~x\in X.$ Equivalently the homomorphism from $G$ to $A(X)$ is injective.
\end{defn}
The action is called free (or semi-regular or fixed point free) if the statement $gx=x$ for some $x\in X \Rightarrow g=e$. In other words no non-trivial element of $G$ fixes a point of $X$ (A much stronger property than faithfulness). The action of any group on itself by left multiplication is free. A finite set may act faithfully on a set of size much smaller than its cardinality. For example the abelian 2-group $({\ZZ}/2{ZZ})^n$ acts faithfully on a set of size $2n.$
\begin{defn}
Let $G$ be a group acting on a set $S$. The action is said to be transitive if for any two elements $x,y\in S$ and there exists $g\in G$ such that $gx=y.$
\end{defn}
The action is simply transitive if it is both transitive and free. This means that given any $x,y\in S$ the element $g$ in definition is unique.
\begin{defn}
The action of $G$ on a set $S$ is called primitive if there is no partition of $S$ preserved by all the elements of $G$ apart from the trivial partition.
\end{defn}
\begin{prop}
\begin{enumerate}
\item The group action is transitive iff it has exactly one orbit i.e., $\exists~x\in S$ such that $\mathcal{O}_x=S.$
\item The action of $G$ on $S$ is free iff all stabilizers are trivial.
\item Let $G$ be a group acting on a set $S$ then this action induce a group morphism $\theta:G\to A(S)$ and $\ker f=\displaystyle\bigcap_{x\in S} G_x.$ If $\ker f=\{e\}$ then the action is faithful.
\item Let $x,y\in S$ and there exists $g\in G$ such that $y=gx$ then, $gG_xg^{-1}=G_y.$
\end{enumerate}
\end{prop}
\proof (4) Let $h\in G_y \Rightarrow hy=y \Rightarrow h(gx)=gx \Rightarrow (g^{-1}hg)x=x \Rightarrow g^{-1}hg\in G_x \Rightarrow h\in gG_xg^{-1} \Rightarrow G_y\subseteq gG_xg^{-1}.$ Let $p\in G_x \Rightarrow px=x \Rightarrow p(g^{-1}y)=g^{-1}y \Rightarrow (gpg^{-1})y=y \Rightarrow gpg^{-1}\in G_y \Rightarrow gG_xg^{-1}\subseteq G_y$ Therefore $G_y=gG_xg^{-1}.$ \qed



\subsubsection{Sylow's theorem}

\begin{lemma}[Key lemma]
If a group $H$ of order $p^n$ ($p$ is prime) acts on a finite set  $S$ and $$ S_0:=\{s\in S:hs=s,\forall~h\in H\},$$ then $$|S|\equiv |S_0|~(mod~p).$$
\end{lemma}
\proof If $H$ acts on $S$ then $|S|=\displaystyle\sum_{\text{distinct}} |\mathcal{O}_x|.$ Now, $|\mathcal{O}_x|=1 \Rightarrow \mathcal{O}_x=\{x\} \Rightarrow hx=x,\forall~h\in H \Rightarrow x\in S_0.$ Therefore we can write \begin{align*}
|S|&=\displaystyle\sum_{|\mathcal{O}_x|=1} |\mathcal{O}_x|+\displaystyle\sum_{|\mathcal{O}_x|>1} |\mathcal{O}_x|\\
&=|S_0|+\displaystyle\sum_{[H:H_x]>1} [H:H_x]\qquad\qquad[\text{as}~[H:H_x]=|\mathcal{O}_x|]
\end{align*}
Now, $1\neq [H:H_x]\big| |H|=p^n \Rightarrow p\big| [H:H_x]$. Therefore, $|S|=|S_0|+pl$ [where $p|\displaystyle\sum_{[H:H_x]>1} [H:H_x]] \Rightarrow |S|\equiv |S_0|~(mod~p).$ \qed 
\begin{theorem}[Cauchy]
If $G$ is a finite group whose order is divisible by  a prime $p$, then $G$ contains an element of order $p.$
\end{theorem}
\proof (Mckay)~ Let $$S:=\{(a_1,\cdots,a_p):a_i\in G,1\leq i\leq p, a_1\cdots a_p=e\}.$$ Then $|S|=|G|^{p-1}$ and $p\big| |G|.$ Now let us consider the following action \begin{align*}
{\ZZ}/p{\ZZ}\times S&\to S\\
(i,(a_1,\cdots,a_p))&\mapsto (a_{i+1},\cdots,a_p,a_1,\cdots,a_i)
\end{align*}
Note that $a_1\cdots a_p=e \Rightarrow a_2\cdots a_p=a_1^{-1} \Rightarrow a_2\cdots a_pa_1=e$ therefore, $a_{i+1}\cdots a_pa_1\cdots a_i=e$ hence $a_{i+1}\cdots a_pa_1\cdots a_i\in S.$ Clearly the above action defines a group action. \begin{align*}
S_0=\{(a_1,\cdots,a_p)\in S:(a_{i+1},\cdots,a_p,a_1,\cdots,a_i)=(a_1,\cdots,a_p),1\leq i\leq p\}
\end{align*}
Therefore, $(a_1,\cdots,a_p)\in S_0 \Leftrightarrow a_1=a_2=\cdots=a_p$ and $a_1\cdots a_p=e$. Since, $(e,\cdots,e)\in S_0, S_0\neq \emptyset$. Therefore, by Key lemma $|S|\equiv|S_0|~(mod~p)$ and $p\big| |S| \Rightarrow |S_0|\equiv 0~(mod~p)$. Again, $|S_0|\neq 0 \Rightarrow |S_0|>1$ then there exists $a\neq e$ such that $(a,\cdots,a)\in S_0 \Rightarrow a^p=e.$ \qed

\begin{defn}
A group in which every element has order $p^n$ (for some $n\geq 0,$ for fixed prime $p$) is called a $p-$group. If $G$ is a group and $H<G$, such that $H$ is a $p-$group, then we call $H$ is a $p-$subgroup of $G.$
\end{defn}
\begin{corollary}
A finite group $G$ is a $p-$group iff $|G|$ is a power of $p.$
\end{corollary}
\proof  Suppose, $G$ is a finite $p-$group, if $|G|=p^lm,$ where $m>1$ and $p\nmid m,$ then there exists a prime $q\neq p$ such that $q|m \Rightarrow q\big| |G|.$ By Cauchy's theorem there is an element $a\in G$ such that $a^q=e$ which is a contradiction. Therefore, $|G|=p^n$ for some $n>0.$ Conversely, if $|G|=p^m$ for some $m>0$ then for any $a\in G$ $o(a)\big| |G|$ i.e., $a^k=e$ for some $0\leq k\leq m.$ \qed
\begin{lemma}
Let $H$ be a $p-$subgroup of $G$ then $$[N_G(H):H]\equiv [G:H]~(mod~p).$$
\end{lemma}
\proof Let $S$ be the set of all left coset of $H$ in $G$ and we consider the action \begin{align*}
\cdot:H\times S&\to S\\
(h,xH)&\mapsto hxH
\end{align*}
Thus, \begin{align*}
S_0&=\{xH\in S:hxH=xH,\forall~h\in H\}\\
&=\{xH\in S:x^{-1}hx\in H,\forall~h\in H\}\\
&=\{xH\in S:x^{-1}Hx=H\}\qquad\qquad[x^{-1}Hx=H \Leftrightarrow x\in N_G(H) \Rightarrow xH\in N_G(H)/H]\\
&=N_G(H)/H
\end{align*}
By Key lemma, $|S|\equiv |S_0|~(mod~p) \Rightarrow [G:H]\equiv [N_G(H):H]~(mod~p).$ \qed

\begin{corollary}
If $H$ is a $p-$subgroup of a finite group $G$ such that $p\big| [G:H]$ then $N_G(H)\neq H.$
\end{corollary}
\proof If $N_G(H)=H\Leftrightarrow [N_G(H):H]=1.$ But $[G:H]\equiv [N_G(H):H]~(mod~p).$ If $p\big| [G:H]$ then $0\equiv [N_G(H):H]~(mod~p) \Rightarrow N_G(H)\neq H.$ \qed

\begin{theorem}[First Sylow theorem]
Let $G$ be a group of $p^nm$ with $n\geq 1$, $p$ is prime and $\gcd(p,m)=1.$ Then $G$ contains a subgroup of order $p^i$ for each $1\leq i\leq n$ and every subgroup of $G$ of order $p^i~ (i<n)$ is normal in $p^{i+1}$ order subgroup.
\end{theorem}
\proof Since $p\big| |G|$ by Cauchy's theorem there is an element $a\in G$ such that $a^p=e$. We take $H_1=\gen{a}$ then $|H_1|=p.$ Let us assume by induction that there exists a group $H_i$ of order $p^i$ where $i<n.$ Since, $i<n$, $p\big| [G:H_i]=p^nm/p^i=p^{n-i}m$ therefore by previous corollary $N_G(H_i)\neq H_i$. As $p\big| |N_G(H_i)/H_i|$ by Cauchy's theorem $\exists~xH_i\in N_G(H_i)/H_i$ such that $o(xH_i)=p.$ Take, $H_i<H_{i+1}<N_G(H_i)$ such that $H_{i+1}/H_i=\gen{xH_i}$ [Note that $H_{i+1}<N_G(H_i) \Rightarrow H_i\mathrel{\unlhd} H_{i+1}$]. Therefore, $|H_{i+1}|=|H_i||H_{i+1}/H_i|=p^i\cdot p=p^{i+1}$ and $H_i\mathrel{\unlhd} H_{i+1}.$ \qed

\begin{corollary}
The center $Z(G)$ of a non-trivial finite $p-$group $G$ contains more than one element.
\end{corollary}
\proof We know that $$|G|=|Z(G)|+\displaystyle\sum [G:C_G(x_i)].$$ Since each $[G:C_G(x_i)]>1$ and divides $|G|=p^n~(n\geq 1)$, $p$ divides each $[G:C_G(x_i)]$ and $|G|$ and therefore $p\big| |Z(G)|$. By Cauchy's theorem $\exists~a\neq e$ such that $a^p=e$ and $a\in Z(G).$ \qed
\begin{defn}
A subgroup $P$ of a group $G$ said to be Sylow $p-$subgroup if $P$ is maximal $p-$subgroup of $G.$
\end{defn}
By Sylow's first theorem Sylow $p-$subgroup exists and order is $p^n$ where $|G|=p^nm,$ with $\gcd(p,m)=1.$
\begin{note}
Converse of Lagrange's theorem is partially true by Sylow's theorem. Converse of Lagrange theorem is not true in general. For example there is no group of order 6 in $A_4.$
\end{note}
\begin{corollary}
Let $G$ be a group of order $p^nm,\gcd(p,m)=1$ where $p\geq 1$ and $p$ is prime. Let $H$ be a $p-$subgroup of $G.$
\begin{enumerate}
\item $H$ is a Sylow $p-$subgroup of $G$ iff $|H|=p^n.$
\item Every conjugate of a Sylow $p-$subgroup is a Sylow $p-$subgroup.
\item If there is only one Sylow $p-$subgroup $P$, then $P\mathrel{\unlhd} G.$
\end{enumerate}

\end{corollary}
\proof (1) Follows form Sylow's first theorem.\\
(2) $|P|=|x^{-1}Px|=p^n,x\in G.$ Then $x^{-1}Px$ is also a Sylow $p-$subgroup by (1).\\
(3) If there exists only one Sylow $p-$subgroup $P,$ then $x^{-1}Px=P,\forall~x\in G \Rightarrow P\mathrel{\unlhd} G.$ \qed

\begin{theorem}[Second Sylow theorem]
If $H$ is a $p-$subgroup of a finite group $G$ and $P$ is any Sylow $p-$subgroup, then there exists $x\in G$ such that $H<xPx^{-1}.$ In particular any two Sylow $p-$subgroups are conjugate.
\end{theorem}
\proof Let $S$ be the set of all left coset of $P$ in $G$ and we consider the following group action \begin{align*}
H\times S&\to S\\
(h,xP)&\mapsto hxP
\end{align*}
Then $S_0=\{xP\in S:hxP=xP,\forall~h\in H\}$ and by Key lemma $|S|\equiv |S_0|~(mod~p)$ but $|S|=[G:H]=m$ and $p\nmid m \Rightarrow |S_0|\not\equiv 0~(mod~p) \Rightarrow |S_0|\neq 0.$
Now, \begin{align*}
xP\in S_0 \Leftrightarrow hxP=xP,\forall~h\in H
\Leftrightarrow x^{-1}hx\in P,\forall~h\in H
\Leftrightarrow x^{-1}Hx\in P
\Leftrightarrow x^{-1}Hx< P.
\end{align*}
Since, $|S_0|\neq 0 \Rightarrow \exists~xP\in S_0$ such that $H<xPx^{-1}$. If $H$ is a Sylow $p-$subgroup then $|H|=|P|=|xPx^{-1}|$ hence $H=xPx^{-1}.$ \qed
\begin{theorem}[Third Sylow theorem]
If $G$ is a finite group and $p$ is a prime, then number of Sylow $p-$subgroup of $G$ divides $|G|$ and is of the form $kp+1$ for some $k\geq 0.$
\end{theorem}
\proof By second Sylow theorem any two Sylow $p-$subgroups are conjugate. So we need to find number of conjugate subgroups of $P$ (where $P$ is a Sylow subgroup) and that is $[G:N_G(P)]$. We know that $[G;N_G(P)]\big| |G|$ by Corollary 4.14. Let $S$ be the set of all Sylow $p-$ subgroups of $G$ and consider the following group action \begin{align*}
P\times S&\to S\\
(x,K)&\mapsto xKx^{-1}
\end{align*}
Therefore, $S_0=\{K\in S:xKx^{-1}=K,\forall~x\in P\}$ and by Key lemma $|S|\equiv |S_0|~(mod~p).$ Let $Q\in S_0 \Leftrightarrow xQx^{-1}=Q,\forall~x\in P \Leftrightarrow P<N_G(Q)$ but both $P$ and $Q$ are Sylow $p-$subgroups of $G$ and hence of $N_G(Q)$ and they are conjugate in $N_G(Q).$ Since, $Q\mathrel{\unlhd} N_G(Q) \Rightarrow P=Q$ therefore, $S_0=\{P\}$ and $|S_0|=1 \Rightarrow |S|=1+kp.$ \qed


\begin{theorem}
If $P$ is a Sylow $p-$subgroup of a group $G$, then $N_G(N_G(P))=N_G(P).$
\end{theorem}
\proof Every conjugate of $P$ is a Sylow $p-$subgroup of $G$ and of any subgroup of $G$ that contains it. Since $P$ is normal in $N_G(P)$, $P$ is the only Sylow $p-$subgroup of $N_G(P).$ Therefore, $x\in N_G(N_G(P)) \Rightarrow xN_G(P)x^{-1}=N_G(P) \Rightarrow xPx^{-1}<N_G(P) \Rightarrow xPx^{-1}=P \Rightarrow x\in N_G(P).$ Hence, $N_G(N_G(P))<N_G(P).$ The other inclusion is obvious. \qed




\begin{qns}
Let $H,K<G$. We define $HK:=\{hk:h\in H,k\in K\}$. If either $H\unlhd G$ or $K\unlhd G$ then $HK$ is a subgroup of $G.$ Also show that $$o(HK)=\dfrac{o(H)o(K)}{o(H\cap K)}.$$ If $H,K\unlhd G$ then $HK\cong H\times K$ provided $H\cap K=\{e\}.$
\end{qns}
Ans. Given $H,K<G$ and $K\unlhd ~~G$ then $gK=Kg$ for all $g\in G$, in particularly $hK=Kh,\forall~h\in H.$ Now, $(h_1k_1)^{-1}h_2k_2=k_1^{-1}h_1^{-1}h_2k_2=h_1^{-1}k_1^{-1}h_2k_2=h_1^{-1}h_2k_1^{-1}k_2 \in HK$ (as $k_1^{-1}k_2\in K,$ and $h_1^{-1}h_2\in H$). Therefore $HK$ is a subgroup of $G$. Define a map \begin{align*}
\theta: H\times K&\to HK\\
(h,k)&\mapsto hk
\end{align*}
where $H,K\unlhd G$ and $H\cap K=\{e\}.$ Consider the commutator $(hkh^{-1})k^{-1}=h(kh^{-1}k^{-1}).$ Since $K\unlhd G$ left hand side is in $K$, similarly $H\unlhd G$, right hand side is in $H$. As $H\cap K=\{e\} \Rightarrow hkh^{-1}k^{-1}=e \Rightarrow hk=kh$ therefore, $HK=KH.$  We will show that $\theta$ is a homomorphism. $\theta((h_1,k_1)(h_2,k_2))=\theta((h_1h_2,k_1k_2))=h_1h_2k_1k_2=h_1k_1h_2k_2=\theta((h_1,k_1))\theta((h_2,k_2)).$ Our next job is to show that $\theta$ is injective. Let $\theta(h_1,k_1)=\theta(h_2,k_2)\Rightarrow h_1k_1=h_2k_2 \Rightarrow h_2^{-1}h_1=k_2k_1^{-1}.$ Since $H\cap K=\{e\}$ we have $h_2^{-1}h_1=e$ and $k_2k_1^{-1}=e$ which give $h_1=h_2$ and $k_1=k_2$ thus $\theta$ is injective. For any $hk\in HK$ we have $(h,k)\in H\times K$ such that $\theta((h,k))=hk \Rightarrow\theta$ is surjective hence bijective. Therefore, $H\times K\cong HK.$ 

\begin{qns}
Every group of order $p^2$ is abelian where $p$ is prime.
\end{qns}
Ans. Let $G$ be a group of order $p^2$ and $Z(G)$ be its center. Then $|Z(G)|=p$ or $p^2$. If $|Z(G)=p^2$ then $G=Z(G)$ hence $G$ is abelian. If $|Z(G)|=p$ then $|G/Z(G)|=p$ which imply $G/Z(G)$ is cyclic hence $G$ is abelian. Therefore, any group of order $p^2$ is abelian.


\begin{qns}
Classify all the groups of order $2p,$ where $p$ is prime.
\end{qns}
Ans. \textbf{Case 1.} If $p\neq 2.$ Let $|G|=2p,$ by Cauchy's theorem there exists an element $e\neq a\in G$ such that $o(a)=p$. Let $H=\gen{a}$ then $[G:H]=2$ hence $H\unlhd G.$ Again by Cauchy's theorem there exists an element $b\in G$ such that $o(b)=2$. Let $K=\gen{b}.$ As $H\unlhd G$ and $H=\gen{a} \Rightarrow bab^{-1}\in H.$ Let $bab^{-1}=a^i$ (say), where $1\leq i\leq p-1.$ Now, $a=b^2ab^{-2}=b(bab^{-1})b^{-1}=ba^{i}b^{-1}=(bab^{-1})^i=a^{i^2} \Rightarrow a^{i^2-1}=e.$ Since, $o(a)=p$ we have $p\big| i^2-1\Rightarrow p\big| (i+1)(i-1) \Rightarrow p\big| i+1$ or $p\big| i-1$ (as $p$ is prime). If $p\big| i-1 \Rightarrow i=1$ if $p\big| i+1 \Rightarrow i=p-1.$ For $i=1$ we have $bab^{-1}=a \Rightarrow ba=ab \Rightarrow o(a)o(b)=o(ab)=2p$ (as $\gcd(o(a),o(b))=1$). Now, $H\unlhd G \Rightarrow HK<G$ and $|HK|=\dfrac{|H||K|}{|H\cap K|}.$ Now we will show that $H\cap K=\{e\}$. If not let $x\in H\cap K \Rightarrow o(x)\big| |H|$ as well as $o(x)\big| |K|$ which is impossible unless $o(x)=1.$ Hence $H$ and $K$ intersect trivially. Therefore, $|HK|=2p $ and $HK=G.$ \begin{align}
G=HK&=\gen{a,b:a^p=b^2=e,ba=ab}\\
G=HK&=\gen{a,b:a^p=b^2=e,bab^{-1}=a^i}
\end{align}
From (1) we get $G$ is abelian and $o(ab)=2p$ therefore, $G$ is cyclic, hence $G\cong {\ZZ}/2p{\ZZ}.$ Form (2) $bab^{-1}=a^i=a^{p-1}=a^{-1} \Rightarrow ba=a^{-1}b$. Hence, $G\cong D_{2p}.$\\
\textbf{Case 2.} When $p=2$, $|G|=2\cdot 2=2^2$. Therefore, any group of order $p^2$ is abelian. If there exists an element $a\in G$ such that $o(a)=4$ then $G\cong {\ZZ}/4{\ZZ}.$ If there does not exists $a\in G$ such that $o(a)=4$ then $o(a)=2,\forall~a\in G\setminus \{e\}$ (by Lagrange theorem). Therefore, $G\cong {\ZZ}/2{\ZZ}\times {\ZZ}/2{\ZZ}.$
\begin{qns}
Classify all the groups of order $pq$ where $p,q$ are primes and $p<q.$
\end{qns}
Ans. By Cauchy's theorem $\exists~a,b\in G$ such that $o(a)=p$ and $o(b)=q.$ Let $H=\gen{a}$ and $K=\gen{b}.$ Since, $[G:K]=p$, least prime dividing the group's order, we have $K\unlhd G.$ Thus $HK<G$ and $|HK|=\dfrac{|H||K|}{|H\cap K|}.$ By similar reason $H\cap K=\{e\}$ hence $|HK|=pq$ and therefore, $HK=G.$ Since, $K\unlhd G,$ $aba^{-1}\in K \Rightarrow aba^{-1}=b^i$ (say) where $1\leq i\leq q-1.$ Now, $b=a^pba^{-p}=a^{p-1}(aba^{-1})a^{-(p-1)}=a^{p-1}b^ia^{-(p-1)}=\cdots=b^{i^p} \Rightarrow b^{i^p-1}=e \Rightarrow q\big| i^p-1 \Rightarrow i^p\equiv 1~(mod~q).$\\
\textbf{Case 1.} If $p\nmid q-1$. Then number of Sylow $p-$subgroup is $1+kp$ with $k\geq 0$ and $1+kp\big| q.$ Since $q$ is prime, $1+kp=1$ or $1+kp=q$. If $1+kp=1$ then number of Sylow $p-$subgroup is 1 and $H\unlhd G.$ Therefore, $G\cong H\times K={\ZZ}/p{\ZZ}\times {\ZZ}/q{\ZZ}.$ Since, $\gcd(p,q)=1 \Rightarrow G={\ZZ}/p{\ZZ}\times {\ZZ}/q{\ZZ}\cong {\ZZ}/pq{\ZZ}.$ If $1+kp=q \Rightarrow kp=q-1 \Rightarrow p\big| q-1$ which contradicts our assumption.


\begin{qns}
Find a normal subgroup of a group of order $12.$
\end{qns}
\begin{qns}
Show that there exists a normal subgroup in a group of order $pqr$ where $p<q<r$ and $p,q,r$ are distinct primes.
\end{qns}

\begin{qns}
Every group of even order has an element of order $2.$
\end{qns}
Ans. Let $G$ be a group such that $|G|=2n$. We construct a set $A=\{x\in G:x=x^{-1}\}$. Then $A\neq \emptyset$ as $e\in A.$ Suppose $|A|=1$ then $|G\setminus A|=2n-1$ and every element of $G\setminus A$ has inverse different from itself thus $G\setminus A$ has even number of element which is impossible therefore $A$ must have $e\neq x\in G$ such that $x=x^{-1} \Rightarrow x^2=e.$ 
\begin{qns}
Show that in any group of order $p^2$ there exists a normal subgroup of order $p$ and also show that normal subgroup of order $p$ lies at the center of the group.
\end{qns}
Ans. Since any group of order $p^2$ is abelian we have $G=Z(G)$ abd by Cauchy's theorem $\exists~a(\neq e)\in G$ such that $a^p=e$. Consider $H=\gen{a}$ then $H\unlhd G$ since $G$ is abelian and $H$ lies at the center of $G.$ 
\begin{qns}
Show that in any group of order $2p$ there exists a normal subgroup of order $p.$
\end{qns}
Ans. By Cauchy's theorem there exists $a\in G$ such that $a^p=e$ and we consider $H=\gen{a}$ then $[G:H]=2$ which imply $H\unlhd G.$
\begin{qns}
Let $G$ be a finite abelian group of order $n$ and $m|n$, then $G$ has a subgroup of order $m.$
\end{qns}
Ans. Given $|G|=n=p_1^{\alpha_1}\cdots p_r^{\alpha_r}$ where $p_i$'s are distinct primes. By Sylow's first theorem we get Sylow $p_i-$subgroup and we denote it by $H_i$ then $|H_i|=p_i^{\alpha_i}.$ Since $G$ is abelian and $H_i\cap H_j=\{e\}$ with $H_iH_j=H_jH_i,\forall~i\neq j$. Therefore, $G\cong H_1\times \cdots\times H_r=H_1\cdots H_r.$ Since $m|n$ let us assume that $m=p_1^{\beta_1}\cdots p_r^{\beta_r}$ where $0\leq \beta_i\leq \alpha_i$ with $1\leq i\leq r.$ Again by Sylow's first theorem we have subgroups of order \\
\\
\\
$G$ is a finite abelian group of order $n=p_1^{\alpha_1}\cdots p_r^{\alpha_r}$ then $G\cong \displaystyle\prod_{i=1}^r {\ZZ}/p_i^{\alpha_i}{\ZZ}.$




\begin{qns}

\end{qns}
\begin{qns}

\end{qns}
\begin{qns}

\end{qns}






















































































































































































































\newpage
%\chapter{Ring Theory}
\section{Ring Theory}
\begin{definition}
Let $R$ be an non empty set equipped with two binary operation $+$ and $\cdot $ then $(R,+,\cdot )$ is said to be ring if 
\newline $(i)$ $(R,+)$ is a commutative group
\newline $(ii)$ $(R,\cdot )$ is a semi-group and
\newline $(iii)$ $a\cdot (b+c)=a\cdot b+a\cdot c$ $\forall a,b,c \in R$
\newline R is said to be commutative ring with identity if $a\cdot b=b\cdot a$ $\forall a,b \in R$ and $1 \in R.$
\end{definition}
For rest of this discussion our assumption is $R$ is a commutative ring with identity unless and otherwise stated.
\begin{remark}
If $G$ is an arbitrary abelian group then $G$ can be made into a ring by defining $ab=0$ for all $a,b\in G.$ This ring has no multiplicative identity.
\end{remark}
\subsection{Sub-ring}
\begin{definition}
Let $S \subseteq R,$ $S$ is a subring of $(R,+,\cdot)$ if $(S,+|_{S\times S}: S\times S \rightarrow S,\cdot |_{S\times S}: S\times S \rightarrow S)$ is a ring.
\end{definition}
\subsection{Ideals}
\begin{definition}
Let $I$ be an non-empty subset of $R.$ $I$ is said to be ideal of $R$ if $a,b\in I$ $\Rightarrow a+b \in I$ and $r \in R,$ $a\in I$ $\Rightarrow ra \in I.$
\end{definition}
\begin{obs}
Every ideal is a subring of $R$ but not every subring is ideal. For example ${\ZZ}$ is a subring of ${\QQ}$ but not ideal. Consider the subring $R[x^2]$ of $R[x]$ which is not an ideal.
\end{obs}
Note that, if $1_{R} \in I$ then $I=R.$
Let $A\subseteq R$ be a subset of $R$. Ideal generated by $A$ means intersection of all ideals containing $A$ say $I.$ Then $A\subseteq I.$ As, $a\in A\Rightarrow a\in I$ and $r\in R,a\in A \Rightarrow ra\in I$ then we have $\{r_1a_1+\cdots +r_na_n|r_i\in R,a_i\in A\}\subseteq I.$ We claim $I=\{r_1a_1+\cdots +r_na_n|r_i\in R,a_i\in A\}$.
\proof Suppose, $x=a_1r_1+\cdots +a_nr_n$ and $y=b_1r'_1+\cdots +b_nr'_n$ then $x+y\in \{r_1a_1+\cdots +r_na_n|r_i\in R,a_i\in A\}.$ Pick, $r\in R$ and $x=a_1r_1+\cdots +a_nr_n\in \{r_1a_1+\cdots +r_na_n|r_i\in R,a_i\in A\}$ then $rx\in \{r_1a_1+\cdots +r_na_n|r_i\in R,a_i\in A\}.$ Therefore, $\{r_1a_1+\cdots +r_na_n|r_i\in R,a_i\in A\}$ is an ideal. Clearly, $A\in \{r_1a_1+\cdots +r_na_n|r_i\in R,a_i\in A\}.$            %imcomplete proof


\begin{definition}
Let $R$ be a ring. An ideal $I$ is said to be finitely generated if it is generated by finite number of elements of $R$ i.e, if $a_1,\cdots ,a_n\in R$ then finitely generated ideal generated by $a_1,\cdots ,a_n$ is $$I=\langle a_1,\cdots ,a_n\rangle=\left\lbrace \displaystyle\sum_{i=1}^n a_ir_i|~r_i\in R\right\rbrace $$
\end{definition}



\begin{definition}
Let $R$ be a ring. An ideal is said to be principal ideal if it is generated by a single element i.e, if $a\in R$ principal ideal generated by $\langle a\rangle=\{ar|~r\in R\}.$
\end{definition}
\begin{definition}
Let $R$ be a ring. An ideal $I\subseteq R$ is said to be prime ideal of $R$ if $ab\in I$ $\Rightarrow a\in I$ or $b \in I.$ Equivalently $I$ is a prime ideal if $a \notin I, b\notin I$ but $ab\in I.$
\newline Define, $specR=\{P\subseteq R|$ $P$ is a prime ideal of $R \}$ 
\end{definition}
\begin{definition}
Let $R$ be a ring. An ideal $M\subseteq R$ is said to be maximal ideal of $R$ if $M\subseteq I$ where $I$ is an ideal implies either $M=I$ or $I=R.$
\newline Define, $maxspecR=\{M\subseteq R|$ $M$ is a maximal ideal of $R \}$ 
\end{definition}
\begin{theorem}
Let $R$ be a commutative ring with identity then $maxspecR$ $\neq \emptyset.$
\end{theorem}
\proof Let $\Sigma=\{I\subseteq R|$ $I$ is an ideal of $R$ and $I\neq R$ \}, then $\Sigma\neq \emptyset$ as $(0)\in \Sigma.$ Let use consider the partial order relation, set inclusion $\subseteq$ on $\Sigma.$ Let $\{I_{\lambda} \}_{\lambda \in \Lambda}$ be a chain in $\Sigma.$ We claim ${\displaystyle\bigcup_{\lambda \in \Lambda}I_{\lambda}}$ is an ideal in $\Sigma.$ Let $x,y \in {\displaystyle\bigcup_{\lambda \in \Lambda}I_{\lambda}}$ then $x\in I_{\alpha}$ and $y\in I_{\beta},$ for some $\alpha, \beta \in \Lambda.$ Since $\{I_{\lambda} \}_{\lambda \in \Lambda}$ is a chain, either $I_{\alpha} \subseteq I_{\beta}$ $\Rightarrow x,y \in I_{\beta}$ or $I_{\beta} \subseteq I_{\alpha}$ $\Rightarrow x,y \in I_{\alpha}.$ Then $x+y\in I_{\alpha}$ or $I_{\beta}$ $\Rightarrow x+y \in {\displaystyle\bigcup_{\lambda \in \Lambda}I_{\lambda}}.$ Choose $r\in R$ and $x\in I_{i}$ then $rx\in I_{i}$ $\Rightarrow rx\in {\displaystyle\bigcup_{\lambda \in \Lambda}I_{\lambda}}.$ Hence ${\displaystyle\bigcup_{\lambda \in \Lambda}I_{\lambda}}$ is an ideal in R. Now, ${\displaystyle\bigcup_{\lambda \in \Lambda}I_{\lambda}}\neq R$ because if ${\displaystyle\bigcup_{\lambda \in \Lambda}I_{\lambda}} =R$ then $I_{i}=R$ for some $i\in \Lambda$ (contradiction). Therefore ${\displaystyle\bigcup_{\lambda \in \Lambda}I_{\lambda}}$ is an upper bound for a chain in $\Sigma.$ By Zorn's Lemma $\Sigma$ has an maximal element say $m$ and $m\neq R$ since $m \in \Sigma.$ If $m\in I$ $I$ is an ideal then either $I=m$ or $I=R.$ Then $m\in maxspecR$ $\Rightarrow maxspecR \neq \emptyset.$ \qed 

\begin{definition}
A non-zero element $r \in R$ is said to be zero divisor if $\exists$ $s(\neq 0) \in R$ such that $rs=0.$
\end{definition}
\begin{example}
In the ring ${\ZZ}/6{\ZZ},$ $\bar{2}$ has zero divisor. More generally ${\ZZ}/n{\ZZ}$ has zero divisor where $n$ is not a prime number.
\end{example}
\begin{definition}
A ring $R$ is said to be integral domain if $R$ does not contain any zero divisor i.e., if $ab=0$ ($a,b \in R$) $\Rightarrow$ either $a=o$ or $b=0.$ \\
Alternatively, A ring $R$ is said to be integral domain if $\gen{0}$ is the prime ideal of $R.$
\end{definition}
\begin{example}
The ring of integers ${\ZZ}$ is an integral domain. For any prime $p,$ ${\ZZ}/p{\ZZ}$ is an integral domain.
\end{example}
\begin{remark*}
An integral domain is also called entire ring(S Lang).
\end{remark*}
\begin{definition}
A ring is said to be field if $\gen{0}$ is the only maximal ideal of $R.$ In other word, $R$ has no non trivial proper maximal ideal.
\end{definition}
\begin{example}
${\QQ},$ ${\RR},$ ${\CC}$.
\end{example}
\begin{lemma}
A finite integral domain is a field.
\end{lemma}
\proof Let $0\neq a_1\in R,$ if $a_1=1$ then it has inverse. Otherwise let $R=\{1,0,a_1,\cdots ,a_n\}$ (i.e., $a_1\neq 1$). Consider, 
\begin{align}
\{a_1,0,a_1^2,\cdots a_1a_n\}\subseteq R
\end{align}
then $a_1a_i\neq 0,1\leq i\leq n$ ($R$ is integral domain). Let $a_i\neq a_j$ and $a_1a_i=a_1a_j$ gives $a_i=a_j$ therefore, $a_i\neq a_j \Rightarrow a_1a_i\neq a_1=a_i.$ Form (1) $$\{a_1,0,a_1^2,\cdots ,a_1a_n\}=R=\{1,0,\cdots ,a_n\}$$ $1\in R$ and $1\neq 0,1\neq a_1$ then $\exists~1\leq i\leq n$ such that $a_1a_i=1 \Rightarrow a_1$ has multiplicative inverse, hence $R$ is a field. \qed
\begin{ex}
Show that $({\ZZ}/p{\ZZ},+,\cdot)$ is an integral domain hence a field for any prime $p.$ 
\end{ex}

\begin{ex}
Show that above definition of field implies that every elements of field is unit.
\end{ex}

\begin{defn}
Let $R$ be a ring then $R$ is said to be simple if $R$ has no proper non trivial ideal i.e., only ideals of $R$ is only zero ideal and $R$ itself. In particular a commutative ring is simple iff it is a field.
\end{defn}
\begin{obs}
Center of a simple ring is necessarily a field.
\end{obs}
\begin{definition}
Let $R,S$ be rings. $f:R \rightarrow S$ is said to be ring homomorphism if 
\newline $(i)$ $f(a+b)=f(a)+f(b)$
\newline $(ii)$ $f(ab)=f(a)f(b)$
\newline $(iii)$ $f(1_{R})=1_{S}$
\end{definition}
\begin{definition}
Let $f: R \rightarrow S$ be a ring homomorphism we define $Kerf=\{r \in R|f(r)=0 \}$ and $Imf=\{s \in S|f(r)=S \}$
\end{definition}
Check that $Kerf, Imf$ is subring of $R$ and $S$ respectively. $Kerf$ is an ideal of $R.$ Pick $x,y \in Kerf \subseteq R$ then $f(x)=0=f(y).$ Now $f(x+y)=f(x)+f(y)=0$ $\Rightarrow x+y \in Kerf.$ Pick $s,s' \in Imf \subseteq S$ then $f(a)=s,f(b)=s'$ for some $a,b \in R.$ Now $f(a+b)=f(a)+f(b)=s+s'$ $\Rightarrow s+s' \in Imf.$ $x,y \in Kerf$ then $f(x)=0=f(y).$ Now $f(x+y)=f(x)+f(y)=0$ $\Rightarrow x+y \in Kerf.$ Pick $r \in R, x \in Kerf,$ $f(rx)=f(r)f(x)=0$ $\Rightarrow rx\in Kerf.$ 
\newline Check that: Let $f: R \rightarrow S$ be a ring homomorphism $J\subseteq S $ is an ideal of $S,$ then $f^{-1}(J)=\{x\in R| f(x)\in J \}$ is an ideal of $R.$ Pick $a,b \in f^{-1}(J)$ then $f(a)=s, f(b)=s'$ for some $p,q \in S$ $\Rightarrow f(a+b=f(a)+f(b)=s+s'$ $\Rightarrow f^{-1}(s+s')=a+b \in f^{-1}(J).$ Take $r \in R, a \in f^{-1}(J)$ $\Rightarrow f(r)=r'$ for some $r'$ and $f(a)=p$ $\Rightarrow f(ra)=f(r)f(a)=r'p \in J$ since $J$ is an ideal $\Rightarrow ra \in f^{-1}(J).$ Hence $f^{-1}(J)$ is an ideal of $R.$
\newline \textbf{Notation:}  $J^{c}=f^{-1}(J)$ is called the contraction of $J.$
\newline Let $f: R \rightarrow S$ be a ring homomorphism, $I\subseteq R$ is an ideal of $R,$ then $f(I)$ is need not be an ideal of $S.$ $\phi: {\ZZ} \hookrightarrow {\QQ}$ by $x \mapsto x.$ $2{\ZZ}$ is an ideal of ${\ZZ}$ but not of ${\QQ}$ as $2 \in 2{\ZZ}$ and $\dfrac{1}{2}\in {\QQ}$ $\Rightarrow 2\cdot \dfrac{1}{2}=1 \notin 2{\ZZ}.$
\newline \textbf{Observation:} If $\phi :R \rightarrow S$ is a surjective ring homomorphism, then $\phi(I)$ is an ideal of $S$ where $I\subseteq R$ is an ideal. Let $a,b \in I.$ $\phi(a)=p, \phi(b)=q$ Now, $\phi(a+b)=\phi(a)+\phi(b)=p+q.$ Let $r' \in R$ and $\phi(a) \in \phi(I)$(then $a\in I$), $\exists$ $r\in R$ such that $\phi(r)=r'$ $\Rightarrow r'\phi(a)=\phi(r)\phi(a)=\phi(ra) \in \phi(I).$ Therefore, $\phi(I)$ is an ideal of $S.$
\newline \textbf{Observation} $(i)$ $I,J$ be two ideals of $R.$ $I+J=\{x+y|x\in I, y\in J \}.$ $I+J$ is an ideal of $R.$
\newline  $(ii)$ $I,J$ be two ideals of $R.$ $IJ= \left\lbrace \displaystyle\sum_{\substack{\text{finite}\\ \text{sum}}}xy: x\in I, y\in J \right\rbrace.$ $IJ$ is an ideal of $R.$
\newline  $(iii)$ $I\cap J$ is an ideal of $R.$
\newline  $(iv)$ $(I:J)=\{x \in R|$ $xJ\subseteq I \}$ is an ideal of $R.$
\newline  Let $f: R \rightarrow S$ be a ring homomorphism. $I\subseteq R,$ $J\subseteq S$ be ideals.
\newline  $(v)$ $f(I)$ is need not be an ideal of $S$ but $I^{e}:=f(I)S=\left\lbrace \displaystyle\sum_{\substack{\text{finite}\\ \text{sum}}}f(r)s:~r\in I, s\in S \right\rbrace $
\newline  $(vi)$ If $f$ is surjective then $f(I)=I^{e}$ hence $f(I)$ is an ideal of $S.$
\newline \textbf{Notation:} $f(I)=I^{e}$ is called the expansion of $I$
\newline  $(vii)$ If $P \in$ $specS$ then $f^{-1}(P) \in$ $specR$ hence $f: R \rightarrow S$ induces a map $f_{\ast}:$  $specS \rightarrow specR$ by $P \mapsto f^{-1}(P)=P^{c}.$
\newline  $(viii)$ $I \subseteq I^{ec}$
\newline  $(ix)$ $J^{ce} \subseteq J$
\newline  $(x)$ $J_{1} \subseteq J_{2}$ in $S$ then $J_{1}^{c} \subseteq J_{2}^{c}$
\newline  $(xi)$ $I_{1} \subseteq I_{2}$ in $R$ then $I_{1}^{e} \subseteq I_{2}^{e}$
\newline  $(xii)$ $I^{ece}=I^{e}$
\newline  $(xiii)$ $J^{cec}=J^{c}$
\proof $(i)$ Pick $a,b \in I+J$ then $a=x_{1}+y_{1}$ and $b=x_{2}+y_{2}$ $\Rightarrow a+b=x_{1}+y_{1}+x_{2}+y_{2}$ $\Rightarrow a+b \in I+J.$ Take $r\in R,$ $r(x+y)=rx+ry\in I+J$ (as I,J are ideals $rx\in I,$ $ry\in J$). Hence $I+J$ is an ideal. \
\newline $(ii)$ Let $a={\displaystyle\sum_{\substack{\text{finite}\\ \text{sum}}}x_{1}y_{1}},$ $b={\displaystyle\sum_{\substack{\text{finite}\\ \text{sum}}}x_{2}y_{2}}.$ $a+b={\displaystyle\sum_{\substack{\text{finite}\\ \text{sum}}}x_{1}y_{1}}+{\displaystyle\sum_{\substack{\text{finite}\\ \text{sum}}}x_{2}y_{2}}$ $={\displaystyle\sum_{\substack{\text{finite}\\ \text{sum}}}xy}$ $\in IJ.$ Let $r \in R$ and $a={\displaystyle\sum_{\substack{\text{finite}\\ \text{sum}}}xy}$ then $ra=r{\displaystyle\sum_{\substack{\text{finite}\\ \text{sum}}}xy}={\displaystyle\sum_{\substack{\text{finite}\\ \text{sum}}}(rx)y} \in IJ.$  \\
\newline $(iii)$ Let $x,y \in I\cap J$ $\Rightarrow x,y\in I$ and $x,y\in J$ $\Rightarrow x+y\in I$ and $x+y\in J$ then $x+y \in I\cap J.$ Now, $r \in R,$ $x\in I\cap J$ then $x\in I,$ $x \in J$ $\Rightarrow rx \in I,$ $rx\in J$ $\Rightarrow rx\in I\cap J.$  
\newline $(iv)$ Take $x,y \in (I:J)$ then $xJ,yJ \subseteq I$ $\Rightarrow (x+y)J\subseteq I$ $\Rightarrow x+y \in (I:J).$ Let $r \in R$ and $x \in (I:J)$ then $xj \subseteq I$ $\Rightarrow (rx)J \subseteq I$ $\Rightarrow rx \in (I:J).$  
\newline $(v)$ Let $x={\displaystyle\sum_{\substack{\text{finite}\\ \text{sum}}}f(r)s}$ and $y={\displaystyle\sum_{\substack{\text{finite}\\ \text{sum}}}f(r')s'}$ then $x+y={\displaystyle\sum_{\substack{\text{finite}\\ \text{sum}}}f(r)s}+{\displaystyle\sum_{\substack{\text{finite}\\ \text{sum}}}f(r')s'}$ $\Rightarrow x+y \in f(I)=I^{e}.$ Let $p\in S$ $\Rightarrow p{\displaystyle\sum_{\substack{\text{finite}\\ \text{sum}}}f(r)s}={\displaystyle\sum_{\substack{\text{finite}\\ \text{sum}}}f(r)(ps)}\in I^{e},$ so $I_{e}$ is an ideal. 
\newline $(vi)$ \qed\\\\
\textbf{Lattice diagram of ideal of $R:$}\\
\begin{center}
\begin{tikzcd}
(I:J)	  &                                           & I+J                                             &                       \\
      & I \arrow[ru, no head] \arrow[lu, no head] &                                                 & J \arrow[lu, no head] \\
      &                                           & I\cap J \arrow[lu, no head] \arrow[ru, no head] &                       \\
      &                                           & IJ \arrow[u, no head]                           &                      
\end{tikzcd}
\end{center}
\begin{qns}
Let $I=\gen{a,b}$ then what will be the generators of $I^2.$ 
\end{qns}
Ans. As $I=\gen{a,b}$, $I=\{ra+sb:r,s\in R\}.$ We know that \begin{align*}
I^2&=\left\lbrace \displaystyle\sum_{\substack{\text{finite}\\ \text{sum}}} xy:x,y\in I \right\rbrace \\
&=\left\lbrace \displaystyle\sum_{\substack{\text{finite}\\ \text{sum}}} (r_1a+s_1b)(r_2a+s_2b):r_i,s_i\in R\right\rbrace \\
&=\left\lbrace \displaystyle\sum_{\substack{\text{finite}\\ \text{sum}}} r_1r_2a^2+s_1s_2b^2+(r_1s_2+s_1r_2)ab:r_i,s_i\in R \right\rbrace\\
&=\gen{a^2,ab,b^2}
\end{align*} 
We have examples of ideals such that $I\neq I^2.$
\begin{theorem}[Multinomial theorem]
$(f_{1}+f_{2}+\cdots +f_{k})^{n}$ $={\displaystyle\sum_{i_{1}+i_{2}+\cdots +i_{k}=n} \dfrac{n!}{i_{1}!i_{2}!\cdots i_{k}!} f_{1}^{i_{1}}f_{2}^{i_{2}}\cdots f_{k}^{i_{k}}}$ where $0\leq i_{l}\leq n,$ $1\leq l\leq k$
\end{theorem}

\begin{definition}[Radical Ideal] 
Let $I\subseteq R$ be an ideal of $R.$ Define $\sqrt{I}=\{x\in R:x^{n} \in I$ for some n $\in {\NN}$ \}
\end{definition}
\textbf{Claim:} $\sqrt{I}$ is an ideal of $R$ called radical of $I.$
\proof Let $x,y \in \sqrt{I}$ $\Rightarrow x^{n_{1}}, y^{n_{2}}\in I$ for some $n_{1},n_{2}>0.$ Now, $(x+y)^{n_{1}+n_{2}}={\displaystyle\sum_{r=0}^{n_{1}+n_{2}}}\binom{n_{1}+n_{2}}{r}x^{r}y^{n_{1}+n_{2}-r}.$ If $r<n_{1},$ then $n_{1}+n_{2}-r>n_{2}$ $\Rightarrow y^{n_{1}+n_{2}-r} \in I$ $\Rightarrow x^{r}y^{n_{1}+n_{2}-r} \in I.$ If $r\geq n_{1},$ then $x^{r}\in I$ $\Rightarrow  x^{r}y^{n_{1}+n_{2}-r} \in I$ therefore $(x+y)^{n_{1}+n_{2}} \in I$ $\Rightarrow x+y \in \sqrt{I}.$ 
Let $r \in R$ then $(rx)^{n_{1}}=r^{n_{1}}x^{n_{1}} \in I$ $\Rightarrow rx \in \sqrt{I}.$ So, $\sqrt{I}$ is an ideal.
\qed
\begin{definition}[Nil Radical] 
Let $R$ be a commutative ring with identity. $nil(R)=\{x \in R|$ $x^{n}=0$ for some $n>0$\} then $nil(R)$ is an ideal of $R$ called nil radical.
\end{definition}
\textbf{Notation:} $\sqrt{0}=nil(R)$
\newline $x,y \in nil(R)$ then $x^{n}=0,$ $y^{m}=0.$ $(x+y)^{n+m}={\displaystyle\sum_{r=0}^{m+n} \binom{k}{r} x^{r}y^{n+m-r}}$ if $k\geq m$ then $x^{k}=0$ if $k<m$ then $n+m-r\geq n$ $\Rightarrow y^{n+m-r}=0.$ Therefore $(x+y)^{m+n}=0$ $\Rightarrow x+y \in nil(R).$ Take $a\in R,$ $x \in nil(R)$ then $(ax)^{n}=a^{n}x^{n}=0,$ so $nil(R)$ is an ideal.
\begin{lemma}
Let $R$ be a commutative ring with identity and $S$ be a subset of $R$ which is disjoint from an $\mathfrak{a}$ then $$\sum=\{I\subseteq R: I~\text{is an ideal of}~R~\text{and}~\mathfrak{a}\subseteq I, I\cap S=\emptyset\}$$ has a maximal element. Moreover, if $S$ is multiplicative then the maximal element is a prime ideal.
\end{lemma}
\proof Clearly $\sum$ is non empty as $\mathfrak{a} \in \sum.$ Now we cosider a chain of ideal $\{I_{\lambda}\}_{\lambda\in \Lambda}$ in $\sum$ then $\displaystyle\bigcup_{\lambda\in \Lambda} I_\lambda$ is an ideal. If $\displaystyle\bigcup_{\lambda\in \Lambda} I_\lambda \cap S\neq \emptyset$ then $I_{\lambda}\cap S\neq \emptyset$ for some $\lambda\in \Lambda$ which is a contradiction therefore, $\displaystyle\bigcup_{\lambda\in \Lambda} I_\lambda\in$ $\sum.$ So, every chain in $\sum$ has a upper bound in $\sum$ by Zorn's lemma $\sum$ has a maximal element say $P$.\\ Let $S$ be a multiplicative set, we show that $P$ is a prime ideal. If not assume $a\notin P,b\notin P$ but $ab\in P$ clearly, $P\subsetneq P+\gen{a}$ and $P\subsetneq P+\gen{b}$. Since, $P$ is maximal in $\sum$, $P+\gen{a}$ and $P+\gen{b}\notin \sum$ and therefore, $S\cap P+\gen{a}\neq \emptyset.$ Let $f_1=p_1+at_1$ and $f_2=p_2+bt_2 \Rightarrow f_1f_2=(p_1+at_2)(p_2+bt_2)=p_1p_2+bp_1t_2+p_2at_1+abt_1t_2\in P$ ($\because ab\in P$) which is a contradiction. Therefore, $P$ is a prime ideal. \qed 

\begin{theorem}
Let $R$ be a commutative ring with identity then $\sqrt{0}={\displaystyle\bigcap_{P \in specR}P}$
\end{theorem}
\proof Let $x \in nil(R),$ then $x^n=0 \in P$ then  $x^n \in P$ for all $P \in specR$ $\Rightarrow x \in P$ for all $P\in specR$ $\Rightarrow x\in {\displaystyle\bigcap_{P \in specR}P}$ $\Rightarrow nil(R) \subseteq {\displaystyle\bigcap_{P \in specR}P}.$ For the other inclusion, we show that if $x\notin \sqrt{0} \Rightarrow x\notin \displaystyle\bigcap_{P\in \spec R} P.$ Let $x \notin nil(R)$ then $x^n \neq 0$ for all $n \in {\NN}.$ Let us consider the set $$\sum=\{I\subseteq R: I~\text{is an ideal of $R$ and}~x^n\notin I,\forall~n\in {\NN}\}$$ then $\sum \neq \emptyset$ as $(0) \in \sum.$ Let $\{I_{\lambda} \}_{\lambda \in \Lambda}$ be a chain in $\sum.$ Claim: $\displaystyle\bigcup_{\lambda \in \Lambda}I_{\lambda}$ $\in \sum.$ If $x^n \in \displaystyle\bigcup_{\lambda \in \Lambda}I_{\lambda}$ $\in \sum$ for some $n\in {\NN}$ then $x^n\in I_{\lambda}$
for some $\lambda \in \Lambda$ a contradiction. Therefore each chain in $\sum$ has an upper bound $\displaystyle\bigcup_{\lambda \in \Lambda}I_{\lambda}$ $\in \sum.$ By Zorn's lemma $\sum$ has a maximal element say $P.$ Claim: $P$ is a prime ideal. If not let $a,b\notin P$ but $ab \in P.$ Then $P\subsetneq P+(a)$ and $P\subsetneq P+(b)$ but $P$ is a maximal element in $\sum \Rightarrow P+(a) \notin \sum$ and $P+(b) \notin \sum$ then $\exists~m,n \in {\NN}$ such that $x^m \in P+(a)$ and $x^n \in P+(b)\Rightarrow x^m=r_{1}+at_{1}$ and $x^n=r_{2}+bt_{2}$ where $r_{1},r_{2} \in P$ and $t_{1},t_{2}\in R \Rightarrow x^m\cdot x^n=(r_{1}+at_{1})(r_{2}+bt_{2})=r_{1}r_{2}+r_{1}bt_{2}+r_{2}at_{1}+abt_{1}t_{2} \in P$ (as $ab \in P$) $\Rightarrow x^{n+m} \in P$ a contradiction as $P\in \sum.$ Therefore $P$ is a prime ideal and $x^n \notin P$ $\forall n\in {\NN}.$ In particular $x^1 \notin P$ $\Rightarrow x\notin {\displaystyle\bigcap_{P \in specR}P}.$ Therefore ${\displaystyle\bigcap_{P \in specR}P} \subseteq nil(R).$ Therefore ${\displaystyle\bigcap_{P \in specR}P}=nil(R).$ \qed
\begin{corollary} 
$\sqrt{I}={\displaystyle\bigcap_{P \in specR}P}$ where $I\subseteq P.$
\end{corollary}
\proof Let $x\in \sqrt{I} \Rightarrow x^n\in I\subseteq P$ for some $n\in {\NN}$ then $x\in P \Rightarrow \sqrt{I}\subseteq P \Rightarrow \sqrt{I} \subseteq \displaystyle\bigcap_{P\in \spec R} P$. Now, assume that $x\notin \sqrt{I} \Rightarrow x^n\notin I,\forall~n\in {\NN}.$ Therefore, $\{1,x,x^2,\cdots \}$ is a multiplicative set disjoint from $I$ and by Lemma 5.28 there is a prime ideal $P$ containing $I$ and not containing $x$. Thus $\sqrt{I}=\displaystyle\bigcap_{P \in specR}P$ where $I\subseteq P.$ \qed


\textbf{Observation:} $(i)$ $I \subseteq \sqrt{I}.$ 
\newline $(ii)$ $I \subseteq J$ $\Rightarrow \sqrt{I} \subseteq \sqrt{J}.$ 
\newline $(iii)$ If $P \in \spec R$ then $\sqrt{P^n}=P$ $\forall$ $n \in {\NN}.$\\
$(iv)$ $\sqrt{I\cap J}=\sqrt{I}\cap \sqrt{J}.$\\
$(v)$ $\sqrt{\sqrt{I}}=\sqrt{I}$\\
$(vi)$ If $I$ is  an ideal then $\sqrt{I^n}=\sqrt{I}$\\
$(vii)$ Suppose, $I,J$ are ideals of $R.$ If $\sqrt{I},\sqrt{J}$ are comaximal then $I,J$ are also comaximal.\\
$(viii)$ If $x\in \sqrt{0}$ then $1+x$ is a unit.\\
$(ix)$ Units of $R=R-\displaystyle\bigcup_{m\in \mspec R} m.$\\
Ans. $(i)$ $x \in I$ $\Rightarrow x^1\in I$ $\Rightarrow x \in \sqrt{I},$ hence $I \subseteq \sqrt{I}.$\\
$(ii)$ Pick $x \in \sqrt{I}$ $\Rightarrow x^n\in I\subseteq J$ $\Rightarrow x \in \sqrt{J}$ $\Rightarrow \sqrt{I} \subseteq \sqrt{J}.$\\
$(iii)$ $P^n \subseteq P$ for all $n \in {\NN}$ then $\sqrt{P^n} \subseteq \sqrt{P}.$ Claim $\sqrt{P}=P.$ We have $P\subseteq \sqrt{P}.$ Let $x \in \sqrt{P}$ $\Rightarrow x^n \in P$ for some $n$ $\Rightarrow x \in P$ (since $P$ is prime ideal) $\Rightarrow \sqrt{P} \subseteq P$ therefore $\sqrt{P}=P.$ Then we have $\sqrt{p^n}\subseteq \sqrt{P}=P.$ Let $x\in P$ $\Rightarrow x^n \in P^n \subseteq \sqrt{P^n}.$ Hence $P\subseteq \sqrt{P^n}$ $\Rightarrow P=\sqrt{P^n}$ \\
$(iv)$ Pick $x\in \sqrt{I\cap J} \Rightarrow x^n\in I\cap J \Rightarrow x^n\in I ~\text{and}~ x^n\in J \Rightarrow x\in \sqrt{I}\cap \sqrt{J}$ hence, $\sqrt{I\cap J}\subseteq \sqrt{I}\cap \sqrt{J}.$ Again, let $y\in \sqrt{I}\cap \sqrt{J} \Rightarrow y\in \sqrt{I} ~\text{and}~ y\in \sqrt{J} \Rightarrow y^n\in I, J \Rightarrow y\in \sqrt{I\cap J} \Rightarrow \sqrt{I}\cap \sqrt{J} \subseteq \sqrt{I\cap J}$ therefore, $\sqrt{I\cap J}=\sqrt{I}\cap \sqrt{J}.$\\
$(viii)$ If $x\in \sqrt{0}$ then $x^n=0$ for some natural number $n$ but this gives $(1+x)(1-x+x^2+\cdots +(-1)^{n-1}x^{n-1})=1-x^n=1$ therefore, $1+x$ is a unit.\\
$(ix)$ As every ideal is contained in some maximal ideal so if we pick any $x\in  \displaystyle\bigcup_{m\in maxspec R} m$ then $x$ must be in some ideal of $R$ so $x$ cannot be a unit.
\begin{definition}
A ring $R$ is said to be reduced if $nil(R)=0.$ 
\end{definition}
\begin{obs}
For any ring $R,$ $R/nil(R)$ is reduced.
\end{obs}
\begin{definition}
Let $R$ be any ring. We define the ideal $Jac(R)={\displaystyle\bigcap_{m \in maxspecR}m}.$
\end{definition}
$Jac(R)$ is called the Jacobson ideal of $R.$

\subsection{Prime avoidance lemma}

\begin{theorem}

Let $P_1,\cdots ,P_n\in \operatorname{spec}R$ then $$I\subseteq \displaystyle\bigcup _{i=1}^n P_i \Rightarrow I\subseteq P_i~\text{for some $i$}.$$

\end{theorem}

\proof We have to prove that if $I\nsubseteq P_i,~\forall~1\leq i\leq n$ then $I\nsubseteq \displaystyle\bigcup_{i=1}^n P_i.$ We proceed by induction on $n.$ If $n=1$ then we are done. Suppose, the statement is true for $n-1$ ideals. We consider $P_2,\cdots ,P_n$ and we have $I\nsubseteq P_i,~2\leq i\leq n$ then by induction hypothesis $I\nsubseteq \displaystyle\bigcup_{i=2}^n P_i$ then $\exists x_i\in I$ such that $x\notin \displaystyle\bigcup_{i=2}^n P_i$ i.e., $x\notin P_i,~2\leq i\leq n.$ If $x_1\notin P_1$ then $x_1\notin \displaystyle\bigcup_{i=1}^n P_i$ and hence $I\nsubseteq \displaystyle\bigcup_{i=1}^n P_i$ and we are done. So we may assume $x_1\in P_1$ and $x_1\notin P_i,~2\leq i\leq n.$ Now we consider $\{P_1,P_2,\cdots ,P_n\}\setminus \{P_2\}$ and by similar approach we get $x_2\in I$ with $x_2\in P_2$ and $x_2\notin P_i,~\{1,\dots ,n\}\setminus \{2\}$ and lastly we get $x_n\in I$ with $x_n\in P_n$ and $x_n\notin P_1,~1\leq i\leq n-1.$ We consider $$x=x_2\cdots x_n+x_1x_3\cdots x_n+x_1x_2x_4\cdots x_n+\cdots +x_1\cdots x_{n-1}$$
then $x\in I.$ We claim that $x\notin \displaystyle\bigcup_{i=1}^n P_i$ i.e., $x\notin P_1,~1\leq i\leq n.$ If $x\in P_i$ for some $i.$ Let $$y_i=x_1\cdots \widehat{x_i}\cdots x_n$$ then $x_i|y_j$ for $i \neq j \Rightarrow y_j\in P_i ~[x_i\in P_i] \Rightarrow  \displaystyle\sum_{\substack{j=1\\ j\neq i}}^n y_j\in P_i \Rightarrow x-\displaystyle\sum_{\substack{j=1\\ j\neq i}}^n y_j\in P_i ~[\because x\in P_i] \Rightarrow y_i\in P_i \Rightarrow x_1\cdots \widehat{x_i}\cdots x_n\in P_i$ but $x_j\notin P_i,~j\neq i$ [since $P_i$ is a prime ideal] $\Rightarrow x\notin \displaystyle\bigcup_{i=1}^n P_i.$ Hence, $I\nsubseteq .$ \qed

\begin{remark}

Prime avoidance lemma is not true for infinite number of prime ideals.

\end{remark}

\begin{example}

Let $R=K[x_1,\cdots ,x_n,\cdots]$ (infinitely many variables). Let $I=(x_1,\cdots ,x_n,\cdots ),P_i=(x_1,\cdots ,x_i),i\in {\NN}$ then $R/P_i\cong K[x_{i+1},x_{i+2},\cdots ,]$ (integral domain) then $P_i\in spec~R.$ But $I\subseteq \displaystyle\bigcup_{i\in {\NN}} P_i$ and $I\subseteq P_i~\forall i\in {\NN}.$

\end{example}

\begin{theorem}[Prime avoidance lemma]

Let $R$ be a commutative ring with 1, $I$ be an ideal of $R$ and $f\in R.$ Suppose $P_1,\cdots,P_r\in\spec R$ such that $f+I=\displaystyle\bigcup_{i=1}^r P_k$ then $\gen{f,I}\subseteq P_i$ for some $i\in\{1,\cdots,r\}.$

\end{theorem}

\proof Let $\displaystyle\sum$ be the collection of all $s\in{\NN}$ such that there exist $t\in R$ and an ideal $J$ of $R$ such that $t+J\subseteq \displaystyle\bigcup_{i=1}^s P_i$ but $\gen{t,J}\nsubseteq P_i,1\leq i\leq s.$ If $\displaystyle\sum\neq\emptyset$ then by well ordering principle of Natural numbers, $\displaystyle\sum$ has a least element say $l\in \displaystyle\sum.$ So there exist $g\in R$ and $\mathfrak{A}\subseteq R$ such that $g+\mathfrak{A}\subseteq \displaystyle\bigcup_{i=1}^l P_i$ but $\gen{g,\mathfrak{A}}\nsubseteq P_i,1\leq i\leq l.$ We note that $l\geq 2$ and $P_i\nsubseteq P_j.$ We claim that $g\in\displaystyle\bigcap_{i=1}^l P_i.$ If not, $g\notin P_{i_0}$ for some $i_0\in\{1,\cdots,l\}$, then $(g+P_{i_0}\mathfrak{A})\cap P_{i_0}=\emptyset$ hence $g+P_{i_0}\mathfrak{A}\subseteq \displaystyle\bigcup_{\substack{j=1\\j\neq i_0}}^l P_j.$ Since $l$ is the minimal element of $\displaystyle\sum$, we have $\gen{g,P_{i_0}\mathfrak{A}}\subseteq P_{j_0}$ for some $j_0\in\{1,\cdots, l\}$ but $j_0\neq i_0.$ Then $P_{i_0}\mathfrak{A}\subseteq P_{j_0} \Rightarrow P_{i_0}\subseteq P_{j_0}$ which is a contradiction (since if $\mathfrak{A}\subseteq P_{j_0}$ then $\gen{g,P_{j_0}\mathfrak{A}}\subseteq P_{j_0}$ implies $g\in P_{j_0} \Rightarrow \gen{g,\mathfrak{A}}\subseteq P_{j_0}$ but $\gen{g,\mathfrak{A}}\nsubseteq P_i$ for all $1\leq i\leq l$ so, $\mathfrak{A}\nsubseteq P_{j_0}$). Therefore, $g\in\displaystyle\bigcap_{i=1}^l P_i \Rightarrow \mathfrak{A}\subseteq \displaystyle\bigcup_{i=1}^l P_i\Rightarrow \mathfrak{A}\subseteq P_s$ for some $1\leq s\leq l.$ Then by our assumption $\gen{g,\mathfrak{A}}\subseteq P_s$ but $\displaystyle\sum\neq \emptyset$ which is a contradiction. Hence our assumption is not true that is $\displaystyle\sum=\emptyset.$ \qed

\begin{prop}

Let $I_1,\cdots, I_r$ be ideals of $R$ and $P\in\spec R$. If $\displaystyle\bigcap_{k=1}^r I_k\subseteq P$ then $I_k\subseteq P$ for some $k\in\{1,\cdots,r\}.$

\end{prop}

\proof Since $\displaystyle\prod_{k=1}^r I_k\subseteq \displaystyle\bigcap_{k=1}^r I_k\subseteq P,$ by definition of prime ideal $I_k\subseteq P$ for some $1\leq k\leq r.$ \qed

\begin{theorem}[Module theoretic version]

Let $R$ be a commutative ring with 1 and $P_1,\cdots, P_m\in\spec R$, $M$ be an $R-$module and $x_1,\cdots,x_n\in M.$ Consider the submodule $N=\gen{x_1,\cdots, x_n}$ of $M$. If $N_{P_j}\nsubseteq P_jM_{P_j},j=1,\cdots,m$ then there exist $a_2,\cdots,a_n\in R$ such that $x_1+\displaystyle\sum_{i=1}^n a_ix_i\notin P_jM_{P_j}.$

\end{theorem}






















































\subsection{Quotient Ring}
\begin{definition}
Let $R$ be a ring and $I\subseteq R$ be a ideal of $R.$ We consider the set $R/I:=\{x+I|$ $x\in R \}$ the set of all left coset of $I$ in the additive group $(R,+).$
\end{definition}
Note that $x+I=y+I$ $\Leftrightarrow x-y \in I.$
\newline Define $+:R/I\times R/I \rightarrow R/I$ by $(x+I,y+I) \mapsto (x+y)+I$ and $\cdot :R/I\times R/I \rightarrow R/I$ by $(x+I,y+I) \mapsto (xy)+I.$ 
\newline Claim: `$+,$' and `$\cdot $' are well-defined. Let $x_{1}+I=y_{1}+I$ and $x_{2}+I=y_{2}+I$ $\Rightarrow x_{i}-y_{i} \in I,$ $i\in \{1,2 \}.$ Suppose, $x_{i}=y_{i}+t_{i}$ where $t_{i} \in I,$ $i\in \{1,2 \}.$ So, $(x_{1}+x_{2})-(y_{1}+y_{2})=t_{1}+t_{2}\in I$ $\Rightarrow (x_{1}+x_{2})-(y_{1}+y_{2}) \in I$ $\Rightarrow (x_{1}+x_{2})+I=(y_{1}+y_{2})+I.$ Therefore, $+$ is well-defined. Now $x_{1}x_{2}=t_{1}y_{2}+t_{1}y_{2}+t_{2}y_{1}+t_{1}t_{2}$ $\Rightarrow x_{1}x_{2}-y_{1}y_{2} \in I$ $\Rightarrow  x_{1}x_{2}+I=y_{1}y_{2}+I.$ Hence, $\cdot$ is well-defined.
\newline Check $(R/I,+,\cdot )$ is a commutative ring with identity $0+I.$ Note that $x+I=0+I \Leftrightarrow x\in I.$


\section{Homomorphism of rings}
\begin{definition}
Let $R,S$ be rings. $f:R \rightarrow S$ is said to be ring homomorphism if 
\newline $(i)$ $f(a+b)=f(a)+f(b)$ $\forall a,b\in R$
\newline $(ii)$ $f(ab)=f(a)f(b)$ $\forall a,b\in R$
\newline $(iii)$ $f(1_{R})=1_{S}$
\newline Here $S$ is said to be $R$-algebra if there exists a ring homomorphism from $R$ to $S.$
\end{definition}
Note that for any ring $R,$ there is a canonical ring homomorphism $f: {\ZZ} \rightarrow R$ defined by $n \mapsto \displaystyle\underbrace{1_{R}+\cdots +1_{R}}_{n~times},$ $-n \mapsto \displaystyle\underbrace{(-1_{R})+\cdots +(-1_{R})}_{n~times}$
\begin{definition}
Let $f: R \rightarrow S$ be a ring homomorphism we define $Kerf=\{r \in R|f(r)=0 \}$ and $Imf=\{s \in S|f(r)=S \}$
\end{definition}
Check that $Kerf$ is an ideal of $R$ and $Imf$ is a subring of $S.$
\begin{definition}
Let $f:R\rightarrow S$ be a ring homomorphism. $f$ is said to be an isomorphism if $\exists$ another ring homomorphism $g:S\rightarrow R$ such that $g\circ f=id_{R}$ and $f\circ g=id_{S}.$ 
\end{definition}
\begin{theorem}
$f:R\rightarrow S$ is an isomorphism iff $f$ is bijective.
\end{theorem}
\proof $(\Rightarrow)$ Obvious.
\newline $(\Leftarrow)$ Suppose, $f$ is bijective then $\exists~ g:S\rightarrow R$ such that $f\circ g=id_{S}$ and $g\circ f=id_{R}.$ We need to check that $g$ is a ring homomorphism. Let $s_{1},s_{2}\in S.$ $(f\circ g)(s_{1}+s_{2})=s_{1}+s_{2}=(f\circ g)(s_{1})+(f\circ g)(s_{2})$ $\Rightarrow f[g(s_{1}+s_{2})-g(s_{1})-g(s_{2})]=0$ (since $f$ is a homomorphism). As $f$ is bijective $f$ is injective also so $g(s_{1}+s_{2})=g(s_{1})+g(s_{2}).$ Again $f(g(s_{1}s_{2}))=s_{1}s_{2}=f(g(s_{1}))f(g(s_{2}))$ $\Rightarrow f[g(s_{1}s_{2})-g(s_{1})g(s_{2})]=0$ $\Rightarrow g(s_{1}s_{2})=g(s_{1})g(s_{2}).$ Hence $g$ is a ring homomorphism. \qed
\begin{theorem}[First Isomorphism Theorem]
Let $f:R\rightarrow S$ be a ring homomorphism. $I\subseteq \operatorname{Ker}f$ be an ideal of $R$ then there exists a unique ring homomorphism $\tilde{f}:R/I\rightarrow S$ such that the diagram commutes i.e., $\tilde{f}\circ \pi=f.$\\
\begin{center}
\begin{tikzcd}
 R \arrow[rr, "f"] \arrow[dd, "\pi"'] &  & S \\
                                      &  &   \\
R /I \arrow[rruu, "\tilde{f}"']       &  &  
\end{tikzcd}
\end{center} 
Moreover, if $I=\operatorname{Ker}f$ then $\tilde{f}$ is injective. Therefore, $R/\operatorname{Ker}f \simeq \operatorname{Im}f$ 
\newline By definition $\operatorname{Im}\tilde{f}=\operatorname{Im}f$ so that if $\tilde{f}$ is surjective then $f$ is also surjective then $R/\operatorname{Ker}f \simeq S.$ 
\end{theorem}
\proof Define, $\tilde{f}:R/I\rightarrow S$ by $(x+I)\mapsto f(x).$ Claim: $\tilde{f}$ is well defined. Let, $x+I$=$y+I$ $\Rightarrow$ $x-y$ $\in I\subseteq \operatorname{Ker}f$ $\Rightarrow$ $f(x-y)=0$ $\Rightarrow f(x)=f(y)$ $\Rightarrow \tilde{f}(x+I)=\tilde{f}(y+I).$ Therefore $\tilde{f}$ is well-defined. [$\tilde{f}(x+y+I)=f(x+y)$ $\Rightarrow f(x)+f(y)$ (as $f$ is a ring homomorphism) $\Rightarrow \tilde{f}(x+y+I)=\tilde{f}(x+I)+\tilde{f}(y+I)$
and $\tilde{f}(xy+I)=f(xy)=f(x)f(y)=\tilde{f}(x+I)\tilde{f}(y+I)$ so, $\tilde{f}$ is a ring homomorphism.] Uniqueness: Suppose $g$ is another ring homomorphism such that $g:R/I\rightarrow S$ such that $g\circ\pi=f.$ Then $f(x)=g\circ\pi(x)$ $\Rightarrow \tilde{f}(x+I)=f(x)=g(x+I)$ so we have $g=\tilde{f}.$ Therefore $\tilde{f}$ is unique. Now suppose $I=\operatorname{Ker}f$ then $\tilde{f}(x+ \operatorname{Ker}f)=\tilde{f}(y+ \operatorname{Ker}f)$ $\Rightarrow f(x)=f(y)$ $\Rightarrow f(x-y)=0$ $\Rightarrow x-y \in \operatorname{Ker}f$ $\Rightarrow x+ \operatorname{Ker}f=y+ \operatorname{Ker}f$ hence $\tilde{f}$ is injective. Therefore, $R/\operatorname{Ker}f \cong \operatorname{Im}\tilde{f}=\operatorname{Im}{f}$ (by definition $\operatorname{Im}\tilde{f}=\operatorname{Im}{f}$). If $\tilde{f}$ is surjective then so $f$ hence $S=\operatorname{Im}\tilde{f}=\operatorname{Im}{f}.$ Therefore, $R/\operatorname{Ker}f \cong S$.  \qed
\begin{qns}
What are the ideals of $R/I?$
\end{qns}
Ans. Ideals of $R/I$ are exactly the set $\Sigma=\{J/I:I\subseteq J\subseteq R ~\text{and $J$ is an ideal of $R$}\}.$ We consider the surjective ring homomorphism $\pi:R\to R/I.$ Let $Q$ be an ideal of $R/I$ then $\pi^{-1}(Q)=Q^c$ is an ideal of $R$ now $0+I\subseteq Q \Rightarrow \pi^{-1}(0+I)\subseteq Q^c \Rightarrow I\subseteq Q^c.$ Claim: $Q^c/I=Q,$ since $\pi$ is surjective $\pi(\pi^{-1}(Q))=Q=\pi(Q^c)=Q^c/I.$ Conversely, if we take $J/I\subseteq R/I$ with $I \subseteq J\subseteq R$ then $(x+I),(y+I)\in J/I$ where $x,y\in J$ as $x+y\in J\Rightarrow (x+I)+(y+I)=(x+y)+I\in J/I.$ Let $r+I\in R/I$ and $x+I\in J/I$ as $J$ is an ideal $r\in R$ and $x\in J$ we have $rx\in J$ so that $rx/I\in J/I.$ Therefore, $J/I$ is an ideal of $R/I.$ 
\begin{theorem}[Third isomorphism theorem]
We have seen that if $I\subseteq J\subseteq R$ then $J/I$ is an ideal of $R/I$ hence prove that $\dfrac{R/I}{J/I}\cong  \dfrac{R}{J}.$
\end{theorem}
\proof Define $\pi_{J}:R\to R/J$ by $x\mapsto x+J$ now, $I\subseteq J=\operatorname{Ker}\pi_{J}$ by first isomorphism theorem there exists an unique ring homomorphism $\theta:R/I\to R/J$ such that the diagram is commutative i.e., $\pi_{J}=\theta\circ \pi_{I}.$ \begin{center}
\begin{tikzcd}
 R \arrow[rr, "\pi_I"] \arrow[dd, "\pi_J"'] &  & R/I \\
                                            &  &     \\
R /J \arrow[rruu, "\theta"']                &  &    
\end{tikzcd}
\end{center}
Claim: $\operatorname{Ker}\theta=J/I.$ Let $x+I\in \operatorname{Ker}\theta \Leftrightarrow \theta(x+I)=0+J \Leftrightarrow x+J=0+J \Leftrightarrow x+J$ hence, $\operatorname{Ker} \theta=J/I$ as $\pi_{J}$ is surjective by first isomorphism theorem $\dfrac{R/I}{J/I}\cong \dfrac{R}{J}.$ \qed

\begin{theorem}
Let $P$ be an prime ideal of an ring $R$ then $R/P$ is an integral domain.
\end{theorem}
\proof As $R$ is commutative ring with $1_R$ and $P\in \operatorname{spec}R$, we get $R/P$ is also commutative ring. Let $(a+P)(b+P)=0+P \Rightarrow ab+P=0+P \Rightarrow ab\in P$ as $P$ is an prime ideal and $ab\in P$ we have either $a\in P$ or $b\in P$ then $a+P=0+P$ or $b+P=0+P$ therefore, $R/P$ is an integral domain.\\
Conversely, suppose, $R/P$ is an integral domain and let $x,y\in R$ such that $xy\in P$ then $xy+P=0+P$ in $R/P$ this implies $(x+P)(y+P)=0+P.$ As $R/P$ is integral domain we have either $x+P=0+P$ or $y+P=0+P$ hence, $x\in P$ or $y\in P$ this gives $P\in \operatorname{spec}R.$ \qed
\begin{theorem}
Let $m\in \operatorname{maxspec}R$  then $R/m$ is a field.
\end{theorem}
\proof As $R$ is commutative ring with $1_R$ and $m\in \operatorname{maxspec}R$, we get $R/m$ is also commutative ring. Let $x+m\in R/m$ with $x+m\neq 0+m \Rightarrow x\notin m \Rightarrow m\subsetneq m+(x)$ since $m$ is a maximal ideal $m+(x)=R\Rightarrow 1\in m+(x) \Rightarrow 1=t+xy$ where $t\in m$ and $y\in R$ then $xy-1\in m\Rightarrow xy+m=1=m\Rightarrow (x+m)(y+m)=1+m$ hence $R/m$ is a field.\\
Conversely, $R/m$ is a field suppose, $m\subseteq I$ if $m\neq I$ $\exists x\in I$ such that $x\in m$ this gives $x+m\neq 0+m$ then $\exists y\neq 0$ such that $(x+m)(y+m)=1+m$ [as $R/m$ is a field] then we have $xy-1\in m\subseteq I$ since $x\in I \Rightarrow xy\in I \Rightarrow xy-(xy-1)\in I \Rightarrow 1\in I \Rightarrow I=R$ hence $m\in \operatorname{maxspec}R.$\\\\
Alternative proof: Let $m\subseteq I\subseteq R$ then $I/m\subseteq R/m$ is an ideal. If $m$ is an maximal ideal then either $I=m$ and $m=R$ then ideals of $R/m$ is $0+m$ or $R/m$ hence $R/m$ is a field. Conversely, if $R/m$ is a field, ideals of $R/m$ is zero ideal and $R/m$ itself then $m\subseteq I\subseteq R$ implies either $m=I$ and $I=R$ hence $m$ is a maximal ideal. \qed


\begin{qns}
Find $\operatorname{spec}(R/I).$
\end{qns}
Ans. Let us take an ideal $P/I$ where $I\subseteq P$ and $P\in \operatorname{spec}R$ then $\dfrac{R/I}{P/I}\cong R/P.$ As $P\in \operatorname{spec}R$ and $R$ is a commutative ring $R/P$ is a integral domain then $P/I\in \operatorname{spec}(R/I).$ Conversely, if $P/I\in \operatorname{spec}(R/I)$ then $\dfrac{R/I}{P/I}=R/P$ is an integral domain hence $P\in \operatorname{spec}R.$ 
\begin{qns}
Find $\operatorname{maxspec}(R/I).$
\end{qns}
Ans. Let us take an ideal $m/I$ where $I\subseteq m$ and $m\in \operatorname{maxspec}R$ then $\dfrac{R/I}{m/I}\cong R/m.$ As $m\in \operatorname{maxspec}R$ and $R$ is a commutative ring $R/m$ is a field then $m/I\in \operatorname{maxspec}(R/I).$ Conversely, if $m/I\in \operatorname{maxspec}(R/I)$ then $\dfrac{R/I}{m/I}=R/m$ is a field hence $m\in \operatorname{maxspec}R.$ 
\begin{corollary}
$\operatorname{maxspec}R\subseteq \operatorname{spec}R$
\end{corollary}
\proof Let $m\in \operatorname{maxspec}R$ $\Rightarrow R/m$ is a field $\Rightarrow R/m$ is an integral domain $\Rightarrow m\in \operatorname{spec}R$ therefore, $\operatorname{maxspec}R\subseteq \operatorname{spec}R.$ \qed
\begin{corollary}
$\sqrt{0}\subseteq \operatorname{Jac}R$
\end{corollary}
\proof Let $x\in \sqrt{0}$ and $m\in \operatorname{maxspec}R\subseteq \operatorname{spec}R$ $\Rightarrow x^n=0\in m\Rightarrow x\in m \Rightarrow m\in \displaystyle\bigcap_{m\in maxspec~R} m \Rightarrow x\in \operatorname{Jac}R.$ \qed
\begin{definition}
A ring is said to be local if it has unique maximal ideal.
\end{definition}

\textbf{Construction of local ring:} Let $R$ be a commutative ring with $1$ and $m\in \operatorname{maxspec}R$ then for any $k\in {\NN}^+$ the ring $R/m^k$ is local. Moreover, $\operatorname{spec}R/m^k=\{m/m^k\}.$ Every prime ideal of $R/m^k$ is of the form $P/m^k$ where $m^k\subseteq P$ and $P$ is a prime ideal of $R.$ Taking radical on both side we get, $\sqrt{m^k}\subseteq \sqrt{P} \Rightarrow m\subseteq P$ since $m$ is an maximal ideal, $m=P$ hence $spec(R/m^k)=\{m/m^k\}.$

\begin{example}
${\ZZ}/p^n{\ZZ}$ is a local ring with maximal ideal $\gen{\bar{p}}.$
\end{example}

\begin{remark}
Let $(R,m,K)$ be a local ring. $x\in R-m \Leftrightarrow x$ is a unit in $R.$
\end{remark}

\begin{defn}
A ring is said to be semi local if it has finite number of maximal ideals.
\end{defn}

\subsection{Characteristic of a Ring}

Let $R$ be a commutative ring with identity. We consider the ring homomorphism \begin{align*}
\theta:&~{\ZZ}\to R\\
&n\mapsto\displaystyle\underbrace{1_R+\cdots +1_R}_{\text{n times}}\\
&-n\mapsto\displaystyle\underbrace{(-1_R)+\cdots +(-1_R)}_{\text{n times}}\\
&0\mapsto 0
\end{align*}
Check that $\theta$ is a ring homomorphism. Then $\operatorname{Ker}\theta=0$ or $m{\ZZ}$ for some $m\in {\ZZ},m>0.$ If $\operatorname{Ker}\theta=0$ we call $\operatorname{Char}R=0,$ if $\operatorname{Ker}\theta=m{\ZZ}$ we call $\operatorname{Char}R=m.$

\begin{remark}
If $\operatorname{Ker}\theta=0$ then ${\ZZ}\hookrightarrow R$ i.e., ${\ZZ}$ is a subring of $R,$ if $\operatorname{Ker}\theta=m{\ZZ}$ then ${\ZZ}/m{\ZZ}\hookrightarrow R.$
\end{remark}

\begin{remark}
If $R$ is an integral domain then $\operatorname{Ker}\theta=0$ or $\operatorname{Ker}\theta=p{\ZZ}$ for some prime $p$ since ${\ZZ}/m{\ZZ}$ is an integral domain iff $m$ is prime. In this case ${\ZZ}/p{\ZZ}\hookrightarrow R$ [Note that ${\ZZ}/p{\ZZ}$ is a field].
\end{remark}

\begin{remark}
If $R$ is a field then $\operatorname{Ker}\theta=0$ or $p{\ZZ}.$ If $\operatorname{Char}R=0$ we say that ${\QQ}\hookrightarrow R.$ Let $p/q\in {\QQ}, q\in {\ZZ}\hookrightarrow R \Rightarrow q^{-1}$ exists in $R$ (since $R$ is a field) therefore, $p,q^{-1}\in R\Rightarrow p/r\in R \Rightarrow {\QQ}\hookrightarrow R.$ If $\operatorname{Char}R=p$ then ${\ZZ}/p{\ZZ} \hookrightarrow R$ these type of field is called prime sub field.
\end{remark}

\begin{problem}
Find all dense subfield of ${\CC}.$
\end{problem}

Ans. We know that ${\CC}$ is a vector space over ${\RR}$ with $\dim_{{\RR}}{\CC}=2.$ If $\alpha=a+ib,b\neq 0$ then $\{1,\alpha\}$ is a basis of ${\CC}$ over ${\RR}.$ Let $F$ be a dense subfield of ${\CC}$ then $\operatorname{Char}{\CC}=0\Rightarrow \operatorname{Char}R=0 \Rightarrow {\QQ}\hookrightarrow F.$ If $F\subseteq {\RR} \Rightarrow \overline{F}\subseteq \overline{{\RR}}$ [Closure of ${\RR}$ in ${\CC}$ is ${\RR}$]. Then we have a contradiction hence $F\nsubseteq {\RR}$ then there exists $\alpha\in F$ such that $\alpha \notin {\RR}.$ If $\alpha=a+ib$ then $b\neq 0$ therefore, ${\QQ}(\alpha)\hookrightarrow F$ [as $\alpha \in F$].\\
Claim: ${\QQ}(\alpha)$ is dense in ${\CC}.$ Let $z\in {\CC}$ since $\{1,\alpha\}$ is a basis of ${\CC}$ over ${\RR}$ then $z=x+y\alpha.$ Now, ${\QQ}$ is dense in ${\RR}$ implies there exists $\{x_n\},\{y_n\}\subseteq {\QQ}$ such that $\lim x_n=x,\lim y_n=y$ therefore, $x_n+y_n\alpha\in {\QQ}(\alpha)$ and $\lim (x_n+y_n\alpha)=x+y\alpha$ hence, ${\QQ}(\alpha)$ is dense in ${\CC}.$


\section{Product ring}
Let $\{R_i\}_{i\in \Lambda}$ be a collection of rings then the cartesian product ring is defined on base set $\displaystyle\prod_{i\in \Lambda} R_i$ (cartesian product). We define,
\begin{align*}
+:\left(\displaystyle\prod_{i\in \Lambda} R_i \times \displaystyle\prod_{i\in \Lambda} R_i \right) \to \displaystyle\prod_{i\in \Lambda} R_i
\end{align*}
by $(a_i,b_i)\mapsto a_i+b_i$ and, \begin{align*}
\cdot :\left(\displaystyle\prod_{i\in \Lambda} R_i\times \displaystyle\prod_{i\in \Lambda} R_i\right)\to \displaystyle\prod_{i\in \Lambda} R_i
\end{align*}
by $(a_i,b_i)\mapsto (a_i)(b_i).$ Check that $\left(\displaystyle\prod_{i\in \Lambda} R_i,+,\cdot\right)$ is a ring. Let $R_1,\cdots ,R_n$ be rings and $I_i\subseteq R_i$ be ideals of $R_i.$ We consider the ring homomorphism, \begin{align*}
&R_1\times \cdots \times R_n \stackrel{\theta}{\longrightarrow} R_1/I_1\times \cdots \times R_n/I_n\\
&(x_1,\cdots ,x_n)\mapsto (x_1+I,\cdots ,x_n+I)
\end{align*}
Clearly $\theta$ is a surjective ring homomorphism and kernel of this map is \begin{align*}
\operatorname{Ker}\theta&=\{(x_1,\cdots ,x_n)\in R_1\times \cdots \times R_n: x_i\in I_i, 1\leq i\leq n\}\\
&=I_1\times \cdots \times I_n
\end{align*}
By first isomorphism theorem $$\dfrac{R_1\times \cdots \times R_n}{I_1\times \cdots \times I_n}\cong R_1/I_1\times \cdots \times R_n/I_n.$$
\begin{qns}
Let $R=R_1\times \cdots \times R_n$ where $R_i's$ are rings. What are the ideals of $R?$
\end{qns}
Ans. Ideals of $R$ is of the form $I_1\times \cdots \times I_n$ where $I_i\subseteq R_i$ are ideals. Let $J\subseteq R$ be an ideals. Consider the map \begin{align*}
&R_1\times \cdots \times R_n \stackrel{\pi_i}{\longrightarrow} R_i\\
&(x_1,\cdots ,x_n)\mapsto x_i
\end{align*}
$\pi_i$ is a surjective ring homomorphism. Let $I_i=\pi_i(J)\subseteq R_i$ is an ideal [since $\pi_i$ is surjective].\\
Claim: $I_1\times \cdots \times I_n=J.$ Let $(x_1,\cdots ,x_n)\in I_1\times \cdots \times I_n \Rightarrow \pi_i((x_1,\cdots ,x_n)=x_i\in \pi_i(J)=I_i$ therefore, $x_i \in \pi_i(J)$ then there exists $a_i\in J$ such that $\pi_i(a_i)=x_i.$ Now, $a_i\in J \Rightarrow (0,\cdots ,1_{R_i},\cdots ,0)a_i\in J \Rightarrow (0,\cdots ,a_i,\cdots ,0)$ [as $\pi_i(a_i)=x_i$] then $\displaystyle\sum_{i=1}^n (0,\cdots ,x_i,\cdots ,0)\in J \Rightarrow (x_1,\cdots ,x_n)\in J \Rightarrow I_1\times \cdots \times I_n \subseteq J.$ Let $(x_1,\cdots ,x_n)\in J \Rightarrow \pi_i((x_1,\cdots ,x_n))\in \pi_i(J)=I_i \Rightarrow (x_1,\cdots ,x_n) \in I_1\times \cdots \times I_n \Rightarrow J\subseteq I_1\times \cdots \times I_n.$ Hence, $J=I_1\times \cdots \times I_n.$
\\
Conversely, if we take $I_1\subseteq R_i$ then $I_1\times \cdots \times I_n$ is an ideal of $R_1\times \cdots \times R_n.$
\begin{obs}
Product ring is never an integral domain.
\end{obs}
\begin{qns}
What is the $\operatorname{spec}(R_1\times \cdots \times R_n)?$
\end{qns}
Ans. \begin{align*}
\operatorname{spec}(R_1\times \cdots \times R_n)&=\{P_1\times \cdots \times R_n:P_1\in \operatorname{spec}R_1\}\\
&\quad\quad \cup \{R_1\times P_2\times \cdots \times R_n:P_2\in \operatorname{spec}R_2\}\\
&\quad\quad \cup \cdots \cup \{R_1\times \cdots \times P_n:P_n\in \operatorname{spec}R_n\}\\
&=\displaystyle\bigcup_{i=n}^n \operatorname{spec} R_i
\end{align*}
Let $Q\subseteq R=R_1\times \cdots \times R_n$ be a prime ideal then $Q=I_1\times \cdots \times I_n$ where $I_i$ is an ideal of $R_i$ then $$R/Q=\dfrac{R_1\times \cdots \times R_n}{I_1\times \cdots \times I_n}\cong R_1/I_1\times \cdots \times R_n/I_n$$ 
As $R/Q$ is an integral domain $\exists ~i$ such that $R_j/I_j=0,~\forall j\neq i,1\leq i\leq n$ this gives $I_j=R_j,~\forall j\neq i,1\leq i\leq n$ and $R/Q=R_i/I_i$ (integral domain) then $I_i\in \operatorname{spec} R_i.$ Let $I_i=P_i\in \operatorname{spec} R_i$ so that \begin{align*}
Q&=R_1\times \cdots \times R_{i-1}\times I_i\times R_{i+1}\times \cdots \times R_n\\
&R_1\times \cdots \times R_{i-1}\times P_i\times R_{i+1}\times \cdots \times R_n
\end{align*}
Conversely, if we take $Q=R_1\times \cdots \times R_{j-1}\times P_j\times R_{j+1}\times \cdots \times R_n$ where $P_j\in \operatorname{spec}R_j$ then $$\dfrac{R_1\times \cdots \times R_n}{R_1\times \cdots \times P_j\times \cdots \times R_n}\cong R_j/P_j \quad \text{(integral domain)}$$ Hence we get $$R_1\times \cdots \times P_j\times \cdots \times R_n\in \operatorname{spec}(R_1\times \cdots \times R_n)$$ we identify $R_1\times \cdots \times P_j\times \cdots \times R_n$ to $P_j$ then $\operatorname{spec}(R_1\times \cdots \times R_n)=\displaystyle\bigcup_{i=n}^n \operatorname{spec} R_i.$
\begin{qns}
What is the $\operatorname{maxspec}(R_1\times \cdots \times R_n)?$
\end{qns}
Ans. \begin{align*}
\operatorname{maxspec}(R_1\times \cdots \times R_n)&=\{m_1\times \cdots \times R_n:m_1\in \operatorname{maxspec}R_1\}\\
&\quad\quad \cup \{R_1\times m_2\times \cdots \times R_n:m_2\in \operatorname{maxspec}R_2\}\\
&\quad\quad \cup \cdots \cup \{R_1\times \cdots \times m_n:m_n\in \operatorname{maxspec}R_n\}\\
&=\displaystyle\bigcup_{i=n}^n \operatorname{maxspec} R_i
\end{align*}
Let $Q\subseteq R=R_1\times \cdots \times R_n$ be a maximal ideal then $Q=I_1\times \cdots \times I_n$ where $I_i$ is an ideal of $R_i$ then $$R/Q=\dfrac{R_1\times \cdots \times R_n}{I_1\times \cdots \times I_n}\cong R_1/I_1\times \cdots \times R_n/I_n$$ 
As $R/Q$ is an field $\exists ~i$ such that $R_j/I_j=0,~\forall j\neq i,1\leq i\leq n$ this gives $I_j=R_j,~\forall j\neq i,1\leq i\leq n$ and $R/Q=R_i/I_i$ (field) then $I_i\in \operatorname{maxspec} R_i.$ Let $I_i=m_i\in \operatorname{maxspec} R_i$ so that \begin{align*}
Q&=R_1\times \cdots \times R_{i-1}\times I_i\times R_{i+1}\times \cdots \times R_n\\
&R_1\times \cdots \times R_{i-1}\times m_i\times R_{i+1}\times \cdots \times R_n
\end{align*}
Conversely, if we take $Q=R_1\times \cdots \times R_{j-1}\times m_j\times R_{j+1}\times \cdots \times R_n$ where $m_j\in \operatorname{maxspec}R_j$ then $$\dfrac{R_1\times \cdots \times R_n}{R_1\times \cdots \times m_j\times \cdots \times R_n}\cong R_j/m_j \quad \text{(field)}$$ Hence we get $$R_1\times \cdots \times m_j\times \cdots \times R_n\in \operatorname{maxspec}(R_1\times \cdots \times R_n)$$ we identify $R_1\times \cdots \times m_j\times \cdots \times R_n$ to $m_j$ then $\operatorname{maxspec}(R_1\times \cdots \times R_n)=\displaystyle\bigcup_{i=n}^n \operatorname{maxspec} R_i.$
\begin{qns}
What will happen in case of arbitrary product $\displaystyle\prod_{i\in I} R_i?$
\end{qns}
Ans. 
\begin{obs}
Product ring is never a local ring.
\end{obs}
\begin{qns}
Give an example of a ring which has five maximal ideals.
\end{qns}
Ans. Take direct product of five local ring.


\subsection{Chinese Remainder Theorem}
Let $I,J$ be two ideals of $R$ and we consider the map \begin{align*}
\theta:&R\to R/I\times R/J\\
&x\mapsto (x+I,x+J)
\end{align*}
then $\operatorname{Ker}\theta=I\cap J.$ Let $x\in \operatorname{Ker}\theta\Leftrightarrow x+I=0+I~\text{and}~x+J=0+J\Leftrightarrow x\in I,J\Leftrightarrow x\in I\cap J.$ By first isomorphism theorem $$\dfrac{R}{I\cap J} \stackrel{\tilde{\theta} }{\hookrightarrow} R/I\times R/J$$
If we consider the comaxilaity condition on $I$ and $J$ i.e., $1\in I+J$ then \\
(i) $IJ=I\cap J$\\
(ii) $\theta$ is surjective hence, \\
(iii) $\dfrac{R}{I\cap J}\cong R/I\times R/J$ (Chinese Remainder theorem)
\proof (i) Let $a\in I\cap J$ and $1\in I\cap J \Rightarrow 1=s+t$ for some $s\in I$ and $t\in J.$ Then $a=a\cdot 1=a(s+t)=as+at \quad[a\in J,s\in I~\text{and}~a\in I,t\in J] \Rightarrow a\in IJ$
then $I\cap J\subseteq IJ.$ Therefore, $IJ=I\cap J.$ \\
(ii) Let $(x+I,y+J)\in R/I\times R/J$ and we consider $t=sx+ry$ [$1=r+s,r\in I,s\in J$] then $t-x=x(s-1)+ry=-rx+ry\in I$ so that $t+I=x+I$ again $t-y=sx+(r-1)y=sx-sy\in J$ then $t+J=y+J.$ This gives $\theta(t)=(t+I,t+J)=(x+I,y+J)$ hence $\theta$ is surjective therefore, $\dfrac{R}{I\cap J}\cong R/I\times R/J.$\\
Let $I_1,\cdots ,In$ be ideals of $R$ such that $1\in I_i+I_j,i\neq j,1\leq i,j\leq n$ then $$\dfrac{R}{I_1\cdots I_n}\cong R/I_1\times \cdots \times R/I_n$$
\proof $n=2$ we are done. Let $I=I_n$ and $J=I_1\cdots I_{n-1}$ by Chinese remainder theorem $\dfrac{R}{I_1\cdots I_n}\cong R/I_n\times R/I_1\cdots I_{n-1}$ by induction $R/I_1\cdots I_{n-1}\cong R/I_1\times \cdots \times R/I_{n-1},$ hence $$\dfrac{R}{I_1\cdots I_n}\cong R/I_1\times \cdots \times R/I_n.$$ \qed
\begin{definition}
Let $R$ be any ring and $e\in R$ is said to be idempotent element if $e^2=e.$
\end{definition}
\begin{definition}
A ring is said to be Boolean ring if every non-identity element is idempotent element.
\end{definition}
\begin{obs}
$R$ is direct product of two rings iff exists a non trivial idempotent $e\in R.$
\end{obs}
\proof Suppose, $e\in R$ and $e\neq 0,1$ be a non trivial idempotent such that $e^2=e \Rightarrow e(1-e)=0.$ $R\cong R/(0)=R/\langle e(1-e)\rangle$ as $1\in \langle e\rangle+\langle 1-e\rangle.$ By Chinese remainder theorem $$R\cong R/eR\oplus R/(1-e)R$$ Now $R\stackrel{\theta}{\longrightarrow } R(1-e)$ by $x\mapsto x(1-e)$ since $\theta$ is surjective homomorphism $\operatorname{Ker}\theta=\{x(1-e):x(1-e)=0\}.$ Now, $x=x\cdot 1=x(e+(1-e))=xe$ [$x(1-e)=0$] and $x\in eR$ and $eR\subseteq \operatorname{Ker}\theta$ then $\operatorname{Ker}\theta=eR$ $$R/eR\cong R(1-e)$$ Similarly, $R/(1-e)R\cong Re$ therefore, $R\cong Re\oplus R(1-e).$\\
Conversely, if $R\cong R_1\oplus R_2$ take $e=(1,0)$ and $e\neq (0,0),(1,1)$ then $e^2=e,$ $e$ is non trivial idempotent. \qed



\newpage
\section{Field of fraction and Localization}
\subsection{Field of fraction}
\begin{defn}
Let $R$ be a ring and $S\subseteq R$ is said to be multiplicatively closed set if
\begin{enumerate}
\item[(i)] $1\in S$
\item[(ii)] $a,b\in S\Rightarrow ab\in S.$
\end{enumerate}
\end{defn}
\begin{eg}
\begin{enumerate}
\item Let $a\in R$ then the set $\{1,a,a^2,\cdots \}$ is a multiplicatively closed set.
\item If $P\in spec~R$ then $R-P$ is a multiplicatively closed set.
\end{enumerate}
\end{eg}
\textbf{Construction of fraction ring.}\\
Let $S$ be a multiplicatively closed set of $R.$ We define a relation $\sim$ on $\Sigma=\{a/b|~a\in R,b\in S\}$ in the following way. $$\dfrac{a}{b}\sim \dfrac{c}{d} ~\text{if and only if}~\exists~ s\in S~\text{such that}~s(ad-bc)=0$$ 
Check that `$\sim$' is an equivalence relation. $$\dfrac{a}{b}\sim \dfrac{a}{b} ~\text{as}~s(ab-ab)=0,~\forall a\in R,b\in S$$
Therefore $\sim$ is reflexive. Now suppose, \begin{align*}
\dfrac{a}{b} \sim \dfrac{c}{d}~\text{then}~\exists s\in S~\text{such that}~s(ad-bc)=0\Rightarrow (-s)(bc-da)=0 \Rightarrow \dfrac{c}{d}\sim \dfrac{a}{b}
\end{align*}
Hence $\sim$ is symmetric. Let $\dfrac{a}{b},\dfrac{c}{d},\dfrac{e}{f}\in \Sigma$ where $a,c,e\in R$ and $b,d,f\in S$ then there exists $s_1,s_2\in S$ such that $$s_1(ad-bc)=0~\text{and}~s_2(cf-ed)=0$$ but this gives \begin{align*}
0&=s_2s_1(adf-bcf)+s_1s_2(bcf-bed)\\
&=s_1s_2(adf-bed)\\
&=s'(af-be)\quad \text{as $s_1,s_2,d\in S \Rightarrow s'=s_1s_2d\in S$ (say)}
\end{align*}
then we have $$\dfrac{a}{b}\sim \dfrac{c}{d},\dfrac{c}{d}\sim \dfrac{e}{f} \Rightarrow \dfrac{a}{b}\sim \dfrac{e}{f}$$ therefore, $\sim$ is symmetric also. Hence $\sim$ is an equivalence relation. We define, $$S^{-1}R:=\{cl(a/b):a/b\in \Sigma\}$$ By abuse of notation we only write $\dfrac{a}{b}$ instead of $cl(a/b).$ Now, \begin{align*}
+:&~S^{-1}R\times S^{-1}R\to S^{-1}R\\
 &\left(\dfrac{a}{b},\dfrac{c}{d}\right)\mapsto \dfrac{ad+bc}{bd}\\
\cdot:&~S^{-1}R\times S^{-1}R\to S^{-1}R\\
 &\left(\dfrac{a}{b},\dfrac{c}{d}\right)\mapsto \dfrac{ad}{bc}
\end{align*}
Show that `$+$' and `$\cdot$' is well defined. Let $\dfrac{a}{b}=\dfrac{a_1}{b_1}$ and $\dfrac{c}{d}=\dfrac{c_1}{d_1}$ then $\exists s_1,s_2\in S$ such that $$s_1(ab_1-a_1b)=0 ~\text{and}~s_2(cd_1-c_1d)=0$$ Now, \begin{align*}
s_1s_2dd_1(b_1a-ba_1)&+s_1s_2bb_1(cd_1-c_1d)=0\\
\dfrac{ad+bc}{bd}&=\dfrac{a_1d_1+b_1c_1}{b_1d_1}
\end{align*}
Therefore, `$+$' is well defined.
Similarly, \begin{align*}
s_1s_2cd_1(ab_1-a_1b)&+s_1s_2a_1b(cd_1-c_d)=0\\
\dfrac{ac}{bd}&=\dfrac{a_1c_1}{b_1d_1}
\end{align*}
Hence `$\cdot$' is well defined also. Therefore, $\left(S^{-1}R,+,\cdot\right)$ is a ring with identity where $0/1$ is additive identity and $1/1$ is multiplicative identity. Net we consider the ring homomorphism \begin{align}
\theta:~&R\to S^{-1}R\\
&a\mapsto a/1 \nonumber
\end{align}
It is called canonical ring homomorphism. 
\begin{remark}
If $0\in S$ then for each element $a/b$ of $S^{-1}R$ have $0(a\cdot 1-b\cdot 0)=0 \Rightarrow a/b=0/1$ therefore, $S^{-1}R$ is a trivial ring.
\end{remark}
\begin{remark}
$\theta(S)\subseteq S^{-1}R$ is a set of units. Let $s/1\in \theta(S) \Rightarrow s/1\cdot 1/s=1/s \cdot s/1=1/1$ $[$ as $s\in S \Rightarrow 1/s \in S^{-1}R].$ Therefore, $\theta(S)\subseteq $ units of $S^{-1}R.$ 
\end{remark}
\begin{remark}
If $S$ contain no zero divisor then $\theta$ is injective. As $\operatorname{Ker}\theta=\{r\in R:r/1=0/1\}.$ Let $r\in \operatorname{Ker}\theta \Leftrightarrow \exists s\in S~\text{such that}~s(r\cdot 1-1\cdot 0)=0 \Leftrightarrow rs=0 \Leftrightarrow r=0.$
\end{remark}
\begin{corollary}
If $R$ is integral domain then $\theta$ is injective.
\end{corollary}
\textbf{Universal property of localised ring.}
\begin{theorem}
Let $f:A\to B$ be a ring homomorphism and $S\subseteq A$ be a multiplicatively closed set such that $f(S)\subseteq $ units in $B$ then there is a unique ring homomorphism $\tilde{f}:S^{-1}A\to B$ such that the diagram is commutative. 
\begin{center}
\begin{tikzcd}
A \arrow[rr, "f"] \arrow[dd, "\theta"'] &  & B \\
                                        &  &   \\
S^{-1}A \arrow[rruu, "\tilde{f}"']      &  &  
\end{tikzcd}
\end{center}
\end{theorem}
\proof Define $\tilde{f}:S^{-1}A\to B$ by $\tilde{f}(a/b)=\dfrac{f(a)}{f(b)}.$ We claim that $\tilde{f}$ is well defined. Let $a/b \sim c/d$ then there exists $s\in S$ such that $s(ad-bc)=0\Rightarrow f(s)(f(a)f(d)-f(b)f(c))=0.$ Since $f(s)$ is unit in $S,$ $f(a)f(d)=f(b)f(c)$ therefore we have $f(a)/f(b)=f(c)/f(d)$ [since $f(b),f(d)$ is units in $B$]. 
\begin{align*}
\tilde{f}(a_1/b_1+a_2/b_2)&=\tilde{f}\left(\dfrac{a_1b_2+b_2a_1}{b_1b_2}\right)=\dfrac{f(a_1b_2+b_1a_2)}{f(b_1b_2)}\\
&=\dfrac{f(a_1)f(b_2)+f(a_2)f(b_1)}{f(b_1)f(b_2)}\\
&=\dfrac{f(a_1)}{f(b_2)}+\dfrac{f(a_2)}{f(b_2)}\\
&=\tilde{f}(a_1/b_1)+\tilde{f}(a_2/b_2)
\end{align*}
Similarly, \begin{align*}
\tilde{f}\left(\dfrac{a_1}{b_1}\cdot \dfrac{a_2}{b_2}\right)=\tilde{f}\left(\dfrac{a_1a_2}{b_1b_2}\right)=\dfrac{f(a_1)f(a_2)}{f(b_1)f(b_2)}=\tilde{f}(a_1/b_1)\tilde{f}(a_2/b_2)
\end{align*}
and $\tilde{f}(1/1)=1_B.$ Therefore, $\tilde{f}\circ \theta(a)=\tilde{f}(a/1)=f(a)/f(1)=f(a)$ for all $a\in A \Rightarrow \tilde{f}\circ \theta=f$ (i.e., the diagram is commutative).


\textbf{Uniqueness.} Suppose, there is another ring homomorphism $g:S^{-1}A\to B$ such that $g\circ \theta=f.$ Let $a/b \in S^{-1}A,a\in A,b\in S \Rightarrow g\circ \theta(b)=f(b) \Rightarrow g(b/1)=f(b).$ Now, $$g(1/s)=g((\theta(s))^{-1})=(g(\theta(s)))^{-1}=1/f(s).$$ Therefore, $g(a/b)=g(a/1\cdot 1/b)=g(a/1)g(1/b)=f(a)/f(b)=\tilde{f}(a/b)~\forall a/b \in S^{-1}B \Rightarrow g=\tilde{f}.$ \qed
\begin{obs}
Let $R$ be an integral domain then there is a smallest field $Q(R)$ containing $R.$
\end{obs}
\proof Let $S=R\setminus \{0\}$ then $S$ is a multiplicatively closed set in $R.$ Now, $R$ is an integral domain implies $\theta:R \to S^{-1}R$ is injective hence $R$ is a subring of $S^{-1}R.$ We note that $S^{-1}R$ is a field. If $a/b\in S^{-1}R$ with $a/b\neq 0/1$ i.e., $sa\neq 0$ for all $s\in R\setminus \{0\} \Rightarrow a\neq 0$ then $b/a \in S^{-1}R$ therefore, $a/b\cdot b/a=1/1 \Rightarrow b/a$ is a inverse of $a/b$ is $S^{-1}R.$\\
Let $R\subseteq F$ and $F$ is a field then $S=R\setminus \{0\}\subseteq F$ is units in $F.$ By Universal property there exists a ring homomorphism $\tilde{i}:S^{-1}R \to F$ such that the diagram is commutative.
\begin{center}
\begin{tikzcd}
R \arrow[rr, "i"] \arrow[dd, "\theta"'] &  & F \\
                                        &  &   \\
S^{-1}R \arrow[rruu, "\tilde{i}"']      &  &  
\end{tikzcd}
\end{center} But, 
$$\tilde{i}(a/b)=\dfrac{i(a)}{i(b)}=\dfrac{a}{b}$$ is not a zero map $\Rightarrow \tilde{i}$ is injective hence $S^{-1}R\subseteq F.$ Therefore, $S^{-1}R$ is the smallest field containing $R.$ \qed\\
\textbf{Notation.} $S^{-1}R:=Q(R)$ is the field of fraction of $R$ (where $R$ is an integral domain).
\begin{obs}
If $R$ is an integral domain, then $S^{-1}R$ is also an integral domain where $S$ is a multiplicatively closed set of $R$ and $0\notin S.$
\end{obs}
\proof Let $a/b,a'/b' \in S^{-1}R.$ If $\dfrac{aa'}{bb'}=\dfrac{0}{1} \Rightarrow \exists t\in S$ such that \begin{align*}
0&=t(aa'\cdot 1-bb'\cdot 0)\\
&=t(ab)\\
&=ab\quad \text{as}~t\in S\\
\end{align*} 
Therefore, $a=0$ or $b=0$ this implies $a/b=0/1$ or $a'/b'=0/1.$ \qed\\
\begin{problem}
Ideals of $S^{-1}R$ is of the form $S^{-1}I$ where $I\subseteq R$ is an ideals of $R$ and $S\cap I=\emptyset.$
\end{problem}
\proof Let $a/b,c/d\in S^{-1}I$ where $a,c\in I$ and $b,d\in S$ then $$\dfrac{a}{b}+\dfrac{c}{d}=\dfrac{ad+bc}{bd}\in S^{-1}I~\text{as}~a\in I,b\in S\Rightarrow ad\in I,~\text{and}~b,d\in S\Rightarrow bd\in S$$
Let $s/t\in S^{-1}R$ and $a/b\in S^{-1}I \Rightarrow \dfrac{s}{t}\cdot \dfrac{a}{b}=\dfrac{sa}{tb}\in S^{-1}I.$ Therefore, $S^{-1}I$ is an ideal of $S^{-1}R.$ Now, if $S\cap I\neq \emptyset$ then $x\in S$ and under canonical map $\theta(x)$ is an unit of $S^{-1}R$ similarly $x\in I$ is a unit in $R$ hence $I=R.$ Therefore, $S\cap I=\emptyset.$\\
Conversely, let $Q\subseteq S^{-1}R$ is an ideal let $J=\theta^{-1}(Q).$ Show that $S^{-1}J=Q,S\cap J=\emptyset.$ So every ideal of $S^{-1}R$ is of the form $S^{-1}J$ where $S\cap J=\emptyset.$ \qed
\begin{problem}
Let $P\in \operatorname{spec}R$ and $S=R-P$ we define $R_P:=S^{-1}R.$ Show that $R_P$ is a local ring with maximal ideal $S^{-1}P.$
\end{problem}
\proof Let $S=R-P$ and $m=S^{-1}P.$ Let $a/s,a'/s\in S^{-1}P \Rightarrow \dfrac{a's+as'}{ss'}\in S^{-1}P,ss'\in S$ ($a's,s'a\in P$ as $a,a'\in P$). Now if $b/r\in S^{-1}R$ and $a/s\in S^{-1}P$ then $\dfrac{b}{r}\cdot \dfrac{a}{s}=\dfrac{ab}{rs}\in S^{-1}P$ (as $rs\in S$ and $ab\in P$) so $S^{-1}P$ is an ideal. We need to show that $S^{-1}R\setminus S^{-1}P$ has only units. Let $a/s\in S^{-1}R\setminus S^{-1}P$ then $a\in S,s\in S \Rightarrow \dfrac{s}{a}\in S^{-1}R\setminus S^{-1}P.$


\begin{problem}
${\ZZ}$ is an integral domain. Consider $S={\ZZ}\setminus p{\ZZ},~p$ prime. Let ${\ZZ}_p=S^{-1}{\ZZ}$ is an integral domain. What is the field of fraction of ${\ZZ}_p?$
\end{problem}
Ans.


\newpage
\section{Polynomial ring}
Let $R$ be a ring and $R[X]$ is the set of all sequences of elements of $R$ $(a_0,a_1,a_2,\cdots )$ such that $a_i=0$ for all but finitely many indices $i.$ 
\begin{enumerate}
\item $R[X]$ is a ring with addition and multiplication defined as $$(a_0,a_1,a_2,\cdots )+(b_0,b_1,b_2,\cdots )=(a_0+b_0,a_1+b_1,a_2+b_2,\cdots )$$ and $$(a_0,a_1,a_2,\cdots )(b_0,b_1,b_2,\cdots )=(c_0,c_1,c_2,\cdots )$$ where $c_i=\displaystyle\sum_{k=0}^i a_kb_{i-k}$
\item If $R$ is a integral domain then $R[X]$ is integral domain.
\item The map \begin{align*}
&R\stackrel{\phi}{\longrightarrow} R[X]\\
&r\mapsto (r,0,0,\cdots )
\end{align*} is an injective ring homomorphism.
\end{enumerate}
\begin{theorem}
Let $R$ be a commutative ring with $1_R$, denote $X=(0,1_R,0\cdots )$ in $R[X]$. Then show that \begin{enumerate}
\item $X^n=(0,0,\cdots ,1_R,0,\cdots )$ (n+1 position)
\item $r\in R, rX^n=(0,0,\cdots ,r,0,\cdots )$ (n+1 position)
\item For each $f\in R[X]$ there is a $n\in {\NN}$ such that $f=a_0+a_1X+\cdots +a_nX^n$. The integer $n$ and the elements $a_i, 0\leq i\leq n$ are unique in the sense that $f=b_0+b_1X+\cdots b_mX^m$ then $m\geq n$ and $a_i=b_i$, $i=0,\cdots ,n$ and $b_k=0, n<k\leq m.$
\end{enumerate}
$a_0$ is constant of $f$ and $n$ is called degree of $f.$
\end{theorem}
\proof (1) True for $n=1$ as $X^1=(0,1_R,\cdots )$ by definition. Let $X^{n-1}=(0,\cdots ,1_R,\cdots )$ ($n^{th}$ position) then $X^n=X^{n-1}X=(0,\cdots ,1_R,\cdots )(0,1_R,\cdots )=(c_0,c_1,\cdots )$. If $0\leq s\leq n$ then $c_s=\displaystyle\sum_{i=0}^s a_ib_{s-i}$ where $(a_0,a_1,\cdots )=(0,\cdots ,1_R,\cdots )~\text{and}~(b_0,b_1,\cdots )=(0,1_R,\cdots )$. If $i\leq s<n$ then $a_i=o \Rightarrow c_s=0$ for $s<n$ and $c_n=a_nb_{n-n}=a_nb_0=0$ and $c_{n+1}=a_nb_{(n+1)-n)}=a_nb_1=1_R.$ If $s>n+1, c_s=a_nb_{s-n}.$ Now, $s>n+1 \Rightarrow s-n>1 \Rightarrow b_{s-n}=0 \Rightarrow c_s=0$. Therefore, $X^n=(0,\cdots ,1_R,0\cdots )$ ($n+1^{th}$ position).\\
(2) $rX^n=(r,0,\cdots )(0,\cdots ,1_R,\cdots )=(0,\cdots ,r,\cdots ).$\\
(3) Let $f=(a_0,a_1,\cdots )\in R[X]$ where $a_i$'s all but finitely many are non zeroes. Then $f=(a_0,\cdots ,0)+(0,a_1,\cdots ,0)+\cdots +(0,\cdots ,a_n,0,\cdots )=\displaystyle\sum_{i=1}^n a_iX^i.$\\
Uniqueness: Let $f=b_0+b_1X+\cdots +b_mX^m=(b_0,b_1,\cdots ,b_m,0,\cdots ))=(a_0,a_1,\cdots a_n,\cdots )$ which imply $b_0=a_0, b_1=a_1,\cdots $. Now, $n$ is the largest integer such that $a_n\neq 0$ if $m<n$ then $b_n=0$ and $a_n=b_n \Rightarrow a_0=0$ which is a contradiction. Therefore, $m\geq n$ and $b_i=0$ for all $n<i\leq m.$ \qed 

\subsection{Substitution or Evaluation}
Let $R$ be a subring of $S$ and $U$ be a subset of $S$ then $R[U]$ is the usbring of $S$ generated by $R$ and $U$ or the smallest subring of $S$ containing both $R$ and $U$. 
\begin{qns}
Ler $R$ be a subring of $S$ and $U,V$ be subset of $S$ then show that $R[U][V]=R[U\cup V]=R[V][U]$.
\end{qns}
If $U$ is finite i.e., $U=\{u_1,\cdots,u_n\}$ then $R[U]=R[u_1][u_1,\cdots,u_n].$ So it suffices to study $R[u]$ i.e., $R$ is a subring of $S$ and $u\in S$ then by definition we know that $R[u]$ is a subring of $S$. As $u\in R[u],u,u^2,\cdots,u^n\in R[u]$ for some $n\in {\NN}$ and $R\in R[u]$ as well so $a_0+a_1u+\cdots+a_nu^n\in R[u],a_i\in R;1\leq i\leq n.$ Let $$R'=\{a_0+a_1u+\cdots+a_nu^n:a_i\in R;0\leq i\leq n,n\in {\NN}\}$$
i.e., $R'$ is the collection of all polynomial ``expression" in $u$ with coefficient from $R$. Then it is easy to see that $R'$ is closed under addition. Let $a_0+\cdots+a_nu^n,b_0+\cdots+b_mu^m\in R[u]$ then $(a_0+\cdots+a_nu^n)(b_0+\cdots+b_mu^m)=\displaystyle\sum_{k=0}^{m+n}\left(\displaystyle\sum_{i=0}^k a_ib_{k-i}\right)u^k\in S$ but $u^k\in R'$ and $\displaystyle\sum_{i=0}^k a_ib_{k-i}\in R'$ hence the product is also in $R'$ therefore, $R'$ is a subring of $S$. Also $R$ is a subring of $R'$ and $u\in R'$ so $R[u]\subseteq R'\subseteq R[u]$ hence $R'=R[u].$
\begin{remark}
Representation of an element of $R[u]$ as polynomial expression may not be unique as in ${\ZZ}[i]$ we write $i^2=-1$ which is two different representation.
\end{remark}
 Clearly we have a ring homomorphism \begin{align*}
 \phi:R[x]&\to R[u]\\
 a_0+\cdots+a_nx^n&\mapsto a_0+\cdots+a_nu^n
 \end{align*}
then it is easy to see that $\phi$ is onto and $R[x]/\ker \phi\cong R[u].$ 
\begin{theorem}[Fundamental morphism property of polynomials]
Let $R_1,R_2$ be two rings and $u\in R_2$. Let $\phi:R_1\to R_2$ be a ring homomorphism then $\phi$ has a unique extension to a ring homomorphism \begin{align*}
\phi_u:R_1[x]\to R_2
\end{align*} such that $\phi_u(x)=u.$
\end{theorem}
\proof We define the map as follows \begin{align*}
\phi_u:R_1[x]&\to R_2\\
a_0+\cdots+a_nx^n&\mapsto \phi(a_0)+\phi(a_1)u+\cdots+\phi(a_n)u^n
\end{align*}
Then its is easy to see that $\phi_u$ is an ring homomorphism and $\phi_u(x)=\phi(1)u=u.$ Let $a\in R_1,\phi_u(a)=a=\phi(a)$ therefore, $\phi_u\big|_{R_1}=\phi.$ If $\psi:R_1[x]\to R_2$ is another ring homomorphism such that $\psi_u\big|_{R_1}=\phi$ and $\phi(x)=u$ then $\psi=\phi_u$(check).


 Now let $R$ be a subring of $S$ and $u\in S$. Take $\phi:R\to S$ be the inclusion map. Then by previous theorem there exists am unique extension $\psi: R[x]\to S$ such that $\psi\big|_R=id$ and $\psi(x)=u.$ This is commonly called ``substitution" or ``evaluation". Let $I=\ker \psi$ then $R[x]/I\cong R[u]$. Let $a\in I\cap R$ then $\psi(a)=0$ but $a\in R$ then $a=0 \Rightarrow I\cap R=\{0\}.$ Therefore, we got $R[u]\cong R[x]/I$ and $I\cap R=\{0\}.$ Conversely, let $I$ be an ideal of $R[x]$ such that $I\cap R=\{0\}$ then consider the natural projection map $\pi:R[x]\to R[x]/I$ then $\pi\big|_R$ is injective. Let $\pi\big|_R(a)=0 \Rightarrow a+I=0+I \Rightarrow a\in I$ and $a\in R\Rightarrow a\in I\cap R=\{0\} \Rightarrow a=0$. So we may think $R\hookrightarrow R[x]/I$ and $R$ can be regarded as a subring of $R[x]/I$. As $R[x]$ is generated by $R$ and $x$, $R[x]/I$ is generated by $R$ and $x+I$ if we write $u=x+I$ then $R[x]/I\cong R[u].$ \qed



\newpage
\section{Euclidean domain, Principal ideal domain, Unique factorization domain}
\subsection*{Factorization in commutative ring}
\begin{definition}
A non-zero element of a commutative ring $R$ is said to be divide an element $b\in R$ (notation $a|b$) if $\exists~ x\in R$ such that $ax=b.$ Elements $a,b$ of $R$ are said to be associates $(a\sim b)$ if $a|b$ and $b|a.$
\end{definition}
\begin{theorem}
Let $a,b ~\& ~u\in R$ be elements of a commutative ring $R$ with identity.\\
$(1)$ $a|b$ iff $(b)\subseteq (a)$\\
$(2)$ $a,b$ are associates iff $(a)=(b)$\\
$(3)$ $u$ is unit iff $u|r$ for all $r\in R$\\
$(4)$ $u$ is unit iff $(u)=R$\\
$(5)$ The relation $a$ is a associates of $b$ is an equivalence relation on $R$\\
$(6)$ If $a=br$ with $r\in R,$ a unit, then $a$ and $b$ are associates. If $R$ is integral domain, converse is also true.
\end{theorem}
\proof $(1)$ Suppose, $a|b$ then $b=ac$ for some $r\in R$ this implies, $b\in (a) \Rightarrow (b)\subseteq (a).$ Conversely, Suppose, $(b)\subseteq (a).$ Now,$b \in (b)\subseteq (a) \Rightarrow b\in (a) \Rightarrow b=ac \Rightarrow a|b.$ \\
$(2)$ Let $a,b$ are associates then $a|b$ and $b|a.$ From previous one we have $(b)\subseteq (a)$ and from later part we have $(a)\subseteq (b),$ hence $(a)=(b).$ Conversely, if $(a)=(b)$ then we have $(b)\subseteq (a)$ and $(a)\subseteq (b).$ Therefore, we have $a,b$ are associates.\\
$(3)$ Suppose, $u$ is unit in $R$ then there exists $v\in R$ such that $uv=1 \Rightarrow u(vr)=r \Rightarrow u|r ~\forall r\in R.$ Conversely, if $u|r,$ for all $r\in R$ then $u|1 \Rightarrow \exists v\in R$ such that $uv=1$ hence $u$ is a unit.\\
$(4)$ If $u$ is a unit then $uv=1$ for some $v\in R.$ Therefore, $1\in (u) \Leftrightarrow (u)=R.$\\
$(5)$ Left as exercise.\\ 
$(6)$ Left as exercise.
\begin{definition}
Let $R$ be a commutative ring with id. An element $c\in R$ is irreducible provided that \\
$(i)$ $c$ is non-zero, non-unit element\\
$(ii)$ $c=ab$ implies $a$ or $b$ is unit.
\end{definition}
\begin{definition}
An element $p$ of $R$ is prime provided\\
$(i)$ $p$ is non-zero, non-unit\\
$(ii)$ $p|ab \Rightarrow p|a$ or $p|b.$
\end{definition}
\begin{definition}
An integral domain $R$ is said to be PID or principal ideal domain if every ideal of $R$ is principally generated.
\end{definition}
\begin{theorem}
Let $p$ and $c$ be non-zero elements in an integral domain $R$.\\
$(i)$ $p$ is prime iff $(p)$ is a non-zero prime ideal.\\
$(ii)$ $c$ is irreducible iff $(c)$ is maximal in set of all proper principal ideals of $R$.\\
$(iii)$ Every prime element of $R$ is irreducible.\\
$(iv)$ If $R$ is a PID, then $p$ is prime iff $p$ is irreducible.\\
$(v)$ Every associates of an irreducible (prime) element of $R$ is irreducible(prime).\\
$(vi)$ The only divisor of an irreducible elements of $R$ are its associates and units of $R.$
\end{theorem}
\proof $(i)$ Let $p$ is prime. Now, if $ab\in (p) \Rightarrow ab=pk \Rightarrow p|ab$ then either $p|a$ or $p|b$ therefore, $(a)\subseteq (p)$ or $(b)\subseteq (p)$ hence $(p)$ is an prime ideal. Conversely, Let $(p)$ is a prime ideal and let $p|ab \Rightarrow ab\in (p) \Rightarrow a\in (p)$ or $b\in (p) \Rightarrow p|a$ or $p|b.$ Therefore, $p$ is a prime element.\\
$(ii)$ Suppose, $c$ is irreducible and $\Sigma=$ set of all principal ideals in $R.$ suppose, $(c)\subseteq (r) \Rightarrow c\in (r) \Rightarrow c=rs$ where $s$ is unit in $R$ therefore, $(c)=(r)$ hence $(c)$ is maximal ideal in $\Sigma.$ Conversely, Let $(c)$ is maximal element in $\Sigma.$ If $c=rs \Rightarrow c\in (r) \Rightarrow (c)\subseteq (r) \Rightarrow c=r \Rightarrow r|c$ then $r=ck$ for some $k\in R$ then we have $$c=rs=cks \Rightarrow c(1-ks)=0$$
Since $R$ is an integral domain $1-ks =0 \Rightarrow s$ is unit therefore, $c$ is irreducible.\\
$(iii)$ Let $p$ be a prime element in $R$ and $p=rs \Rightarrow rs\in (p) \Rightarrow r\in (p)$ or  $s\in (p)$ then $r=pl_1$ or $s=pl_2.$ If $r=pl_1$ then $p=pl_1s\Rightarrow p(1-l_1s)=0 \Rightarrow 1-l_1s=0 \Rightarrow s$ is unit therefore, $p$ is irreducible. If $s=pl_2$ by similar way we get $p$ is irreducible.\\
$(iv)$ If $R$ is integral domain then prime element implies irreducible element. If $R$ is PID we need to show that irreducible implies prime. Let $c$ be an irreducible element then $(c)$ is maximal in $\Sigma$ then $(c)$ is maximal ideal then $(c)$ is prime ideal therefore, $c$ is prime element.\\
$(v)$ Suppose, $a$ is irreducible and $b$ is associates of $a$ then $(b)=(a).$ but $(a)$ is maximal in $\Sigma$ therefore, $(b)$ is also maximal hence $b$ is irreducible.\\
$(vi)$ Let $a$ be an irreducible and $b|a.$ Suppose, $b$ is not an unit then $(b)\in \Sigma.$ Now, $b|a$ implies $(a)\subseteq (b)$ since $(a)$ is maximal, $(b)=(a).$ Therefore, $b$ and $a$ are associates. \qed
\begin{ex}
Show that ${\ZZ}$ is a PID.
\end{ex}
\proof Let $I\subseteq R$ be an ideal. If $I=\gen{0}$ then we are done. Let say $I\neq \{0\}$ and $n$ be the least positive integer in $I.$ Clearly, $\gen{n}\subseteq I.$ Suppose, $x\in I$ then by division algorithm $\exists~q,r\in I$ such that $x=nq+r$ where $r=0$ or $r<n.$ As, $r=x-nr\in I$ then $r$ must be zero otherwise it contradict the definition of $n.$ Therefore, $x=nq$ and $I=\gen{n}=n{\ZZ}.$ \qed

\subsection{Division in ${\ZZ}[i]$}
Let $\alpha,\beta\in {\ZZ}[i] $ with $\beta\neq 0$ then there exists $\gamma,\delta\in {\ZZ}[i]$ such that $\alpha=\gamma \cdot \beta +\delta$ where $\delta=0$ or $|\delta|<|\beta|$. We want to find $\gamma\in {\ZZ}[i]$ such that $\bigg|\dfrac{\alpha}{\beta}-\gamma\bigg|<1$ i.e., $\bigg|\dfrac{\gamma}{\beta}\bigg|<1.$ Consider the complex number $\dfrac{\alpha}{\beta} \in {\CC}.$ Maximum distance of $\dfrac{\alpha}{\beta}$ from a point in ${\ZZ}[i]$ is $\leq \dfrac{1}{2}(\text{diagonal})=1/2\cdot \sqrt{2}=\sqrt{2}<1.$ Hence, $\gamma\in {\ZZ}[i]$ and $\bigg|\dfrac{\alpha}{\beta}-\gamma\bigg|<1$ therefore, $\delta=\alpha-\beta\gamma.$
\begin{definition}
Let $R$ be a ring. A norm is a function $N:R\to {\NN}$ such that $N(0)=0.$
\end{definition}
\begin{definition}
Let $R$ be a commutative ring with 1. $R$ is Euclidean ring if there exists a norm function $N:R\setminus \{0\}\to {\NN}$ such that \begin{enumerate}
\item If $a,b\in R$ and $ab\neq 0$ then $N(a)\leq N(ab)$,
\item If $a,b\in R$ and $b\neq 0$ then there exists $q,r$ such that $a=qb+r$ with $r=0$ or $N(r)<N(b).$
\end{enumerate}
\end{definition}
\begin{definition}
An Euclidean ring which is integral domain is Euclidean domain.
\end{definition}
\begin{eg}
Let $F$ be a field then $F$ is a ED. Let $a,b(\neq 0)\in F$ then $a=(ab^{-1})b+0$ hence $N:F\to {\NN}$ sends everyone to zero.
\end{eg}
\begin{ex}
Show that ${\ZZ}[i]$ is an Euclidean domain.
\end{ex}

\begin{proposition}
Every Euclidean ring is principal ideal ring.
\end{proposition}
\proof If $0\neq I\subseteq R$ be an ideal. Choose $a\in I$ such that $N(a)$ is the least integer in the set $\{N(a):a\neq 0,a\in I\}$. If $b\in I$ then $b=aq+r$ with $r=0$ or $N(r)<N(a)$. If $r\neq 0$ then $r=b-aq\in I$ is a contradiction. Therefore, $r=0$ then $b=aq \Rightarrow b\in \gen{a} \Rightarrow I\subseteq \gen{a}.$ Again, $a\in I \Rightarrow \gen{a}\subseteq I$. Therefore, $I=\gen{a}.$ \qed

\begin{proposition}
If $K$ is a field then $K[x]$ is a Euclidean domain.
\end{proposition}
\proof Let $N:K[x]\setminus \{0\} \to {\NN}$ by $N(f(x))=\deg f.$ Suppose,
$f,g\in K[x]\setminus \{0\}, K$ is integral domain then $K[x]$ is also integral domain hence $fg\neq 0.$ Let $f(x)=a_nx^n+\cdots +a_1x+a_0$, $g(x)=b_mx^m+\cdots +b_1x+b_0$ so that $fg=a_0b_0+\cdots a_nb_mx^{n+m}$ where $a_nb_m\neq 0.$ Clearly, $\deg (fg)\geq \deg f.$ $g(x)\neq 0.$ If $\deg f<\deg g$ then $f=0\cdot g+f$ so we assume that $\deg f>\deg g.$ If $m\leq n$ then $$f=\dfrac{a_n}{b_m}x^{n-m}(b_mx^m+\cdots +b_0)+r_1(x)$$ where $$r_1(x)=f-\dfrac{a_n}{b_m}x^{n-m}g(x)$$ and $\deg r_1(x)<\deg f$. If $\deg r_1(x)<\deg g(x)$ the we are done. Otherwise, let $r_1(x)=\dfrac{r_s}{b_m}x^{s-m}g(x)+r_2(x)$ where $\deg r_2(x)<\deg r_1(x).$ Continuing this process we get $f(x)=p(x)g(x)+r(x)$ where $r(x)=0$ or $\deg r(x)<\deg f(x).$ \qed
\begin{remark*}
\begin{enumerate}
\item The same proof works for any commutative ring $R$ and in $R[x]$ provided that $g(x)$ has leading term unit i.e., in particular if $g$ is monic otherwise division is not possible for example ${\ZZ}[x]$ is not an Euclidean domain.
\item The quotient and remainder is unique.
\end{enumerate}
\end{remark*}
\begin{definition}
Let $R$ be a commutative ring and $a,b$ be non zero elements in $R$ then greatest common divisor (gcd) of $a,b$ is a non zero element $d$ such that \begin{enumerate}
\item $d|a$ and $d|b$
\item If $c|a$ and $c|b$ then $c|d.$
\end{enumerate}
\end{definition}
\textbf{Notation:} $d=\gcd (a,b).$\\
Check that, \begin{enumerate}
\item If $d=\gcd (a,b)$ and $u$ is a unit then $du$ is also a gcd.
\item Let $R$ be a domain and $d,d'$ are both gcd of $a,b$ then there exist a unit $u\in R$ such that $d=ud'.$
\end{enumerate} 
\subsection{Algorithm for finding gcd of two elements in a Euclidean domain}
Let $R$ be an Euclidean domain and $a,b\in R$  be two non zero elements then, 
\begin{align*}
a&=q_0b+r_0,\quad &N(r_0)&<N(b)\\
b&=q_1r_0+r_1,\quad &N(r_1)&<N(r_0)\\
r_0&=q_2r_1+r_2,\quad &N(r_2)&<N(r_1)\\
&\vdots   &\vdots\\
r_{n-2}&=q_nr_{n-1}+r_n,\quad &N(r_n)<&N(r_{n-1})\\
r_{n-1}&=q_{n+1}r_n
\end{align*}
where $r_n$ be the last non zero remainder. Note that the process must stop after finite stage as $$N(b)>N(r_0)>N(r_1)>\cdots $$ is a strictly decreasing chain of positive integers.\\
Claim: $r_n=\gcd(a,b)$ \begin{enumerate}
\item $r_n|a$ and $r_n|b$ (start from the bottom and go up)
\item If $e|a$ and $e|b$ then $e|r_n$ (start from top and go down)
\end{enumerate}
Show that $\gen{a,b}=\gen{r_n}$.\\
\textbf{Conclusion:} In an Euclidean domain, gcd exists. Further if $d=\gcd (a,b)$ then $\exists~x,y\in R$ such that $ax+by=d.$

\begin{theorem}
Every Euclidean domain is Principal ideal domain.
\end{theorem}
\proof Let $R$ be an Euclidean domain and $N$ be the standard Euclidean norm. Suppose, $I$ is an ideal of $R$. If $I=\gen{0}$ then we are done so let us assume that $I\neq \gen{0}.$ $$\Sigma=\{N(a):a\in I\}$$
By well ordering principal of natural number, $I$ has a element that has minimum norm say $a$ then $\gen{a}\subseteq I.$ Now, pick $b\in I$ then by division algorithm, there exists $q,r\in R$ such that $b=aq+r$ with $r=0$ or $N(r)<N(a).$ If $r\neq 0$ then $r=b-aq\in I$ with norm less than $a$ which contradicts the choice of $a$. Hence, $r=0$ and $I\subseteq \gen{a}.$ Therefore, $I=\gen{a}.$ \qed
\begin{lemma}
Any two non zero element $a,b\in R$ where $R$ is Euclidean domain having gcd $d$ then $d=ax+by$ for some $x,y\in R.$
\end{lemma}
\proof Let, $I=\{as+bt:s,t\in R\}$. Suppose, $d\in I$ has minimum norm and $d=ax+by$ for some $x,y\in R.$ As, $d\in I,$ by division algorithm, $a=dq_1+r_1$ with $r_1=0$ or $N(r_1)<N(d).$ Now, $r_1=a-dq_1\in I$ and $N(r_1)<N(d)$ which is contradiction, therefore, $r_1=0$ and $d|a_1$. Similarly we can show $d|b.$ Hence, $d$ is a common divisor of $a,b.$ Moreover, $d=ax+by$ so every common divisor of $a,b$ is also divide $d$ hence $d=\gcd (a,b).$ \qed
\begin{eg}
${\ZZ}[\sqrt{-5}]=\{a+b\sqrt{-5}:a,b\in {\ZZ}\}$. ${\ZZ}[\sqrt{-5}]$ is not an PID, hence not an ED.
\end{eg}
Note that ${\ZZ}[\sqrt{-5}]$ has a norm \begin{align*}
N:{\ZZ}[\sqrt{-5}]\to {\NN}\\
z\mapsto |z|^2
\end{align*}
and $N$ is multiplicative i.e., $N(z_1z_2)=N(z_1)N(z_2).$\\\\
\textbf{Units of ${\ZZ}[\sqrt{-5}]$.} Let $a+b\sqrt{-5}\in {\ZZ}[\sqrt{-5}]$ and $N(a+b\sqrt{-5})=1 \Rightarrow a^2+5b^2=1$ this gives, $a=\pm 1$ and $b=0.$ Therefore, $1,-1$ are the only norm one element. Now, let $\alpha$be a unit in ${\ZZ}[\sqrt{-5}]$ then there exists $\beta$ such that \begin{align*}
\alpha\beta&=1\\
N(\alpha\beta)&=N(1)\\
N(\alpha)N(\beta)&=1
\end{align*}
Hence, $N(\alpha)=1$ gives only units of ${\ZZ}[\sqrt{-5}]$ are $1$ and $-1$.
\begin{qns*}
Show that ${\ZZ}[\sqrt{-5}]$ is not a PID.
\end{qns*}
Ans. Consider the ideal $I=\gen{2,1+\sqrt{-5}}$ we claim that $I$ is not principal ideal. First we show that $I\neq {\ZZ}[\sqrt{-5}]$. Let \begin{align*}
1&=2\alpha+\beta(1+\sqrt{-5})\\
&=2(a+b\sqrt{-5})+(c+d\sqrt{-5})(1+\sqrt{-5})\\
&=(2a+c-5d)+(2b+d+c)\sqrt{-5}
\end{align*}
Comparing the coefficient of real and imaginary part, we get $2a+c-5d=1$ and $2b+d+c=0$. From this two equation we get $2a=1$ which is a contradiction.\\
Suppose, $\gen{2,1+\sqrt{-5}}=\gen{\alpha}$ for some $\alpha\in {\ZZ}[\sqrt{-5}]$. As, $2\in \gen{\alpha}$ there exists $\beta\in {\ZZ}[\sqrt{-5}]$ such that $2=\alpha\beta$ similarly,
$1+\sqrt{-5}=\gamma\alpha$ for some $\gamma\in {\ZZ}[\sqrt{-5}].$ Applying norm, we get $4=N(2)=N(\alpha)N(\beta)$ then $N(\alpha)=1,2$ and 4. $N(\alpha)= 1$ will imply $\alpha$ is a unit and $I={\ZZ}[\sqrt{-5}].$ Again, from second equation we get $6=N(\alpha)N(\gamma)$. From these two equation we get $N(\alpha)=2$. Let, $\alpha=a+b\sqrt{-5}$ so that $2=N(\alpha)=a^2+5b^2$ but no integer solution can be found for this equation. Hence $I$ is not principally generated, therefore, ${\ZZ}[\sqrt{-5}]$ is not a PID.
\begin{qns}
Give an example of an element which is irreducible but not prime.
\end{qns}
Ans. In ${\ZZ}[\sqrt{-5}]$, 2 is irreducible but not prime. Let, $2=(a+b\sqrt{-5})(c+d\sqrt{-5})$. Applying norm, we have $4=(a^2+5b^2)(c^2+5d^2)$ then either $a^2+5b^2=1$ or $c^2+5d^2=1$ (as we saw there is no element of norm 2 in ${\ZZ}[\sqrt{-5}]$). Therefore, either $a+b\sqrt{-5}$ is unit or $c+d\sqrt{-5}$ is unit. Hence, 2 is irreducible.\\
We show that $2$ is not a prime as $2|6=2|(1+\sqrt{-5})(1-\sqrt{-5})$ but $2\nmid (1+\sqrt{-5})$ and $2\nmid (1-\sqrt{-5})$. Therefore, 2 is not a prime element.

\begin{definition}
An integral domain is said to be a Factorization domain if every non zero, non unit element can be factored as product of irreducible elements.
\end{definition}
\begin{definition}
A factorization domain is said to be Unique Factorization Domain (UFD) if every non zero, non unit element $a$ can be factored as $a=p_1\cdots p_n$ where $p_i$'s are irreducible. If $a=q_1\cdots q_m$ where $q_i$'s are irreducible then $m=n$ and $p_i$ is associated with $q_{\sigma (i)}$ for some $\sigma\in S_n.$
\end{definition}
\textbf{Note:} ${\ZZ}[\sqrt{-5}]$ is not an UFD since, $6=2\cdot 3=(1+\sqrt{-5})(1-\sqrt{-5})$ are two different factorization of $6$. We claim that \begin{enumerate}
\item 2,3,$1+\sqrt{-5}$ and $1-\sqrt{-5}$ are irreducible,
\item $2\nsim 1+\sqrt{-5}$ or $1-\sqrt{-5}$ and $3\nsim 1+\sqrt{-5}$ or $1-\sqrt{-5}$. 
\end{enumerate}
Suppose, 2 is reducible then $2=(a+b\sqrt{-5})(c+d\sqrt{-5})\Rightarrow 4=(a^2+5b^2)(c^2+5d^2)$ then one of them must be a unit. Similarly if $3=(a+b\sqrt{-5})(c+d\sqrt{-5})$ then $9=(a^2+5b^2)(c^2+5d^2)$ implies one of them must be unit. Suppose, $1+\sqrt{-5}=(a+b\sqrt{-5})(c+d\sqrt{-5})$ then $6=(a^2+5b^2)(c^2+5d^2)$ which gives either $a^2+5b^2=1,2,3~\text{or}~6$. As we know there is no element of norm 2,3 in ${\ZZ}[\sqrt{-5}]$ either $a^2+5b^2=1~\text{or}~6$ which give $a+b\sqrt{-5}$ is unit or $c+d\sqrt{-5}$ is unit. Suppose, $2\sim 1+\sqrt{-5}$ which imply $2=u(1+\sqrt{-5})$ for some unit $u$. After applying norm, we have $4=6$ which is impossible.

\begin{proposition}
In a Principal ideal domain, every non-zero prime ideal is maximal.
\end{proposition}
\proof Let $P=\gen{p}$ be a non-zero proper prime ideal. Assume that $P\subseteq M$, where $M$ is a maximal ideal. As $R$ is a PID, $M=\gen{m}.$ Since, $p\in \gen{m} \Rightarrow p=\lambda m.$ We know that in an integral domain, prime element implies irreducible element therefore, $\lambda$ is unit or $m$ is unit. $m$ is not unit as $\gen{m}=R$, so $\lambda$ is unit then $\gen{p}=\gen{m}.$ \qed
\begin{remark*}
$\gen{m}=\gen{\lambda m}$ if $\lambda$ is a unit. Clearly, $\gen{\lambda m}\subseteq \gen{m}$. $\lambda$ is unit implies $m=\lambda^{-1}(\lambda m) \Rightarrow \gen{m}\subseteq \gen{\lambda m}.$ Therefore, $\gen{m}=\gen{\lambda m}.$
\end{remark*}

\begin{proposition}
In a Principal ideal domain, gcd of two elements exists.
\end{proposition}
\proof Consider the ideal generated by $\gen{a,b}.$ As, $R$ is PID, $\gen{a,b}=\gen{d}$ for some $d\in R\setminus \{0\}.$ $a,b\in \gen{d}$ therefore, $d|a$ and $d|b$. Let $e\in R\setminus \{0\}$ such that $e|a,e|b$ then $a,b \in \gen{e} \Rightarrow \gen{a,b}\subseteq \gen{e} \Rightarrow \gen{d}\subseteq \gen{e} \Rightarrow e|d.$ So, $d$ is gcd of $a,b$. Moreover, if $d=\gcd (a,b)$ then there exists $x,y\in R\setminus \{0\}$ such that $d=ax+by.$ \qed



\begin{qns*}
Show that in an integral domain gcd may not exist.
\end{qns*}
Ans. In ${\ZZ}[\sqrt{-5}],$ $\gcd (6,2(1+\sqrt{-5}))$ doesn't exists. Let $\alpha=6=2\cdot 3$ and $\beta=2(1+\sqrt{-5})$. If possible, let $\gamma=\gcd (\alpha,\beta)$ then $\gamma|6 \Rightarrow N(\gamma)|36 \Rightarrow N(\gamma)=1,2,3,4,6,9,12,18,36$ and $\gamma|2(1+\sqrt{-5}) \Rightarrow N(\gamma)|24 \Rightarrow N(\gamma)=1,2,3,4,6,8,12,24.$
Clearly $2|\alpha$ and $2|\beta$ as $\gamma$ is gcd of $\alpha$ and $\beta$, $2|\gamma \Rightarrow 4|N(\gamma)$. Similarly, $(1+\sqrt{-5})|\gamma \Rightarrow N(1+\sqrt{-5})|N(\gamma) \Rightarrow 6|N(\gamma).$ From above relation only possible value for $N(\gamma)$ is 12. Let $\gamma=a+b\sqrt{-5} \Rightarrow 12=N(\gamma)=a^2+5b^2$ but there is no integer solution exists. Therefore, $\gamma$ doesn't exist.

\begin{obs}
Let $X=\{a_1,\cdots ,a_n\}$ and $d=\gcd (X)$ then \begin{enumerate}
\item If $R$ is a principal ideal domain, then $d=a_1r_1+\cdots +a_nr_n,$
\item If $R$ is Unique factorization domain, then $\gcd (X)$ exists.
\end{enumerate}
\end{obs}
\begin{qns*}
Are irreducibles prime in UFD?
\end{qns*}
Ans. Yes. Let $p$ be an irreducible and assume that $p|ab \Rightarrow ab=\lambda p$. Factorizations of the elements of the above equation gives $$(ua_1\cdots a_n)(vb_1\cdots b_m)=(w\lambda_1\cdots \lambda_k)p$$ Then $p$ must be either $a_i$ or $b_i$ which imply either $p|a$ or $p|b$. Thus $p$ is prime.
\begin{proposition}
In a UFD, gcd of two element exist.
\end{proposition}
\proof Let $R$ be a UFD and $a,b\in R\setminus \{0\}$ then unique factorizations of $a$ and $b$ into irreducible element gives \begin{align*}
a&=up_1^{\alpha_1}p_2^{\alpha_2}\cdots p_n^{\alpha_n}\\
b&=vp_1^{\beta_1}p_2^{\beta_2}\cdots p_n^{\beta_n}
\end{align*}
where $u,v$ are units and $\alpha_i,\beta_i\geq 0$ for all $i\in \{1,\cdots ,n\}.$ Let $d=p_1^{\gamma_1}p_2^{\gamma_2}\cdots p_n^{\gamma_n}$ where $\gamma_i=\min \{\alpha_i,\beta_i\}$ for all $1\leq i\leq n.$ Then $d=\gcd (a,b).$ \qed

Note that gcd of $2,x$ exists and is equal to 1 but 1 can't be written as $1=2f(x)+xg(x).$

\begin{theorem}
Let $R$ be a UFD and $P$ be any non zero prime ideal then there exists a non zero prime ideal $P_1$ such that $P_1\subsetneq P$ and $P_1$ is principal.
\end{theorem}
\proof Since $P$ is a non zero prime ideal, we pick $r\in P.$ Then $r=\pi_1^{s_1}\cdots \pi_k^{s_k}\in P \Rightarrow \pi_i^{s_i}\in P$ for some $i$ which imply $\pi_i\in P \Rightarrow (\pi_i)\subsetneq P$. This proof follows from the fact that irreducibles are prime in UFD. \qed

\begin{corollary}
In a UFD $R$, $P\subsetneq R$ is a prime ideal. Then $P$ is principal iff $P=0$ or $P\supsetneq P_1,$ $P_1$ is prime implies $P_1=0.$
\end{corollary}
\proof If $P=0$ then $P=\gen{\emptyset}.$ Let $P$ is a non zero prime ideal which is also principally generated then by previous theorem there exists a prime ideal $P_1=\gen{\pi}\subsetneq P$. Both $P,P_1$ are maximal in the set of all principally generated ideals of $R$ (as $p,\pi$ are irreducible in $R$) we have $P_1=0.$ Conversely, let $P\neq 0$ is a prime ideal and $P_1\subsetneq P$ such that $P_1$ is prime and $P_1$ is zero ideal.


\begin{qns}
Is ${\ZZ}[x]$ PID?
\end{qns}
Ans. No. Since the ideal $\gen{p,x}$ is not principal where $p\in {\ZZ}$ is prime. We know that $\gen{p,x}$ is prime in ${\ZZ}[x]$ and $(p)\subsetneq \gen{p,x}$ is a non zero prime ideal. By previous corollary $\gen{p,x}$ is not principal.

\begin{lemma}

Let $R$ be a PID and $I_1\subseteq I_2\subseteq I_3\subseteq \cdots $ be a chain of ideals then there exist $n\in {\NN}$ such that $I_n=I_k$ for all $n\geq k.$

\end{lemma}

\proof Let $I=\displaystyle\bigcup_{n\in {\NN}} I_n$ then $I$ is an ideal and $I=\gen{a}$ for some $a\in R$ as $R$ is PID. Therefore, $a\in I=\displaystyle\bigcup_{n\in {\NN}} I_n \Rightarrow a\in I_n$ for some $n\in {\NN} \Rightarrow \gen{a}\subseteq I_n \Rightarrow I\subseteq I_n.$ Clearly, $I_n\subseteq I$ hence $I=I_n \Rightarrow I_n=I_{n+1}=I_{n+2}\cdots $. \qed

\begin{theorem}

Every Principal ideal domain is Unique factorization domain.

\end{theorem}

\proof There are two different proof of this theorem and we present it one by one.
Let $R$ be a PID and $r\in R$ be a non zero, non unit element. If $r$ is irreducible then we stop. If not $r=r_1r_2$ where $r_,r_2$ is not unit. If $r_1,r_2$ are irreducible then we stop. If not, say $r_1$ is reducible then $r_1=r_{11}r_{12}$ where neither $r_{11}$ nor $r_{12}$ is unit then $\gen{r}\subsetneq \gen{r_1}$(as $r_2$ is not unit) and $\gen{r_1}\subsetneq \gen{r_{11}}$ (since $r_{12}$ is not an unit). Also note that they all are proper ideal and $$\gen{r}\subsetneq \gen{r_1}\subsetneq \gen{r_{11}}\subsetneq \cdots $$
This is an increasing chain of ideal and as $R$ is PID, it stabilize after some finite step. Thus we conclude that $r$ is a product of finitely many irreducible. We claim that such factorization is unique. Let $r=p_1\cdots p_n=q_1\cdots q_m$ be two factorization of $r$ into irreducibles. $p_1$ is irreducible and $R$ is PID then $p_1$ is prime thus $p_1|q_1\cdots q_m \Rightarrow p_1|q_i$ for some $i$. Renaming if necessary, let $p_1|q_1 \Rightarrow q_1=\lambda _1 p_1.$ As, $q_1$ is irreducible, $\lambda _1$ is unit and $r=p_1\cdots p_n=(\lambda _1p_1)\cdots q_m \Rightarrow p_2\cdots p_n=\lambda _1q_2\cdots q_m$ then $p_2|q_i$ for some $2\leq i\leq m$. Renaming if necessary, let say $p_2|q_2 \Rightarrow p_2=\lambda _2q_2$ where $\lambda _2$ is an unit. Therefore, $n=m$ and $p_i\sim q_i$ for all $i.$

\textit{Another proof.} We consider $$\sum=\{x\in R: x~\text{is non-zero, non-unit and can't be written as finite product of irreducibles}\}$$ If $\sum \neq \emptyset$, let $a\in \sum,$ since $a$ is non-zero,non-unit $\gen{a}\subseteq R$ is proper so there is an irreducible $c_a\in R$ such that $\gen{a}\subseteq \gen{c_a}.$ Therefore, $c_a|a \Rightarrow a=c_ax_a.$ If $x_a$ is a unit then $c_a$ and $a$ is associates so $a$ is irreducible. Hence, $x_a$ can't be a unit. If $x_a$ is finite product of irreducible then also this is a contradiction. Again $x_a=0 \Rightarrow a=0$, so $x_a\in \sum.$ Note that, $\gen{a}\subsetneq \gen{x_a}.$ Now we define a map \begin{align*}
f:\sum &\to \sum\\
a&\mapsto x_a
\end{align*}
$f$ is well defined. [Suppose, $a=c_ax_a=c_ay_a \Rightarrow x_a=y_a$]
Now, define \begin{align*}
\phi:{\NN}&\to \sum\\
n&\mapsto f^n(a):=x_{\phi(n)}
\end{align*}
Then \begin{equation}
\gen{a}\subsetneq \gen{x_{\phi(1)}}\subsetneq \gen{x_{\phi(2)}}\subsetneq \cdots 
\end{equation}

Since $R$ is PID, the chain must be stationary i.e., $\exists~n\in {\NN}$ such that $\gen{x_{\phi(n)}}=\gen{x_{\phi(i)}},\forall~i\geq n$. Therefore, $\sum$ is empty.\\
Let $a=c_1\cdots c_n=d_1\cdots d_m$ where $c_i,d_i$ are irreducible and $c_i|d_{i_j}$ for some $i_j\in \{1,\cdots ,m\}$. Since, both are ireeducible $c_i$ and $d_{i_j}$ are associates. Lrt $c_i=u_id_{i_j}$ and $d_{i_j}=v_ic_i$. If $n>m$ then we get $c_{m+1} \cdots c_n=u_1\cdots u_m$ which is a contradiction. If $m>n$, $v_1\cdots v_n=\text{product of some $d_i$'s}$, again a contradiction. Therefore, $n=m.$  \qed

\begin{lemma}

Let $I$ be an ideal of $R$ and $f\in R$ be an element of $R$. Suppose, $J=I+\gen{f}$ and $J$ and $(I:J)$ both are principal then $I$ is principal and $I=J\cdot (I:J).$

\end{lemma}

\proof Let $x\in I\subseteq J=\gen{a+ft}$ (say) where $a\in I,t\in R$ then $x=s(a+ft).$ Let $y\in J$ then $y=r(a+ft)$ where $r\in R.$ $sy=sr(a+ft)=xr\in I \Rightarrow sJ\subseteq I$ as $y\in J$ is chosen arbitrarily then $s\in (I:J)$ thus $x\in J\cdot (I:J)$. Therefore, $I\subseteq J(I:J).$ Let $\sum a_is_i\in J(I:J)$ where $a_i\in J,s_i\in (I:J)$ then $a_is_i\in I \Rightarrow \sum a_is_i\in I \Rightarrow J(I:J)\subseteq I$. Hence, $I=J(I:J).$ Let $(I:J)=\gen{\beta}$ then $I=\gen{(a+ft)\beta},$ hence $I$ is principal. \qed

\begin{theorem}

Let $R$ be an integral domain. $R$ is principal ideal domain iff every prime ideal is principal.

\end{theorem}

\proof $(\Rightarrow)$ Trivial.\\

$(\Leftarrow)$ Suppose, $R$ is an integral domain in which every prime ideals are principal. Let $\sum=\{I\subseteq R:I~\text{is an ideal of}~R~\text{and $I$ is not principal}\}.$ If $\sum \neq \emptyset$ then consider $\{I_{\lambda}\}_{\lambda\in \Lambda}$ be a chain in $\sum$. If $\displaystyle\bigcup_{\lambda\in \Lambda} I_{\lambda}=\gen{a}$ then $a\in I_{\lambda}$ for some $\lambda\in \Lambda \Rightarrow I_{\lambda}=\gen{a}$ is a contradiction as $I_{\lambda}\in \sum.$ Therefore, $\displaystyle\bigcup_{\lambda\in \Lambda} I_{\lambda}$ is not a principal ideal, hence $\displaystyle\bigcup_{\lambda\in \Lambda} I_{\lambda}\in $ $\sum$. Every chain in $\sum$ has an upper bound. Therefore, by Zorn's lemma $\sum$ has a maximal element say $P.$ We will show that $P$ is a prime ideal. If not suppose $a\notin P,b\notin P$ and $ab\in P$. Let $J=P+\gen{a}$ then $P\subsetneq J$ and $b\in (P:J)$ [$\because ab\in P$] imply $P\subsetneq (P:J)$ [as $b\in (P:J)$ but $b\notin P$]. Since $P$ is maximal element of $\sum$, $J$ and $(P:J)$ both are principal ideal and by previous lemma $P=J(P:J)$ is principal. This gives a contradiction that $P\in \sum.$ Therefore, $P$ is a prime then $P$ is principal again contradiction as $P\in \sum.$ Hence, $\sum =\emptyset$ and $R$ is PID. \qed

















\begin{qns*}
Which odd primes of ${\ZZ}$ can be expressed as sum of two squares?
\end{qns*}
Ans. All odd primes of the form $4k+1.$
\begin{qns*}
Suppose, $p$ be a prime of the form $4k+1$. Can $p$ be written as sum of two squares?
\end{qns*}
Ans. Suppose $p$ is not irreducible in ${\ZZ}[i]$ i.e., $p=(a+bi)(c+di) \Rightarrow N(p)=N(a+bi)N(c+di) \Rightarrow p^2=(a^2+b^2)(c^2+d^2)$. If one of the factor becomes 1 then $p$ is irreducible. If $p$ is not irreducible then $a^2+b^2=c^2+d^2=p$. So its enough to show that $p$ is not irreducible in ${\ZZ}[i].$ As ${\ZZ}[i]$ is UFD, we will show that $p$ is not prime in ${\ZZ}[i].$ So we wish to show that if $p=4k+1$ then $X^2+1\equiv 0~(mod~p)$ has a solution.









\newpage
\section{Noetherian and Artinian ring}
\begin{definition}
Let $R$ be a ring and $\Sigma$ be the set of all ideals of $R.$ We say $\Sigma$ satisfy a.c.c (ascending chain condition) if for any increasing chain $I_0\subseteq I_1\subseteq I_2\subseteq \cdots$ in $\Sigma$ there exists $n\in {\NN}$ such that $I_n=I_{n+1}=\cdots$ and we say $R$ satisfies d.c.c (descending chain condition) if for any decreasing chain $I_0\supseteq I_1\supseteq I_2\supseteq \cdots$ in $\Sigma,$ there exists $n\in {\NN}$ such that $I_n=I_{n+1}=\cdots.$ 
\end{definition}
\begin{obs}
$R$ satisfies acc iff every ideal in $R$ is finitely generated.
\end{obs}
\proof Suppose, $R$ satisfies acc. If $\exists~I\in \Sigma$ such that $I$ is not finitely generated. Let $a_0\in I,a_1\in I- a_0R,a_2\in I-(a_0R+a_1R)$ and so on. Note that $I$ is not finitely generated this implies, $$I-\left(\displaystyle\sum_{i=1}^n a_iR\right)\neq \emptyset$$ If $I-\left(\displaystyle\sum_{i=1}^n a_iR\right)=\emptyset \Rightarrow I\subseteq \displaystyle\sum_{i=1}^n a_iR\subseteq I$ as $a_i\in I$ for all $i$ then $I=\displaystyle\sum_{i=1}^n a_iR$, we arrive at a contradiction. So we get a sequence of increasing ideal $J_0=a_0R,\cdots ,J_n=\displaystyle\sum_{i=1}^n a_iR$ and $$J_0\subseteq J_1\subseteq \cdots \subseteq J_n\subseteq \cdots $$ and $J_{n-1}\subsetneq J_n,$ contradiction. Hence $I$ is finitely generated.\\
Conversely, If every ideal of $R$ is finitely generated and we have a chain $$I_0\subseteq I_1\subseteq \cdots \subseteq I_n\subseteq \cdots $$ then $\displaystyle\bigcup_{i=1}^{\infty} I_i$ is finitely generated ideal. Let $\displaystyle\bigcup_{i=1}^{\infty} I_i=\langle a_1,\cdots ,a_r\rangle,a_i\in R$ then there exists $j_k$ such that $a_k\in I_{j_k},k\in \{1,\cdots ,r\}.$ Let $n=\max \{j_k|~1\leq k\leq r\}$ then $\displaystyle\bigcup_{i=1}^{\infty} I_i=\langle a_1,\cdots ,a_r\rangle=I_n$ this implies $I_n=I_{n+1}=\cdots $ hence $R$ satisfies acc.
\begin{eg}
${\ZZ}$ satisfies acc since every ideal of ${\ZZ}$ is principal.
\end{eg}
\begin{definition}
A commutative ring with identity which satisfies acc is called Noetherian ring, and if it satisfies dcc we call it Artinian ring.
\end{definition}
\begin{ex}
${\ZZ}$ is not an Artinian ring. ${\ZZ}/n{\ZZ},$ any field is Artinian ring. Let $K$ be a field then $K[t]/t^n$ is Artinian for every integer $n>1.$
\end{ex}
Let $R=K[x]/(x^2)$ then ideals are is of the form $I/(x^2)$ where $(x^2)\subseteq I$ i.e., $x^2\in I.$ Let $I=\gen{f(x)}$ then $f(x)|x^2 \Rightarrow f(x)=x $ or $x^2$ therefore, ideals are $0,x/(x^2)$ this implies $R$ is Artinian. $R$ is Noetherian ring and $K[x]/x^2$ is infinite ring as $R$ is a vector space over $K$ of dimension $2.$ Therefore, $R$ is infinite.
\begin{problem}
Is $C[0,1]$ Noetherian?
\end{problem}
Ans. Let $\Sigma_x=\{f\in C[0,1]:f(t)=0,~\forall t\in [0,x],~0\leq x\leq 1\}.$ If $x>y$ then $\Sigma_x\supsetneq \Sigma_y$ so if we choose $r_1\leq r_2\leq \cdots$ where $r_i's$ are rationale then we have $$\Sigma_0\subsetneq \Sigma_{r_1}\subsetneq \Sigma_{r_2}\subsetneq \cdots $$ therefore, $C[0,1]$ is not Noetherian then $C[0,1]$ is not Artinian also.\\
\textbf{Note:} A ring is Artinian implies it is also Noetherian (with dim 0). Artinian ring but not Noetherian is not possible.
\begin{problem}
Show that if $K$ be a field then $K[x]$ is PID.
\end{problem}
Ans. Let $I\subseteq K[x]$ be an ideal. If $I=0$ then it is principal. If $I\neq 0$ then we consider the set $$\Gamma=\{n:n=\deg f,~f\in I\}$$ By well ordering principle of natural number $\Gamma$ has an least element  say $m_0$ then there exists $g\in I$ such that $\deg g=m_0.$ If $f\in I$ by division algorithm $f=gq+r$ where $r=0$ or $\deg r< \deg g.$ As $f,g\in I \Rightarrow r\in I$ if $r\neq 0$ then we get a contradiction as $\deg r<\deg g.$ Therefore, $\deg g=m_0$ is the least element in $\Gamma \Rightarrow f=gq \Rightarrow I\subseteq \gen{g}$ but $g\in I \Rightarrow I=\gen{g}.$ \qed
\begin{problem}
Is $K[x_1,\cdots ,x_n]$ Noetherian?
\end{problem}
Ans. Yes (Hilbert basis theorem)
\begin{obs}
If $R$ is an Artin integral domain, then $R$ is a field.
\end{obs}

\proof Let $x\neq 0$ the we consider the chain of ideals $$(x)\supseteq (x^2)\supseteq (x^3)\supseteq \cdots $$ then there exists $n\in {\NN}$ such that $(x^n)=(x^{n+1})=\cdots \Rightarrow x^n\in (x^{n+1}) \Rightarrow x^n=yx^{n+1} \Rightarrow x^n(xy-1)=0$ since $R$ is integral domain $xy-1=0 \Rightarrow xy=1 \Rightarrow x$ is a unit hence $R$ is a field. \qed

\begin{lemma}
If $R$ is an Artin ring then maxspec $R$ is finite.
\end{lemma}

\proof Suppose, $\{m_1,m_2,\cdots \}\subseteq \operatorname{maxspec}R$ (an infinite set) and consider the chain $$m_1\supseteq m_1\cap m_2\supseteq m_1\cap m_2\cap m_3\supseteq \cdots$$ then for some $k$ we get $m_1\cap\cdots \cap m_k=m_1\cap\cdots \cap m_k\cap m_{k+1} \Rightarrow m_{k+1}\supseteq m_1\cap\cdots \cap m_k$ by prime avoidance lemma $m_{k+1}\supseteq m_i$ for some $i\in \{i,\cdots ,k\}$ but $m_{k+1},m_i$ both maximal this implies $\operatorname{maxspec}R$ is finite. \qed

\begin{remark}
Any Artin ring is semi local.
\end{remark}

\begin{obs}
If $(R,m)$ is a Artin local ring then $m^k=0$ for some $k\in {\NN}.$
\end{obs}





























































\newpage
\section{Zariski topology}
Let $R$ be a commutative ring with 1 and $I\subseteq R$ is an ideal of $R$. Define $$V(I)=\{P\in \spec R:I\subseteq P\}$$
Then the followings hold: \begin{enumerate}
\item $V(0)=\spec R$
\item $V(1)=\emptyset$
\item If $I\subseteq J$ then $V(J)\subseteq V(I)$ 
\item $V(I)\cup V(J)=V(IJ)=V(I\cap J)$
\item $\displaystyle\bigcup_{i=1}^n V(I_i)=V\left(\displaystyle\prod_{i=1}^n I_i\right)=V\left(\displaystyle\bigcap_{i=1}^n I_i\right)$
\item $V(I)=V(\sqrt{I})$
\item Let $\{I_{\alpha}\}_{\alpha\in \Lambda}$ be a collection of ideals in $R$ then $\displaystyle\bigcap_{\alpha\in \Lambda} V(I_{\alpha})=V\left(\displaystyle\sum_{\alpha\in \Lambda} I_{\alpha}\right)$
\end{enumerate}

\proof (3) Let $I\subseteq J$. Pick $P\in V(J) \Rightarrow J\subseteq P \Rightarrow I\subseteq P \Rightarrow P\in V(I) \Rightarrow V(J)\subseteq V(I).$\\
(4) Let $P\in V(I)\cup V(J) \Rightarrow P\in V(I)$ or $P\in V(J)$ then $I\subseteq P$ or $J\subseteq P \Rightarrow IJ\subseteq P \Rightarrow P\in V(IJ) \Rightarrow V(I)\cup V(J)\subseteq V(IJ).$ For reverse inclusion, let $P\in V(IJ) \Rightarrow IJ\subseteq P \Rightarrow I\subseteq P$ or $J\subseteq P$ then $P\in V(I)$ or $P\in V(J) \Rightarrow P\in V(I)\cup V(J) \Rightarrow V(IJ)\subseteq V(I)\cup V(J)$. Therefore, $V(I)\cup V(J)=V(IJ).$
We have $IJ\subseteq I\cap J \Rightarrow V(I\cap J)\subseteq V(IJ).$ For other case let $I\cap J\subseteq I$ and $I\cap J\subseteq J$ then $V(I)\subseteq V(I\cap J)$ and $V(J)\subseteq V(I\cap J) \Rightarrow V(I)\cup V(J)\subseteq V(I\cap J)\subseteq V(IJ)=V(I)\cup V(J)$. Hence, $$V(I)\cup V(J)=V(IJ)=V(I\cap J)$$
(5) Inductively from (4).\\
(6) We know $I\subseteq \sqrt{I} \Rightarrow V(\sqrt{I})\subseteq V(I).$ Let $P\in V(I) \Rightarrow I\subseteq P \Rightarrow \sqrt{I}\subseteq \sqrt{P}=P \Rightarrow P\in V(\sqrt{I}) \Rightarrow V(I)\subseteq V(\sqrt{I}).$\\
(7) $I_{\alpha}\subseteq \displaystyle\sum_{\alpha\in \Lambda} I_{\alpha} \Rightarrow V\left(\displaystyle\sum_{\alpha\in \Lambda} I_{\alpha}\right)\subseteq V(I_{\alpha}) \Rightarrow V\left(\displaystyle\sum_{\alpha\in \Lambda} I_{\alpha}\right)\subseteq \bigcap_{\alpha\in \Lambda} V(I_{\alpha}).$  Let $P\in \displaystyle\bigcap_{\alpha\in \Lambda} V(I_{\alpha}) \Rightarrow P\in V(I_{\alpha}),\forall~\alpha\in \Lambda \Rightarrow I_{\alpha}\subseteq P,\forall~\alpha\in \Lambda \Rightarrow \displaystyle\sum_{\alpha\in \Lambda} I_{\alpha} \subseteq P \Rightarrow P\in V\left(\displaystyle\sum_{\alpha\in \Lambda} I_{\alpha}\right) \Rightarrow \bigcap_{\alpha\in \Lambda} V(I_{\alpha})\subseteq V\left(\displaystyle\sum_{\alpha\in \Lambda} I_{\alpha}\right).$ Hence $\displaystyle\bigcap_{\alpha\in \Lambda} V(I_{\alpha})=V\left(\displaystyle\sum_{\alpha\in \Lambda} I_{\alpha}\right)$.\qed\\\\\\
Let $X=\spec R$ and $\tau=\{\spec R\setminus V(I):I~\text{is an ideal of}~R\}$ then check that $(X,\tau)$ is a topological space. If $I=\emptyset$ then $V(I)=\spec R$ so $\emptyset\in \tau,$ if $I=R$ then $V(I)=\emptyset$ so $X\in \tau.$ Let $\{U_{\alpha}\}_{\alpha\in \Lambda} \in \tau$ then $U_{\alpha}=V(I_{\alpha})^c$ as we have \begin{align*}
\displaystyle\bigcap_{\alpha\in \Lambda} V(I_{\alpha})&=V\left(\displaystyle\sum_{\alpha\in \Lambda} I_{\alpha}\right)\\
\left[\displaystyle\bigcap_{\alpha\in \Lambda} V(I_{\alpha})\right]^c&=\left[V\left(\displaystyle\sum_{\alpha\in \Lambda} I_{\alpha}\right)\right]^c\\
\displaystyle\bigcup_{\alpha\in \Lambda}U_{\alpha}&=V\left(\displaystyle\sum_{\alpha\in \Lambda} I_{\alpha} \right)^c
\end{align*}
Thus $\displaystyle\bigcup_{\alpha\in \Lambda}U_{\alpha}\in \tau$. Let $\{U_I\}_{i=1}^n \in\tau \Rightarrow U_i=V(I_i)^c,1\leq i\leq n$ then \begin{align*}
U_1\cap U_2\cap\cdots\cap U_n&= V(I_1)^c\cap\cdots\cap V(I_n)^c\\
&=[V(I_1)\cup\cdots\cup V(I_n)]^c\\
&=\left[V\left(\displaystyle\sum_{i=1}^n I_i\right)\right]^c\in \tau
\end{align*}
Therefore, $\tau$ is a topology on $\spec R$ this topology is called as Zariski topology on $spec R.$

\begin{lemma}
Show that $V(I)=V(J)$ if and only if $\sqrt{I}=\sqrt{J}.$ 
\end{lemma}
\proof $(\Rightarrow)$ Suppose, $V(I)=V(J)$. Now we know that \begin{align*}
\sqrt{I}=\displaystyle\bigcap_{P\in V(I)} P\quad\text{and}\quad \sqrt{J}=\displaystyle\bigcap_{P\in V(J)} P
\end{align*}
Since $V(I)=V(J)$ we get $\sqrt{I}=\sqrt{J}.$\\
$(\Leftarrow)$ Suppose, $\sqrt{I}=\sqrt{J}$ and $P\in V(I) 
\Leftrightarrow I\subseteq P \Leftrightarrow \sqrt{I}\subseteq \sqrt{P}=P \Leftrightarrow \sqrt{J}\subseteq P \Leftrightarrow P\in V(J).$
Thus $V(I)=V(J).$ \qed\\\\
Let $m\in \mspec R$ then $V(m)=\{m\}.$ Similarly, $$V(m^r)=\{P\in spec R:m^r\subseteq P\} =\{P\in spec R:\sqrt{m^r}\subseteq \sqrt{P}\}=\{P\in spec R:m\subseteq P\}=\{m\}$$

\begin{obs}
In a commutative ring any two maximal ideals are comaximal. Moreover, any power of two comaximal ideals are comaximal. Let $m_1,m_2\in \mspec R$ then $V(m_1+m_2)=\emptyset$ as $m_1+m_2=1$ i.e., they are comaximal ideals in $R.$ Similarly $m_1^{r_1}$ and $m_2^{r_2}$ are also comaximal because \begin{align*}
V(m_1^{r_1}+m_2^{r_2})=V(m_1^{r_1})\cap V(m_2^{r_2})=\{m_1\}\cap \{m_2\}=\phi=V(1)
\end{align*}
therefore, $1\in m_1^{r_1}+m_2^{r_2}.$ It follows from the previous lemma as $1=\sqrt{1}=\sqrt{m_1^{r_1}+m_2^{r_2}}=\sqrt{\sqrt{m_1^{r_1}}+\sqrt{m_2^{r_2}}}=\sqrt{m_1+m_2}.$
\end{obs}


 Let $R$ be a commutative ring with 1 such that $\mspec R=\{m_1,\cdots,m_r\}$ and $$m_1^{\alpha_1}\cdots m_r^{\alpha_r}=0$$ Then \begin{align*}
 R=R/(0)=R/m_1^{\alpha_1}\cdots m_r^{\alpha_r}\cong R/m_1^{\alpha_1}\times \cdots \times R/m_r^{\alpha_r}
 \end{align*}
i.e., $R$ can be written as finite product of local rings. 
\begin{ex}
Try to find an example of a ring in which product of maximal ideals is 0.
\end{ex}
\begin{eg}
Give an example of a ring $R$ such that $R$ is not integral domain but $nil (R)=\{0\}.$ 
\end{eg}
Ans. Product of any two integral domain.

































\newpage
\section{Exercise}
\begin{ex}
$\dfrac{R[x]}{(x-a)}\cong R$ where $a\in R.$
\end{ex}
\proof \begin{align*}
R[x]&\stackrel{\theta}{\longrightarrow} R\\
f(x)&\mapsto f(a)
\end{align*}
Then $\theta$ is a ring homomorphism. Let $c\in R$ then $ \theta(x-a+c)=c$ therefore, $\theta$ is surjective. $\ker \theta=\{f(x)\in F[x]:f(a)=0\}$ then clearly, $\gen{x-a}\subseteq \ker \theta.$ Let $f(x)\in \ker \theta \Rightarrow f(x)=q(x)(x-a)+r(a) \Rightarrow \theta(f(x))=\theta(q(x))\theta(x-a)+\theta(r(a)) \Rightarrow (r(a))=0 \Rightarrow \ker \theta \subseteq \gen{x-a}.$ Thud, $\ker \theta=\gen{x-a}.$ By first isomorphism theorem $R[x]/\gen{x-a}\cong R.$ \qed

\begin{ex}
Let $C[0,1]=\{f:[0,1]\to {\RR}:f~\text{is continuous}\}$ be a ring and $c\in [0,1].$ Define a map \begin{align*}
\phi:C[0,1]&\to {\RR}\\
f\quad&\mapsto f(c)
\end{align*}
Show that $\phi$ is a surjective ring homomorphism. Let $m_c=\{f\in C[0,1]:f(c)=0\}$. Also show that $C[0,1]/m_c\cong {\RR}.$ 
\end{ex}
\proof \begin{align*}
\phi:C[0,1]&\to {\RR}\\
f\quad&\mapsto f(c)
\end{align*} Clearly, $\phi$ is a ring homomorphism. Pick any $a\in R$ then the constant map $f(x)=a \in C[0,1]$ which imply $\phi(f(x))=a$ therefore, $\phi$ is surjective. $\ker \phi=\{f(x)\in C[0,1]:f(c)=0\}=m_c$ by first isomorphism theorem, $C[0,1]/m_c\cong {\RR}.$ \qed

\begin{ex}
$\dfrac{{\RR}[x]}{\langle x^2+1\rangle}\cong {\CC}.$
\end{ex}
\proof  \begin{align*}
{\RR}[x]&\stackrel{\theta}{\longrightarrow} {\CC}\\
f(x)&\mapsto f(i)
\end{align*}
Then $\theta$ is a ring homomorphism. Let $a+ib\in {\CC}$ then $ \theta(a+bx)=a+ib$ therefore, $\theta$ is surjective. $\ker \theta=\{f(x)\in {\RR}[x]:f(i)=0\}$ then clearly, $\gen{x^2+1}\subseteq \ker \theta.$ Let $f(x)\in \ker \theta \Rightarrow f(x)=q(x)(x^2+1)+ax+b \Rightarrow \theta(f(x))=\theta(q(x))\theta(x^2+1)+\theta(ax+b) \Rightarrow b+ia=0 \Rightarrow \ker \theta \subseteq \gen{x^2+1}.$ Thud, $\ker \theta=\gen{x^2+1}.$ By first isomorphism theorem ${\RR}[x]/\gen{x^2+1}\cong {\CC}.$ \qed



\begin{ex}
$\dfrac{{\CC}[x]}{\langle x^2+1\rangle}\cong {\CC}\times {\CC}.$
\end{ex}
\proof As, $x^2+1=(x+i)(x-i)$ and $1=\dfrac{1}{2i}[(x+i)-(x-i)]$ which imply $\gen{x-i}$ and $\gen{x+i}$ are co-maximal. By Chinese remainder theorem ${\CC}[x]/\gen{x^2+1}={\CC}[x]/\gen{x+i}\gen{x-i}={\CC}/\gen{x+i}\times {\CC}[x]/\gen{x_i}={\CC}\times {\CC}.$
\begin{ex}
$\dfrac{{\ZZ}[x]}{\langle 2x-1\rangle}\cong {\ZZ}\left[1/2\right].$
\end{ex}
\proof \begin{align*}
{\ZZ}[x]&\stackrel{\theta}{\longrightarrow} {\ZZ}[1/2]\\
f(x)&\mapsto f(1/2)
\end{align*}
As, ${\ZZ}\subseteq {\QQ} \Rightarrow {\ZZ}[x]\subseteq {\QQ}[x]$ and $f(x)\in \ker \theta \Leftrightarrow f(1/2)=0.$

\begin{ex}
$\dfrac{{\RR}[x]}{\langle 2x-1\rangle}\cong {\RR}.$
\end{ex}
\proof By problem 12.1
\begin{ex}
$\dfrac{{\RR}[x]}{\langle x^3-x\rangle}\cong  {\RR}\times {\RR}\times {\RR}$
\end{ex}
\proof $(x^3-x)=x(x-1)(x+1)$ thus $1=\dfrac{1}{2}[(x+1)-(x-1)]=[x-(x-1)]=[(x+1)-x]$ therefore, $\gen{x},\gen{x-1},\gen{x+1}$ are mutually co-maximal ideal hence by chiense remainder theorem, ${\RR}[x]/\gen{x^3-x}\cong {\RR}[x]/\gen{x}\times {\RR}[x]/\gen{x-1}\times {\RR}[x]/\gen{x+1}\cong {\RR}\times {\RR}\times {\RR}.$ \\ Find the pre-image of (1,2,3). Let $f(x)\in {\RR}[x]$ such that $f(x)+\gen{x^3+1}=(1,2,3)$ then $f(x)=a+bx+cx^2$ where $a,b,c\in {\RR}$ then $f(0)=a=1,f(1)=a+b+c=3$ and $f(-1)=a-b+c=2$. Solving the equations we have $f(x)=1+\dfrac{x}{2}+\dfrac{3x^2}{2}$. \qed
\begin{ex} Parabola
\begin{align*}
\theta:&~K[x,y]\to K[t]\\
&x\mapsto t\\
&y\mapsto t^2
\end{align*}
Find $\operatorname{Ker}\theta.$ 
\end{ex}
\begin{ex}
Twisted cubic curve
\begin{align*}
\theta:&~K[x,y,z]\to K[t]\\
&x\mapsto t\\
&y\mapsto t^2\\
&z\mapsto t^3
\end{align*}
Find $\operatorname{Ker}\theta.$ 
\end{ex}
Ans. $\ker \theta=\gen{y-x^2,z-x^3}.$ Let $f(x,y,z)\in \ker \theta,$ by division algorithm \begin{align*}
f(x,y,z)&=(z-x^3)g(x,y,z)+r(x,y)\\
r(x,y)&=(y-x^2)h(x,y)+r_1(x)
\end{align*}
Therefore, $$f(x,y,z)=(z-x^3)g(x,y,z)+(y-x^2)h(x,y)+r_1(x)$$ Applying $\theta$ on both side, we get, $$0=0+r_1(t) \Rightarrow r_1(x)=0$$ Thus, $f(x,y,z)=(z-x^3)g(x,y,z)+(y-x^2)h(x,y) \Rightarrow f(x,y,z) \in \gen{z-x^3,y-x^2}.$ Conversely, $z-x^3 \in \ker \theta, y-x^2\in \ker \theta \Rightarrow \gen{z-x^3,y-x^2}\subseteq \ker \theta \Rightarrow \ker \theta=\gen{z-x^3,y-x^2}.$ \qed



\begin{ex}
\begin{align*}
\theta:&~K[x,y,z]\to K[t]\\
&x\mapsto t^2\\
&y\mapsto t^3\\
&z\mapsto t^5
\end{align*}
Find $\operatorname{Ker}\theta.$
\end{ex}
\begin{ex}
\begin{align*}
\theta:K[x,y]&\to K[t]\\
&x\mapsto t^a\\
&y\mapsto t^b
\end{align*}
Where $K$ is any field and $\gcd (a,b)=1.$ Show that $\operatorname{Ker}\theta=\langle y^a-x^b\rangle.$
\end{ex}
\proof Let $f(x,y)\in \ker \theta.$ Since, $y^a-x^b$ is monic in $y$ $$f(x,y)=(y^a-x^b)g(x,y)+(r_0(x)+r_1(x)y+\cdots +r_{a-1}(x)y^{a-1}))$$ Applying $\theta$ both side, $$0=0+r_0(t^a)+r_1(t^a)t^b+\cdots +r_{a-1}(t^a)t^{b(a-1)}$$
Let $r_i(x)=a_{0,i}+a_{1,i}x+\cdots +a_{k_i,i}x^{k_i},0\leq i\leq a-1,a_{ij}\in K$
then \begin{align*}
r_i(t^a)&=a_{0,i}+a_{1,i}t^a+\cdots +a_{k_i,i}t^{ak_i}\\
r_i(t^a)t^{ib}&=a_{0,i}t^{ib}+a_{1,i}t^{a+ib}+\cdots +a_{k_i,i}t^{ak_i+b}
\end{align*}
therefore, \begin{align}
0=\displaystyle\sum_{i=0}^{a-1} r_i(t^a)t^{ib}&=\displaystyle\sum_{i=0}^{a-1} (a_{0,i}t^ib+a_{1,i}t^{a+ib}+\cdots +a_{k_i,i}t^{ak_i+ib}) \nonumber\\
&=(a_{0,0}+a_{0,1}t^b+\cdots +a_{0,a-1}t^{a-1}b)+(a_{1,0}t^a+a_{1,1}t^{a+b}+\cdots +a_{1,a-1}t^{a+(a-1)b}) \nonumber\\
&+\cdots +(a_{k_0,0}t^{ak_0}+a_{k_1,1}t^{ak_1+b}+\cdots +a_{k_{a-1},a-1}t^{ak_{a-1}+(a-1)b})
\end{align}
Let $k_1a+j_1b=k_2a+j_2b$ with $k_1>k_2 \Rightarrow (k_1-k_2)a=(j_2-j_1)b.$ As, $\gcd (a,b)=1$, $a|j_2-j_1$ but $0\leq j_2\leq a-1,0\leq j_1\leq a-1$ and $j_2>j_1 \Rightarrow j_2=j_1 \Rightarrow k_2=k_1$. Therefore, no two powers of $t$ are same in the expression (3) which implies, $a_{ij}=0,\forall~ij \Rightarrow r_i(x)=0,0\leq i\leq a-1 \Rightarrow f(x,y)=(y^a-x^b)g(x,y) \Rightarrow f(x,y) \in \gen{y^b-x^a}.$ Again, $\theta(y^b-x^a)=t^{ab}-t^{ab}=0 \Rightarrow y^b-x^a \in \ker \theta \Rightarrow \ker \theta =\gen{y^b-x^a}.$ \qed
\begin{ex}
\begin{align*}
K[x_1,x_2,\cdots ,x_n]&\stackrel{\theta}{\longrightarrow} K[t]\\
x_1&\mapsto t\\
x_2&\mapsto t^{a_2}\\
\qquad \vdots\\
x_n&\mapsto t^{a_n}
\end{align*}
Find $\operatorname{Ker}\theta.$
\end{ex}
\begin{ex}
Let $X$ be a compact Hausdroff space. Maximal ideals of $C(X):=\{f:X\to {\RR}:f\text{~is continuous}\}$ is the set $\Sigma=\{m_c:c\in X\}$ where $m_c=\{f\in C(X):f(c)=0\}.$
\end{ex}
\proof At first we show that $m_c$ is a maximal ideal for each $c\in X.$ We fix $c\in X$ and consider the map \begin{align*}
C(X)&\stackrel{\theta_c}{\longrightarrow} {\RR}\\
f&\mapsto f(c)
\end{align*}
Clearly, $\theta_c$ is surjective as for any $a\in {\RR}$ take constant map $f(x)=a,\forall~x\in X$ then $\theta_c(f)=a$ and $\ker \theta_c=m_c$ (by definition of $\theta_c$) therefore, $C(X)/m_c \cong {\RR}$ which imply $m_c$ is a maximal ideal. Let $m\in \mspec C(X)$ and suppose, $m\neq m_c,\forall~c\in X.$ the for each $c\in X, \exists~f_c\in m$ such that $f_c(x)\neq 0.$ Since, $f_c$ is continuous there exists an open set $U_c\subseteq X$ such that $f_c(U_c)\neq 0$ and $X\subseteq \displaystyle\bigcup_{c\in X} U_c$, since $X$ is compact, $\exists~c_1,\cdots c_n\in X$ such that $X\subseteq \displaystyle\bigcup_{i=1}^n U_{c_i}$. Now, consider the function $f=f_{c_i}^2+\cdots f_{c_n}^2$ and ley $y\in X\subseteq  \displaystyle\bigcup_{i=1}^n U_{c_i}$ then $y\in U_{c_i}$ for some $i$ which imply $f_{c_i}(y)\neq 0 \Rightarrow f(y)\neq 0$. Since $y\in X$ is arbitrary, $f(y)\neq 0,\forall~y\in X$ and $1/f$ is unit and $f\in m$ as $f_{c_i}\in m$ which is a contradiction. Therefore, $m=m_x$ for some $c\in X.$
\begin{Obs*}
\begin{align*}
X&\stackrel{\phi}{\longrightarrow} \mspec C(X)\\
c&\mapsto m_c
\end{align*}
We show that $\phi$ is a bijection. We already showed $\phi$ is surjective. Suppose, $x\neq y,$ $X$ is compact, Hausdorff therefore, $X$ is normal. By Uryshon lemma $\exists~f,g\in C(X)$ such that $f(x)=0,f(y)=1$ and $g(x)=1,g(y)=0 \Rightarrow m_x\neq m_y.$
\end{Obs*}\qed

\begin{ex}
Let $a=(a_1,\cdots ,a_n)\in {\CC}^n$ and $m_a=\langle x_1-a_1,\cdots ,x_n-a_n\rangle.$ Show that $m_a$ is a maximal ideal of ${\CC}[x_1,\cdots ,x_n].$
\end{ex}
\proof \begin{align*}
{\CC}[x_1,\cdots ,x_n]&\stackrel{\theta}{\longrightarrow} {\CC}\\
f(x_1,\cdots ,x_n)&\mapsto f(a_1,\cdots ,a_n)
\end{align*}
By Taylor's theorem \begin{align*}
f(x_1,\cdots ,x_n)=f(a_1,\cdots ,a_n)&+\dfrac{1}{1!}\left[\displaystyle\sum_{i=1}^n (x_i-a_i)\dfrac{\partial f}{\partial x_i}(a_1,\cdots ,a_n)\right]\\
&+\dfrac{1}{2!}\left[\displaystyle\sum_{1\leq i,j\leq n} (x_i-a_i)(x_j-a_j)\dfrac{\partial^2 f}{\partial x_i\partial x_j}(a_1,\cdots ,a_n)\right]+\cdots
\end{align*}
Now, if $f\in \ker \theta \Rightarrow f(a_1,\cdots ,a_n)=0,$ therefore,  \begin{align*}
f(x_1,\cdots ,x_n)=0&+\dfrac{1}{1!}\left[\displaystyle\sum_{i=1}^n (x_i-a_i)\dfrac{\partial f}{\partial x_i}(a_1,\cdots ,a_n)\right]\\
&+\dfrac{1}{2!}\left[\displaystyle\sum_{1\leq i,j\leq n} (x_i-a_i)(x_j-a_j)\dfrac{\partial^2 f}{\partial x_i\partial x_j}(a_1,\cdots ,a_n)\right]+\cdots \in \gen{x_1-a_1,\cdots ,x_n-a_n}
\end{align*}
Hence, $\ker \theta\subseteq m_a.$ Since, $x_i-a_i\in \ker \theta \Rightarrow m_a\in \ker \theta \Rightarrow m_a=\ker \theta.$\\ Let $p\in {\CC}$ then $\theta(x_1-a_1+p)=p \Rightarrow \theta $ is surjective thus $${\CC}[x_1,\cdots ,x_n]/m_a\cong {\CC}$$ so, $m_a$ is a maximal ideal. \qed
\begin{remark*}
Converse of the theorem (for ${\CC}$) is also true i.e., any maximal ideal of ${\CC}[x_1,\cdots ,x_n]$ is of the form $m_a$ for some $a\in {\CC}^n.$
\end{remark*}
\begin{remark*}
Converse of the theorem is not true for ${\RR}$ (a non algebraically closed field).
\begin{eg*}
$\gen{x^2+1}\subseteq {\RR}[x]$ is maximal ideal but not of the form $\gen{x-c}$ for some $c\in {\RR}.$
\end{eg*}
\end{remark*}

\begin{ex}
Let $R$ be an integral domain then so $R[x].$
\end{ex}
\proof Let $f(x)=a_0+\cdots +a_nx^n;a_i\in R,0\leq i\leq n$ where $a_n\neq 0.$ Suppose, $g(x)=b_0+\cdots +b_mx^m$ be the least degree polynomial such that  
$f(x)g(x)=0$ (where $f(x),g(x)\neq 0$) then we have $$a_0b_0+a_nb_mx^{m+n}=0$$ comparing the coefficient from both side of the equation, $a_nb_m=0$ but $a_n,b_m\neq 0$ then we get a contradiction as $R$ is assumed to be an integral domain. \qed

\begin{ex}
Suppose, $R$ is a commutative ring with identity and $f(x)=a_0+a_1x+\cdots +a_nx^n \in R[x]$ then $f$ is an unit of $R[x]$ iff $a_0$ is unit and $a_1,\cdots ,a_n$ are nilpotent. $f$ is nilpotent iff all of $a_i's$ are nilpotent.
\end{ex}
\proof Let us consider the homomorphism \begin{align*}
R[x]&\stackrel{\theta}{\longrightarrow} R/P[x]\\
a_0+\cdots +a_nx^n&\mapsto (a_0+P)+\cdots +(a_n+P)x^n
\end{align*}
Suppose, $f\in R[x]$ is a unit then $\theta(f)$ is unit in $R/P[x]$ and as $R/P$ is integral domain, $R/P[x]$ is also integral domain hence, $\theta(f)\in R/P$ [units of $R[x]$ = units of $R$ if $R$ is an integral domain]. Thus, $(a_0+P)+\cdots +(a_n+P)x^n\in R/P \Rightarrow a_i+P=0+P;1\leq i\leq n \Rightarrow a_i\in P$. Since, $P$ is chosen arbitrarily, $a_i\in \displaystyle\bigcap_{P\in \spec R} P=\sqrt{0};1\leq i\leq n$. Therefore, $a_1,\cdots ,a_n$ are nilpotent. As $f$ is unit there exists $g\in R[x]$ such that $fg=1$. Comparing the coefficient we have $a_0b_0=1 \Rightarrow a_0$ is unit. Conversely, if $a_0$ is a unit and $a_1,\cdots ,a_n$ are nilpotent then $a_0+a_1x+\cdots +a_nx^n$ is an unit.\\\\
Now, suppose, $f(x)$ is nilpotent. As $R/P$ is integral domain, $a_i+P=0+P;0\leq i\leq n$ then $a_i\in \sqrt{0}$ for all $i$ so that all $a_i$'s are nilpotent. Conversely, if $a_0,\cdots ,a_n$ is nilpotent then $a_0+a_1x+\cdots +a_nx^n$ is nilpotent i.e., $f(x)$ is nilpotent. \qed
\begin{proposition}
If $R$ is an integral domain and $f(x)\in R[x]$ is an unit then $f\in R.$
\end{proposition}
\proof Let $f(x)=a_0+\cdots +a_nx^n;n\geq 1$ and $g(x)=b_0+\cdots +b_mx^m$ (least degree) and $fg=1.$ Thus, $(a_0+\cdots +a_nx^n)(b_0+\cdots +b_mx^m)=1 \Rightarrow a_nb_m=0$. Since $n\geq 1, a_n\neq a_0.$ Therefore, $a_nb_m=0 \Rightarrow a_n=0$ or $b_m=0$ which is a contradiction. So, $n=0$ i.e., $f\in R.$ \qed
\begin{ex}
$K[[x]]=\left\lbrace\displaystyle\sum_{i=0}^{\infty} a_ix^i:a\in K\right\rbrace$ is ring of formal power series, where $$\displaystyle\sum_{i=0}^{\infty} a_ix^i+\displaystyle\sum_{i=0}^{\infty} b_ix^i=\displaystyle\sum_{i=0}^{\infty} (a_i+b_i)x^i$$ and $$\left(\displaystyle\sum_{i=0}^{\infty} a_ix^i\right)\left(\displaystyle\sum_{i=0}^{\infty} b_ix^i\right)=\left(\displaystyle\sum_{i=0}^{\infty} c_ix^i\right) \quad\text{where}~c_k=\displaystyle\sum_{i=0}^{k} a_ib_{k-i}$$ Show that $K[[x]]$ is a ring, $K[x]\subseteq K[[x]].$ Find $\operatorname{maxspec}K[[x]].$ \\
$$K[[x,x^{-1}]]=\left\lbrace\displaystyle\sum_{i=-\infty}^{\infty} a_ix^i:a_i\in K\right\rbrace$$ is ring of Laurent series where addition and multiplication are similar as formal power series ring.
\end{ex}
\begin{ex}
Let $K$ be a field, $\phi:K[x\to K[x]$ is an automorphism then $\phi(x)=\lambda x+\mu, \lambda\in K^*,\mu \in K.$
\end{ex}


\begin{obs}
If $\ch R=p>0$ ($p$ is prime) then the map $\phi: R\to R$ defined by $\phi(x)=x^p$ is a homomorphism. Also note that $p|\binom{p^n}{i},1\leq i\leq p^n-1,n\in {\NN}.$
\end{obs}
\proof Let $a,b\in R$ then $\phi(a+b)=(a+b)^p=\displaystyle\sum_{i=1}^p \binom{p}{i} a^{p-i}b^i$ where $p|\binom{p}{i},1\leq i\leq p-1$ then $\phi(a+b)=a^p+b^p=\phi(a)+\phi(b).$ Clearly $\phi(ab)=(ab)^p=a^pb^p=\phi(a)\phi(b),\phi(1)=1$ and $\phi(0)=0$ thus $\phi$ is an ring homomorphism.
\begin{ex}
Is $\phi$ injective?
\end{ex}
Ans. No in general as consider the ring $R={\ZZ}/p{\ZZ}[x]/\gen{x}^p$ then $\phi(\bar{x})=0$ so kernel is not trivial.
\begin{ex}
Let $\operatorname{Char}K=p>0$ and $K$ is a field. Choose $r\geq 2$ such that $p\nmid r.$ Define, \begin{align*}
\phi:&~K[x,y]\to K[t]\\
&x\mapsto -t-t^{rp}\\
&y\mapsto t^{p^2}
\end{align*}
Find $\operatorname{Ker}\phi$ and show that $\phi$ is surjective.
\end{ex}
\proof The map \begin{align*}
\phi:K[x,y]&\to K[t]\\
x&\mapsto -t-t^{rp}\\
y&\mapsto t^{p^2}
\end{align*}
Then $\phi(x^p)=(-1)(t+r^{rp})^p=-(t^p+t^{rp^2})$ and $\phi(y^r)=t^{rp^2} \Rightarrow \phi(y^r+x^p)=-t^p \Rightarrow \phi(y^r+x^p)^r=-t^{pr}.$ Now, $\phi(-x+(y^r+x^p)^r)=-\phi(x)+\phi(y^r+x^p)^r=t-t^{rp}-t^{pr}=t.$ So, $\phi$ is surjective. And $\phi(x^r+y^p)^p=-t^{p^2}=-\phi(y) \Rightarrow \phi(y+(x^r+y^p)^p)=0.$




\begin{ex}
Let $F$ be a field of characteristic $p>0.$ Is $ \phi:F\to F$ defined by $x\mapsto x^p$ surjective?
\end{ex}
\begin{ex}
Give an example of countably infinite field with finite characteristic.
\end{ex}
Ans. We know that $K={\ZZ}/p{\ZZ}(x)$ is a field with $\ch R=p.$ Note that if a ring is countable then its fraction ring is also countable. So its enough to show that ${\ZZ}/p{\ZZ}[x]$ is countably infinite. Clearly, ${\ZZ}/p{\ZZ}[x]$ is a vector space over ${\ZZ}/p{\ZZ}$ and basis is $\{1,x,\cdots,x^n:n\in {\NN}\}$ so cardinality of ${\ZZ}/p{\ZZ}[x]$ is $p\times \aleph=\aleph.$ 

\begin{ex}
$(15{\ZZ}:6{\ZZ})=?$
\end{ex}
Ans Recall that $(I:J)=\{x\in R:xJ\subseteq I\}$ then by definition of colon ideal $(15{\ZZ}:6{\ZZ})=\{x\in {\ZZ}:(6x)\subseteq (15)\}=\{x\in {\ZZ}:6x=15y\}=\{x\in {\ZZ}:2x=5y\}=\{x\in {\ZZ}:x=5\lambda\}=5{\ZZ}.$
\begin{ex}
Are ${\RR}$ and ${\CC}$ isomorphic as ring?
\end{ex}
\begin{ex}
Are ${\QQ}$ and ${\ZZ}$ isomorphic as ring?
\end{ex}
\begin{ex}
Is $({\RR}^2,+)\cong ({\RR},+)?$
\end{ex}
\begin{ex}
Show that $\langle 2,x\rangle$ is a maximal ideal in ${\ZZ}[x].$
\end{ex}
\begin{ex}
$\dfrac{{\ZZ}[x]}{\langle n,x\rangle}\cong {\ZZ}/n{\ZZ}$
\end{ex}
\begin{ex}
$R$ is a ring. Assume that every prime ideal is finitely generated then every ideal is finitely generated.
\end{ex}
\begin{ex}

\end{ex}
See \textbf{Appendix I} for more exercise on ring theory.






























































\newpage
%\chapter{Module}
\section{Module Theory}
\subsection{Introduction}
\textbf{Motivation for Module theory:} Why should we study module theory? One can say that while defining vector space, we take a non-empty abelian group $V$ and a field $K$ and define something called scalar multiplication $\cdot :K\times V\to V$ by $(\alpha,v)\mapsto \alpha\cdot v$ which follow some properties which we are not listed here. So one can think we can replace field and instead of field what happened if we take a ring not necessarily commutative and not necessarily having multiplicative identity. The result turns out to be a motivation for the study of module theory but we do something apart form this. So we first prove this theorem analogous to Cayley's theorem in group theory. Before this we construct a ring from an abelian group which is pretty much obvious and easy to do.\\

Let $(M,+)$ be an abelian group and $End~(M):=\{\phi:M\to M: \phi~\text{is a group homomorphism}\}$. Now we define addition and multiplication in this that $(End~(M),+,\cdot)$ is a ring. Define addition on $End~(M)$ as \begin{align*}
+:End~(M)\times End~(M)&\to End~(M)\\
(\phi,\psi)&\mapsto (\phi+\psi)
\end{align*}
now, $(\phi+\psi)(m)=\phi(m)+\phi(m),\forall~m\in M$ then check that $(End~(M),+)$ is a group. Now, define multiplication as mapping composition i.e., $\phi,\psi\in End~(M)$ and $(\phi )\cdot (\psi)=\phi \circ \psi,$ check that $(End~(M),+,\cdot)$ is ring with unity as identity function.
\begin{theorem}
Let $R$ be a ring then $R$ is isomorphic to a subring of an endomorphism ring of an abelian group.
\end{theorem}

\proof As $R$ is a ring, $(R,+)$ is an abelian group then $(End~(R),+,\cdot)$ is a ring. Now define \begin{align*}
R&\stackrel{\phi}{\longrightarrow}End~(R)\\
a&\mapsto \tau_a
\end{align*}
where $\tau_a:R\to R$ defined by $\tau_a(x)=ax$. As, $\tau_a$ is a group homomorphism, $\tau_1\in End~(R).$ Now, $\tau_{a+b}(x)=(a+b)x=ax+bx=\tau_a(x)+\tau_b(x)=(\tau_a+\tau_b)(x),\forall~x\in R$ then $\tau_{a+b}=\tau_a+\tau_b$ and $\tau_{ab}(x)=abx=a\tau_b(x)=(\tau_a)(\tau_b)(x),\forall~x\in R$ hence, $\tau_{ab}=\tau_a\circ \tau_b.$ Now, we show that $\phi$ is an injective ring homomorphism. let $a,b\in R$ then $\phi(a+b)=\tau_{a+b}=\tau_a+\tau_b=\phi(a)+\phi(b)$ and $\phi(ab)=\tau_{ab}=\tau_a\tau_b=\phi(a)\phi(b)$ and $\phi(1)=\tau_1=id_R$. Therefore, $\phi$ is a ring homomorhism. Let $\phi(a)=\phi(b) \Rightarrow \tau_a=\tau_b\Rightarrow ax=bx,$ in particularly if $x=1$ then $a=b$ so $\phi$ is injective. \qed\\
Let $R$ be a ring and $(M,+)$ is an abelian group. Suppose that there is a ring morphism $\phi:R\to End~(M)$ such that \begin{enumerate}
\item $\phi(a)(m+n)=\phi(a)(m)+\phi(a)(n),\forall~a\in R$ and $\forall~m,n\in M$ as $\phi(a):M\to M$ is a morphism,
\item $\phi(a+b)=\phi(a)+\phi(b) \Rightarrow \phi(a+b)(m)=\phi(a)m+\phi(b)m,\forall~ a,b\in R$ and $\forall~m\in M,$
\item $\phi(ab)=\phi(a)\phi(b) \Rightarrow \phi(ab)(m)=\phi(a)[\phi(b)\cdot (m)],\forall~m\in M$,
\item If $R$ has unity then $\phi(1)=1$ which is identity map and $\phi(1)m=m,\forall~m\in M.$
\end{enumerate}







\begin{definition}
Let $(M,+)$ be an abelian group and $R$ be an ring. Define, $\cdot : R\times M \rightarrow M$ by $(r,m)\mapsto rm$ be a map called scalar multiplication. $M$ is said to be an left $R-module$ if it satisfies the following properties:
\newline 1) $(r+s)m=rm+sm$ $\forall r,s\in R$ and $\forall m\in M$
\newline 2) $r(m+n)=rm+rn$ $\forall r\in R$ and $\forall m,n \in M$
\newline 3) $(rs)m=r(sm)$ $\forall r,s\in R$ and $\forall m\in M$
\newline If $R$ has identity element then $1_{R}m=m$ $\forall m\in M,$ then $M$ is said to be unitary module. 
If $M$ is defined over an division ring then $M$ is said to be a vector space over $R.$
\end{definition}

\begin{definition}
Let $M$ be an $R-module.$ $N\subseteq M$ be a subgroup of $M$ and $\cdot |_{R\times N}:R\times N \rightarrow N$ with $(N,\cdot |_{R\times N})$ satisfies the module property.
\end{definition}
\begin{theorem}
Let $M$ be an $R-module$ and $N\subseteq M.$ $N$ is submodule iff,
\newline 1) $n_{1},n_{2}\in N$ $\Rightarrow n_{1}+n_{2}\in N$
\newline 2) $r\cdot n \in N$ $\forall r\in R$ and $\forall n\in N.$
\end{theorem}
\proof Obvious.\qed\\

\begin{note}

Note that if $R$ is module over itself, then submodules of $R$ are precisely the ideals of $R.$ Now, suppose $S$ is an $R-$algebra. If $f:R\to S$ is surjective then the submodules of $S$ are the precisely the ideals of $S.$ To see this, let $N$ be an submodule of $S.$ Then for any $n\in N$ and for any $r\in R$, $r\cdot n:=f(r)n\in N.$ Then $N$ is an ideal of $S$ as for any $s\in S$ there exists $r\in R$ such that $f(r)=s$ then $r\cdot n=f(r)n=sn\in N.$

\end{note}

\begin{eg}

Take $R={\ZZ}$ and $S={\QQ}$ and $i:{\ZZ}\to {\QQ}$ be the inclusion map. $2{\ZZ}$ is an submodule of ${\QQ}$ but not an ideal of ${\QQ}.$

\end{eg}










Let $M$ be an $R-$module and $A\subseteq M$, the intersection of all submodule containing $A$ is the smallest submodule containing $A$. We say that the submodule generated by $A$ and denotes by $\gen{A}$. Let $$LC(A):=
\{r_1a_1+\cdots +r_na_n:r_i\in R,a_i\in A\}$$ then show that \begin{enumerate}
\item $LC(A)$ is a submodule of $M$,
\item $A\subseteq LC(A)$,
\item $\gen{a}= LC(A).$
\end{enumerate}
Special case If $|A|<\infty$ then we say $
\gen{A}$ is finitely generated i.e., if $A=\{a_1,\cdots ,a_n\}$ then $\gen{A}:=\{r_1a_1+\cdots +r_na_n:r_i\in R,a_i\in A\}$\\
\textbf{Notation.} $\gen{A}=Ra_1+\cdots +Ra_n.$
\begin{defn}
Let $R$ be an commutative ring with identity and $M$ be an $R-$module, $I\subseteq R$ be ann ideal of $R$. $$IM:=\left\lbrace \displaystyle\sum_{\substack{\text{finite}\\ \text{sum}}} a_im_i:a_i\in I,m_i\in M \right\rbrace$$ then $IM$ is an $R-$module and $IM\subseteq M.$
\end{defn}
Let $\displaystyle\sum_{\substack{\text{finite}\\ \text{sum}}} a_im_i,\displaystyle\sum_{\substack{\text{finite}\\ \text{sum}}} b_jm'_j\in IM$ where $a_i,b_j\in I$ and $m_i,m'_j\in M$ then $\displaystyle\sum_{\substack{\text{finite}\\ \text{sum}}} a_im_i+\displaystyle\sum_{\substack{\text{finite}\\ \text{sum}}} b_jm'_j=\displaystyle\sum_{\substack{\text{finite}\\ \text{sum}}} c_jm''_j\in IM$ and $r\in R$ and $\displaystyle\sum_{\substack{\text{finite}\\ \text{sum}}} a_im_i\in IM \Rightarrow \displaystyle\sum_{\substack{\text{finite}\\ \text{sum}}} (ra_i)m_i\in IM$ as $ra_i\in I$ as $I$ is an ideal, therefore, $IM$ is an $R-$module.

\begin{definition}
Let $M,N$ be $R-modules.$ $f:M \rightarrow N$ is said to be module homomorphism if 
\newline 1) $f(m_{1}+m_{2})=f(m_{1})+f(m_{2})$ $\forall m_{1},m_{2} \in M$
\newline 2) $f(rm)=rf(m)$ $\forall r\in R$ and $\forall m\in M.$
\end{definition}
We define $\operatorname{Ker}f=\{m\in M|$ $f(m)=0$\}. Note that $f$ is injective iff $\operatorname{Ker}f=\{0\}.$ Suppose $f$ is injective. We know that $f(0)=0$ so $\operatorname{Ker}f=\{0\}.$ Conversely, $\operatorname{Ker}f=\{0\}.$ Now, $f(x)=f(y)$ $\Rightarrow f(x-y)=0$ $\Rightarrow x-y \in \operatorname{Ker}f=\{0\}$ $\Rightarrow x=y$ hence $f$ is injective. If $f:M\to N$ be an $R-$module homomorphism then $\ker f$ is submodule of $M$ and $\im f$ is submodule of $N.$\\\\
We define $$\operatorname{Hom}_{R}(M,N):=\{f:M\rightarrow N:f~\text{is a $R-$module homomorphism}\}$$
\newline Note that, $$\cdot : R\times \operatorname{Hom}_{R}(M,N)\rightarrow \operatorname{Hom}_{R}(M,N)$$ defined by $(r,f)\mapsto rf$ where $(rf)(m)=r\cdot (f(m))$ $\forall m\in M.$ Then it is easy to check that $\operatorname{Hom}_{R}(M,N)$ is a left $R-module.$
\begin{remark*}
Let $N\subseteq M$ be a submodule then $N$ is kernel of some $R-$module homomorphism for some suitable $R-$module.
\end{remark*}

\begin{definition}
A module homomorphism $f:M\rightarrow N$ is said to be an isomorphism if $~\exists ~g:N\rightarrow M$ module homomorphism such that $f\circ g=id_{N}$ and $g\circ f=id_{M}$  
\end{definition}
Note that $f:M\rightarrow N$ be a module homomorphism is isomorphism iff $f$ is bijective. 
\begin{proof}
$(\Rightarrow)$ Obvious.
\newline $(\Leftarrow)$ Suppose, $f$ is bijective then $\exists~ g:N\rightarrow M$ such that $f\circ g=id_{N}$ and $g\circ f=id_{M}.$ We need to check that $g$ is a module homomorphism. Let $n_{1},n_{2}\in N.$ $(f\circ g)(n_{1}+n_{2})=n_{1}+n_{2}=(f\circ g)(n_{1})+(f\circ g)(n_{2})$ $\Rightarrow f[g(n_{1}+n_{2})-g(n_{1})-g(n_{2})]=0$ (since $f$ is a homomorphism). As $f$ is bijective $f$ is injective also so $g(n_{1}+n_{2})=g(n_{1})+g(n_{2}).$ Again let $r\in R$ and $n\in N.$ $f(g(rn))=rn=rf(g(n))=f(rg(n))$ $\Rightarrow f[g(rn)-rg(n)]=0$ this implies $g(rn)=rg(n).$ Hence $g$ is a module homomorphism.
\end{proof}
\textbf{Notation:} If $M,N$ be $R-modules.$ If $M$ and $N$ are isomorphic then $M\simeq N$
\begin{theorem}
Show that $\operatorname{Hom}_{R}(R,M)\simeq M$
\end{theorem}
\begin{proof}
Define, $\theta: \operatorname{Hom}_{R}(R,M)\rightarrow M$ by $\theta(f)=f(1).$ $\theta(f_{1}+f_{2})=(f_{1}+f_{2})(1)=f_{1}(1)+f_{2}(1)=\theta(f_{1})+\theta(f_{2}).$ Now, $\operatorname{Ker}f=\{f~|$ $\theta(f)=0$\} $\Rightarrow \operatorname{Ker}f=\{f~|$ $ f(1)=0$\}. Let $r\in R$ $\Rightarrow f(r)=rf(1)=0$ $\Rightarrow f\equiv 0$ $\Rightarrow \theta$ is injective. Let $m\in M.$ We consider the map $f\in \operatorname{Hom}_{R}(R,M)$ such that $f(r)=rm$ (check that $f$ is a module homomorphism) $\Rightarrow f(1)=m=\theta(f)$ hence, $\theta$ is surjective. Then $\theta$ is an isomorphism. Therefore, $\operatorname{Hom}_{R}(R,M)\simeq M.$
\end{proof}
\subsection{Quotient Module}
Let $N\subseteq M$ be a submodule. Consider the quotient group $(M/N,+)$ and the scalar multiplication map \begin{align*}
\cdot :&R\times M/N\to M/N\\
&(r,x+N)\mapsto (rx+N)
\end{align*}
Claim: `$\cdot$' is well defined. Let $x+N=y+N \Rightarrow x-y\in N \Rightarrow r(x-y)\in N \Rightarrow rx+N=ry+N,$ hence `$\cdot$' is well defined. Check that $M/N$ is an $R-$module. Let $\pi:M\to M/N$ defined by $\pi(x)=x+N$ is a surjective homomorphism whose kernel is $N.$


\subsubsection{Isomorphism theorems for module}
\begin{theorem}[First isomorphism theorem]
Let $M_1,M_2$ be two $R-$modules and $f:M_1\to M_2$ be $R-$module homomorphism. Suppose, $N\subseteq \operatorname{Ker}f$ be a submodule of $M_1$ then there exists an unique $R-$module homomorphism $\tilde{f}:M_1/N\to M_2$ such that the diagram commutes i.e., $f=\tilde{f}\circ \pi.$\\
\begin{center}
\begin{tikzcd}
M_1 \arrow[rr, "f"] \arrow[dd, "\pi"'] &  & M_2 \\
                                       &  &     \\
M_1/N \arrow[rruu, "\tilde{f}"']       &  &    
\end{tikzcd}
\end{center}
Moreover, if $I=\operatorname{Ker}f$ then $\tilde{f}$ is injective hence $M_1/\operatorname{Ker}f\cong \operatorname{Im}f.$ If $\tilde{f}$ is surjective then $f$ is also surjective then $M_1/\operatorname{Ker}f\cong M_2.$
\end{theorem}
\proof Define, $\tilde{f}:M_1/N\rightarrow M_2$ by $(x+N)\mapsto f(x).$ Claim: $\tilde{f}$ is well defined. Let, $x+N$=$y+N$ $\Rightarrow$ $x-y$ $\in N\subseteq \operatorname{Ker}f$ $\Rightarrow$ $f(x-y)=0$ $\Rightarrow f(x)=f(y)$ $\Rightarrow \tilde{f}(x+N)=\tilde{f}(y+N).$ Therefore $\tilde{f}$ is well-defined. [$\tilde{f}(x+y+N)=f(x+y)$ $\Rightarrow f(x)+f(y)$ (as $f$ is a ring homomorphism) $\Rightarrow \tilde{f}(x+y+N)=\tilde{f}(x+N)+\tilde{f}(y+N)$
and $\tilde{f}(xy+N)=f(xy)=f(x)f(y)=\tilde{f}(x+N)\tilde{f}(y+N)$ so, $\tilde{f}$ is a ring homomorphism.] Uniqueness: Suppose $g$ is another ring homomorphism such that $g:M_1/N\rightarrow M_2$ such that $g\circ\pi=f.$ Then $f(x)=g\circ\pi(x)$ $\Rightarrow \tilde{f}(x+N)=f(x)=g(x+N)$ so we have $g=\tilde{f}.$ Therefore $\tilde{f}$ is unique. Now suppose $N=\operatorname{Ker}f$ and let $x+N\in \operatorname{Ker}\tilde{f} \Rightarrow \tilde{f}(x+N)=0=f(x) \Rightarrow x\in \operatorname{Ker}f \Rightarrow x+N=x+\operatorname{Ker}f=0$ [as $N=\operatorname{Ker}f$] so $\tilde{f}$ is injective. Therefore, $M_1/\operatorname{Ker}f \cong \operatorname{Im}\tilde{f}=\operatorname{Im}{f}$ (by definition $\operatorname{Im}\tilde{f}=\operatorname{Im}{f}$). If $\tilde{f}$ is surjective then so $f$ hence $M_2=\operatorname{Im}\tilde{f}=\operatorname{Im}{f}.$ Therefore, $M_1/\operatorname{Ker}f \cong M_2$.  \qed\\\\
\textbf{Observation.} $\dfrac{M+N}{N}\cong \dfrac{M}{N\cap M}$\\ 
\proof \begin{align*} M&\stackrel{i}{\hookrightarrow}M+N\stackrel{\pi}{\rightarrow}\dfrac{M+N}{N}\\
x&\mapsto x+0\mapsto x+N
\end{align*}
therefore $f=\pi \circ i,$ let $x\in \operatorname{Ker}f\subseteq M \Leftrightarrow x+N=0 \Leftrightarrow x\in N\Leftrightarrow x\in M\cap N$ hence $\dfrac{M}{M\cap N}\hookrightarrow \dfrac{M+N}{N}.$ Let $m+n\in M+N$ then $(m+n)+N=m+N \Rightarrow f(m)=m+N=(m+n)+N$ gives the surjectivity of $f$ hence we get our desire result. \\

\textbf{Observation.} Suppose $P\subseteq N\subseteq M$ then $N/P=\{x+P:x\in N\}$ is a submodule of $M/P.$ Show that $\dfrac{M/P}{N/P}\cong \dfrac{M}{N}.$ Define $f:M/P\to M/N$ by $(x+P)\mapsto (x+N)$ therefore, $f$ is well defined by first isomorphism theorem. $\operatorname{Ker}f=\{x+P:f(x+P)=0+N\}=\{x+P:x+N=0+N\}=N/P$ as $f$ is surjective we have $\dfrac{M/P}{N/P}\cong \dfrac{M}{N}.$

\newpage
\subsection{Chinese remainder theorem}
Let $M$ be an $R-$ module and $I,J$ be two ideals of $R$. Recall that $IM$ is a submodule of $M$. Now we consider the map \begin{align*}
\theta: M&\to M/IM\times M/JM\\
m&\mapsto (m+IM,m+JM)
\end{align*}
then it is easy to show that $\theta$ is a $R-$module morphism. $$\ker \theta=\{m\in M: \theta(m)=(0+IM,0+JM)\}=IM\cap JM.$$


 In general, $(I\cap J)M\subseteq IM\cap JM$ as $I\cap J\subseteq I\Rightarrow (I\cap J)M\subseteq IM$ similarly, $(I\cap J)M\subseteq JM$ hence $(I\cap J)M\subseteq IM\cap JM.$ 
 \begin{qns}
 Find an example where $(I\cap J)M\neq IM\cap JM.$ 
 \end{qns}

Now back to our discussion, we assume that $I,J$ are comaximal, i.e, $I+J=1.$ Then what we can show that the $\theta$ described above is surjective as well as $(I\cap J)M=IM\cap JM$ and we show this one by one. Since $I+J=1 \Rightarrow x+y=1,x\in I,y\in J.$ Let $(m_1+IM,m_2+JM)\in M/IM\times M/JM$ \begin{align*}
m_2x+m_1y-m_1&=m_2x+m_1(y-1)\\
&=m_2x-m_1x\in IM\\
(m_2x+m_1y)+IM&=m_1+IM
\end{align*} 
Similarly, \begin{align*}
m_2x+m_1y-m_2&=m_2(x-1)+m_1y\\
&=m_1-m_2y\in JM\\
(m_2x+m_1y)+JM&=m_2+JM
\end{align*}
Therefore, $\theta(m_2x+m_1y)=(m_1+IM,m_2+JM)$ which proves that $\theta$ is surjective. By first isomorphism theorem $$M/(IM\cap JM)\cong M/IM\times M/JM.$$
As we assume $I,J$ are comaximal, let $m\in IM\cap JM$ then $$m=1\cdot m=(x+y)m,\qquad\qquad x+y=1,x\in I,y\in J.$$ Thereofore, $m\in JM,x\in I \Rightarrow xm\in (IJ)M$. Similarly, $m\in IM,y\in J \Rightarrow ym\in (IJ)M \Rightarrow m\in (IJ)M.$ Now $$(IJ)M\subseteq (I\cap J)M\subseteq IM\cap JM\subseteq (IJ)M.$$ Thus if $I$ and $J$ are comaximal we have $(IJ)M=(I\cap J)M=IM\cap JM.$

Let $V$ be a vector space over a field $F$ and $T:V\to V$ be a linear operator. Let $p=p_1^{r_1}\cdots p_k^{r_k}$ be the minimal polynomial of $T$. Now $V$ is an $F[x]$ module via $T$. Clearly, $(p)$ is an ideal of $F[x]$ and $(p)V$ is a subspace of $V$. Then $$V/(p)V=V/(p_1^{r_1}\cdots p_k^{r_k})V=V/(p_1^{r_1})V\cdots (p_k^{r_k})V\cong V/(p_1^{r_1})V\times \cdots\times V/(p_k^{r_k})V.$$

\subsection{Module Structure}
\begin{enumerate}
\item Let $R,S$ be two rings and $f:R\to S$ be a ring homomorphism (i.e., $S$ is a $R$ algebra). If $M$ is an $S-$module then $M$ is an $R-$module (via $f$) as, $$\cdot: R\times M\to M$$ where $(r,m)\mapsto f(r)m$ be the scalar multiplication map.
\item If $S$ is a $R-$algebra then scalar multiplication map defined by $\cdot: R\times S\to S$, $\cdot (r,s)=f(r)s$ which makes $S$ is an $R-$module.
\item Let $T:V\to V$ be a linear operator on a vector space $V$ over a field $K$ then $V$ is a $K[x]$ module via \begin{align*}
&K[x]\times V\to V\\
&(f(x),v)\mapsto f(T)v
\end{align*}
\item Let $R$ be a commutative ring and $M$ be a left $R-$module. Define, \begin{align*}
\cdot:M\times R&\to M\\
(m,r)&\mapsto m\cdot r:=rm
\end{align*} where $rm$ is usual left scalar multiplication.
Check that $M$ is a right $R-$module.
\end{enumerate}

\begin{definition}
Let $M$ be an $R-$module we define, $$\operatorname{Ann}_R M=\{r\in R:rm=0\text{~for all}~ m\in M\}$$ is an ideal of $R.$ An $R-$module $M$ is said to be faithful $R-$module if $\operatorname{Ann}M=0.$
\end{definition}
Let $r_1,r_2\in \operatorname{Ann}M \Rightarrow r_1m=0=r_2m \text{~for all}~ m\in M \Rightarrow (r_1+r_2)m=0 \text{~for all}= m\in M \Rightarrow r_1+r_2\in \operatorname{Ann}M.$ Similarly let $r\in R$ and $r_1\in \operatorname{Ann}M$ then $r(r_1m)=r\cdot m=0\text{~for all}~ m\in M$ then $rr_1\in \operatorname{Ann}M.$ Therefore, $\operatorname{Ann}M$ is an ideal of R.
\begin{problem}
Let $V$ be a vector space over a field $K$ and let $T:V\to V$ be a linear operator, then $V$ is a $K[x]$ module via $T.$ Show that $V$ is not faithful $K[x]$ module i.e., $\operatorname{Ann}V\neq 0$
\end{problem}
Ans. By Cayley-Hamilton theorem.
\begin{ex}
Let, $$\pi: R\to R/Ann~M$$ be the projection map. Show that $M$ is a faithful $R/Ann~M$ module.
\end{ex}
Ans. Define the scalar multiplication map $$\cdot :R/Ann~M\times M\to M$$ by $(r+Ann~M,m)\mapsto rm.$ Claim: '$\cdot$' is well defined, Let $r_1+Ann~M=r_2+Ann~M \Rightarrow r_1-r_2\in Ann~M \Rightarrow (r_1-r_2)m=0\text{~for all}~ m\in M$ then $r_1m=r_2m$ hence $\cdot$ is well defined. Now, $$Ann_{(R/Ann~M)}M=\{r+Ann~M:(r+Ann~M)m=0 \text{~for all}~ m\in M\}$$ Let $r+Ann~M\in Ann_{R/Ann~M}M \Rightarrow rm=0 \text{~for all}~ m\in M \Rightarrow r\in Ann~M \Rightarrow r+Ann~M=0+Ann~M$ in $R/Ann~M$ this implies $M$ is a faithful $R/Ann~M$ module. 
\begin{qns}
Let $R,S$ be two rings and $S$ is an $R-$algebra. Let $M$ be an $R-$module then under what condition $M$ is a $S-$module $?$
\end{qns}
\begin{obs}
Let $f:R \to S$ be a surjective ring homomorphism and $\operatorname{Ker}\subseteq Ann_R M$ then if $M$ is a $R-$module then $M$ is an $S-$module.
\end{obs}
\proof Define \begin{align*}
\cdot: &~S\times M\to M\\
&(s,m)\mapsto f(r)m
\end{align*}
and suppose, $s$ has two pre image i.e., $s=f(r_1)=f(r_2) \Rightarrow f(r_1-r_2)=0 \Rightarrow r_1-r_2\in \operatorname{Ker}f\subseteq Ann_R M \Rightarrow (r_1-r_2)m=0 \text{~for all}~ m\in M \Rightarrow r_1m=r_2m$ i.e., the map is well defined.\qed

\subsection{Nakayama Lemma}

\begin{definition}
Let $M$ be an $R-module.$ $S$ be a subset of $M.$ We say that $S$ is a generating set for $M$ if any $m\in M$ can be written as finite $R$ linear combination of elements in $S$ that is there exist $\{s_{1},\cdots ,s_{r}\}\subseteq S$ such that for any $m\in M$ can be written as $m=\displaystyle\sum_{i=1}^r r_{i}s_{i}$ where $r_{i} \in R,1\leq i\leq r$.
\end{definition}

\textbf{Notation.} $M=\gen{S}.$

\begin{remark}
Suppose $M$ and $N$ be $R-$Modules and $f:M\to N$ be an epimorphism. If $S$ is a generating set of $M$ then $f(S)$ is also generating set of $N.$
\end{remark}

\begin{definition}
Let $S\subseteq M$ be a generating set of $M.$ $S$ is said to be minimal generating set if any proper subset of $S$ doesn't generate $M.$
\end{definition}
Note that $S_{1}=\langle 1\rangle ={\ZZ}$ and $S_{2}=\langle 3,5\rangle ={\ZZ}$ but $|S_{1}|\neq |S_{2}|$ i.e, cardinality of minimal generating set may not be equal.
\begin{exercise}
Let $R$ be a commutative ring with identity. $m\in maxspecR.$ Show that $spec(R/m^k)=\{m/m^k\}$ where $k\in {\NN},$ hence $R/m^k$ is a local ring.
\end{exercise}
\proof Every prime ideal of $R/m^k$ is of the form $P/m^k$ where $m^k\subseteq P$ and $P$ is a prime ideal of $R.$ Taking radical on both side we get, $\sqrt{m^k}\subseteq \sqrt{P} \Rightarrow m\subseteq P$ since $m$ is an maximal ideal, $m=P$ hence $spec(R/m^k)=\{m/m^k\}.$ \qed

\begin{defn}
A Module $M$ is said to be finitely generated if there exists $\{m_1,\cdots,m_k\} \subseteq M$ such that $M=\gen{m_1,\cdots,m_k}.$
\end{defn}

\begin{note}
A module $M$ is finitely generated if and only if there exists a surjection from $R^k\to M$ for some $k\in{\NN}.$
\end{note}

\subsubsection{NAK (version 1)}
\begin{lemma}
Let $M$ be an finitely generated $R-$module and $I\subseteq R$ is an ideal of $R$ if $IM=M$ then there exists $x\in I$ such that $(1+x)M=0.$
\end{lemma}
Recall that $IM=\left\lbrace \displaystyle\sum_{\substack{\text{finite}\\ \text{sum}}}rm:r\in I,m\in M\right\rbrace$ and $IM\subseteq M$ is a submodule.\\
\proof Suppose $M=\langle m_1,\dots ,m_k\rangle$ and $IM=M$ then for $m_i\in IM, 1\leq i\leq k$ \begin{align*}
m_1=r_{11}m_1+r_{12}m_2+\cdots +&r_{1k}m_k\\
m_2=r_{21}m_1+r_{22}m_2+\cdots +&r_{2k}m_k\\
\vdots \quad\quad\quad\quad\quad\quad\vdots\quad\quad\quad\quad\quad &\vdots\\
m_k=r_{k1}m_1+r_{k2}m_2+\cdots +&r_{kk}m_k\\
\end{align*}
Then we can write it $$\underbrace{\begin{pmatrix}
&r_{11}-1\quad &r_{12} \quad &\cdots  &r_{1k}\\
&r_{21}\quad &r_{22}-1 &\cdots  &r_{2k}\\
&\vdots \quad &\vdots &\ddots &\vdots\\
&r_{k1} \quad &r_{k2} &\cdots  &r_{kk}-1
\end{pmatrix}}_{A}
\begin{pmatrix}
m_{1}\\
\vdots\\
m_k
\end{pmatrix}=\begin{pmatrix}
0\\
\vdots\\
0
\end{pmatrix}$$
Multiplying by $\operatorname{adj}A$ we get $$(\operatorname{adj}A)A\begin{pmatrix}
m_{1}\\
\vdots\\
m_k
\end{pmatrix}=\begin{pmatrix}
0\\
\vdots\\
0
\end{pmatrix}$$ This gives $$\det A\begin{pmatrix}
m_{1}\\
\vdots\\
m_k
\end{pmatrix}=\begin{pmatrix}
0\\
\vdots\\
0
\end{pmatrix}$$
Hence, $(\det A)\cdot m_i=0$ for all $i \Rightarrow (\det A)\cdot M=0.$ We note that $\det A$ is of the form $1+x$ where $x\in I,$ therefore $(1+x)M=0.$


 We prove that $\det A$ is of the form $1+x$ where $x\in I.$ We proceed by induction on $k.$ It is true for $k=1.$ Now, $$\det \begin{pmatrix}
&r_{11}-1 \quad &\cdots  &r_{1k}\\

&\vdots \quad  &\ddots &\vdots\\
&r_{k1} \quad &\cdots  &r_{kk}-1
\end{pmatrix}=(r_{11}-1)\det A_{11}-r_{12}\det A_{12}+\cdots +(-1)^kr_{1k}\det A_{1k}$$
By induction $\det A_{11}=1+x_1,x_1\in I$ then $(r_{11}-1)(1+x_1)+y=1+x$ where $y\in I$ (as $r_{1i}\in I, 2\leq i\leq k$) \qed
\subsubsection{NAK (version 2)}
\begin{lemma}
Let $(R,m)$ be a local ring and $M$ is finitely generated $R-$module such that $mM=M$ then $M=0.$
\end{lemma}
\proof $\exists~ x\in m$ such that $(1+x)M=0$ (by version 1). Now, $(R,m)$ is a local ring and $x\in m$ then $1+x$ is a unit, let $y(1+x)=1.$ Therefore, $y(1+x)M=y\cdot 0\Rightarrow M=0.$ \qed
\subsubsection{NAK (version 3)}
\begin{lemma}
Let $(R,m)$ be a local ring and $N\subseteq M,$ $M$ is finitely generated. Suppose, $N+mM=M$ then $N=M$
\end{lemma}
\proof $N+mM=M \Rightarrow \dfrac{N+mM}{N}\cong \dfrac{M}{N} \Rightarrow m(M/N)\cong (M/N) \Rightarrow M/N=0.$
Therefore, $N=M$ [Note that $M$ is finitely generated implies $M/N$ is also finitely generated] \qed

\subsection{Finitely generated $R-$module}

Let $(R,m,K)$ be a local ring and $M$ is an $R-$module then $M/mM$ is a $K$ vector space under the scalar multiplication map \begin{align*}
\cdot:&R/m\times M/mM \to M/mM\\
&(r+m,m_1+mM)\mapsto rm_1+mM
\end{align*}
Then check that the map is well defined. Let $r_1+m=r_2+m \Rightarrow r_1-r_2=\alpha \in m$ then $r_1=r_2+\alpha.$ On the other hand let $m_1+mM=m_2+mM \Rightarrow m_1-m_2\in mM \Rightarrow m_1-m_2=\displaystyle\sum_{i=1}^l r_i'\beta_i \Rightarrow m_1=m_2+\displaystyle\sum_{i=1}^l r_i'\beta_i.$ Then $$r_1m_1=(r_2+\alpha)(m_2+\displaystyle\sum_{i=1}^l r_i'\beta_i)=r_2m_2+\left( \displaystyle\sum_{i=1}^l (r_2r_i)\beta_i+\alpha m_2+\displaystyle\sum_{i=1}^l (r_i'\alpha)\beta_i \right)$$ Everything within the parentheses in above expression lies in $mM$ hence $r_1m_1-r_2m_2\in mM \Rightarrow r_1m_1+mM=r_2m_2+mM.$\\
Next we consider the map \begin{align*}
\pi:&M\to M/mM\\
&m_1\to m_1+mM
\end{align*}
Suppose, $M$ is a finitely generated $R-$module and $S\subseteq M$ be a minimal generating set of $M.$ Show that $\pi(S)$ is a basis of the $K$ vector space $M/mM.$ Let $S=\{m_1,\cdots ,m_l\}$ and let $\alpha\in M \Rightarrow \alpha=\displaystyle\sum_{i=1}^l r_im_i$ then \begin{align*}
\alpha+mM&=\left(\displaystyle\sum_{i=1}^l r_im_i \right) +mM\\
&=\displaystyle\sum_{i=1}^l (r_i+m)(m_i+mM)
\end{align*}
Therefore, we get $\operatorname{span}\pi(S)=M/mM.$ Next we show that $\pi(S)$ is linearly independent. Suppose, $$\displaystyle\sum_{i=1}^l (r_i+m)(m_i+mM)=0$$ If $r_j+m\neq 0+m$ for some $j\in \{1,\cdots, l\}$ then $r_j\notin m.$ Since $R$ is local, $r_j$ is a unit then $\exists u_j\in R$ such that $u_jr_j=1$ then \begin{align*}
(r_j+m)(m_j+mM)&=-\displaystyle\sum_{\substack{i=1\\ i\neq j}}^k (r_i+m)(m_i+mM)\\
m_j+mM&=-\displaystyle\sum_{\substack{i=1\\ i\neq j}}^k (u_j+m)(r_i+m)(m_i+mM)\\
&=\left(-\displaystyle\sum_{\substack{i=1\\ i\neq j}}^k (u_jr_im_i)\right)+mM
\end{align*}
Therefore, $m_j+\displaystyle\sum_{\substack{i=1\\ i\neq j}}^k (u_jr_im_i)\in mM$ then \begin{equation} 
 m_j+\displaystyle\sum_{\substack{i=1\\ i\neq j}}^k (u_jr_im_i)=\displaystyle\sum_{i=1}^l a_im_i,~a_i\in m,1\leq i\leq l \end{equation}
We claim that $S\setminus \{m_j\}$ is also a generating set of $M,$ hence we get a contradiction. From $(1),$ $(1-a_j)m_j=\displaystyle\sum_{\substack{i=1\\ i\neq j}}^l (a_i-u_jr_i)m_i$ as $a_j\in m \Rightarrow 1-a_j$ is unit then $\exists b_j\in R$ such that $(1-a_j)b_j=1$ therefore, $m_j=\displaystyle\sum_{\substack{i=1\\ i\neq j}} (a_i-u_jr_i)b_jm_i.$ Let $s_i=(a_i-u_jr_i)b_j,1\leq i\leq l,i\neq j.$ Let $s\in M$ and \begin{align*} 
s&=\displaystyle\sum_{i=1}^l t_im_i=\displaystyle\sum_{\substack{i=1\\ i\neq j}}^l t_im_i+t_jm_j\\
&=\displaystyle\sum_{\substack{i=1\\ i\neq j}}^l t_im_i+t_j\left(\displaystyle\sum_{\substack{i=1\\ i\neq j}}^l s_im_i\right)\\
&=\displaystyle\sum_{\substack{i=1\\ i\neq j}}^l (t_i+t_js_i)m_i, \quad \text{contradiction}
\end{align*}
Therefore, $\pi(S)$ is linearly independent hence $\pi(S)$ is a basis. 
\begin{corollary}
Let $S_1,S_2\subseteq M$ be two minimal generating set of $M.$ Then $\pi(S_1)$ and $\pi(S_2)$ are two bases of $M/mM$ this implies, $|\pi(S_1)|=|\pi(S_2)| \Rightarrow |S_1|=|S_2|.$
\end{corollary}
Note that, $|S|=|\pi(S)|$ as $\pi(S)$ is linearly independent in $M/mM.$
\subsubsection{Construction of minimal generating set of M, where M is finitely generated R-module and R is a local ring}
$M$ is finitely generated module a over a local ring $(R,m,K)$ then $M/mM$ is finite dimensional $K$ vector space. Let $\{m_1+mM,\cdots ,m_r+mM\}$ be the basis of $M/mM$ over $K$ where $m_i\in M,1\leq i\leq r.$ Consider the set $S=\{m_1,\cdots ,m_r\}\subseteq M$ then $\langle S\rangle \subseteq M$ is a sub module, let $N=\langle S\rangle.$\\
Claim: $M=N.$ Suppose, $\alpha\in M \Rightarrow \alpha+mM\in M/mM$ so we can write \begin{align*}
\alpha+mM&=\displaystyle\sum_{i=1}^r (r_i+m)(m_i+mM)\\
&=\displaystyle\sum_{i=1}^r r_im_i+mM
\end{align*}
Therefore, $\alpha-\displaystyle\sum_{i=1}^r r_im_i\in mM \Rightarrow \alpha\in N+mM \Rightarrow M\subseteq N+mM,$ again $N$ and $mM$ is a sub modules of $M$ so that $N+mM\subseteq M.$ Therefore, $M=N+mM.$ By NAK $M=N.$
\begin{definition}
Let $M$ be finitely generated $R-$module over a local ring $R$ then $\mu(M)=|S|$ where $S\subseteq M$ is a minimal generating set for $M.$
\end{definition}
By previous discussion $\mu(M)$ is well defined.
\subsection{Product module and Free module}
\begin{defn}
Let $\{M_i\}_{i\in I}$ be a set of $R-$modules then, $\displaystyle\prod_{i\in I} M_i$ is a module with component wise scalar multiplication i.e., if $a=(a_i)\in \displaystyle\prod_{i\in I} M_i,$ $b=(b_i)\in \displaystyle\prod_{i\in I} M_i$ then $a+b=(a_i+b_i)$ and if $r\in R \Rightarrow ra=(ra_i).$ $\displaystyle\prod_{i\in I} M_i$ is called direct product of $\{M_i\}_{i\in I}.$
\end{defn}

\begin{defn}
We define direct sum of module $\displaystyle\bigoplus_{i\in I} M_i$ is a sub module of $\displaystyle\prod_{i\in I} M_i$ where  elements are $a=(a_i),$ all but finitely many components are zero.
\end{defn}
Note that if $I$ is finite then $\displaystyle\prod_{i\in I} M_i=\displaystyle\bigoplus_{i\in I} M_i$.

\begin{defn}
Let $M$ be an $R-$module. $S\subseteq M$ is said to be linearly independent if $\{s_1,\cdots ,s_r\}\subseteq S$ (any finite subset) and $\displaystyle\sum_{i=1}^r t_is_i=0,t_i\in R,1\leq i\leq r \Rightarrow t_i=0, ~\forall~i.$ $S$ is said to be basis of $M$ if $S$ is linearly independent and $\langle S\rangle=M.$
\end{defn}

\begin{defn}
An $R-$module $M$ is said to be free if it has a basis.
\end{defn}

\begin{problem}
$\displaystyle\bigoplus_{i\in I}R_i$ where $R_i=R,\text{~for all}~ i\in I$ is a free $R-$module with a basis $\{e_i\}_{i\in I}$ where $e_i=(0,\cdots ,0,1,0,\cdots)$ (ith position).
\end{problem}
Ans. If $m\in \displaystyle\bigoplus_{i\in I}R_i\Rightarrow m=\displaystyle\sum_{i=1}^k c_ie_{l_i}$ for some $k\in {\ZZ}^+.$ Suppose, $$\displaystyle\sum_{i=1}^k t_{l_i}e_{l_i}=0 \Rightarrow (0,\cdots ,t_{l_1},\cdots ,t_{l_2},\cdots ,t_{l_k},\cdots)=(0,0,\cdots,0,\cdots) \Rightarrow t_{l_i}=0, \text{~for all}~i\in I.$$

\begin{obs}
Note that for any $m_0\in M$ if we can write $m_0=\displaystyle\sum_{i=1}^rt_is_i=\displaystyle\sum_{i=1}^r t_i's_i,1\leq i\leq r$ this implies, $\displaystyle\sum_{i=1}^r(t_i-t_i')s_i=0 \Rightarrow t_i=t_i', 1\leq i\leq r.$ So each element $m_0\in M$ has unique representation.
\end{obs}


\begin{problem}
Show that $\displaystyle\prod_{i\in {\NN}} M_i,~M_i={\ZZ}\text{~for all}~ i\in {\NN}$ is not a free ${\ZZ}$ module.
\end{problem}

We discuss product module and free module further in \textbf{Appendix A} section.

\subsubsection{Universal property}
\begin{theorem}[Universal property]
Let $F$ be a free module with a basis $S$ and $N$ be any $R-$module with a set map $\tau:S\to N$ then, there exists unique module homomorphism $\tilde{\tau}:F\to N$ such that the diagram commutes.\begin{center}
\begin{tikzcd}
F \arrow[rrdd, "\tilde{\tau}"]             &  &   \\
                                           &  &   \\
S \arrow[rr, "\tau"] \arrow[uu, "i", hook] &  & N
\end{tikzcd}
\end{center}
\end{theorem}
\proof Let $m\in F$ then $m=\displaystyle\sum_{i=1}^n r_is_i$ (unique representation). Define, \begin{align*}
\tilde{\tau}&:F\to N\\
&m\mapsto \displaystyle\sum_{i=1}^n r_i\tau(s_i)
\end{align*}
Note that $\tilde{\tau}$ is well defined as $m$ has unique representation. Then $\tilde{\tau}(s)=\tau(s)~\forall s\in S \Rightarrow \tilde{\tau}\circ i=\tau.$\\
Let $g:F\to N$ be another module homomorphism such that $g\circ i=\tau \Rightarrow g(s)=\tau(s) ~\forall s\in S.$ Now, $$g(m)=g\left(\displaystyle\sum_{i=1}^n r_is_i\right)=\displaystyle\sum_{i=1}^n r_ig(s_i)=\displaystyle\sum_{i=1}^n r_i\tau(s_i)=\tilde{\tau}(m)$$ Therefore, $g=\tilde{\tau}.$ \qed

\begin{theorem}
Let $M$ be an $R-$module then there exists a free $R-$module $F$ and a surjective module homomorphism between $F$ and $M.$
\end{theorem}
\proof Let $S\subseteq M$ be a generating set. By Zermelo's theorem ``every set can be well ordered", we may think $S$ as a index set. Consider $F=\displaystyle\bigoplus_{i\in S}R_i$ where $R_i=R$ then $F$ is a free module with basis $\{e_i\}_{i\in S}=T$(say). Then we have a set map $\tau:T\to S(\subseteq M)$ defined by $\tau(e_s)=s.$ By Universal property of free module there exists a unique module homomorphism $\tilde{\tau}:F\to M$ such that the diagram commutes.
\begin{center}
\begin{tikzcd}
F \arrow[rrdd, "\tilde{\tau}"]             &  &   \\
                                           &  &   \\
T \arrow[rr, "\tau"] \arrow[uu, "i", hook] &  & M
\end{tikzcd}
\end{center}
Claim: $\tilde{\tau}$ is surjective. Let $m\in M$ then $\exists~\{s_1,\dots ,s_r\}\subseteq S$ such that \begin{align*}
m&=\displaystyle\sum_{i=1}^r r_is_i=\displaystyle\sum_{i=1}^r r_i\tau(e_{s_i})\\
&=\tilde{\tau}\left(\displaystyle\sum_{i=1}^r r_ie_{s_i}\right)
\end{align*}
Therefore, $\tilde{\tau}$ is surjective. \qed
\subsection{Exact Sequence}

\begin{defn}[Chain Complex]

Let $\{M_i,\phi_i\}_{i\in {\NN}}$ be a collection where $M_i$'s are $R-$modules and $\phi_i:M_i\to M_{i-1}$ be $R-$linear map, then the sequence,
 $$\mathbb{M}_{\sbullet}\equiv \cdots\to M_i\stackrel{\phi_i}{\longrightarrow} M_{i-1}\stackrel{\phi_{i-1}}{\longrightarrow} M_{i-2}\longrightarrow \cdots $$
  is called chain complex if $$\operatorname{Im}\phi_i\subseteq \operatorname{ker}\phi_{i-1}~\text{for all}~ i\in {\NN}$$ The chain complex is said to be exact if $$\operatorname{Im}\phi_i= \operatorname{ker}\phi_{i-1}~\text{for all}~ i\in {\NN}.$$

\end{defn}

\begin{defn}

Let, $H_i(\mathbb{M}_{\sbullet})=\dfrac{\operatorname{ker}\phi_{i}}{\operatorname{Im}\phi_{i+1}}$ where $\mathbb{M}_{\sbullet}$ is a chain complex, is called $i^{th}$ homology group of $H_i(\mathbb{M}_{\sbullet})$. Therefore, $\mathbb{M}_{\sbullet} $ is exact if and only if $H_i(\mathbb{M}_{\sbullet})=0$ for all $i\in{\NN}.$

\end{defn}

\begin{defn}[Short exact sequence]

Let $M_1,M_2,M_3$ be $R-$modules then the sequence $$0\longrightarrow M_1\stackrel{\phi_1}{\longrightarrow} M_2\stackrel{\phi_2}{\longrightarrow} M_3\longrightarrow 0$$ is called short exact sequence if it is an exact sequence i.e., \\$(i)$ $\phi_1$ is injective,\\
$(ii)$ $\operatorname{Im}\phi_1=\operatorname{ker}\phi_2,$\\
$(iii)$ $\phi_2$ is surjective.

\end{defn}

\begin{eg}

Let $f:M\to N$ be a module homomorphism then $$0\longrightarrow \operatorname{ker}f\stackrel{i}{\longrightarrow}M\stackrel{f}{\longrightarrow}\operatorname{Im}f\longrightarrow 0, \quad\operatorname{Im}f\subseteq N$$
the above sequence is exact.

\end{eg}

\begin{eg}

Let $M$ be an $R-$module then there exists a free module $F$ such that $$F\stackrel{\pi}{\longrightarrow} M\longrightarrow 0$$ is a surjection. Therefore, $$0\longrightarrow \operatorname{ker}\pi\stackrel{i}{\longrightarrow}  F\stackrel{\pi}{\longrightarrow} M\longrightarrow 0$$ is exact sequence and $F/\operatorname{ker}\pi\cong M.$

\end{eg}

\begin{defn}

Let $\mathbb{L}_{\sbullet}\equiv \{M_i,\phi_i\}$ and $\mathbb{T}_{\sbullet}\equiv\{N_i,\psi_i\}$ be two chain complexes. $\{\tau_i:M_i\to N_i\}$ is called a chain map if \begin{center}

\begin{tikzcd}
\mathbb{L}_{\sbullet}\equiv\cdots \arrow[r] & M_{i+1} \arrow[dd, "\tau_{i+1}"] \arrow[rr, "\phi_{i+1}"] &                                                & M_i \arrow[dd, "\tau_i"] \arrow[rr, "\phi_i"] &                                                & M_{i-1} \arrow[dd, "\tau_{i-1}"] \arrow[r, "\phi_{i-1}"] & \cdots \\
                 &                                                           & {} \arrow[loop, distance=2em, in=215, out=145] &                                               & {} \arrow[loop, distance=2em, in=215, out=145] &                                                          &        \\
\mathbb{T}_{\sbullet}\equiv \cdots \arrow[r] & N_{i+1} \arrow[rr, "\psi_{i+1}"]                          &                                                & N_i \arrow[rr, "\psi_i"]                      &                                                & N_{i-1} \arrow[r, "\psi_{i-1}"]                          & \cdots

\end{tikzcd}

\end{center}
each diagram is commutative i.e., $\tau_i\circ \phi_{i+1}=\psi_{i+1}\circ \tau_{i+1}$ for all $i\in{\NN}.$

\end{defn}

\begin{defn}

Let, $f:M\to N$ be a $R-$linear map then $\operatorname{Coker}f=N/\im f$ and $$0\longrightarrow \operatorname{ker}f\stackrel{i}{\longrightarrow}  M\stackrel{f}{\longrightarrow} N\stackrel{\pi}{\longrightarrow}\operatorname{coker}f\longrightarrow 0$$ is an exact sequence.

\end{defn}

\begin{defn}

Let $\{M^i,\phi^i\}_{i\in {\NN}}$ be a collection where $M^i$'s are $R-$modules and $\phi^i:M^i\to M^{i+1}$ be $R-$linear map, then the sequence, $$\mathbb{M}^{\sbullet}\equiv \cdots\to M^i\stackrel{\phi^i}{\longrightarrow} M^{i+1}\stackrel{\phi^{i+1}}{\longrightarrow} M^{i+2}\longrightarrow \cdots $$ is called co-chain complex if $$\operatorname{Im}\phi^i\subseteq \operatorname{ker}\phi^{i+1},~\text{for all}~ i\in {\NN}$$ The co-chain complex is said to be exact if $$\operatorname{Im}\phi^i= \operatorname{ker}\phi^{i+1},~\text{for all}~ i\in {\NN}.$$

\end{defn}

\begin{defn}

Let, $H^i(\mathbb{M}^{\sbullet})=\dfrac{\operatorname{ker}\phi^{i}}{\operatorname{Im}\phi^{i-1}}$ where $\mathbb{M}^{\sbullet}$ is a co-chain complex, is called $i^{th}$ co-homology group of $H^i(\mathbb{M}^{\sbullet})$. Therefore, $\mathbb{M}^{\sbullet} $ is exact if and only if $H^i(\mathbb{M}^{\sbullet})=0$ for all $i\in{\NN}.$

\end{defn}

\begin{defn}

Let $\mathbb{L}^{\sbullet}\equiv \{M^i,\phi^i\}$ and $\mathbb{T}^{\sbullet}\equiv\{N^i,\psi^i\}$ be two co-chain complexes. $\{\tau^i:M^i\to N^i\}$ is called a co-chain map if \begin{center}
\begin{tikzcd}

\mathbb{L}^{\sbullet}\equiv \cdots \arrow[r] & M^i \arrow[dd, "\tau^{i}"] \arrow[rr, "\phi^{i}"] &                                                & M^{i+1} \arrow[dd, "\tau^{i+1}"] \arrow[rr, "\phi^{i+1}"] &                                                & M^{i+2} \arrow[dd, "\tau^{i+2}"] \arrow[r, "\phi^{i+2}"] & \cdots \\
                                      &                                                   & {} \arrow[loop, distance=2em, in=215, out=145] &                                                           & {} \arrow[loop, distance=2em, in=215, out=145] &                                                          &        \\
                                      
\mathbb{T}^{\sbullet}\equiv \cdots \arrow[r] & N^i \arrow[rr, "\psi^i"]                          &                                                & N^{i+1} \arrow[rr, "\psi^{i+1}"]                          &                                                & N^{i+2} \arrow[r, "\psi^{i+2}"]                          & \cdots

\end{tikzcd}

\end{center}
each diagram is commutative i.e., $\psi^i\circ \tau^i=\tau^{i+1}\circ \phi^{i+1}$ for all $i\in{\NN}.$

\end{defn}

\begin{defn}

Let $\mathcal{C}$ and $\mathcal{D}$ be two chain complexes and $f:\mathcal{C}\to \mathcal{D}$ be a chain map.

\begin{center}

\begin{tikzcd}
\mathcal{C}\equiv\cdots \arrow[r] & M_{i+1} \arrow[r, "g_{i+1}"] \arrow[d, "f_{i+1}"] & M_i \arrow[r, "g_i"] \arrow[d, "f_i"] \arrow[ld, "s_i"] & M_{i-1} \arrow[r, "g_{i-1}"] \arrow[d, "f_{i-1}"] \arrow[ld, "s_{i-1}"] & \cdots \arrow[r] & M_0 \arrow[r] \arrow[d, "f_0"] & 0 \\

\mathcal{D}\equiv\cdots \arrow[r] & N_{i+1} \arrow[r, "h_{i+1}"']                     & N_i \arrow[r, "h_i"']                                   & N_{i-1} \arrow[r, "h_{i-1}"']                                           & \cdots \arrow[r] & M_0 \arrow[r]                  & 0
\end{tikzcd}

\end{center}

$f$ is said to be null homotopic if there exists a map $s_i:M_i\to N_{i+1}$ such that $$f_i=h_{i+1}\circ s_i+s_{i-1}\circ g_i$$ for all $i\in{\NN}.$ If $p$ and $q$ be two chain maps. Then $p$ and $q$ are said to be homotopic if $p-q$ is null homotopic.

\end{defn}

\begin{prop}

Let $p,q:\mathcal{C}\to \mathcal{D}$ be two homotopic chain maps. Then $H_i(\mathcal{C})\xrightarrow[q_{*i}]{p_{*i}} H_i(\mathcal{D})$ are the same maps.

\end{prop}

\proof Consider the diagram \begin{center}

\begin{tikzcd}
\cdots \arrow[r]                   & \Ker \alpha_{i+1} \arrow[d, hook] \arrow[r, "\alpha_{i+1}"] & \Ker \alpha_i \arrow[d, hook] \arrow[r, "\alpha_i"] & \Ker\alpha_{i-1} \arrow[d, hook] \arrow[r] & \cdots \\

\mathcal{C}\equiv \cdots \arrow[r] & M_{i+1} \arrow[r, "\alpha_{i+1}"] \arrow[d, "p_{i+1}"]      & M_i \arrow[r, "\alpha_i"] \arrow[d, "p_i"]          & M_{i-1} \arrow[r] \arrow[d, "p_{i-1}"]     & \cdots \\

\mathcal{D}\equiv\cdots \arrow[r]  & N_{i+1} \arrow[r, "\beta_{i+1}"]                            & N_i \arrow[r, "\beta_i"]                            & N_{i-1} \arrow[r]                          & \cdots
\end{tikzcd}

\end{center}

Let $x\in\Ker\alpha_i \Rightarrow \alpha_i(x)=0 \Rightarrow p_{i-1}\circ \alpha_i(x)=0 \Rightarrow \beta_i\circ p_i(x)=0 \Rightarrow p_i(x)\in\Ker\beta_i$. Therefore $p_i$ maps an element of $\Ker\alpha_i$ into $\Ker\beta_i$. Now,

\begin{center}

\begin{tikzcd}
\Ker\alpha_i \arrow[d, "\tilde{\pi_i}"'] \arrow[r, "p_i"]    & \Ker\beta_i \arrow[r, "\pi_i"] & \dfrac{\Ker\beta_i}{\im\beta_{i+1}} \\
\dfrac{\Ker\alpha_i}{\im\alpha_{i+1}} \arrow[rru, "p_{*i}"'] &                                &                                    
\end{tikzcd}

\end{center}

We want to show that $\im\alpha_{i+1}\subseteq \Ker(\pi_i\circ p_i).$ Let $y\in \im\alpha_{i+1},$ then there exists $y'\in M_{i+1}$ such that $\alpha_{i+1}(y')=y \Rightarrow p_i\circ \alpha_{i+1}(y')=p_i(y) \Rightarrow \beta_{i+1}\circ p_{i+1}(y')=p_i(y) \Rightarrow (\pi_i\circ \beta_{i+1})\circ p_{i+1}(y')=(\pi_i\circ p_i)(y)\Rightarrow 0=(\pi_i\circ p_i)(y) \Rightarrow y\in\Ker(\pi_i\circ p_i).$ Therefore there exists a well defined group morphism $p_{*i}:\dfrac{\Ker\alpha_i}{\im\alpha_{i+1}}\to \dfrac{\Ker\beta_i}{\im\beta_{i+1}}.$ Similar computation for $q_i$ leads to another map namely $q_{*i}$, hence we have two following maps  \begin{center}

\begin{tikzcd}

p_{*i}:\dfrac{\Ker\alpha_i}{\im\alpha_{i+1}} \arrow[r] & \dfrac{\Ker\beta_i}{\im\beta_{i+1}} & \text{and} & p_{*i}:\dfrac{\Ker\alpha_i}{\im\alpha_{i+1}} \arrow[r] & \dfrac{\Ker\beta_i}{\im\beta_{i+1}} \\

m_i+\im\alpha_{i+1} \arrow[r, maps to]                 & p_i(m_i)+\im\beta_{i+1}             &            & m_i+\im\alpha_{i+1} \arrow[r, maps to]                 & q_i(m_i)+\im\beta_{i+1}            

\end{tikzcd}

\end{center}
We need to show $p_{*i}=q_{*i}.$ Pick $m_i\in \Ker \alpha_i$ then $(p_i-q_i)(m_i)=\beta_{i+1}\circ s_i(m_i)+s_{i-1}\circ\alpha_i(m_i)=\beta_{i+1}\circ s_i(m_i)\in \im\beta_{i+1} \Rightarrow p_{*i}=q_{*i}.$ \qed




































































\subsection{Hom functor}

Recall, suppose $M,N$ be two $R-$modules, $$\Hom{M,N}:=\{f:M\to N|~f~\text{is a R-linear map}\}$$ is also a $R-$module. Let $f:M_1\to M_2$ be module homomorphism and $\phi\in \Hom{M_2,N}$ then $\phi \circ f\in \Hom{M,N}.$ Thus $f$ induces a module homomorphism\begin{align*}
f^*:\Hom{M_2,N}&\to \Hom{M,N}\\
\phi&\mapsto \phi\circ f
\end{align*} i.e., $f^*(\phi)=\phi\circ f,$
\begin{center}
 \begin{tikzcd}
M_1 \arrow[rr, "f"] \arrow[rrdd, "\phi \circ f"'] &  & M_2 \arrow[dd, "\phi"]                         \\
                                                  &  & {} \arrow[loop, distance=2em, in=150, out=210] \\
                                                  &  & N                                             
\end{tikzcd}\\
\end{center}
 and if $\psi\in \Hom{N,M_1}$ then \begin{center}
\begin{tikzcd}
N \arrow[rr, "\psi"] \arrow[rrrr, "f\circ \psi", bend right] &  & M_1 \arrow[rr, "f"] &  & M_2
\end{tikzcd}
\end{center}
Thus $f$ induces a module homomorphism \begin{align*}
f_*:\Hom{N,M_1}&\to \Hom{N,M_2}\\
\psi&\mapsto f\circ \psi
\end{align*}
i.e., $f_*(\psi)=f\circ \psi.$
\begin{proposition}
For any $R-$module $N,$ $\Hom{-,N}$ is contra variant left exact functor and $\Hom{N,-}$ is covariant left exact functor i.e., for any exact sequence \begin{align}
0\longrightarrow M_1\stackrel{f}{\longrightarrow} M_2\stackrel{g}{\longrightarrow}M_3\longrightarrow 0
\end{align} of $R-$module, the induce sequence \begin{align}
0\longrightarrow \Hom{M_3,N}\stackrel{g^*}{\longrightarrow}\Hom{M_2,N}\stackrel{f^*}{\longrightarrow}\Hom{M_1,N}
\end{align}
is exact and \begin{align}
0\longrightarrow \Hom{N,M_1}\stackrel{f_*}{\longrightarrow}\Hom{N,M_2}\stackrel{g_*}{\longrightarrow}\Hom{N,M_3}
\end{align}
is exact.
\end{proposition}
\proof We show that $g^*$ is injective. Let $g^*(\phi)=0 \Rightarrow \phi \circ g=0 \Rightarrow \phi \circ g(m_2)=0$ for all $m_2\in M_2.$ Let $m_3\in M_3$ since $g$ is surjective $\exists m_2'\in M_2$ such that $g(m_2')=m_3 \Rightarrow \phi(g(m_2'))=\phi(m_3) \Rightarrow \phi(m_3)=0 \Rightarrow \phi =0$ [since $m_3$ is chosen arbitrarily] therefore, $\Ker g^*=\{0\}$ implies $g^*$ is injective.\\
Now we show that $\im g^*=\Ker f^*.$ Let $f^*\circ g^*(\phi)=f^*(\phi \circ g)=\phi\circ (g\circ f)=0$ [$g\circ f=0$ as the sequence (3) is exact] therefore, $\im g^*\subseteq \Ker f^*.$ Let $\phi \Ker f^*$ where $\phi \in \Hom{M_2,N}$ i.e.,  $\phi \circ f=0.$
\begin{center}
\begin{tikzcd}
0 \arrow[rr] &  & M_1 \arrow[rr, "f"] \arrow[rrdd, "\phi\circ f"] &  & M_2 \arrow[rr, "g"] \arrow[dd, "\phi"] &  & M_3 \arrow[rr] &  & 0 \\
             &  &                                                 &  &                                        &  &                &  &   \\
             &  &                                                 &  & N                                      &  &                &  &  
\end{tikzcd}
\end{center}
Let $m_3\in M_3,$ since $g$ is surjective $\exists m_2\in M_2$ such that $g(m_2)=m_3.$ Let us define a function \begin{align*}
\psi:M_3&\to N\\
m_3&\mapsto \phi (m_2)
\end{align*}
Let $m_3=g(m_2)=g(m_2') \Rightarrow g(m_2-m_2')=0 \Rightarrow m_2-m_2' \in \Ker g=\im f$ then $\exists m_1\in M_1$ such that $m_2-m_2' =f(m_1) \Rightarrow m_2=m_2'+f(m_1) \Rightarrow \phi(m_2)=\phi(m_2')+\phi \circ f(m_1) \Rightarrow \phi(m_2)=\phi(m_2')$ [as $\phi \circ f=0$]
therefore, $\psi$ is well defined $R-$linear map. \\
Claim: $g^*(\psi)=\phi.$ Let $\psi \circ g(m_2)=\psi(m_3)=\phi(m_2),~\forall m_2\in M_2$ therefore $g^*(\psi)=\phi$ hence $\im g^*=\Ker f^*.$ \\
Now we will show that $f_*$ is injective. Let $f_*(\tau)=0 \Rightarrow f \circ \tau =0  \Rightarrow f(\tau (n))=0$ for all $n\in N.$ Since $f$ is injective, we have $\tau (n)=0$ for all $n\in N$ then $\tau =0,$ hence $f_*$ is injective. Next we need to show that $\im f_*=\Ker g_*.$ Let $\tau' \in \im f_*$ where $\tau' \in \Hom{N,M_2}$ then $\tau'=f_*(\tau) \Rightarrow \tau'(n)=f(\tau(n))$ for all $n\in N.$ Now, $\im f=\Ker g$ and $\tau'(n)\in \im f=\Ker g \Rightarrow g(\tau'(n))=0,~\forall n\in N \Rightarrow g_*(\tau')=0 \Rightarrow \tau' \in \Ker g_* \Rightarrow \im f_*\subseteq \Ker g_*.$ Suppose, $\tau' \in \Ker g_* \Rightarrow g_*(\tau')=0 \Rightarrow g \circ \tau' =0 \Rightarrow g(\tau'(n))=0,~\forall n\in N.$ Clearly $\tau'(n)\in \Ker g=\im f.$\\

\begin{theorem}
Let $\{M_i\}_{i\in\Lambda}$ be a collection of $R-$module and $N$ is also be an $R-$module then
\begin{enumerate}
\item $\Hom{\displaystyle\bigoplus_{i\in\Lambda}M_i,N}\cong \displaystyle\prod_{i\in \Lambda}\Hom{M_i,N}.$
\end{enumerate}
\end{theorem}









\newpage

\subsection{Tensor Product}
\begin{defn}
Let $M_1,\cdots,M_k,N$ be $R-$modules. A map $f:M_1\times \cdots\times M_k\to N$ is said to be linear in ith variable if, given fixed $m_j,j\neq i$, the map $$T:M_i\to N$$ defined by $T(m)=f(m_1,\cdots,m_{i-1},m,m_{i+1},m_k)$ is linear. The map $f$ is said to be multilinear if it is linear in each variable.
\end{defn}

Let $M, N$ be two $R-$modules. Consider the free module $F$ generated by the $M\times N$ over $R$, then the elements of $F$ are of the form $\displaystyle\sum_{\text{finite sum}} r_ix_i$ where $r_i\in R$ and $x_i\in M\times N.$ Let $D$ be the submodule of $F$ generated by the elements of the form \begin{align*}
&(m_1+m_2,n)-(m_1,n)-(m_2,n)\\
&(m,n_1+n_2)-(m,n_1)-(m,n_2)\\
&(rm,n)-r(m,n)\\
&(m,rn)-r(m,n)
\end{align*}
where $m,m_1,m_2\in M$, $n,n_1,n_2\in N$ and $r\in R.$ Let $T=F/D.$ We denote $T=M\otimes_R N$ and $T$ is said to be Tensor product of $M$ and $N$. We denote $(m,n)+D\in F/D$ by $m\otimes n$ and we have a map \begin{align*}
M\times N&\stackrel{\pi}{\to} T\\
(m,n)&\mapsto m\otimes n
\end{align*}

We will show that $\pi$ is bilinear map. $\pi((m_1+m_2,n))=(m_1+m_2,n)+D.$ Since $$(m_1+m_2,n)-(m_1,n)-(m_2,n)\in D$$
$\pi((m_1+m_2,n))=(m_1+m_2,n)+D=(m_1,n)+D+(m_2,n)+D=\pi(m_1,n)+\pi(m_2,n)$ for all $m_1,m_2\in M$ and for all $n\in N$. Similarly we can show that $\theta$ satisfies the property of bilinear map.

\begin{theorem}[Universal Property]
For every bilinear map $\beta:M\times N\to P$ where $P$ is an $R-$ module, there exists an unique $R-$linear map $\tilde{\beta}:M\otimes_R N\to P$ such that the diagram commutes.

\begin{center}
\begin{tikzcd}
M\times N \arrow[rr, "\beta"] \arrow[dd, "\pi"'] &  & P \\
                                                 &  &   \\
M\otimes_R N \arrow[rruu, "\tilde{\beta}"']      &  &  
\end{tikzcd}
\end{center}


 More over, if $(T',\theta')$ be another pair with such property then there exists a module isomorphism $M\otimes_RN\to T'.$
\end{theorem}

\proof Define $\tilde{\beta}:M\otimes_RN\to P$ by $\tilde{\beta}(m\otimes n)=\beta(m,n)$ and extend it linearly. Let $m_1\otimes n_1=m_2\otimes n_2 \Rightarrow (m_1,n_1)-(m_2,n_2)\in D.$ Since 

 
 By our construction $\tilde{\beta}$ is bilinear. Suppose $\gamma:M\otimes_RN\to P$ be another $R-$linear map such that the diagram commutes. Then $\gamma(m\otimes n)=\beta(m,n)=\tilde{\beta}\pi(m,n)=\tilde{\beta}(m\otimes n).$ Hence $\gamma=\tilde{\beta}.$
 
  Now we assume that there exists another pair $(T',\theta')$ with same property, then

\begin{center}
\begin{tikzcd}
M\times N \arrow[rr, "\pi"] \arrow[dd, "\theta'"'] &  & M\otimes_R N &  &  & M\times N \arrow[dd, "\pi"'] \arrow[rr, "\theta'"] &  & T' \\
                                                   &  &              &  &  &                                                    &  &    \\
T' \arrow[rruu, "\tilde{\pi}"']                    &  &              &  &  & M\otimes_R N \arrow[rruu, "\tilde{\theta'}"']      &  &   
\end{tikzcd}
\end{center}

where $\tilde{\pi}$ and $\tilde{\theta'}$ are $R-$ linear map. Since the diagrams commutes, we have $\tilde{\pi}\circ \theta'=\pi$ (from first diagram) and $\tilde{\theta'}\circ \pi=\theta'$ (from second diagram). Hence $(\tilde{\theta'}\circ \tilde{\pi})\circ \theta'=\theta'$ and $(\tilde{\pi}\circ\tilde{\theta'})\circ \pi=\pi.$ Again we consider the following diagrams 

\begin{center}
\begin{tikzcd}
M\times N \arrow[rr, "\pi"] \arrow[dd, "\pi"']   &  & M\otimes_R N &  &  & M\times N \arrow[dd, "\theta'"'] \arrow[rr, "\theta'"] &  & T' \\
                                                 &  &              &  &  &                                                        &  &    \\
M\otimes_R N \arrow[rruu, "\id_{M\otimes_R N}"'] &  &              &  &  & T' \arrow[rruu, "\id_{T'}"']                           &  &   
\end{tikzcd}
\end{center}

By Universal property, we have $\tilde{\theta'}\circ \tilde{\pi}=id_{T'}$ and $\tilde{\pi}\circ \tilde{\theta'}=\id_{M\otimes_R N}.$ \qed\\

\textbf{Tensor product of algebras.} Let $A$ and $B$ be $R-$algebra, We consider the module $C=A\otimes_R B$. Let us define a mapping $\beta:A\times B\times A\times B\to C$ by $\beta(a,b,a',b')=aa'\otimes bb'.$ Since $\beta$ is multilinear, $\beta$ induce a mapping $\tilde{\beta}:C\otimes_R C\to C$. This $\tilde{\beta}$ corresponds a bilinear mapping $\gamma:C\times C\to C$ given by $\gamma(a\otimes b,a'\otimes b')=aa'\otimes bb'.$ Since $\gamma$ is well define, it defines a multiplication on $C$ and therefore $C$ becomes a commutative ring with unity, $1\otimes 1$ being the multiplicative identity. Since $A$ and $B$ are $R-$algebra, there exists $f:R\to A$ and $g:R\to B$ two ring morphisms. Now we define $\psi:R\to A\otimes_R B$ by $\psi(r)=f(r)\otimes g(r).$ Let $r_1,r_2\in R$ then $\psi(r_1+r_2)=f(r_1+r_2)\otimes g(r_1+r_2)=f(r_1)\otimes g(r_1+r_2)+f(r_2)\otimes g(r_1+r_2).$



We note that $C$ is both $A$ and $B$ algebra as $\mu_A:A\to A\otimes_R B$ is defined by $\mu_A(a)=a\otimes 1_B$ and $\mu_B:B\to A\otimes_R B$ is defined by $\mu_B(b)=1_A\otimes b.$ It is easy to check that both $\mu_A$ and $\mu_B$ is a ring homomorphism.

\begin{theorem}[Properties of Tensor product]
Let $M,N,P$ and $\{M_i\}_{i\in\Lambda}$ be $R-$modules, $I\subseteq R$ be a ideal of $R,$ $S$ be a multiplicatively closed set in $R$ then we have
\begin{enumerate}
\item $M\otimes_R N\cong N\otimes_R M.$
\item $(M\otimes_R N)\otimes_R P\cong M\otimes_R(N\otimes_R) P.$
\item $M\otimes_R R\cong M.$
\item $M\otimes_R R/I \cong M/IM.$
\item $M\otimes_R S^{-1}R\cong S^{-1}M.$
\item $\left(\displaystyle\bigoplus_{i\in\Lambda} M_i\right) \otimes_R N\cong \displaystyle\bigoplus_{i\in\Lambda} (M_i\otimes_R N).$
\end{enumerate}
\end{theorem}

\proof \begin{enumerate}
\item Consider the diagram 

\begin{center}
\begin{tikzcd}
M\times N \arrow[rr, "\alpha", shift left] \arrow[dd, "\pi"'] &  & N\times M \arrow[dd, "\tilde{\pi}"] \arrow[ll, "\beta", shift left] \\
                                                              &  &                                                             \\
M\otimes_R N \arrow[rr, "\alpha'", shift left]                &  & N\otimes_R M \arrow[ll, "\beta'", shift left]              
\end{tikzcd}
\end{center}

where $\alpha((m,n))=(n,m)$ and $\beta((n,m))=(m,n).$ We claim that $\tilde{\pi}\circ\alpha$ is bilinear. Let $(m_1+m_2,n)\in M\times N$, $\tilde{\pi}\alpha((m_1+m_2,n))=\tilde{\pi}(n,m_1+m_2)=n\otimes (m_1+m_2)=n\otimes m_1+n\otimes m_2=\tilde{\pi}\alpha((m_1,n))+\tilde{\pi}\alpha((m_2,n))$ for all $m_1,m_2\in M$ and for all $n\in N.$ Similarly other properties can be shown. By Universal property, we have a module morphism $\alpha':M\otimes_R N\to N\otimes_R M.$ Similarly the map $\beta\circ \pi$ is also bilinear so we have a $R-$ linear map $\beta':N\otimes_R M\to M\otimes_R M.$ We just need to show that $\alpha'\circ \beta=\id_{N\otimes_R M}$ and $\beta'\circ \alpha'=\id_{M\otimes_R N}$ which is easy, $\alpha'\circ \beta'(n\otimes m)=\alpha'(m\otimes n)=n\otimes n$ and $\beta'\circ \alpha'(m\otimes n)=\beta'(n\otimes m)=m\otimes n.$

\item 

\item Let $f:M\times R\to M$ be the map where $f(m,r)=rm.$ Since $M$ is an $R-$module, $f$ is bilinear, hence $f$ induce a map $\tilde{f}:M\otimes_R R\to R$ such that the diagram commutes

\begin{center}
\begin{tikzcd}
M\times R \arrow[rr, "f"] \arrow[dd, "\pi"'] &  & M \\
                                             &  &   \\
M\otimes_R R \arrow[rruu, "\tilde{f}"']      &  &  
\end{tikzcd}
\end{center}

where $\tilde{f}\circ \pi=f \Rightarrow f(m,r)=\tilde{f}\pi(m,r) \Rightarrow rm=\tilde{f}(m\otimes r)$ and $\tilde{f}$ is $R-$linear. Let $g:M\to M\otimes_R R$ defined as $g(m)=m\otimes 1.$ It is easy to show that $g$ is $R-$linear and $\tilde{f}\circ g=\id_{M}$ and $g\circ \tilde{f}=\id_{M\otimes_R R}.$

\item Let $f:M\times R/I\to M/IM$ be the bilinear map defined by $f(m,r+I)=rm+IM.$ By Universal property there exists a well define module morphism $\tilde{f}:M\otimes_R R/I\to M/IM$ such that the diagram commutes, 

\begin{center}
\begin{tikzcd}
M\times R/I \arrow[rr, "f"] \arrow[dd, "\pi"'] &  & M/IM \\
                                             &  &   \\
M\otimes_R R/I \arrow[rruu, "\tilde{f}"']      &  &  
\end{tikzcd}
\end{center}

where $\tilde{f}(m\otimes (r+I))=rm+IM.$ Let $g:M/IM\to M\otimes_R R/I$ be the map $g(m+IM)=m\otimes (1+I)$. Then $g$ is an $R-$linear map and $g\circ \tilde{f}=\id_{M\otimes_R R/I}$ and $\tilde{f}\circ g=\id_{M/IM}.$

\item Consider \begin{center}
\begin{tikzcd}
M\times S^{-1}R \arrow[rr, "f"] \arrow[dd, "\pi"'] &  & S^{-1}M \\
                                             &  &   \\
M\otimes_R S^{-1}R \arrow[rruu, "\tilde{f}"']      &  &  
\end{tikzcd}
\end{center}

where $f\left(m,\dfrac{r}{s}\right)=rm/s.$ First we need to check $f$is well defined. Let $\dfrac{r_1}{s_1}=\dfrac{r_2}{s_2}$ then there exists some $s\in S$ such that $s(r_1s_2-s_1r_2)=0 \Rightarrow s(r_1s_2-s_1r_2)m=0\Rightarrow s(r_1s_2m-s_1r_2m)=0 \Rightarrow \dfrac{r_1m}{s_1}=\dfrac{r_2m}{s_2}.$ It is obvious that $f$ is bilinear. Then there exits a unique module morphism $\tilde{f}:M\otimes_R S^{-1}R \to S^{-1}M$ where $\tilde{f}\left(m\otimes \dfrac{r}{s}\right)=\dfrac{rm}{s}.$ Define $g:S^{-1}M\to M\times S^{-1}R$ by $g\left(\dfrac{m}{s}\right)=m\times \dfrac{1}{s}.$ $g$ is well defined module morphism and $g=\tilde{f}^{-1}.$

\item Let $\theta_i:M_i\to \displaystyle\bigoplus_{i\in\Lambda} M_i$ be the inclusion map. Define \begin{align*}
f:\left(\displaystyle\bigoplus_{i\in\Lambda} M_i\right)\times N&\to \displaystyle\bigoplus_{i\in\Lambda} (M_i\otimes_R N)\\
((m_i)_{i\in\Lambda},n)&\mapsto (m_i\otimes n)_{i\in\Lambda}.
\end{align*}
We will show that $f$ is bilinear. $f((m_i)_{i\in\Lambda}+(m'_i)_{i\in\Lambda},n)=f((m_i+m'_i)_{i\in\Lambda},n)=((m_i+m'_i)\otimes n)_{i\in\Lambda}=(m_i\otimes n)_{i\in\Lambda}+(m'_i\otimes n)_{i\times \Lambda}=f((m_i)_{i\in\Lambda},n)+f((m'_i)_{i\in\Lambda},n).$ Similarly other properties can be shown. Hence we have a map $\tilde{f}:\left(\displaystyle\bigoplus_{i\in\Lambda} M_i\right)\otimes_R N\to \displaystyle\bigoplus_{i\in\Lambda} (M_i\otimes_R N)$ defined by $\tilde{f}((m_i)_{i\in\Lambda}\otimes n)=(m_i\otimes n)_{i\in\Lambda}.$ Define $g: \displaystyle\bigoplus_{i\in\Lambda} (M_i\otimes_R N)\to \left(\displaystyle\bigoplus_{i\in\Lambda} M_i\right)\otimes_R N$ by $g((m_i\otimes n_i)_{i\in\Lambda})=\displaystyle\sum_{i\in\Lambda}(\theta_i(m_i)\otimes n_i)$. Note that $g$ is $R-$linear. Now, $g\circ\tilde{f}((m_i)_{i\in\Lambda}\otimes n)=g((m_i\otimes n)_{i\in\Lambda})=\displaystyle\sum_{i\in\Lambda} (\theta_i(m_i)\otimes n)=\left(\displaystyle\sum_{i\in\Lambda} \theta_i(m_i)\right)\otimes n=(m_i)_{i\in\Lambda}\otimes n\Rightarrow g\circ \tilde{f}=\id_{\left(\displaystyle\bigoplus_{i\in\Lambda} M_i\right)\otimes_R N}.$ Let $(m_i\otimes n_i)_{i\in\Lambda}\in \displaystyle\bigoplus_{i\in\Lambda} (M_i\otimes_R N),$ then $\tilde{f}\circ g((m_i\otimes n_i)_{i\in\Lambda})=\tilde{f}\left(\displaystyle\sum_{i\in\Lambda}(\theta_i(m_i)\otimes n_i)\right)=\displaystyle\sum_{i\in\Lambda}\tilde{f}(\theta_i(m_i)\otimes n_i)=\displaystyle\sum_{i\in\Lambda}=\theta_i(m_i)\otimes n_i=(m_i\otimes n_i)_{i\in\Lambda} \Rightarrow \tilde{f}\circ g=\id_{\displaystyle\bigoplus_{i\in\Lambda} (M_i\otimes_R N)}.$

\end{enumerate}
\qed

\begin{remark}
Let $f:A\to B$ be a ring homomorphism. Suppose $M$ is an $A-$module and $N$ is an $B-$module. Then $M\otimes_A N$ has both $A$ and $B$ module structure, \begin{align*}
 B\times M\otimes_A N&\to M\otimes_A N\\
 (n,m\otimes n)&\mapsto m\otimes bm
\end{align*}
\end{remark}

\begin{theorem}
Let $B$ be an $A$ algebra, $M$ be an $A-$module and $N,P$ be $B-$modules. Then $$(M\otimes_A N)\otimes_B P\cong M\otimes_A(N\otimes_B P).$$
\end{theorem}

\proof It is suffices to establish the isomorphism as $B-$module.

\qed

\begin{theorem}[Hom-Tensor adjunction]
Let $M,N,P$ be $R-$modules. Then $$\Hom{M\otimes_RN,P}\cong \Hom{M,\Hom{N,P}}.$$
\end{theorem}

\proof Define \begin{align*}
\psi:\Hom{M\otimes_RN,P}&\to \Hom{M,\Hom{N,P}}\\
f&\mapsto \psi(f)
\end{align*}
where $\psi(f)(m)(n)=f(m\otimes n)$ and \begin{align*}
\phi:\Hom{M,\Hom{N,P}}&\to \Hom{M\otimes_RN,P}\\
g&\mapsto \phi(g)
\end{align*}
where $\phi(g)(m\otimes n)=g(m)(n).$ We shall now show that $\phi(g)$ is well defined. Consider the diagram 

\begin{center}
\begin{tikzcd}
M\times N \arrow[rr, "f"] \arrow[dd, "\pi"'] &  & P \\
                                             &  &   \\
M\otimes_R N \arrow[rruu, "\tilde{f}"']      &  &  
\end{tikzcd}
\end{center}

where $f(m,n)=g(m)(n)$. We claim that $f$ is bilinear. $f(m_1+m_2,n)=g(m_1+m_2)(n)=g(m_1)(n)+g(m_2)(n)=f(m_1,n)+f(m_2,n)$ for all $m_1,m_2\in M$ and for all $n\in N.$ Now $f(m,n_1+n_2)=g(m)(n_1+n_2)=g(m)(n_1)+g(m)(n_2)=f(m,n_1)+f(m,n_2)$ for all $m\in M$ and for all $n_1,n_2\in N.$ Pick $r\in R,m\in M$ and $n\in N$, $f(rm,n)=g(rm)(n)=rg(m)(n)=rf(m,n)$ and $f(m,rn)=g(m)(rn)=rg(m)(n)=rf(m,n).$ By Universal property $\tilde{f}$ is well defined map such that $\tilde{f}\circ \pi=f$ and $\tilde{f}=\phi(g).$ Now it is easy to show that $\phi\circ \psi=\id_{\Hom{M\otimes_RN,P}}$ and $\psi\circ \phi=\id_{\Hom{M,\Hom{N,P}}}.$ \qed

\begin{theorem}
Let $B$ be an $A$ algebra, $M$ be an $A-$module and $N,P$ be $B$ modules. Then $$\Homb{M\otimes_A N,P}\cong \Homa{M,\Homb{N,P}}.$$
\end{theorem}

\proof Note that $\Homa{M,\Homb{N,P}}$ is an $B-$module, \begin{align*}
B\times \Homa{M,\Homb{N,P}}&\to \Homa{M,\Homb{N,P}}\\
(b,f)&\mapsto (bf)
\end{align*}
where $(bf):M\to \Homb{N,P}$ is defined by $(bf)(m):=b\cdot(f(m)).$

Now, we define $$\theta:\Homa{M,\Homb{N,P}}\to \Homb{M\otimes_A N,P}$$ where $\theta(f)(m\otimes n)=f(m)(n).$ We will show that $\theta(f)$ is well defined. 

\begin{center}
\begin{tikzcd}
M\times N \arrow[rr, "\alpha"] \arrow[dd, "\pi"'] &  & P \\
                                             &  &   \\
M\otimes_R N \arrow[rruu, "\tilde{\alpha}"']      &  &  
\end{tikzcd}
\end{center}

Where $\alpha(m,n)=f(m)(n).$ We claim that $\alpha$ is $A-$linear in first component and $B-$linear in second component. Let $m_1,m_2\in  M$ and $m\in N,$ $\alpha(m_1+m_2,n)=f(m_1+m_2)(n)=f(m_1)(n)+f(m_2)(n)=\alpha(m_1,n)+\alpha(m_2,n).$ Let $m\in M, n_1,n_2\in N$ then $\alpha(m,n_1+n_2)=f(m)(n_1+n_2)=f(m)(n_1)+f(m)(n_2)=\alpha(m,n_1)+\alpha(m,n_2).$ Now, for all $a\in A,m\in M,n\in N$, $\alpha(am,n)=f(am,n)=af(m)(n)=a\alpha(m,n)$ and for all $b\in B, m\in M,n\in N,$ $\alpha(m,bn)=f(m)(bn)=bf(m)(n)=b\alpha(m,n)$. Hence $\alpha$ is $A-$linear in first component and $B-$linear in second component. Hence $\theta(f)$ is a well defined $B-$linear map. Let \begin{align*}
\psi:\Homb{M\otimes_A N,P}&\to \Homa{M,\Homb{N,P}}\\
g&\mapsto \psi(g)
\end{align*}
where $\psi(g):M\to \Homb{N,P}$ is the map $\psi(g)(m)(n)=g(m\otimes n).$ It is easy to show that $\psi$ is a $B-$linear map and $\psi\circ \theta=\id_{\Homa{M,\Homb{N,P}}}$ and $\theta\circ \psi=\id_{\Homb{M\otimes_A N,P}}.$ \qed

\begin{corollary}
Let \begin{equation}
0\to M'\to M\to M''\to 0
\end{equation} 
be an exact sequence of $R-$modules. Let $N$ be another $R-$module then the sequence 
\begin{equation}
M'\otimes_R N\to M\otimes_R N\to M''\otimes_R N\to 0
\end{equation} 
is exact.
\end{corollary}

\proof Let $P$ be any $R-$module. Since (11) is exact, the sequence \begin{equation}
\Hom{M'',\Hom{N,P}}\to \Hom{M,\Hom{N,P}}\to \Hom{M',\Hom{N,P}}\to 0
\end{equation}
is exact and by Theorem 14.53 we have $$\Hom{M''\otimes_R N,P}\to \Hom{M\otimes_R N,P}\to \Hom{M'\otimes_R N,P}\to 0$$ is exact. Hence we have (12). \qed

\subsubsection{Flat module}

\begin{defn}
A module $N$ is said to be flat $R-$module if for every short exact sequence of $R-$modules $$0\to M'\to M\to M''\to 0$$ we have the following short exact sequence $$0\to M'\otimes_R N\to M\otimes_R N\to M''\otimes_R N\to 0.$$
\end{defn}

\begin{remark}
\begin{enumerate}

\item An $R$-mdoule $N$ is said to be flat if and only if for every short exact sequence $$0\to M'\to M\to M''\to 0$$ of $R-$modules, we have the following exact sequence $$0\to M'\otimes_R N\to M\otimes_R N.$$

\item An $R-$module $N$ is said to be flat if for every exact sequence $$\sum\equiv \cdots\to M_i\to M_{i+1}\to M_{i+2}\to\cdots$$ of $R-$modules, we have the following exact sequence $$\sum\otimes_R N\equiv \cdots\to M_i\otimes_R N\to M_{i+1}\otimes_R N\to M_{i+2}\otimes_R N\to\cdots.$$
\end{enumerate}
\end{remark}

\begin{defn}
An $R-$module $N$ is said to be faithfully flat module if it is a flat module and any sequence of $$\sum\equiv \cdots\to M_i\to M_{i+1}\to M_{i+2}\to\cdots$$ of $R-$modules, $\displaystyle\sum\otimes_R N$ is exact implies $\displaystyle\sum$ is an exact sequence.
\end{defn}

\begin{defn}
Let $S$ be an $R-$algebra. $S$ is said to be flat over $R$ if $S$ is a flat $R-$module.
\end{defn}

\begin{eg}
Let $S$ be a multiplicatively closed set of a ring $R$ then $S^{-1}R$ is a flat $R-$module.
\end{eg}

\begin{qns}
Let $I$ be an ideal of $R$. Is $R/I$ flat?
\end{qns}

\begin{lemma}
Let $M,N$ be flat $R-$modules then $M\otimes_R N$ and $M\oplus N$ is also flat.
\end{lemma}

\proof \begin{enumerate}
\item Let $$0\to M_1\to M_2\to M_3\to 0$$ be an exact sequence of $R-$modules, since $M$ is flat, the following sequence $$0\to M_1\otimes_R M\to M_2\otimes_R M\to M_3\otimes_R M\to 0$$ is exact and so the sequence $$0\to (M_1\otimes_R M)\otimes_R N\to (M_2\otimes_R M)\otimes_R N\to (M_3\otimes_R M)\otimes_R N\to 0.$$ Hence $$0\to M_1\otimes_R (M\otimes_R N)\to M_2\otimes_R (M\otimes_R N)\to M_3\otimes_R (M\otimes_R N)\to 0$$ is exact. Therefore, $M\otimes_R N$ is flat.\\

\item Since $M$ and $N$ are flat the sequences $$0\to M_1\otimes_R M\xrightarrow{\alpha_M} M_2\otimes_R M\xrightarrow{\beta_M} M_3\otimes_R M\to 0$$ and $$0\to M_1\otimes_R N\xrightarrow{\alpha_N} M_2\otimes_R N\xrightarrow{\beta_N} M_3\otimes_R N\to 0$$ are exact. Therefore the sequence $$0\to M_1\otimes_R M\oplus M_1\otimes N\xrightarrow{(\alpha_M,\alpha_N)} M_2\otimes_R M\oplus M_2\otimes N\xrightarrow{(\beta_M,\beta_N)} M_3\otimes_R M\oplus M_3\otimes N\to 0$$ is exact. So we have the following exact sequence, $$0\to M_1\otimes_R (M\oplus N)\to M_2\otimes_R (M\oplus N)\to M_3\otimes_R (M\oplus N)\to 0.$$
Hence $M\oplus N$ is a flat $R-$module.
\end{enumerate}
\qed

\begin{remark}
Let $S$ be a flat $R-$ algebra and $N$ be a flat $S-$module. Then $N$ is a flat $R-$module.
\end{remark}

\proof Let $$0\to M_1\to M_2\to M_3\to 0$$ be an exact sequence of $R-$modules. Since $S$ is flat $R-$module, $$0\to M_1\otimes_R S\to M_2\otimes_R S\to M_3\otimes_R S\to 0$$ is an exact sequence of $R-$module. Since $S$ is an $R-$algebra, each $M_i\otimes_R S,1\leq i\leq 3$ also has $S-$module structure. So the above sequence is an exact sequence of $S-$module. Since $N$ is flat $S-$module, $$0\to (M_1\otimes_R S)\otimes_S N\to (M_2\otimes_R S)\otimes_S N\to (M_3\otimes_R S)\otimes_S N\to 0$$ is exact and so the sequences are 

\begin{center}
\begin{tikzcd}
 0\ar[r] & M_1\otimes_R (S\otimes_S N) \ar[r] & M_2\otimes_R (S\otimes_S N) \ar[r] & M_3\otimes_R (S\otimes_S N) \ar[r]&  0 \\
0\ar[r]  & M_1\otimes_R N \ar[u, phantom, "\nvisom"] \ar[r]& M_2\otimes_R N \ar[u,phantom, "\nvisom"]\ar[r]  & M_3\otimes_R N \ar[u, phantom, "\nvisom"] \ar[r]& 0
\end{tikzcd}
\end{center}
 
therefore, $N$ is a flat $R-$module. \qed

\begin{theorem}
Let $M$ and $N$ be two $S^{-1}R$ modules, then $M$, $N$ are also $R-$modules via $\psi:R\to S^{-1}R$. Then $M\otimes_{S^{-1}R} N\cong M\otimes_R N.$
\end{theorem}

\proof We note that $M\otimes_R N$ is an $S^{-1}R-$module. We will show that $M\otimes_R N$ and $M\otimes_{S^{-1}R} N$ is same as $S^{-1}R-$module, hence they are same as $R-$module also. In $M\otimes_R N$, $$\dfrac{a}{s}(m\otimes n)=\dfrac{am}{s}\otimes n=\dfrac{am}{s}\otimes \dfrac{ns}{s}=\dfrac{sm}{s}\otimes \dfrac{an}{s}=m\otimes \dfrac{an}{s}.$$ Thus $\dfrac{a}{s}m\otimes n=m\otimes \dfrac{an}{s}$ in $M\otimes_R N$. So they are same as $S^{-1}R-$module. \qed

\begin{theorem}
Let $S$ be an $R-$algebra and $M$ be an $S-$module. A necessary and sufficient condition for $M$ to be flat over $R$ is that for every $p\in \spec S,$ $M_p$ is flat $R_q-$module where $q=p\cap R.$
\end{theorem}

\proof First we note that $M_p$ is an $R_q$ module. As $S$ is an $R-$algebra, there exists $f:R\to S$ and $f(p)\subseteq q$ then by Universal property of localization there exists an unique morphism $f_p:R_q\to S_p$ to make $S_p$ an $R_q-$algebra. Now $S_p\otimes_S M\cong M_p$. Thus $M_p$ is an $S_p-$module hence $M_p$ is an $A_q-$module. Suppose $M$ is flat. Consider the exact sequence of $R_q-$modules (also as $R-$modules) \begin{equation}
0\to M_1\to M_2\to M_3\to 0
\end{equation}

By previous theorem, \begin{equation}
M_p\otimes_{R_q} M_i\cong M_p\otimes_R M_i,1\leq i\leq 3.
\end{equation} 
Now From (14) $$0\to M_1\otimes_R M\to M_2\otimes_R M\to M_3\otimes_R M\to 0$$ is an exact sequence of $S-$mdoule (since $M$ is an $S-$module). As $S_p$ is flat over $S$ we have the following exact sequences 

\begin{center}
\begin{tikzcd}
 0\ar[r] & (M_1\otimes_R M)\otimes_S S_p \ar[r] & (M_2\otimes_R M)\otimes_S S_p \ar[r] & (M_3\otimes_R M)\otimes_S S_p \ar[r]&  0 \\
0\ar[r]  & M_1\otimes_R (M\otimes_S S_p) \ar[u, phantom, "\nvisom"] \ar[r]& M_2\otimes_R (M\otimes_S S_p) \ar[u,phantom, "\nvisom"]\ar[r]  & M_3\otimes_R (M\otimes_S S_p) \ar[u, phantom, "\nvisom"] \ar[r]& 0\\
0\ar[r]  & M_1\otimes_R M_p \ar[u, phantom, "\nvisom"] \ar[r]& M_2\otimes_R M_p \ar[u,phantom, "\nvisom"]\ar[r]  & M_3\otimes_R M_p \ar[u, phantom, "\nvisom"] \ar[r]& 0
\end{tikzcd}
\end{center}

From (15) we have the following exact sequence $$0\to M_1\otimes_{R_q} M_p\to M_2\otimes_{R_q} M_p\to M_3\otimes_{R_q} M_p\to 0.$$ Thus $M_p$ is a flat $R_q$ module.

 Conversely, let $M_p$ be flat over $R_q$ for all $p\in \spec S$ and $q=p\cap R.$ Consider the exact sequence of $R-$modules $0\to N'\xrightarrow{\phi} N$ then $$0\to\Ker(\phi\otimes 1)\xrightarrow{i}N'\otimes_R M\xrightarrow{\phi\otimes 1} N\otimes_R M$$ where $\Ker(\phi\otimes 1), N'\otimes_R M$ and $N\otimes_R M$ are $S-$modules and $S_p$ is flat over $S$. Thus we have the exact sequence 
 
 \begin{center}
\begin{tikzcd}
 0\ar[r] & (\Ker(\phi\otimes 1))\otimes_S S_p \ar[r] & (N'\otimes_R M)\otimes_S S_p \ar[r] & (N\otimes_R M)\otimes_S S_p~\text{is exact} \\

0\ar[r]  & (\Ker(\phi\otimes 1))_p \ar[u, phantom, "\nvisom"] \ar[r]& N'\otimes_R (M\otimes_S S_p) \ar[u,phantom, "\nvisom"]\ar[r]  & N\otimes_R (M\otimes_S S_p) \ar[u, phantom, "\nvisom"]~\text{is exact}\\

0\ar[r]  & (\Ker(\phi\otimes 1))_p \ar[u, phantom, "\nvisom"] \ar[r]& N'\otimes_R M_p \ar[u,phantom, "\nvisom"]\ar[r]  & N\otimes_R M_p \ar[u, phantom, "\nvisom"]~\text{is exact}\\

0\ar[r]  & (\Ker(\phi\otimes 1))_p \ar[u, phantom, "\nvisom"] \ar[r]& N'\otimes_R (R_q\otimes_{R_q} M_p) \ar[u,phantom, "\nvisom"]\ar[r]  & N\otimes_R(R_q\otimes_{R_q} M_p) \ar[u, phantom, "\nvisom"]~\text{is exact}\\

0\ar[r]  & (\Ker(\phi\otimes 1))_p \ar[u, phantom, "\nvisom"] \ar[r]& (N'\otimes_R R_q)\otimes_{R_q} M_p \ar[u,phantom, "\nvisom"]\ar[r]  & (N\otimes_R R_q)\otimes_{R_q} M_p \ar[u, phantom, "\nvisom"]~\text{is exact}\\

0\ar[r]  & (\Ker(\phi\otimes 1))_p \ar[u, phantom, "\nvisom"] \ar[r]& N'_q\otimes_{R_q} M_p \ar[u,phantom, "\nvisom"]\ar[r]  & N_q\otimes_{R_q} M_p \ar[u, phantom, "\nvisom"]~\text{is exact}\\
\end{tikzcd}
 \end{center}

Again we have the exact sequence $0\to N_q\to N_q$, since $R_q$ is flat over $R$. As $M_p$ is flat over $R_q$, the following sequence $$0\to N'_q\otimes_{R_q} M_p\to N_q\otimes_{R_q} M_p$$ 
is exact. Therefore, $(\Ker(\phi\otimes 1))_p=0$ for all $p\in \spec S$. By Local-global property, $\Ker(\phi\otimes 1)=0$. So the sequence $0\to N'\otimes_R M\to N\otimes_R M$ is exact. \qed

\begin{lemma}
Let $M$ be an $R-$module. For $p\in \mspec R,$ we have the map $\theta_p:M\to M_p$ given by $m\mapsto \dfrac{m}{1}.$ Let $x\in M$ such that $\theta_p(x)=0$ for all $p\in\mspec R$ then $x=0.$
\end{lemma}

\proof Let $x\neq 0$ then $\operatorname{Ann}_R(x)\neq R$ so there exists $m\in \mspec R$ such that $\operatorname{Ann}_R (x)\subseteq m.$ Consider the map $\theta_m:M\to M_m$. Since $\theta_m(x)=0 \Rightarrow \dfrac{x}{1}=\dfrac{0}{1}\Rightarrow u(x\cdot 1-0\cdot 1)=0 \Rightarrow ux=0 \Rightarrow u\in \operatorname{Ann}_R(x)$ which is a contradiction. Hence $x=0.$ \qed

\begin{theorem}[Local-global property]

Let $M$ be an $R-$module. Then the followings are equivalent:

\begin{enumerate}

\item $M=0.$

\item $M_p=0$ for all $p\in\spec R.$

\item $M_m=0$ for all $m\in\mspec R.$

\end{enumerate}

\end{theorem}

\proof $(3)\Rightarrow (1)$ \qed


\begin{lemma}

Let $N\subseteq M$ be an $R-$module and $P$ be a flat $R-$module. Then $\dfrac{M\otimes_R P}{N\otimes_R P}\cong M/N\otimes_R P.$

\end{lemma}

\proof Consider the exact sequence $0\to N\to M\to M/N\to 0$. Since $P$ is flat, the resulting sequence $$0\to N\otimes_R P\to M\otimes_R P\to M/N\otimes_R P\to 0$$ is exact. \qed

\begin{corollary}

Let $M,N$ be $R-$modules and $f\in\Hom{M,N}.$ Then the followings are equivalent. 

\begin{enumerate}

\item $f$ is injective (surjective).

\item $f_p$ is injective (surjective) for all $p\in\spec R.$

\item $f_m$ is injective (surjective) for all $m\in\mspec R.$

\end{enumerate}

\end{corollary}

\proof 

\qed

\subsection{Projective module}

\begin{theorem}

Let $P$ be an $R-$module. Then the followings are equivalent:

\begin{enumerate}

\item $\Hom{P,-}$ is an exact functor that is given any exact sequence of $R-$modules, $$0\to M'\to M\to M''\to 0$$ the sequence \begin{equation}
0\to \Hom{P,M'}\to \Hom{P,M}\to \Hom{P,M''}\to 0
\end{equation} is exact.

\item  Given \begin{center}

\begin{tikzcd}
            & P \arrow[d,"\psi"]   &   \\
M \arrow[r,"g"] & M'' \arrow[r] & 0
\end{tikzcd}

\end{center}
we have $\phi:P\to M$ such that the diagram commutes that is $g\circ \phi=\psi.$
 \begin{center}
 \begin{tikzcd}
                 & P \arrow[d, "\psi"] \arrow[ld, "\phi"'] &   \\
M \arrow[r, "g"] & M'' \arrow[r]                           & 0
\end{tikzcd}
 \end{center}

\item There exist an $R-$module $Q$ such that $P\oplus Q$ is free.

\item For any epimorphism $f:M\to P$, there exists $s:P\to M$ such that $f\circ s=\id_P.$  
 
\end{enumerate}

\end{theorem}

\proof $(1)\Rightarrow (2)$ Since (16) is exact $g_*(\alpha)=\beta \Rightarrow g\circ \alpha=\beta$. Take $\alpha=\phi$ and $\beta=\psi.$

$(2)\Rightarrow (1)$ We just need to show that $g_*$ is surjective. Let $\gamma\in \Hom{P,M''}$. By (2) there exists $\phi\in \Hom{P,M}$ such that $g\circ \phi=\gamma\Rightarrow g_*(\phi)=\gamma.$

$(2)\Rightarrow (3)$ Given $P$, there exists a free module $F$ and a surjective map $f:F\to P.$

\begin{center}

\begin{tikzcd}
            &                  &                  & P \arrow[d, "\id"] \arrow[ld, "g"'] &   \\
0 \arrow[r] & \Ker f \arrow[r] & F \arrow[r, "f"] & P \arrow[r]                         & 0
\end{tikzcd}

\end{center}

Since $f\circ g=\id_P$ the above sequence is split exact. Hence $F=P\oplus \Ker f$. So $Q=\Ker f$ is the desired module.

$(3) \Rightarrow (2)$ Consider the diagram 

\begin{center}

\begin{tikzcd}
                 & F \arrow[d, "\pi"] \arrow[ldd, "\tilde{\alpha}"'] &   \\
                 & P \arrow[d, "\psi"]                               &   \\
M \arrow[r, "g"] & M'' \arrow[r]                                     & 0
\end{tikzcd}

\end{center}

Let $S\subseteq F$ be a basis, define $\alpha:S\to M$ given by $\alpha(x)=\tau_x$ where $\tau_x\in g^{-1}(\psi\circ (x))$ is a fixed element. Then there exists $\tilde{\alpha}:F\to M$ such that $\tilde{\alpha}\circ g=\psi\circ \pi.$ Then $\tilde{\alpha}|_P:P\to M$ is the required map.

$(2)\Rightarrow (4)$ Obvious.

$(4)\Rightarrow (3)$ Given $P$, there exists a free module $F$ and $f:F\to P$ is a surjection. Then there is also a map $s:P\to F$ such that $f\circ s=\id_P.$ Since the following sequence $$0\to\Ker f\to F\to P\to 0$$ is split exact, $F\cong P\oplus \Ker f.$ \qed

\begin{defn}
Any $R-$module $P$ which satisfies any one of the above condition is called projective module.
\end{defn}

\begin{remark}
Any free module $F$ is projective since $F=F\oplus 0.$ But converse is not true. Let $R={\ZZ}/6{\ZZ}$ and $P={\ZZ}/3{\ZZ}.$ Note that $P$ is an $R-$module, take $Q={\ZZ}/2{\ZZ}$. Then $P\oplus Q=R$ hence $P$ is a projective module over $R$ but $P$ is not free. If $P$ is free $R$ module then ${\ZZ}/3{\ZZ}\cong ({\ZZ}/6{\ZZ})^{|S|}$ where $S$ is a basis of $P$. Therefore $3=|{\ZZ}/3{\ZZ}|=|S||{\ZZ}/6{\ZZ}|=6|S|$ which is impossible.
\end{remark}

\begin{note}
Therefore we have the following implication 
\begin{center}
\begin{tikzcd}
\text{Free} \arrow[r, Rightarrow] & \text{Projective} \arrow[r, Rightarrow] & \text{Flat}
\end{tikzcd}
\end{center}
but the reverse implications are not true. Let $F$ be a free module, then $F\cong \displaystyle\bigoplus_{i\in\Lambda} R_i$ where $R_i=R$ for all $i\in\Lambda$ and \begin{equation}
0\to M'\to M\to M''\to 0
\end{equation}
be an exact sequence of $R-$modules. Then we have $$0\to M'\otimes_R R_i\to M\otimes_R R_i\to M''\otimes_R R_i\to 0$$ is an exact sequence of $R-$modules for all $i\in\Lambda.$ Hence $$0\to \bigoplus_{i\in\Lambda} (M'\otimes_R R_i)\to \bigoplus_{i\in\Lambda} (M\otimes_R R_i)\to \bigoplus_{i\in\Lambda} (M''\otimes_R R_i)\to 0$$ is exact. Therefore $$0\to M'\otimes_R F\to M\otimes_R F\to M''\otimes_R F\to 0$$ is exact that is $F$ is a flat module. Now let $P$ be a projective module then there exist an $R-$module $Q$ such that $P\oplus Q$ is free. By previous result we have $$0\to (M'\otimes_R P)\oplus (M'\otimes_R Q)\to (M\otimes_R P)\oplus (M\otimes_R Q)\to (M''\otimes_R P)\oplus (M''\otimes_R Q)\to 0$$ is exact. Therefore $$0\to M'\otimes_R P\to M\otimes_R P\to M''\otimes_R P\to 0$$ is exact and $P$ is flat. Note that ${\QQ}$ is flat ${\ZZ}$ module since ${\QQ}=S^{-1}{\ZZ}$ where $S={\ZZ}\setminus\{0\}$ but ${\QQ}$ is not projective. Suppose ${\QQ}$ is projective ${\ZZ}-$module then ${\QQ}$ is a free ${\ZZ}-$module which is a contradiction.

\end{note}

\begin{defn}

Let $R-$ be a ring.n A projective module is said to be stably free if there exists a free module $Q$ such that $P\oplus Q$ is free.

\end{defn}

\begin{eg}

\begin{enumerate}

\item Any free module.

\end{enumerate}

\end{eg}

\begin{qns}

Give an example of a module $M$ and a free module $F$ such that $F\oplus M\cong M.$

\end{qns}

\textit{Ans.} Let $F=R^n$, $M=\displaystyle\bigoplus_{i\in{\NN}} R_i$ where $R_i=F^n$ for all $i\in{\NN}.$

\begin{theorem}

Let $(R,m)$ be a local ring. Then any finitely generated projective $R-$module $P$ is free over $R.$

\end{theorem}

\proof Let $S\subseteq P$ be a minimal generating set. Let $S=\{x_1,\cdots,x_n\}$ then $\overline{S}=\{x_1+mP,\cdots,x_n+mP\}$ is the basis of $P/mP$ over $R/m.$ Since $P=\gen{S}$ there exists a surjective map $\phi:R^n\to P$. Consider the exact sequence \begin{equation}
0\to \Ker \phi\xrightarrow{i} R^n\xrightarrow{\phi}P\to 0.
\end{equation} Then we have 

\begin{center}

\begin{tikzcd}
 & \Ker\phi\otimes_R R/m\ar[r,"\tilde{i}"] & R^n\otimes_R R/m \ar[r,"\tilde{\phi}"] & P\otimes_R R/m  \ar[r] &0 \\
 & \dfrac{\Ker\phi}{m\Ker\phi}\ar[u,phantom, "\nvisom"]\ar[r,"\tilde{i}"] & (R/m)^n \ar[u,phantom, "\nvisom"]\ar[r,"\tilde{\phi}"] &P/mP \ar[u, phantom, "\nvisom"]\ar[r] &0
\end{tikzcd}

\end{center}

Since $\dim (R/m)^n=n=\dim P/mP$, $\tilde{\phi}$ is an isomorphism $\dfrac{\Ker \phi}{m\Ker \phi}=0.$ Since $P$ is projective (17) is split exact. Therefore $R^n\cong \Ker \phi\oplus P$ and hence $\Ker \phi$ is finitely generated. By NAK, $\Ker\phi=0$. Hence $P$ is free. \qed

\begin{prop}

Let $R$ be a commutative ring with 1 and $\phi:R^k\to R^n$ be an endomorphism. Then $n\leq k.$

\end{prop}

\proof Let $m\in\mspec R.$ Consider the exact sequence \begin{equation}
0\to \Ker\phi\xrightarrow{i} R^k\xrightarrow{\phi} R^n\to 0.
\end{equation}
of $R-$ modules. We have 

\begin{center}

\begin{tikzcd}
& \Ker\phi\otimes_R R/m \ar[r,"\tilde{i}"] & R^k\otimes_R R/m \ar[r,"\tilde{\phi}"] & R^n\otimes_R R/m \ar[r] & 0 \\
&\Ker\phi\otimes_R R/m \ar[u, phantom, "\nvisom"] \ar[r,"\tilde{i}"]& (R/m)^k \ar[u,phantom, "\nvisom"] \ar[r,"\tilde{\phi}"] & (R/m)^n \ar[u,phantom, "\nvisom"]\ar[r] &0
\end{tikzcd}

\end{center}

Since $(R/m)^k$ is vector space over $R/m$ and the map $\tilde{\phi}$ is onto, by Rank-Nullity theorem $n\leq k.$ \qed

\begin{theorem}

Let $R$ be a commutative ring with 1 such that $R^m\cong R^n$ then $m=n.$

\end{theorem}

\proof Let $\psi:R^m\to R^n$ be the isomorphism then there exists $\phi:R^n\to R^m$ such that $\phi\circ \psi=\id_{R^m}$ and $\psi\circ \phi=\id_{R^n}.$ Since $\psi$ is onto, $n\leq m$ and $\phi$ is onto implies $m\leq n.$ Hence $m=n.$ \qed \\

For a commutative ring $R$ with 1, we define $\rk{R^n}=n$. For a finitely generated free module $F$, there exists $n\in {\RR}$ such that $F\cong R^n$. So we define $\rk{F}=n.$ Let $P$ be a finitely generated projective module over $R.$ Define $\text{rank}:\spec R\to P$ given by $p\mapsto \rk{(P_p)}.$ 
 
\begin{note}

Let $P$ be a projective module, then there exists $Q$ such that $P\oplus Q\cong F$ where $F$ is a free module. Let $p\in \spec R.$ then $(P\oplus Q)\otimes_R R_p\cong F\otimes_R R_p \Rightarrow P_p\otimes_R Q_p\cong F_p.$ Since $P_p$ is a finitely generated over a local ring in $R_p$, and $F_p$ is free $R_p$ module, therefore $P_p$ is projective $R_p$ module and hence $P_p$ is free over $R_p$. So $\rk{(P_p)}$ is well defined. Note that if $R$ is local then the rank function is constant.

\end{note}

\begin{theorem}

Let $R$ be a semi local ring and $P$ be a finitely generated projective module over $R$ of constant rank then $P$ is free.

\end{theorem}

\proof Let $\mspec R=\{m_1,\cdots, m_r\}$ and $J=\displaystyle\bigcap_{i=1}^r m_i$ be the Jacobson radical. By Chinese Remainder theorem $P/JP\cong P/m_1P\times\cdots\times P/m_rP$ and $R/J\cong R/m_1\times\cdots\times R/m_r$ and $P/JP$ is $R/J$ module. Let $S=\{s_1,\cdots,s_k\}$ be a minimal generating set of $P$ over $R$. We claim that $\overline{S}=\{s_1+JP,\cdots,s_k+JP\}$ be the minimal generating set of $P/JP$ over $R/J.$ If not, we assume that $P/JP$ is generated by $\{s_1+JP,\cdots,s_{k-1}+JP\}$. Let $N=\gen{s_1,\cdots,s_{k-1}}.$  Pick $x\in P$ then $x+JP=\displaystyle\sum_{i=1}^{k-1} (r_i+J)(s_i+JP) \Rightarrow x-\displaystyle\sum_{i=1}^{k-1} r_is_i\in JP \Rightarrow x\in N+JP \Rightarrow P=N+JP.$ By NAK, $P=N$ which is a contradiction. So our claim is proved. Thus $P/JP$ is free $R/J$ module. Now we consider the exact sequence \begin{equation}
0\to \Ker f\to R^k\to P\to 0.
\end{equation}

Since $P$ is projective, this above sequence is split exact and therefore $\Ker f$ is finitely generated. From (19) \begin{center}

\begin{tikzcd}
& \Ker f\otimes_R R/J \ar[r,"i\otimes 1"]  & R^k\otimes_R R/J \ar[r,"f\otimes 1"] & P\otimes_R R/J \ar[r] & 0\\
&\dfrac{\Ker f}{J\Ker f} \ar[u, phantom, "\nvisom"]\ar[r,"i\otimes 1"] & (R/J)^k \ar[u, phantom, "\nvisom"]\ar[r,"f\otimes 1"] & P/JP\ar[u, phantom, "\nvisom"] \ar[r] &0
\end{tikzcd}

\end{center}

We claim that $\{s_1+JP,\cdots,s_k+JP\}$ is a $R/J$ basis of $P/JP$. If we prove the claim then $f\otimes 1$ is an isomorphism and $\Ker f/J\Ker f=0 \Rightarrow \Ker f=0$ by NAK and $P\cong R^k$ hence $P$ is free. 

\textit{\textbf{Proof of the claim.}}












\begin{note}

Let $F_i$ be free $R_i$ module of same rank for all $1\leq i\leq k$, then $F=F_1\times\cdots\times F_k$ is free $R_1\times\cdots\times R_k$ module. That is $F_i\cong (R_i)^l$ for some $l\in{\NN},1\leq i\leq n.$ Then $F=F_1\times\cdots\times F_k\cong (R_1)^l\times\cdots\times (R_k)^l\cong (R_1\times\cdots\times R_k)^l.$ We will prove this by induction on $k.$ Let $\theta:R_1^l\times R_2^l\to (R_1\times R_2)^l$ defined by $((x_1,\cdots,x_l),(x_1',\cdots,x_l'))\mapsto ((x_1,x_1'),\cdots,(x_l,x_l'))$ be the required isomorphism.

\end{note}

\begin{note}

Since $P$ is projective of constant rank, let $P_m\cong (R_m)^l$ for all $m\in mspec R$ and for some $l\in{\NN}.$ Let $P/mP\cong (R/m)^s$ for some $s\in {\NN}.$ Then $P/mP\otimes_R R_m\cong (R/m)^s\otimes_R R_m \Rightarrow \dfrac{P_m}{mP_m}\cong \left(\dfrac{R_m}{mR_m}\right)^s\cong \left(\dfrac{R_m}{mR_m}\right)^l\Rightarrow l=s.$ Hence for any $m\in\mspec R$, $P/mP\cong (R/m)^l.$ Therefore $P/JP\cong \displaystyle\prod_{i=1}^r P/m_iP\cong \displaystyle\prod_{i=1}^r (R/m_i)^l\cong \left(\displaystyle\prod_{i=1}^r R/m_i\right)^l\cong (R/J)^l.$

\end{note}

\begin{qns}

Let $R$ be a semi local ring and $F$ be a finitely generated free module over $R$. Is any minimal generating set of $F$ an $R-$basis of $F?$.

\end{qns}


\begin{defn}

Let $M$ be an $R-$module. $M$ is said to be finitely presented if there exists finitely generated free modules $F_1$ and $F_2$ such that the following sequence is exact $$F_1\to F_2\to M\to 0.$$

\end{defn}


\begin{note}
Suppose $M$ is a finitely generated module over $R$. If $\Ker f$ is finitely generated then we have the following sequence

\begin{center}
\begin{tikzcd}
R^k \arrow[rr, "i\circ \phi"] \arrow[rd, "\phi"] &                                   & R^n \arrow[r, "f"] & M \arrow[r] & 0 \\
                                                 & \Ker f \arrow[ru, "i"] \arrow[rd] &                    &             &   \\
                                                 &                                   & 0                  &             &  
\end{tikzcd}
\end{center}

 is exact because $\Ker \phi=\im\phi=\im (i\circ \phi).$ Thus a finitely generated module may not be finitely presented. If $R$ is Noetherian then it is true. Conversely any finitely presented module is finitely generated.

\end{note}

\begin{theorem}

Let $R$ be a ring and $M, N$ be $R-$modules and $S$ be a flat $R-$algebra. Suppose $M$ is of finite presentation then we have $$\Hom{M,N}\otimes_R S\cong \Homs{M\otimes_R S, N\otimes_R S}.$$

\end{theorem}

\proof Since $M$ is of finite presentation, there exists two finitely generated free module $R^p$ and $R^q$ such that \begin{equation}
R^p\to R^q\to M\to 0 
\end{equation}
is exact. Then for any $R-$module $N$ the following sequence \begin{equation}
0\to \Hom{M,N}\to \Hom{R^q,N}\to \Hom{R^p,N}
\end{equation}
is exact. As $S$ is flat, $$0\to \Hom{M,N}\otimes_R S\to \Hom{R^q,N}\otimes_R S\to \Hom{R^p,N}\otimes_R S$$ is exact. Now consider the diagram 

\begin{center}

\begin{tikzcd}
0 \arrow[r] & {\Hom{M,N}\otimes_R S} \arrow[d, "\lambda_M"] \arrow[r] & {\Hom{R^q,N}\otimes_R S} \arrow[d, "\lambda_{R^q}"] \arrow[r] & {\Hom{R^p,N}\otimes_R S} \arrow[d, "\lambda_{R^p}"] \\
0 \arrow[r] & {\Homs{M\otimes_R S,N\otimes_R S}} \arrow[r]            & {\Homs{R^q\otimes_R S,N\otimes_R S}} \arrow[r]                & {\Homs{R^p\otimes_R S,N\otimes_R S}}               
\end{tikzcd}

\end{center}

where $\lambda_M: \Hom{M,N}\otimes_R S\to \Homs{M\otimes_R S,N\otimes_R S}$ is defined by $\lambda_M(f\otimes s)=\tilde{f}$ and $\tilde{f}:M\otimes_R S\to N\otimes_R S$ is defined by $\tilde{f}(m\otimes s)=f(m)\otimes s.$ By Universal property $\tilde{f}$ is well defined. Since $\Hom{R^q,N}\otimes_R S\cong (\Hom{R,N})^q\otimes S\cong N^q\otimes S=(N\otimes_R S)^q$ and $\Homs{R^q\otimes_R S,N\otimes_R S}\cong\Homs{S^q,N\otimes_R S}\cong (N\otimes_R S)^q.$ Thus the mappings $\lambda_{R^q}$ and $\lambda_{R^p}$ are isomorphism. Since the bottom sequence of the above diagram is exact and the diagram ia commutative, $\lambda_M$ is also an isomorphism. \qed

\begin{corollary}

Let $M$ and $N$ be $R-$modules with $M$ be of finite presentation. Then for each $p\in\spec R$, $$(\Hom{M,N}_p\cong \text{Hom}_{R^p}(M_p,N_p).$$

\end{corollary}

\proof Take $S=R_p.$ \qed


\begin{theorem}

Let $R$ be any ring and $M$ be a finitely presented. Then the followings are equivalent:
\begin{enumerate}

\item The map $\theta:M\otimes_R M^*\to R$ defined by $\theta(m,f)=f(m)$ is an isomorphism.

\item There exists an $R-$module $N$ such that $M\otimes_R N\cong R.$

\item $M_m\cong R_m$ for all $m\in\mspec R.$

\item $M_p\cong R_p$ for all $p\in \spec R.$

\item $M$ is projective of rank 1.

\end{enumerate}

\end{theorem}

\proof $(1)\Rightarrow (2)$ Take $N=M^*.$\\
$(2)\Rightarrow (3)$ $M\otimes_R N \cong R\Rightarrow M_m\otimes_R N_m \cong R_m \Rightarrow M_m\otimes_{R_m} N_m\cong R_m \Rightarrow (M_m\otimes_{R_m} N_m)\otimes_{R_m} \dfrac{R_m}{mR_m}\cong \dfrac{R_m}{mR_m}\Rightarrow M_m\otimes_{R_m} \dfrac{N_m}{mN_m}\cong \dfrac{R_m}{mR_m}\Rightarrow \dfrac{M_m}{mR_m}\otimes_{R_m} \dfrac{N_m}{mN_m}\cong \dfrac{R_m}{mR_m}\footnote{As $K(m):=\dfrac{R_m}{mR_m}$ and $K(m)^l\otimes K(m)^s\cong K(m)^{ls}.$}.$ Therefore, $\dfrac{M_m}{mM_m}\cong \dfrac{R_m}{mR_m}.$ By NAK $M_m=\gen{x},x\in M_m \Rightarrow M_m\cong \dfrac{R_m}{Ann_{R_m}(x)} \Rightarrow Ann_{R_m}(x)(M_m\otimes_{R_m} N_m)=0 \Rightarrow Ann_{R_m}(x) R_m=0 \Rightarrow Ann_{R_m}(x)=0 \Rightarrow M_m=R_m.$\\

$(3)\Rightarrow (4)$ Further localization.\\

$(4)\Rightarrow (5)$ By definition.\\

$(5)\Rightarrow (1)$ Since $M$ is of finite presentation, $(\Hom{M,R})_m\cong \text{Hom}_{R_m}(M_m,R_m)$ for all $m\in\mspec R,$ that is $(M^*)_m\cong (M_m)^*.$ Now $M$ is projective of rank 1 so $M_m\cong R_m.$ So we have $M_m\otimes_{R_m} (M_m)^*\cong R_m\otimes_{R_m} (R_m)^*\cong R_m.$ Again from the above equation, \begin{align*}
M_m\otimes_{R_m} (M_m)^* &\cong M_m\otimes_{R_m} (M^*)_m\\
&\cong M_m \otimes_R (M^*)_m\\
&\cong M_m\otimes_R (M^*\otimes_R R_m)\\
&\cong (M\otimes_R R_m)\otimes_R (M^*\otimes_R R_m)\\
&\cong (M\otimes_R M^*)\otimes_R (R_m\otimes_R R_m)\\
&\cong (M\otimes_R M^*)\otimes_R R_m\\
&\cong (M\otimes_R M^*)_m
\end{align*}
  
Hence $(M\otimes_R M^*)_m\cong R_m$ for all $m\in \mspec R.$ By Local-global property $M\otimes_R M^*\cong R.$

\begin{note}

Let $I$ and $J$ be two ideals of $R$ then  $R/I\otimes_R R/J\cong \dfrac{R/I}{J(R/I)}\cong \dfrac{R/I}{(J+I)/I}\cong \dfrac{R}{I+J}.$ (Check this isomorphism as ring.)

\end{note}


\textbf{Picard group.} Let $\sum$ be the isomorphism classes of projective $R-$modules of rank 1. Define \begin{align*}
\cdot:\sum\times\sum&\to \sum\\
([P],[Q])&\mapsto [P\otimes_R Q]
\end{align*}

We need to show that $\left(\sum,\cdot\right)$ is a group with inverse of $[P]$ is $[P^*].$ This group is called Picard group of $R$ and it is denoted by $Pic~R.$ Let $P,Q$ be two projective module of rank 1 then $$(P\otimes_R Q)\otimes_R R_m\cong P_m\otimes_R Q_m\cong P_m\otimes_{R_m} Q_m\cong R_m\otimes_{R_m} R_m\cong R_m.$$ Thus $P\otimes_R Q$ is also a projective module of rank 1. By Corollary 14.88 $(M^*)_p\cong (M_p)^*\cong (R_p)^*\cong R_p$ for all $p\in\spec R.$ Therefore $M$ is projective of rank 1 implies $M^*$ is also projective of rank 1.\\

\textbf{Free, Projective and Flat resolution.}

\begin{defn}

Let $M$ be an $R-$module. A free (or projective or flat) resolution of $M$ over $R$ is an exact sequence of $R-$modules $$\cdots\to P_2\xrightarrow{f_2} P_1\xrightarrow{f_1} P_0\xrightarrow{f_0} M\to 0$$
where each $P_i$ is a free (or projective or flat respt.) $R-$module.
\end{defn}

\begin{lemma}

Let $M$ be an $R-$module. Then projective resolution of $M$ over $R$ exists.

\end{lemma}

\proof For any module $M$, there exists a free module $F$ and a surjective map $F_0\xrightarrow{f_0} M\to 0.$ Consider the $\Ker f_0,$ then there exists a free module $F_1$ with the diagram 

\begin{center}

\begin{tikzcd}
F_1 \arrow[rr, "f_1=i\circ \pi"] \arrow[rd, "\pi_1"] &                                           & F_0 \arrow[r, "f_0"] & M \ar[r] &0 \\
                                                     & \Ker f_0 \arrow[rd] \arrow[ru, "i", hook] &                      &   \\
                                                     &                                           & 0                    &  
\end{tikzcd}

\end{center}

The above diagram is exact since $\Ker f_0=\im \pi_1=\im i\circ \pi_1=\im f_1$ since $i$ is the inclusion map and $\pi_1$ is onto. Next we consider $\Ker f_1$, then there exists $F_2$ such that 

\begin{center}

\begin{tikzcd}
\cdots \arrow[r] & F_2 \arrow[rr, "f_2"] \arrow[rd, "\pi_2"] &                                           & F_1 \arrow[rr, "f_1=i\circ \pi"] \arrow[rd, "\pi_1"] &                                           & F_0 \arrow[r, "f_0"] & M \ar[r] &0\\
                 &                                           & \Ker f_1 \arrow[rd] \arrow[ru, "i", hook] &                                                      & \Ker f_0 \arrow[rd] \arrow[ru, "i", hook] &                      &   \\
                 &                                           &                                           & 0                                                    &                                           & 0                    &  
\end{tikzcd}

\end{center}

Inductively we can construct a free resolution of $M$. Since every free module is projective and therefore flat, we have a projective (or flat) resolution. \qed

\textbf{Tor and Ext.}

\begin{defn}

Let $M$ be an $R-$module. We consider a projective resolution of $M$ that is $$\mathcal{C}\equiv \cdots\to P_2\xrightarrow{f'_2} P_1\xrightarrow{f'_1} P_0\xrightarrow{f'_0} M\to 0.$$ Let $N$ be another $R-$module. We consider, \begin{enumerate}

\item $$\mathcal{C}\otimes_R N\equiv \cdots\to P_2\otimes_R N\xrightarrow{f_2} P_1\otimes_R N\xrightarrow{f_1} P_0\otimes_R N\xrightarrow{f_0} M\otimes_R N\to 0$$ where $f_i=f'_i\otimes 1$ for all $i\in{\NN}.$ Then we define $\text{Tor}_i^R (M,N):=H_i(\mathcal{C}\otimes_R N)=\dfrac{\Ker f_i}{\im f_{i+1}}.$

\item $$\Hom{\mathcal{C},N}\equiv \cdots\xleftarrow{f^*_2}\Hom{P_1,N}\xleftarrow{f^*_1}\Hom{P_0,N}\xleftarrow{f^*_0}\Hom{M,N}\leftarrow 0.$$

we define $\text{Ext}^{~i}_R(M,N):=H^i(\Hom{\mathcal{C,N}})=\dfrac{\Ker f^*_{i+1}}{\im f^*_i}.$

\end{enumerate}

\end{defn}


\begin{remark}

These definition doesn't depend on the choice of resolution of $M.$

\end{remark}










































































































































































































































































































































































































































%\begin{tikzcd}
 %0\ar[r] & A \ar[r] & B \ar[r] & C \ar[r] & 0 \\
  %\mathbb{Z}_{?} \ar[u, phantom, "\visom"] \ar[r]& \mathbb{Z}_{12} \ar[u,phantom, "\nvisom"] \ar[r, phantom, "\isom"] & \mathbb{Z}_{24} \ar[u, phantom, "\nvisom"] \\
%\end{tikzcd}

\newpage
\subsection{Appendix A}
\begin{defn}
Let $\{M_i\}_{i\in I}$ be a family of $R-$module. A product of this family is a pair $(P,\{f\}_{i\in I})$ where $P$ is an $R-$module and $f_i:P\to M_i$ is a morphism for all $i$ such that if we have another pair $(M,\{g_i\}_{i\in I})$ where $M$ is an $R-$module and $g_i:M\to M_i$ is a morphism for all $i\in I$ then there exists unique module homomorphism $h:M\to P$ such that the following diagram commutes i.e., $f_i\circ h=g_i,\forall~i\in I.$
\begin{center}
\begin{tikzcd}
P \arrow[rr, "f_i"]                    &  & M_i \\
                                       &  &     \\
M \arrow[uu, "h"] \arrow[rruu, "g_i"'] &  &    
\end{tikzcd}
\end{center}
\end{defn}
\textbf{Existence of product.} We have $\{M_i\}_{i\in I}$ and $P:=\displaystyle\prod_{i\in I} M_i=\{(m_i)_{i\in I}:m_i\in M_i\}$ and define addition and multiplication component wise i.e., \begin{align*}
(m_i)_{i\in I}+(n_i)_{i\in I}&=(m_i+n_i)_{i\in I}\qquad&[(m_i)_{i\in I},(n_i)_{i\in I}\in \displaystyle\prod_{i\in I} M_i ]\\
r(m_i)_{i\in I}&= (rm_i)_{i\in I}\qquad&[(m_i)_{i\in I}\in \displaystyle\prod_{i\in I} M_i,r\in R]
\end{align*}
then $P$ satisfies all module axiom hence $P$ is an $R-$module. Define, \begin{align*}
P&\stackrel{f_i}{\longrightarrow}M_i\\
(m_i)_{i\in I}&\mapsto m_i
\end{align*}
then its is easy to verify that $f_i$ is a surjective module homomorphism. Now let $(M,\{g_i\}_{i\in I})$ be a pair where $M$ is an $R-$module and $g_i:M\to M_i$ is a morphism for all $i$.
\begin{center}
\begin{tikzcd}
\left(\displaystyle\prod_{i\in I} M_i=\right)P \arrow[rr, "f_i"] &  & M_i \\
                                                                 &  &     \\
M \arrow[uu, "h"] \arrow[rruu, "g_i"']                           &  &    
\end{tikzcd}
\end{center}
Now, we need to define the map $h:M\to P$ in such a way that the above diagram commutes so commutitivity of the diagram forces us to define \begin{align*}
h:M&\to P\\
m&\mapsto (g_i(m))_{i\in I}
\end{align*}
i.e., $h(m)$ is an element of $\displaystyle\prod_{i\in I} M_i$ whose $i^{th}$ co-ordinate is $g_i(m).$ Hence, \begin{align*}
(f_i\circ h)(m)=f_i(h(m))=f_i((g_i(m))_{i\in I})=g_i(m),\forall~i\in I
\end{align*}
Therefore, $f_i\circ h=g_i,\forall~i\in I.$ \\
Let $h_1: M\to P$ be another morphism such that $f_i\circ h_1=g_i,\forall~i\in I.$ Then $i^{th}$ component of $h(m)$ is $f_i(h(m))=(f_i\circ h)(m)=g_i(m)=(f_1\circ h_1)(m)=f_i(h_1(m))=h_1(m),\forall~m\in M$ hence $h$ is unique.\\\\
\textbf{Uniqueness of product.} \begin{theorem}
Let $(P',\{f_i'\}_{i\in I})$ be another product of $\{M_i\}_{i\in I}$ then there exists an unique isomorphism $h':P'\to P$ such that the following diagram commutes for all $i.$ \begin{center}
\begin{tikzcd}
P \arrow[rr, "f_i"]                      &  & M_i \\
                                         &  &     \\
P' \arrow[uu, "h"] \arrow[rruu, "f_i'"'] &  &    
\end{tikzcd}
\end{center}
\end{theorem}
\proof As $(P,\{f_i\}_{i\in I})$ is a product by definition $\exists! ~h:P\to P$ such that the diagram commutes,\begin{center}
\begin{tikzcd}
P \arrow[rr, "f_i"]                      &  & M_i \\
                                         &  &     \\
P' \arrow[uu, "h"] \arrow[rruu, "f_i'"'] &  &    
\end{tikzcd}
\end{center}
Similarly as $(P',\{f_i'\}_{i\in I})$ is a product then $\exists!~h':P\to P'$ such that \begin{center}
\begin{tikzcd}
P' \arrow[rr, "f_i'"]                   &  & M_i \\
                                        &  &     \\
P \arrow[uu, "h'"] \arrow[rruu, "f_i"'] &  &    
\end{tikzcd}
\end{center}
the diagram commutes. Then $f_i'=f_i\circ h$ and $f_i=f_i'\circ h,\forall~i\in I \Rightarrow f_i=f_i\circ (h\circ h'),\forall~i\in I$ but \begin{center}
\begin{tikzcd}
P \arrow[rr, "f_i"]                       &  & M_i &  &  & P \arrow[rr, "f_i"]                            &  & M_i \\
                                          &  &     &  &  &                                                &  &     \\
P \arrow[uu, "id_P"] \arrow[rruu, "f_i"'] &  &     &  &  & P \arrow[uu, "h\circ h'"] \arrow[rruu, "f_i"'] &  &    
\end{tikzcd}
\end{center}
these two diagram also commutes so by the definition of product $id_P=h\circ h'.$ Similarly, $f_i'=f_i\circ h=f_i'\circ (h'\circ h)$ then \begin{center}
\begin{tikzcd}
P' \arrow[rr, "f_i'"]                            &  & M_i \\
                                                 &  &     \\
P' \arrow[uu, "h'\circ h"] \arrow[rruu, "f_i'"'] &  &    
\end{tikzcd}
\end{center}
From the above diagram we have $h'\circ h=id_{P'}.$ Therefore, $h$ is an isomorphism. \qed
\begin{ex}
Show that if $(P,\{f_i\}_{i\in I})$ exists then each $f_i$ is surjective.
\end{ex}
\proof Let us take $M=M_i$ and \begin{align*}
&g_i:M_i\to M_i~\text{be the indentity map}\\
&g_j:M_i\to M_j~\text{be the zero map}
\end{align*}
then we have these tuple $(M_j,\{g_j\})$ and for each $i$ we have \begin{center}
\begin{tikzcd}
P \arrow[rr, "f_i"]                     &  & M_i \\
                                        &  &     \\
M_i \arrow[uu, "h"] \arrow[rruu, "id"'] &  &    
\end{tikzcd}
\end{center}
and $f_i\circ h=id \Rightarrow f_i$ is surjective. \qed

\begin{defn}
Let $\{M_i\}_{i\in I}$ be a family of $R-$module. A coproduct of this family of this family is a pair $(C,\{f_i\}_{i\in I})$ where $C$ is an $R-$module and $f_i:M_i\to C$ is a morphism for all $i\in I$ such that if there is another pair $(M,\{g_i\}_{i\in I})$ where $M$ is an $R-$module and $g_i:M_i\to M$ is a morphism $\forall~i\in I$ then there exists a module morphism $h:C\to M$ such that the following diagram commutes for all $i\in I.$
\begin{center}
\begin{tikzcd}
M_i \arrow[dd, "g_i"] \arrow[rr, "f_i"] &  & C \arrow[lldd, "h"] \\
                                        &  &                     \\
M                                       &  &                    
\end{tikzcd}
\end{center}
\end{defn}

\textbf{Existence of coproduct.} Recall that we have this $R-$module $\displaystyle\prod_{i\in I} M_i$ and we consider the following set \begin{align*}
C=\{(m_i)_{i\in I}\in \displaystyle\prod_{i\in I} M_i: m_i=0~\text{for all but finitely many values of}~i\} 
\end{align*}
Then $C$ is a submodule of $P$. As $(m_i)_{i\in I},(n_i)_{i\in I}\in C$ then $m_i=0$ for all but finitely many values of $i$ and $n_j=0$ for all but finitely many values of $i$ then $(m_i)_{i\in I}+(n_i)_{i\in I}\in C$ as $(m_i)+(n_i)=0$ for all but finitely many values of $i$ and if we pick any $r\in R$ then $r(m_i)$ also has all but finitely many terms are zero hence $C$ is a submodule of $P$. Now, we define a map \begin{align*}
f_i:M_i&\to C\\
m_i&\mapsto (m_j)_{j\in I}\quad\text{where}~m_j=m_i~\text{if}~i=j, 0~\text{otherwise}
\end{align*}
Then clearly $f_i$'s are $R-$module homomorphism so $\ker f_i=\{m_i\in M_i:f(m_i)=0\}\Rightarrow \{0\}$ therefore, $f_i$'s are injective. 


\begin{defn}
Let $R$ be a ring and $S$ be a non empty set. A free $R-$module on $S$ is an $R-$module $F$ such that for any $R-$module $M$ and every set map $g:S\to M$ there exists an unique $R-$module homomorphism $h:F\to M$ such that the following diagram commutes.
\begin{center}
\begin{tikzcd}
S \arrow[rr, "f"] \arrow[dd, "g"] &  & F \arrow[lldd, "h"] \\
                                  &  &                     \\
M                                 &  &                    
\end{tikzcd}
\end{center}
\end{defn}

\begin{theorem}
If $(F,f)$ is a free module on $S$. Then \begin{enumerate}
\item $f$ is injective,
\item $\im f$ generates $F.$
\end{enumerate}
\end{theorem}
\proof (1) Take any $R$ module (can take $M=R$) and $x,y\in S$ with $x\neq y.$ Define, $g:S\to M$ such that $g(x)\neq g(y)$. By the property of free module there exists $h:F\to M$ such that $h\circ f=g$ . Then $h(f(x))=g(x)$ and $h(f(y))=g(y)$ thus $f$ is injective.\\
(2) Let $A$ be the submodule of $F$ generated by $\im f$. We have to show $A=F.$ \begin{center}
\begin{tikzcd}
S \arrow[dd, "f"] \arrow[rr, "\tilde{f}"]             &  & \text{Im}~f \arrow[rr, "i", hook] &  & A \arrow[rr, "i_A"] &  & F \\
                                                      &  &                                   &  &                     &  &   \\
F \arrow[rrrruu, "h"] \arrow[rrrrrruu, "i_A\circ h"'] &  &                                   &  &                     &  &  
\end{tikzcd}
\end{center}
We will show that $i_A$ is surjective. We have $h\circ f=i\circ \tilde{f} \Rightarrow i_A\circ (h\circ f)=i_A\circ (i\circ \tilde{f}=f.$ But \begin{center}
\begin{tikzcd}
S \arrow[rr, "f"] \arrow[dd, "f"] &  & F &  &  & S \arrow[rr, "f"] \arrow[dd, "f"] &  & F \\
                                  &  &   &  &  &                                   &  &   \\
F \arrow[rruu, "id_F"]            &  &   &  &  & F \arrow[rruu, "i_A\circ h"]      &  &  
\end{tikzcd}
\end{center}
By the uniqueness of the morphism we have $id_F=i_A\circ h$ hence $i_A$ is surjective.
\begin{theorem}[Uniqueness of free module]
Let $(F,f)$ be a $R-$free module on a set $S(\neq \emptyset)$ then $(F',f')$ is also free module iff there exists an unique isomorphism $j:F\to F'$ such that $h\circ f=f'.$
\end{theorem}
\proof Let $M$ be an $R-$module and $g:S\to M$ is a set map then \begin{center}
\begin{tikzcd}
  &  & F' \arrow[lldd]                                     &  &                                                   \\
  &  &                                                     &  &                                                   \\
M &  & S \arrow[rr, "f"] \arrow[ll, "g"'] \arrow[uu, "f'"] &  & F \arrow[lluu, "h"] \arrow[llll, "k"', bend left]
\end{tikzcd}
\end{center}
%incomplete

\textbf{Existence of free module.} Let \begin{align*}
F:=\{\theta:S\to R:\theta(s)=0~\text{for all most all}~s\in S\}
\end{align*}
then $F$ is an $R-$module. Let $\theta_1,\theta_2\in F$ then define $\theta_1+\theta_2$ as $(\theta_1+\theta_2)(s)=\theta_1(s)+\theta_2(s)$ then clearly $\theta_1+\theta_2\in F.$ Let $r\in R$ and $\theta\in F$ then $r\theta:=(r\theta)(s)=r\theta(s)$ and $r\theta\in F.$ Define, \begin{align*}
f:S&\to F\\
s&\mapsto \chi_s
\end{align*}
where $\chi_s(t)=\begin{cases}
1,\quad\text{if}~t=s\\
0,\quad\text{if}~t\neq s
\end{cases}$. Let $M$ be an $R-$module and $g:S\to M$ be a set map. Define $h:F\to M$ by $h(\theta)=\displaystyle\sum_{s\in S} \theta(s)g(s)$ then clearly $h$ is a module morphism. Let $t\in S$ \begin{align*}
\left(\displaystyle\sum_{s\in S} \theta(s)\chi_s\right)(t)=\theta(t)\chi_t(t)=\theta(t)\cdot 1=\theta(t)
\end{align*}
As $t\in S$ is arbitrary, $\theta=\displaystyle\sum_{s\in S} \theta(s)\chi_s.$ Now, $(h\circ f)(s)=h(f(s))=h(\chi_s)=\displaystyle\sum_{s\in S} \chi_s(t)g(t)=g(s).$ Therefore, $h\circ f=g.$ Let $h':F\to M$ is another module homomorphism such that $h'\circ f=g.$ To show $h'=h$ we proceed as follows \begin{align*}
h'(\theta)&=h'\left(\displaystyle\sum_{s\in S} \theta(s)\chi_s\right)\\
&=\displaystyle\sum_{s\in S} \theta(s)h'(\chi_s)\\
&=\displaystyle\sum_{s\in S} \theta(s)h(f(s))\\
&=\displaystyle\sum_{s\in S} \theta(s)g(s)\\
&=h(\theta)
\end{align*}
Thus $h=h'$. \qed
\begin{theorem}
Let $(F,f)$ be a free module on $S$. Then $\im f $ is a basis of $F.$
\end{theorem}
\proof We already showed $\im f$ generates $F.$ Let $f(x_1),\cdots,f(x_m)$ be distinct elements of $\im f$. Let $\alpha_i,\beta_i\in R$ be such that $$\alpha_1f(x_1)+\cdots+\alpha_nf(x_n)=\beta_1f(x_1)+\cdots+\beta_nf(x_n)\Rightarrow \displaystyle\sum_{i=1}^n (\alpha_i-\beta_i)f(x_i)=0$$ Then $\displaystyle\sum_{i=1}^n (\alpha_i-\beta_i)\chi_{x_i}=0 \Rightarrow \displaystyle\sum_{i=1}^n (\alpha_i-\beta_i)\chi_{x_i}(x_i)=0 \Rightarrow \alpha_i-\beta_i=0 \Rightarrow \alpha_i=\beta_i;1\leq i\leq n.$ \qed\\

 Let $\{A_i\}_{i\in I}$ be a family of submodule of an $R-$module $M$ then $\displaystyle\sum_{i\in I} A_i$ is a submodule of $M$ generated by $\displaystyle\bigcup_{i\in I} A_i$ then check that $$\displaystyle\sum_{i\in I} A_i=\left\lbrace\displaystyle\sum_{i\in I} a_i:a_i\in A_i,a_i=0~\text{for all most all~}i\right\rbrace$$
Let $M$ be an $R-$module and $\{M_i\}_{i\in I}$ be a family of submodules of $M$ then for $j\in I$ let $\theta_j:M_j\to M$ be the inclusion map then we have unique module homomorphism $h:\displaystyle\bigoplus_{i\in I} M_i\to M$ such that \begin{center}
\begin{tikzcd}
M_j \arrow[rr, "\theta_j"] \arrow[dd, "\tau_j"]      &  & M \\
                                                     &  &   \\
\displaystyle\bigoplus_{i\in I}M_i \arrow[rruu, "h"] &  &  
\end{tikzcd}
\end{center}
the above diagram commutes then $$h((x_i)_{i\in I})=\displaystyle\sum_{\substack{\text{finite}\\ \text{sum}}} \theta_i(x_i)=\displaystyle\sum_{\substack{\text{finite}\\ \text{sum}}} x_i$$
And $\im h$ is the submodule $\displaystyle\sum_{i\in I} M_i$ of $M$ generated by $\displaystyle\bigcup_{i\in I} M_i.$
\begin{defn}
If $h$ is isomorphism then $\displaystyle\bigoplus_{i\in I} M_i \cong M$ and $\displaystyle\bigoplus_{i\in I} M_i$ is called internal direct sum of $\{M_i\}_{i\in I}.$
\end{defn}
\begin{qns}
When does it happen?
\end{qns}
Ans. $h$ is an isomorphism iff each $x\in M$ can be written uniquely as $x=\displaystyle\sum_{\substack{\text{finite}\\ \text{sum}}} x_i,x_i\in M.$
\begin{theorem}
Let $M$ be an $R-$module and $\{M_i\}_{i\in I}$ be a family of submodules of $M$. Then the followings are equivalent: \begin{enumerate}
\item $\displaystyle\sum_{i\in I} M_i$ is the direct sum $\displaystyle\bigoplus_{i\in I} M_i,$
\item $\displaystyle\sum_{i\in I} x_i=0 \Rightarrow x_i=0,\forall~i\in I,$ 
\item For all $i\in I$, $M_j\cap \displaystyle\sum_{i\neq j} M_i=\{0\}.$
\end{enumerate}
\end{theorem}







\newpage
%\chapter{Field Theory}
\section{Field theory}
\begin{defn}
Let $K,L$ be fields with $K\subseteq L$ then we say $L$ is an extension field of $K$. Here $K$ is called the base field.
\end{defn}
\textbf{Notation.} $L|K.$\\
Note that if $L$ is an extension field of $K$ then $L$ is a vector space over $K$. Define, \begin{align*}
\cdot:K\times L&\to L\\
(c,\alpha)&\mapsto c\alpha\qquad\qquad\text{[usual ring multiplication in $L$]}
\end{align*}
\begin{defn}
$L$ is said to be a finite extension if $\dim_K L$ is finite.
\end{defn}
\textbf{Notation.} $[L:K]:=\dim_K L.$

\begin{defn}
Let $L|K$, an element $\alpha\in L$ is said to be algebraic over $K$ if it satisfies a polynomial $f(x)\in K[x]$ i.e., $f(\alpha)=0$. An extension $L|K$ is said to be an algebraic extension if any $\alpha\in L$ is algebraic over $K.$
\end{defn}
\textbf{Fact from ring theory.} \begin{enumerate}
\item Let $A$ be a commutative ring with 1, $m\in \mspec A\Leftrightarrow A/m$ is field.
\item If $A$ is pid then $\spec A\setminus \{0\}=\mspec A.$
\item $A[X]$ is pid if and only if $A$ is field.
\end{enumerate}

\begin{obs}
Suppose $L|K$ and $[L:K]<\infty$ then $L|K$ is algebraic.
\end{obs}
\proof Let $\alpha\in L.$ We consider $S=\{1,\alpha,\alpha^2,\cdots\}$. Since $\dim_K L<\infty$ and $S$ is linearly independent there exists $n\in {\NN}$ such that $$\alpha^n+c_1\alpha^{n-1}+\cdots+c_n=0;c_i\in K,1\leq i\leq n$$ that is $f(\alpha)=0$ where $f(X)=X^n+c_1X^{n-1}+\cdots+c_n\in K[X].$ Therefore, $L|K$ is algebraic. \qed
\begin{remark}
Note that converse is not true that is an algebraic extension need not be a finite extension.
\end{remark}
Let $L|K$ be a field extension and $\alpha\in L$. We define $$K[\alpha]:=\{f(\alpha):f(X)\in K[X]\}\subseteq L$$
Then $K[\alpha]$ is the smallest subring containing $L$ and $\alpha$. Since $K[\alpha]\subseteq L$ and $L$ is a field, $K[\alpha]$ is an integral domain. We define, $$K(\alpha):=\left\lbrace\dfrac{f(\alpha)}{g(\alpha)}:f(X),g(X)\in K[X],g(\alpha)\neq 0\right\rbrace$$ Then $K(\alpha)$ is the quotient field of $K[\alpha]$. Let $S\subseteq L$, if $S=\{\alpha_1,\cdots,\alpha_n\};$ a finite set then $$K[S]:=\{f(\alpha_1,\cdots,\alpha_n):f(X_1,\cdots,X_n)\in K[X_1,\cdots,X_n]\}.$$ If $S$ is infinite then\\
Show that $K[S]$ is the smallest subring in $L$ containing $K$ and $S.$ We define $K(S):=Q(K[S])$ (as $K[S]\subseteq L$, it is an integral domain).
\begin{qns}
Show that if $S\{\alpha_1,\cdots,\alpha_n\}$ then $$K(S)=\left\lbrace \dfrac{f(\alpha_1,\cdots,\alpha_n)}{g(\alpha_1,\cdots,\alpha_n)}:f,g\in K[X_1,\cdots,X_n],g(\alpha_1,\cdots,\alpha_n)\neq 0\right\rbrace.$$
\end{qns}
Ans.\\
If $S$ is infinite then\\

\begin{eg}[counterexample of the remark 15.5]
We consider $$S=\{\sqrt{p}:p~\text{is prime}\}\subseteq{\RR}$$ Then ${\QQ}(S)|{\QQ}$ is algebraic as each $\sqrt{p}$ satisfies $X^2-p\in {\QQ}[X]$ but $[{\QQ}(S):{\QQ}]$ is not finite.
\end{eg}
\begin{lemma}
We consider the field extension $L\subseteq K\subseteq F$ then $[L:F]=[L:K][K:F].$
\end{lemma}
\proof

\begin{theorem}
Let $F|K$ is a field extension and $\alpha\in F$. Then $K(\alpha)|K$ is an algebraic extension iff $K[\alpha]=K(\alpha).$
\end{theorem}
\proof Suppose, the extension is algebraic. Then consider the following map \begin{align*}
\theta:K[X]&\to K[\alpha]\subseteq F\\
f(X)&\mapsto f(\alpha)
\end{align*}
Now, $\ker\theta=\{f(X)\in K[X]:f(\alpha)=0\}\neq 0$ (since $\alpha$ is algebraic over $K$) therefore, $K[X]/\ker\theta\hookrightarrow K[\alpha]$ and $K[X]/\ker\theta$ is an integral domain thus $\ker\theta$ is a prime ideal. Since $K[X]$ is pid, $\ker\theta=\gen{p(X)},p(X)$ is monic irreducible polynomial in $K[X]$ (note that $p(X)$ is the least degree polynomial in $\ker\theta$). Let $f(\alpha)\in K[\alpha]$, we consider the polynomial $f(X)\in K[X]$ so that $\theta(f(X))=f(\alpha)$ hence $\theta$ is surjective. Therefore, $K[X]/\gen{p(X)}\cong K[\alpha].$ As $K[X]/\gen{p(X)}$ is field, $K[\alpha]$ is also field hence $K[\alpha]=K(\alpha).$ Conversely, suppose $K[\alpha]=K(\alpha).$ Consider $\dfrac{1}{\alpha}\in K(\alpha)$ then $\dfrac{1}{\alpha}\in K[\alpha]$ thus $\dfrac{1}{\alpha}=f(\alpha)$ where $f(X)\in K[X]$ so $\alpha$ satisfies the polynomial $Xf(X)-1\in K[X].$ Therefore, $\alpha$ is algebraic over $K$. Hence $K(\alpha)|K$ is algebraic. \qed

\begin{theorem}
Let $F|K$ be a field extension and $\alpha\in F$ is algebraic over $K$. Let $p(X)$ be the minimal polynomial of $\alpha$ Then $[K(\alpha):K]=\deg p(X).$
\end{theorem}
\proof Let $\deg p(X)=n$. We claim that $S=\{1,\alpha,\alpha^2,\cdots,\alpha^{n-1}\}$ is a basis for $K(\alpha)|K.$ Let $c_0+c_1\alpha+\cdots+c_{n-1}\alpha^{n-1}=0,c_i\in K,0\leq i\leq n-1.$ If $c_i\neq 0$ for some $i$ then $\alpha$ satisfies a polynomial over $K$ whose degree is strictly less that $n$ which is a contradiction since $p(X)$ is minimal polynomial of $\alpha$ over $K$. Therefore, $c_i=0,0\leq i\leq n-1.$ As $K(\alpha)|K$ is algebraic, $K[\alpha]=K(\alpha)$. We take $f(\alpha)\in K[\alpha],f(X)\in K[X]$ then $f(\alpha)=c_0+c_1\alpha+\cdots+c_m\alpha^m$ for some $f(X)=c_0+c_1X+\cdots+c_mX^m.$ By division algorithm, $f(X)=q(X)p(X)+r(X)$ where $r(X)=0$ or $\deg r(X)<\deg p(X)$ thus we have $f(\alpha)=r(\alpha)$ and $f(\alpha)=r(\alpha)\in \Span{1,\alpha,\alpha^2,\cdots,\alpha^{n-1}}$ (as $\deg r(X)<\deg p(X)$ or $r(X)=0$). Therefore, $S$ is a basis and $[K(\alpha):K]=n=\deg p(X).$ \qed
\begin{corollary}
Let $F|K$ be a field extension and $\alpha\in F$. Suppose, $\alpha$ is algebraic over $K$ then $K(\alpha)|K$ is algebraic.
\end{corollary}
\proof $[K(\alpha):K]$ is finite hence algebraic. \qed

\begin{prop}
Let $F|K$ be a field extension and $\alpha$ is transcendental (that is not algebraic) over $K$ iff $K[X]=K[\alpha].$
\end{prop}
\proof We define \begin{align*}
\theta:K[X]&\to K[\alpha]\\
f(X)&\mapsto f(\alpha)
\end{align*}
Note that $\theta$ is surjective. $\alpha$ is transcendental iff $\ker\theta=\{0\}.$ Hence $K[X]\cong K[\alpha].$ \qed\\
Note that \\\\


\subsection{Field automorphism and Galois extension}
Let $F$ be a field, $$\aut F:=\{\sigma:F\to F:\sigma~\text{is a field automorphism}\}.$$ Clearly, $\aut F$ is a group under mapping composition. Let $F|K$ be a field extension, $\aut_K F<\aut F$ where $$\aut_K F:=\{\sigma\in\aut F:\sigma|_K=\id_K\}.$$
Any element of $\aut_K F$ is called $K-$automorphism. $\aut_K F$ is called the Galois group of the extension $F|K.$ Let, $F,E$ be extension field of $K$ and $\sigma:F\to E$ be a field homomorphism such that $\sigma|_K=\id$ Then $\sigma$ is called $K-$homomorphism.
\begin{center}
\begin{tikzcd}
F \arrow[d, no head] \arrow[r, "\sigma"] & E \arrow[d, no head] \\
K \arrow[r]                              & K                   
\end{tikzcd}
\end{center}
\begin{qns}
Compute the Galois group of ${\RR}|{\QQ}$ that is, $\aut_{\QQ}{\RR}$.
\end{qns}
Ans. Let $\sigma\in \aut_{\QQ}{\RR}.$ Suppose $x\in {\RR}$ and $x>0$ then $\sigma(x)=\sigma((\sqrt{x})^2=\sigma(\sqrt{x})^2\geq 0$. Again $\sigma(0)=0$ and $\sigma$ is an isomorphism, hence $\sigma(x)>0$ whenever $x>0.$ Let $x,y\in {\RR}$ and $x>y$ then $x-y>0$ and it follows that $\sigma(x)>\sigma(y).$ We claim that $\aut_{\QQ}{\RR}=\{\id_{\RR}\}$. If not then there exists $\sigma\in \aut_{\QQ}{\RR}$ such that $\sigma(x)\neq x.$ Therefore, $\sigma(x)>x$ or $\sigma(x)<x$ (by Law of trichotomy). If $\sigma(x)>x$ we choose a rational $r\in {\QQ}$ between $\sigma(x)$ and $x$ that is $\sigma(x)>r>x$ (by density property of rationals) then $r-x>0$ and $\sigma(r)>\sigma(x)$ which implies $r>\sigma(x)$ contradicting our assumption. By similar argument we can show that if $\sigma(x)<x$ then we arrived at a contradiction. Therefore, $\aut_{\QQ}{\RR}=\{\id_{\RR}\}.$

\begin{qns}
Compute the Galois group of ${\QQ}(\sqrt{2})|{\QQ}$ that is $\aut_{\QQ}{\QQ}(\sqrt{2}).$
\end{qns}
Ans. $\mathcal{B}=\{1,\sqrt{2}\}$ is a basis of the extension ${\QQ}(\sqrt{2})|{\QQ}$ so any element of ${\QQ}{\sqrt{2}}$ is of the form $\{a+b\sqrt{2}:a,b\in {\QQ}\}$. Let $\sigma\in \aut_{\QQ}{\QQ}(\sqrt{2})$ and $\sigma(\sqrt{2})=c+d\sqrt{2}$. Now, $$2=\sigma(2)=\sigma((\sqrt{2}))^2=(\sigma(\sqrt{2}))^2=(c+d\sqrt{2})^2=c^2+2d^2+2cd\sqrt{2}.$$ Therefore we get $2cd\sqrt{2}=0$ which gives either $c=0$ or $d=0$. The case $d=0$ is impossible since $c^2=2$ has no rational solution. Therefore, $c=0$ in which case $d=1,-1$. Thus $\sigma(\sqrt{2})$ will be either $\sqrt{2}$ or $-\sqrt{2}.$ Then $\sigma(a+b\sqrt{2})=a\pm b\sqrt{2}$ (as $a,b\in {\QQ}$). So there are only two automorphism $\sigma_1,\sigma_2$ with $\sigma_1=\id$ and $\sigma_2(\sqrt{2})=-\sqrt{2}.$ Therefore, $\aut_{\QQ}{\QQ}(\sqrt{2})=\{\id,\sigma_2\}\cong {\ZZ}/2{\ZZ}.$

\begin{qns}
Compute the Galois group of the extensions ${\CC}|{\RR}$ and ${\CC}|{\QQ}.$
\end{qns}
Ans.\\
\begin{theorem}
Let $F$ be an extension field of $K$ and $f\in K[X]$. If $u\in F$ is a root of $f$ and $\sigma\in \aut_KF$, then $\sigma(u)\in F$ is also a root of $f.$
\end{theorem}
\proof If $f(X)=a_nX^n+\cdots+a_0$ with $a_n\neq 0;a_i\in K,1\leq i\leq n.$ $u$ is a root of $f$ therefore, $0=f(u)=a_nu^n+\cdots+a_0$. Applying $\sigma$ both side we have $$0=\sigma(f(u))=\sigma(a_nu^n+\cdots+a_0)=a_n\sigma(u)^n+\cdots+a_0.$$ Therefore, $\sigma(u)$ is also a root of $f(X).$ \qed
\begin{theorem}
Let $F$ be an extension field of $K$, $E$ is an intermediate field and $H<\aut_KF$. We define \begin{enumerate}
\item $H':=\{u\in F:\sigma(u)=u~\text{for all}~\sigma\in H\}\subseteq F;$
\item $E':=\{\sigma\in \aut_KF:\sigma(u)=u~\text{for all}~u\in E\}=\aut_E F.$ 
\end{enumerate}
Show that $H'$ is a subfield of $F$ containing $K$ and $E'$ is a subgroup of $\aut_KF.$ $H'$ is called the fixed field of $H$ in $F.$
\begin{center}
\begin{tikzcd}
F \arrow[d, no head] \arrow[r] & F'=\{\id\} &  & \{\id\}'=F \arrow[d, no head] & \{\id\} \arrow[l] \arrow[d, "\bigwedge", phantom] \\
E \arrow[d, no head] \arrow[r] & E'=\aut_EF &  & H' \arrow[d, no head]         & H \arrow[l] \arrow[d, "\bigwedge", phantom]       \\
K \arrow[r]                    & K'=\aut_KF &  & K                             & \aut_KF                                          
\end{tikzcd}
\end{center}
\end{theorem}
Note that in general $(\aut_KF)'\neq K$ but if $u\in K$ then $\sigma(u)=u$ for all $\sigma\in\aut_KF$ therefore $u\in (\aut_KF)' \Rightarrow K\subseteq (\aut_KF)'.$ For example take ${\RR}|{\QQ}$ then $\aut_{\QQ}{\RR}=\{\id\}$ and $(\aut_{\QQ}{\RR})'={\RR}\neq {\QQ}.$
\proof \begin{enumerate}
\item Pick $u_1,u_2\in H'$ then $\sigma(u_1)=u_1$ and $\sigma(u_2)=u_2$ for all $\sigma\in H.$ $\sigma(u_1-u_2)=\sigma(u_1)-\sigma(u_2)=u_1-u_2\in H$ for all $\sigma\in H.$ And $\sigma(u_1^{-1}u_2)=\sigma(u_1)^{-1}\sigma(u_2)=u_1^{-1}u_2$ for all $\sigma\in H$. Therefore, $H'$ is a subfield of $F.$ Since $\aut_KF$ fixes $K$, any subgroup of $\aut_KF$ will also fix $K$, hence $K\subseteq H'.$
\item Let $\sigma_1,\sigma_2\in \aut_EF$, we have $\sigma_1(u)=u$ and $\sigma_2(u)=u$ for all $u\in E.$ $(\sigma_1)^{-1}\sigma_2(u)=\sigma_1^{-1}(u)=u$ for all $u\in E$ which gives $(\sigma_1)^{-1}\sigma_2\in E'$ therefore, $E'=\aut_EF<\aut_KF.$
\end{enumerate}
\qed

\begin{defn}
A field extension $F|K$ is called Galois extension if $(\aut_KF)'=K.$
\end{defn}
In our example ${\RR}|{\QQ}$ is not Galois. We take ${\QQ}(\sqrt{2})|{\QQ}$. $\aut_{\QQ}{\QQ}(\sqrt{2})=\{\id,\sigma\};\sigma(\sqrt{2})=-\sqrt{2}.$ Let $a+b\sqrt{2}\in (\aut_{\QQ}{\QQ}(\sqrt{2})'$ then $\sigma(a+b\sqrt{2})=a+b\sqrt{2} \Rightarrow a-b\sqrt{2}=a+b\sqrt{2}\Rightarrow b=0$. Therefore, $(\aut_{\QQ}{\QQ}(\sqrt{2}))'={\QQ}.$ Hence ${\QQ}(\sqrt{2})|{\QQ}$ is Galois.

\begin{lemma}
Prove the followings:
\begin{enumerate}
\item $F'=\{\id\}$ and $K'=\aut_KF;$
\item $\{\id\}'=F;$
\item If $L\subseteq M $ then $M'<L';$
\item If $H<J$ then $J'\subseteq H'$;
\item $L\subseteq L''$ and $H<H'';$
\item $L'=L'''$ and $H'=H'''.$
\end{enumerate}
\end{lemma}
\proof \begin{enumerate}
\item $F'=\{\sigma\in\aut_KF:\sigma(u)=u~\text{for all}~u\in F\}=\{\id\}$ and $K'=\aut_KF$ by definition.
\item $\{\id\}'=\{u\in F:\id(u)=u\}=F.$
\item Let $L\subseteq M.$ Let $\sigma\in\aut_MF$ then $\sigma(m)=m$ for all $m\in M$ in particular, $\sigma(l)=l$ for all $l\in L$ thus $\sigma\in \aut_LF$ and $M'<L'.$
\item Let $H<J$, Take $u\in J'$ therefore, $\sigma_J(u)=u$ for all $\sigma_J\in J$, in particular, $\sigma_H(u)=u$ for all $\sigma_H\in H$ thus $u\in H'$ and $J'\subseteq H'.$
\item $L''=\{u\in F:\sigma(u)=u~\text{for all}~\sigma\in L'\}$. Let $u\in L$ and $\sigma\in L'$ then $\sigma(u)=u$ and this is true for all $\sigma\in L'$ and for all $u\in L$ therefore, $u\in L''$ and $L\subseteq L''.$\\
$H''=\{\sigma\in\aut_KF:\sigma(u)=u~\text{for all}~u\in H'\}$. Let $\sigma\in H$ and $u\in H'$ then $\sigma(u)=u$ for all $\sigma\in H$ and this is true for all $u\in H'$ therefore, $\sigma\in H''$ and $H<H''.$
\item From (5), $L\subseteq L''$ and By (3), $L'''<L'$ again $L'<L'''$ by (5) hence $L'=L'''.$ By (5) $H<H''$ and by (4) $H'''\subseteq H'$ and by (5) $H'\subseteq H'''$ therefore, $H'=H'''.$
\end{enumerate}
\qed

\begin{defn}
Let $K\subseteq L\subseteq F$ be the extension. $L$ is said to be closed subfield of $F$ with respect to the extension field $F|K$ if $L''=L.$ Let $H<\aut_KF$, $H$ is said to be closed subgroup if $H''=H$. In particular the extension $F|K$ is Galois if $K=K''.$
\end{defn}
\begin{lemma}
Let $K\subseteq L\subseteq M\subseteq F$ be the extension and $[M:L]$ is finite then $[L':M']\leq [M:L]$. In particular if $F|K$ is finite then $|\aut_KF|\leq [F:K].$
\end{lemma}
\proof We proceed by induction on $n=[M:L].$ If $n=1$ then $M=L$ and consequently $L'=M'$ and we are done. Suppose $n>1$ and this lemma is true for all $i<n.$ Let $u\in M-L,$ and consider the following extension $K\subseteq L\subseteq L(u)\subseteq M\subseteq F.$\\
\textbf{Case 1.} $L(u)=M,$ since $[L:M]$ is finite, $u$ is algebraic over $L$. Let $f(X)$ be the irreducible polynomial of $u$ and $\sum=\{v\in F:f(v)=0\}$ then $|\sum|<\deg f=n$ and $S$ be the set of all left cosets of $M'$ in $L'$. Define \begin{align*}
\theta:S&\to \sum\\
\tau M'&\mapsto \tau(u)
\end{align*}
Since $u$ is a root of $f(X)\in L[X]$ and $\tau\in L'=\aut_LF$ therefore $\tau(u)$ is also a root of $f$. We claim that $\theta$ is well defined and injective. Let $\tau M'=\sigma M' \Leftrightarrow \sigma^{-1}\tau\in M'=[L(u)]' \Leftrightarrow \sigma^{-1}\tau(u)=u \Leftrightarrow \tau(u)=\sigma(u)$. Therefore, $|S|\leq |\sum|\leq n$ hence $[L':M']\leq n.$\\
\textbf{Case 2.} If $L(u)\subsetneq M$ then consider the extension $L\subseteq L(u)\subseteq M.$ Let $[L(u):L]=r$ and $[M:L(u)]=n/r.$ Since $u\notin L,$ $r$ must be bigger that 1 and $n/r<n.$ By induction hypothesis $[L(u)':M']\leq [M:L(u)]$ and $[L':L(u)']\leq [L(u):L].$ Multiplying both of them we get $[L':L(u)'][L(u)':M']\leq [M:L(u)][L(u):L] \Rightarrow [L':M']\leq [M:L].$ \qed
\begin{lemma}
Let $F$ be an extension field of $K$ and let $H, J$ be subgroups of the Galois group $\aut_KF$ with $H<J.$ If $[J:H]$ is finite, then $[H':J']\leq [J:H].$
\end{lemma}
\proof Let $[J:H]=n$ and suppose $[H':J']>n$ then there exists $\{u_1,\cdots,u_{n+1}\}\in H'$ such that $\{u_1,\cdots,u_{n+1}\}$ is linearly independent over $J'.$ Let $\{\tau_1,\cdots,\tau_n\}$ be a complete set of representatives of all left cosets of $H$ in $J$ that is $J=\displaystyle\bigcup_{i=1}^n \tau_i H$ and $\tau_i^{-1}\tau_j\notin H$ if $i\neq j.$ We consider the linear system with $n$ equation and $n+1$ unknowns:\begin{align*}
\tau_1(u_1)&X_1+\cdots+\tau_1(u_{n+1})X_{n+1}=0\\
A\equiv\qquad\qquad&\vdots \qquad\qquad\qquad\qquad \vdots\\
\tau_n(u_1)&X_1+\cdots+\tau_n(u_{n+1})X_{n+1}=0
\end{align*}
The system has a non trivial solution. Among all such non trivial solution we choose $(a_1,\cdots,a_{n+1})$ with a minimal number of non zero's. After reindexing, if necessary we may assume $a_{r+1}=\cdots=a_{n+1}=0$ and $a_i\neq 0,1\leq i\leq r$ then $r\geq 1.$ Since $c(a_1,\cdots,a_{n+1}),c\in J'$ is also a solution we may further assume that $a_1=1_F.$ Let $\tau_1=\id$ (as $\{\tau_1,\cdots,\tau_n\}$ is complete list of representatives of left cosets of $H$ in $J$). From the first equation we get $$u_1a_1+\cdots+u_ra_r=0.$$
$\{u_1,\cdots,u_r\}$ is linearly independent over $J'$ and $a_i\neq 0,1\leq i\leq r$ there exists $i\in \{1,\cdots,r\}$ such that $a_i\notin J'.$n We have $a_1=1_F$ which imply $a_i\in J'$ so with out loss of generality we may assume that $a_2\notin J'$ therefore, there exists $\sigma\in J$ such that $\sigma(a_1)\neq a_2$. Now we consider another system of linear equations:
\begin{align*}
\sigma(\tau_1(u_1))&X_1+\cdots+\sigma(\tau_1(u_{n+1}))X_{n+1}=0\\
B\equiv\quad\qquad\qquad&\vdots \qquad\qquad\qquad\qquad \vdots\\
\sigma(\tau_n(u_1))&X_1+\cdots+\sigma(\tau_n(u_{n+1}))X_{n+1}=0
\end{align*}
Since $(a_1,\cdots,a_{n+1})$ is a solution of $A$, $(\sigma(a_1),\cdots,\sigma(a_{n+1}))$ is a solution of $B.$ We claim that $A$ and $B$ has same solution. If we prove the claim then $(\sigma(a_1),\cdots,\sigma(a_{n+1}))$ is a solution of $A$ also and therefore, $(a_1,\cdots,a_{n+1})-(\sigma(a_1),\cdots,\sigma(a_{n+1}))$ is also a solution of $A$. Since $a_1=1_F,\sigma(a_1)=1_F$ and $\sigma(a_2)\neq a_2$ we have a non trivial solution of $A$ namely $(0,a_2-\sigma(a_2),\cdots,a_r-\sigma(a_r),0,\cdots,0)$ which is a contradiction as $(a_1,\cdots,a_r,0,\cdots,0)$ is a non trivial solution with minimal number of zero's. Hence $[H':J']\leq [J:H].$ Now we proof the claim. At first notice that $\{\sigma\tau_1,\cdots,\sigma\tau_n\}$ is also complete list of representatives of left cosets of $H$ in $J$ (as $(\sigma\tau_i)^{-1}(\tau\sigma_j)=\tau_i^{-1}\tau_j\notin H$ if $i\neq j$). Therefore, $J=\displaystyle\bigcup_{i=1}^n \sigma\tau_iH.$ Again if $\zeta H=\theta H\Leftrightarrow \theta^{-1}\zeta\in H \Leftrightarrow \theta^{-1}\zeta(u_i)=u_i \Leftrightarrow \theta(u_i)=\zeta(u_i),1\leq i\leq n+1$ (as $\{u_1,\cdots,u_{n+1}\}\subseteq H']$). Since $\{\tau_1H,\cdots,\tau_nH\}$ and $\{\sigma\tau_1H,\cdots,\sigma\tau_nH\}$ are both complete list of of cosets of $H$, $\sigma\tau_iH=\tau_jH$ and this gives $\sigma\tau_i(u_l)=\tau_j(u_l)$ for all $l\in\{1,\cdots,n+1\}.$ Let $\tau_1=\sigma\tau_k;1\leq k\leq n$ so we get $\tau_1(u_1)=\sigma\tau_k(u_1),\cdots,\tau_1(u_{n+1})=\sigma\tau_k(u_{n+1})$ therefore, the first equation of $A$ becomes $\sigma\tau_k(u_1)X_1+\cdots+\sigma\tau_k(u_{n+1})X_{n+1}=0$. This shows that $A$ and $B$ are the same system of linear equation therefore there solution is also same. \qed

\begin{lemma}
Let $K\subseteq L\subseteq M\subseteq F$ and $\{\id\}<H<J<\aut_KF$ then \begin{enumerate}
\item If $L$ is closed and $[M:L]$ is finite then $M$ is closed and $[L':M']=[M:L],$
\item If $H$ is closed and $[J:H]$ is finite, then $J$ is closed and $[H':J']=[J:H],$
\item IF $F|K$ is finite and Galois then all intermediate fields and all subgroups of $\aut_KF$ are closed and $|\aut_KF|=[F:K].$
\end{enumerate}
\end{lemma}
\proof \begin{enumerate}
\item $[L':M']\leq [M:L]$ by previous lemma and $[M:L]$ is finite implies $[L':M']$ is also finite hence $[M'':L'']\leq [L':M']\leq [M:L]$. Now $M\subseteq M''$ thus $[M:L]\leq [M'':L]\leq [L':M']\leq [M'':L]$. Therefore, $[L':M']=[L:M]$ and $M=M''.$
\item By previous lemma, $[H:J']\leq [J:H]$ hence $[H':J']$ is also finite since $[H:J]$ is finite. Therefore, $[J'':H'']\leq [H':J']\leq [J:H]$. As $H$ is closed, $H=H''$ and $[J'':H]\leq [H':J']\leq [J:H].$ We have $J<J"$ from previous lemma, then $[J:H]\leq [J'':H]\leq [H':J']\leq [J:H]$. Thus $[H':J']=[J:H]$ and $J''=J.$
\item In particular if $F|K$ is Galois, $K=K''$ so $[K':F']=[F:K] \Rightarrow |\aut_KF|=[F:K].$
\end{enumerate}
\qed


\begin{defn}
Let $L\subseteq E\subseteq F$, $E$ is said to be stable (relative to $F|K$) if $\sigma(E)\subseteq E$ for all $\sigma\in\aut_KF.$
\end{defn}
Note that $\sigma\in\aut_KF$ then $\sigma^{-1}\in \aut_KF$. If $E$ is stable, $\sigma(E)\subseteq E$ and $\sigma^{-1}(E)\subseteq E \Rightarrow E\subseteq \sigma(E) \Rightarrow \sigma(E)=E.$

\begin{lemma}
Let $F|K$ be the field extension, \begin{enumerate}
\item If $K\subseteq E\subseteq F$ and $E$ is stable intermediate field, then $E'=\aut_EF\unlhd \aut_KF$;
\item If $H\unlhd \aut_KF$ then $H'$ is stable.
\end{enumerate}
\end{lemma}
\proof \begin{enumerate}
\item Let $\tau\in \aut_KF$ and $x\in E$, $\sigma\in \aut_EF=E'.$ Since $E$ is stable, $\tau^{-1}(x)\in E$ so $\sigma(\tau^{-1}(x))=\tau^{-1}(x) \Rightarrow \tau\sigma\tau^{-1}(x)$ and this is true for any $x\in E$ therefore, $\tau\sigma\tau^{-1}\in E'$ hence $E'\unlhd \aut_KF.$
\item Let $x\in H',]tau\in \aut_KF$. Let $\sigma\in H,$ as $H\unlhd \aut_KF$, $\tau^{-1}\sigma\tau\in H$ therefore, $\tau^{-1}\sigma\tau(x)=x \Rightarrow \sigma(\tau(x))=\tau(x) \Rightarrow \tau(x)\in H'$ (since $\sigma$ is chosen arbitrarily). Hence $H'$ is stable.
\end{enumerate}
\qed

\begin{lemma}
Let $K\subseteq E\subseteq F$ be the extension and $F|K$ is Galois. If $E$ is stable then $E|K$ is Galois.
\end{lemma}
\proof Let $u\in E\setminus K$ then $u\in F\setminus K.$ Since $K=K'',$ there exists $\sigma\in\aut_KF$ such that $\sigma(u)\neq u.$ Now $\sigma|_E:E\to E$ is the restriction map and we can do this because $E$ is stable. Then $\sigma|_E(u)\neq u$. Therefore, $E|K$ is Galois. \qed

\begin{lemma}
Let $K\subseteq E\subseteq F$ such that $E|K$ is algebraic and Galois then $E$ is stable.
\end{lemma}
\proof Let $u\in E$. Since $E|K$ is algebraic, there exists an irreducible monic polynomial $f(X)\in K[X]$ such that $f(u)=0.$ Let $\deg f=n$ and $u_1,\cdots,u_r~(r\leq n)$ be all roots of $f$ lies in $E$. We consider the polynomial $g(X)=(X-u_1)\cdots(X-u_r)\in E[X].$ If $\sigma\in\aut_KE$ then $\sigma$ simply permutes the set $\{u_1,\cdots,u_r\}$. Now, \begin{align*}
g(X)&=X^r-(u_1+\cdots+u_r)X^{r-1}+\displaystyle\sum_{1\leq i<j\leq n}u_iu_jX^{r-2}+\cdots+(-1)^{r-1}\displaystyle\prod_{i=1}^r u_i\\
&=X^r+a_1X^{r-1}+\cdots+a_r
\end{align*}
where $a_1=-(u_1+\cdots+u_r),\cdots,a_r=(-1)^{r-1}\displaystyle\prod_{i=1}^ru_i.$ Since $a_i$'s are symmetric functions of $u_i$'s, $\sigma(a_i)=a_i,\forall~\sigma\in\aut_KE \Rightarrow a_i\in (\aut_KE)'$ for all $1\leq i\leq r.$ Since $E|K$ is Galois, $(\aut_KE)'=K \Rightarrow a_i\in K$ for all $1\leq i\leq r.$ Therefore, $g(X)\in K[X].$ since $u\in\{u_1,\cdots,u_r\}\Rightarrow g(u)=0$. As $f$ is the minimal polynomial of $u$ in $K[X] \Rightarrow f|g$ but $\deg g\leq \deg f \Rightarrow g=f$ (as both are monic). So we get that there are $n$ distinct roots of $f$ lies in $E.$ Now, $\tau\in\aut_KF,\tau(u)$ is also a root of $f \Rightarrow \tau(u)\in E.$ Therefore, $E$ is stable. \qed

\begin{lemma}
Let $K\subseteq E\subseteq F$ be the extension and $E$ is stable, then $\aut_KF/\aut_EF$ is isomorphic to the group of all $K-$automorphisms of $E$ that are extendable to $F.$
\end{lemma}
\proof We define, \begin{align*}
\theta:\aut_KF&\to \aut_KE\\
\sigma&\mapsto \sigma|_E\qquad\qquad\text{(since $E$ is stable)}
\end{align*}
$\ker\theta=\{\sigma\in\aut_KF:\sigma|_E=\id\}=\aut_EF$ and $\dfrac{\aut_KF}{\aut_EF}\cong \im\theta=\{\psi\in\aut_KE:\text{there exists}~\tilde{\psi}\in\aut_KF~\text{such that}~\tilde{\psi}|_E=\psi\}$.\qed

\begin{theorem}[Fundamental theorem of Galois theory]
If $F$ is a finite dimensional Galois extension of $K$, then there is a one to one corresponding between the set of all intermediate fields of $F|K$ and the set of all subgroups of the Galois group $\aut_KF$ (given by $E\mapsto E'=\aut_EF)$ such that \begin{enumerate}
\item the relative dimension of two intermediate fields is equal to the relative index of the corresponding subgroups. In particular $|\aut_KF|=[F:K]$ that is 
\begin{center}
\begin{tikzcd}
F \arrow[d, no head] \arrow[r] \arrow[ddd, "\text{Finite Galois extension}"', no head, bend right=35]  &F'=\{\id\} \arrow[d, "\bigwedge", phantom] \\
M \arrow[d, no head] \arrow[r]                                                                         &M'=\aut_MF \arrow[d, "\bigwedge", phantom]      \\
L \arrow[d, no head] \arrow[r]                                                                         &L'=\aut_LF \arrow[d, "\bigwedge", phantom]      \\
K \arrow[r]                                                                                            &K'=\aut_KF                                
\end{tikzcd}
\end{center}

\item $K\subseteq E\subseteq F$ is a finite Galois extension then $F|E$ is Galois but $E|K$ is Galois iff $E'\unlhd \aut_KF$ and $\aut_KE\cong \aut_KF/E'.$
\end{enumerate}
\end{theorem}

\proof \begin{enumerate}
\item Since $F|K$ is Galois, every intermediate field of $F|K$ and every subgroup of $\aut_KF$ is closed and by lemma 15.23, $|\aut_KF|=[F:K].$
\item Since $F|K$ is Galois, $E$ is closed and therefore, $F|E$ is Galois since $E=(\aut_EF)'=(E')'.$ Now we assume that $E|K$ is Galois, since $E|K$ is finite, it is algebraic and by lemma 15.27 $E$ is stable which implies $E'\unlhd \aut_KF.$ Conversely, let $E'\unlhd \aut_KF$ then $E''$ is stable and since $E$ is closed, $E=E''$, hence $E|K$ is Galois (as $F|K$ is Galois) by lemma 15.26.
\end{enumerate}


\begin{theorem}[Artin]
Let $F$ be a field and $G<\aut F, K=\{x\in F:\sigma(x)=x~\text{for all}~\sigma\in G\}$ that is fixed field of $G$. Then $F|K$ is Galois. If $G$ is finite then $[F:K]$ is finite and $\aut_KF=G.$
\end{theorem}
\proof Let $u\in F\setminus K$ then $u\notin G'$ so there exists $\sigma\in G<\aut_KF$ (Note that $\sigma\in G,x\in K \Rightarrow \sigma(x)=x\Rightarrow \sigma\in\aut_KF$) such that $\sigma(u)\neq u.$ Therefore, $F|K$ is Galois. Suppose, $G$ is finite. Now, $\{\id\}<G<\aut_KF$ and $[G:\{\id\}]=|G|=$ finite therefore, by Lemma 15.22 $[\{\id\}':G']\leq |G| \Rightarrow [F:K]\leq |G|.$ We have proved that $F|K$ is Galois thus $[F:K]=|\aut_KF|$ (by Theorem 15.29) therefore, $|\aut_KF|\leq G\leq |\aut_KF|\Rightarrow \aut_KF=G.$\qed

\begin{lemma}
Suppose $F|K$ is finite and $[F:K]=|\aut_KF|$ then $F|K$ is Galois.
\end{lemma}
\proof Let $G=\aut_KF<\aut F$ then $K\subseteq K''\subseteq F$ and by Artin's theorem $F|K''$ is Galois with Galois group $G$. By Theorem 15.29 $[F:K'']=|G|=|\aut_KF|=[F:K]=[F:K''][K'':K] \Rightarrow [K'':K]=1 \Rightarrow K=K''.$ Hence $F|K$ is Galois.\qed 

\begin{defn}
Let $F$ be a field and $f(X)\in F[X]$ be a polynomial of positive degree. $f$ is said to be split over $F$ if $$f(X)=u_0(X-u_1)\cdots (X-u_n),u_i\in F,0\leq i\leq n.$$
\end{defn}

\begin{defn}
\begin{enumerate}
\item 
Let $K$ be a field and $f\in K[X]$ be a polynomial of positive degree. An extension field $F$ of $K$ is said to be a splitting field over $K$ of then polynomial $f(X)$ if $f$ splits over $F$ and $F=K(u_1,\cdots,u_n)$ where $u_1,\cdots,u_n$ are the roots of $f$ in $F$.
\item Let $S$ be a set of polynomials of positive degree in $K[X]$. An extension field $F$ of $K$ is said to be splitting field over $K$ of the set $S$ of polynomials if every polynomial in $S$ splits over $F$ and $F$ is generated by the set of all roots of all polynomials in $S$ over $K$.
\end{enumerate}
\end{defn}



















\newpage
%\chapter{Appendix I}
\section{Appendix I}

\begin{ex}
Suppose that the set $R$ is equipped with two binary operation `$+$' and `$\cdot$' such  that $(R,+)$ is a a group and $(R,\cdot)$ is a semi group with identity element $1$ and $x(y+z)=xy+xz$ for all $x,y,z\in R.$ Show that $(R,+,\cdot)$ is a ring i.e., show that the operation `$+$' is necessarily commutative.
\end{ex}
\begin{ex}
Let $R$ be a ring whose additive group $(R,+)$ is cyclic. Show that $R$ is a commutative ring.\\
or, $p$ be a prime number. Show that any ring of order $p$ is commutative ring.
\end{ex}
\begin{ex}
Give an example of  a non commutative ring of order $4.$
\end{ex}
\begin{ex}
Show that a ring of order $6$ is commutative. Does there exists an integral domain of order $6?$ Justify your answer
\end{ex}
\begin{ex}
Let $R$ be a ring with unique element $e\neq 0$ such that $ex=x$ for all $x$ in $R.$ Show that $e$ is an identity element of $R.$ Does the above result hold if $e$ is not unique? Justify.
\end{ex}
\begin{ex}
Let $R$ be a Boolean ring. Show that \\
$(i)$ $\ch R=2,$\\
$(ii)$ $R$ is a commutative ring. 
\end{ex}
\begin{ex}
If $R$ is a finite non-zero Boolean ring then show that $R$ has $2^n$ elements for some positive integer $n.$ Hence conclude that there is no Boolean ring of order $6.$
\end{ex}
\begin{ex}
Let $R$ be a ring with identity. Show that $R$ is a Boolean ring if and only if $a(a+b)b=0$ for all $a,b\in R.$
\end{ex}
\begin{ex}
Let $R$ be a finite ring with identity and $a,b\in R$ such that $ab=1$ show that $ba=1.$
\end{ex}
\begin{ex}
Are the rings $C[0,1]$ and $D[0,1]$ integral domain? Justify your answer.
\end{ex}
\begin{ex}
Let $R$ be a ring and $C(R)$ be the center of $R$. If $x^2-x\in C(R)$ then show that $R$ is commutative.
\end{ex}
\begin{ex}
If $R$ is division ring then show that $C(R)$ is a field.
\end{ex}		
\begin{ex}
Let $R$ be a ring with identity such that $R$ has no divisor of zero. If every sub ring of $R$ iss ideal of $R$ then show that $R$ is a commutative ring.
\end{ex}
\begin{ex}
Let $R$ is an integral domain such that $IJ=I\cap J$ for every ideal $I,J$ of $R$. Show that $R$ is a field.
\end{ex}
\begin{ex}
Show that any integral domain with finite number of ideals is field. Hence conclude  that any finite integral domain is a field.
\end{ex}
\begin{ex}
Let $a,b$ two elements in a ring $R$ and $m,n\in {\NN}$ with $\gcd(m,n)=1$ such that $a^m=b^m$ and $a^n=b^n$. Show that $a=b.$
\end{ex}
\begin{ex}
Let $R$ be a ring with identity and $a\in R$,\\
$(i)$ If $a$ has left(or right) inverse but no right(left) inverse then show that $a$ has atleast two left(right) inverse.
$(ii)$ If $a$ has more than a left (or right) inverse then it has infinite number of such inverses.
\end{ex}
\begin{ex}
Let $R$ be a ring such that $R$ has no non zero nilpotent element. If $e\in R$ is an idempotent element then show that $e\in C(R).$
\end{ex}
\begin{ex}
Show that units of ring ${\ZZ}[i]$ forms a cyclic group.
\end{ex}
\begin{ex}
Let $R$ be a ring with identity and $a,b\in R$. If $ 1-ab$ is a unit in $R$ then show that $1-ba$ is also a unit in $R.$
\end{ex}
\begin{ex}
Let $A$ be a left ideal and $B$ be a right ideal of $R$. If $A\cap B$ is an ideal of $R?$ Justify your answer.
\end{ex}
\begin{ex}
$(i)$ Let $R$ be a commutative ring with identity then show that set $N$ of all nilpotent elements forms an ideal of $R$. Also show that the ring $R/N$ has no non zero nilpotent elements. Is commutativity essential? Justify your answer.
\end{ex}
\begin{ex}
Show that a finite ring $R$ with left and right non zero divisor has identity.
\end{ex}
\begin{ex}
If $R$ is a finite ring with identity then show that each non zero element is either a one sided zero divisor or a unit in $R.$
\end{ex}
\begin{ex}
Do zero divisor of a ring form an ideal? Justify.
\end{ex}
\begin{ex}
Show that ${\ZZ}/n{\ZZ}\cong {\ZZ}_n$ for any integer $n>0.$
\end{ex}
\begin{ex}
Show that any epimorphism from the ring of integers onto itself is an isomorphism.
\end{ex}
\begin{ex}
Show that $(2{\ZZ},+,\cdot)$ is not isomorphic to $(3{\ZZ},+,\cdot).$
\end{ex}
\begin{ex}
Is quotient ring of an integral domain always an integral domain?
\end{ex}
\begin{ex}
Show that there is no ring isomorphism from $({\CC},+,\cdot)$ to $({\RR},+,\cdot).$
\end{ex}
\begin{ex}
Let $F=\left\lbrace \begin{pmatrix}
x\quad x\\
x\quad x
\end{pmatrix}:x\in {\RR}\right\rbrace$. Show that $F$ is a field isomorphic to field of real numbers ${\RR}.$
\end{ex}
\begin{ex}
Consder the ring ${\ZZ}$ and two ideals $I=m{\ZZ}$ and $J=n{\ZZ}$. Show that \\
$(i)$ $I+J=\gcd(m,n){\ZZ}\quad\quad\quad$			$(ii)$ $I\cap J=\lcm{m,n}{\ZZ}$\\
$(iii)$ $IJ=(mn){\ZZ}\quad\quad\quad$				$(iv)$ $I\subseteq J$ iff $m|n.$
\end{ex}
\begin{ex}
Let $R=\left\lbrace \begin{pmatrix}
a \quad b\\
0\quad a
\end{pmatrix}\bigg|a,b\in {\RR}\right\rbrace$ and $I=\left\lbrace \begin{pmatrix}
0 \quad b\\
0\quad 0
\end{pmatrix}\bigg|b\in {\RR}\right\rbrace.$ Show that $I$ is a maximal ideal of $R$ and $R/I\cong R.$
\end{ex}
\begin{ex}
Let $f:R\to R'$ be a non-trivial homomorphism from a field onto a ring $R'$. Show that $R'$ is a field.
\end{ex}
\begin{ex}
Let $R$ be a commutative ring with identity and $R'$ be an integral domain with identity. If $f:R\to R'$ be a non-zero homomorphism, then show that $f(1_R)=1_{R'}.$\\
In the above problem, is $R'$ is an integral domain essential? Justify your answer.
\end{ex}
\begin{ex}
$(i)$ Are fields ${\QQ}$ and ${\RR}$ are isomorphic? Justify.\\
$(ii)$ Are fields ${\RR}$ and ${\CC}$ isomorphic? Justify.\\
$(iii)$ Are the rings ${\ZZ}_6$ and ${\ZZ}_2\times {\ZZ}_3$ isomorphic? Justify.
\end{ex}
\begin{ex}
Let $R$ be a ring with identity such that $(xy)^2=x^2y^2$ for all $x,y\in R$. Show that $R$ is commutative ring. Is the above result true , if $R$ does not have an identity?
\end{ex}
\begin{ex}
Let $R$ be a finite ring with $m$ elements and $\ch R=n.$ Show that $n|m.$ 
\end{ex}
\begin{ex}
Let $R$ be a commutative ring with identity such that $\ch R=p$ ($p-$prime). Show that $(a+b)^p=a^p+b^p$ for all $a,b\in R.$ Is the above result true if $p$ is not a prime.
\end{ex}
\begin{ex}
Show that the field of rational ${\QQ}$ has no proper subfields.
\end{ex}
\begin{ex}
Let $R$ be a ring with identity and $a$ is a nilpotent element of $R$ then show that $1-a$ and $1+a$ is unit in $R.$
\end{ex}
\begin{ex}
Find all idempotent elements of the ring $M_2({\RR}).$
\end{ex}
\begin{ex}
Let $R$ be a ring with identity. Show that by means of an example that $A$ is an ideal of $B$ and $B$ is an ideal of $R$ but $A$ is not an ideal of $R.$
\end{ex}
\begin{ex}
Give an example to show that quotient of a ring with zero divisors may be an integral domain.
\end{ex}
\begin{ex}
Let $R=\left\lbrace \begin{pmatrix}
a\quad b\\
0\quad 0
\end{pmatrix}:a,b\in {\RR}\right\rbrace$ be a ring. Show that $R$ has no identity element but $R$ has infinite number of left identity element.
\end{ex}
\begin{ex}
Let $R$ be the set of differentiable functions $f:{\RR}\o {\RR}$. Prove that $R$ is commutative ring with identity when addition and multiplication are defined by $(f+g)(x)=f(x)+g(x)$ and $(fg)(x)=f(x)g(x)$ for all $x\in {\RR}$. Show that $S=\{f\in R:f(0)=0\}$ is an ideal of $R$. If $T=\{f\in R: Df(0)=0\}$, where $Df$ is derivative of $f$, then show that $T$ is subring of $R$ which is not an ideal of $R$ also show that $S\cap T$ is an ideal of $R.$
\end{ex}
\begin{ex}
Show that the rings ${\ZZ}[\sqrt{2}]$ and ${\ZZ}[\sqrt{3}]$ are not isomorphic.
\end{ex}
\begin{ex}
Let $F$ be a field. Show that $(F,+)$ and $(F\setminus \{0\},\cdot)$ is not isomorphic.
\end{ex}
\begin{ex}
Let $R$ be a commutative ring and let $B$ be the set of all idempotent elements of $R$. Define addition `$\oplus$' and multiplication `$\odot$' on $B$ by $x\oplus y=x+y-2xy$ and $x\odot y=xy.$ Show that $(B,\oplus,\odot)$ is a Boolean ring.
\end{ex}
\begin{ex}
Show that every Boolean ring with identity can be embedded in Boolean ring with identity.
\end{ex}
\begin{ex}
Show that any ring without identity can be embedded in any ring with identity.
\end{ex}
\begin{ex}
Show that every integral domain is embedded in a field.
\end{ex}
\begin{ex}
Let $R=\left\lbrace \begin{pmatrix}
a \quad 0\\
0\quad 0
\end{pmatrix}:a\in {\RR}\right\rbrace$ and $R'=M_2({\RR})$ be two rings. Then both $R$ and $R'$ have identity elements namely $1_R=\begin{pmatrix}
1 \quad 0\\
0\quad 0
\end{pmatrix}$ and $1_{R'}=\begin{pmatrix}
1 \quad 0\\
0 \quad 1
\end{pmatrix}$ respectively. Show that the inclusion mapping $f:R\to R'$ does not satisfy $f(1_R)=1_{R'}.$
\end{ex}
\begin{ex}
Ler $R\neq 0$ be a commutative ring with identity having no zero divisor. If every proper subring of $R$ is finite then show that $R$ is a field.
\end{ex}
\begin{ex}
Define two binary operation `$\oplus$' and `$\odot$' on ${\ZZ}$ by $$a\oplus b=a+b-1~\text{and}~a\odot b=a+b-ab~\forall a,b\in {\ZZ}$$ Show that $({\ZZ},\oplus,\odot)$ is a ring isomorphic to ${\ZZ},+,\cdot).$ 
\end{ex}
\begin{ex}
Show that the rings ${\ZZ}$ and ${\ZZ}\times {\ZZ}$ are not isomorphic. Is the groups $({\ZZ},+)$ and $({\ZZ}\times {\ZZ},+)$ isomorphic?
\end{ex}
\begin{ex}
Let $f:R\to R'$ be a ring homomorphism where $R,R'$ are commutative ring with identity. Does $f$ maps $(i)$ idempotent elements to idempotent elements?\\
$(ii)$ nilpotent elements to nilpotent elements?\\
$(iii)$ Zero divisors to zero divisors?
\end{ex}
\begin{ex}
Let $e$ be a central idempotent element of $R$ i.e, $e\in C(R)$ and $e^2=e.$ Show that $eR$ and $(1-e)R$ are ideals of $R.$ Also show that $R=eR\oplus (1-e)R.$
\end{ex}
\begin{ex}
Let $f:R\to R'$ be a ring epimorphism. Justify each of the followings(either prove or disprove).
\begin{enumerate}
\item If $R$ is commutative then $R'$ is commutative.
\item If $R$ has identity $1_R$ and $R'$ has identity $1_{R'}$ then $f(1_R)=1_{R'}.$
\item If $R$ has zero divisor then $R'$ has zero divisor.
\item If $R$ is an integral domain then $R'$ is also an integral domain.
\item If $R$ is a field then $R'$ is a field.
\end{enumerate}
\end{ex}
\begin{ex}
Find the total number of ring homomorphism from ${\ZZ}/4{\ZZ}$ to ${\ZZ}/6{\ZZ}$. 
\end{ex}
\begin{ex}
Let $P_1, P_2\in \spec R.$ Is $P_1\cap P_2\in \spec R?$ Justify.
\end{ex}
\begin{ex}
Let $R$ be a commutative ring with identity and $m\in \mspec R$ such that $m^2=0$ then show that $R$ is local.
\end{ex}
\begin{ex}
Let $R$ be a commutative ring with identity in which every ideal of $R$ is prime ideal then show that $R$ is a field.
\end{ex}
\begin{ex}
Let $R$ be a commutative ring with identity in which $A,B$ be two distinct maximal ideals of $R$ then show that $AB=A\cap B.$
\end{ex}
\begin{ex}
Let $R=C[0,1]$ and for $r\in [0,1]$, $m_r=\{f\in C[0,1]:f(r)=0\}$. Show that $m_r\in \mspec R.$\\
Note that maximal ideals of $R$ is of the form $m_r$ for some $r\in [0,1].$
\end{ex}
\begin{ex}
Is the ideal $I=\{f\in C[0,1]:f(0)=0=f(1)\}$ is a maximal ideal of $C[0,1]?$
\end{ex}
\begin{ex}
Let $R$ be a commutative ring with identity such that for every $x\in R$ satisfies $x^n=x$ for some $n>1$. Show that very prime ideal of $R$ is maximal ideal.
\end{ex}
\begin{ex}
Let $R$ and $R'$ be two commutative ring with identity and $f:R\to R'$ be an epimorphism. Show that\\
(1) $\ker f$ is a prime ideal of $R$ if $R'$ is an integral domain.\\
(2) $\ker f$ is a maximal ideal of $R$ if $R'$ is a field.
\end{ex}
\begin{ex}
Let $R$ be a ring and $P$ be an ideal of $R$ containing an ideal $I$ of $R$. Show that\\
(1) $P\in \spec R$ iff $P/I\in \spec (R/I).$\\
(2) $P\in \mspec R$ iff $P/I \in mspec (R/I).$
\end{ex}
\begin{ex}
Let $(R,m,K)$  be a local ring then only idempotent elements of $R$ is $0,1.$
\end{ex}
\begin{ex}
Let $f:R\to S$ be a ring homomorphism. Let $P\in \spec S$ then show that $f^{-1}P\in \spec R.$ Does the above result true if $P$ is a maximal ideal.
\end{ex}
\begin{ex}
Suppose, $A,B$ are comaximal ideals of $R$ and $A,C$ are comaximal ideals of $R.$ Show that $A,BC$ are comaximal.
\end{ex}
\begin{ex}
Show that $R=\left\lbrace \begin{pmatrix}
0\quad a\\
0\quad b
\end{pmatrix}:a,b\in {\QQ}\right\rbrace$ is a non commutative subring of $M_2({\QQ}).$ If $I=\{A\in R:A^2=0\}$ then show that $I$ is an ideal of $R$ and $R/I\cong {\QQ}.$
\end{ex}
\begin{ex}
Show that ${\ZZ}\times \{0\}$ is a prime ideal of ${\ZZ}\times {\ZZ}$ but not a maximal ideal of ${\ZZ}\times {\ZZ}.$
\end{ex}
\begin{ex}
Let $R$ be a commutative ring with identity then prove that there exists an epimorphism from $R$ onto some field.
\end{ex}
\begin{ex}
Let $R$ be a commutative regular ring with identity then every prime ideal of $R$ is a maximal ideal.
\end{ex}
\begin{ex}
Show that there does not exists \\
(1) An epimorphism from ${\ZZ}/24{\ZZ}$ onto the ring ${\ZZ}/7{\ZZ}.$\\
(2) A monomorphism from ${\ZZ}/6{\ZZ}$ to ${\ZZ}/11{\ZZ}.$
\end{ex}
\begin{ex}
Show that the ring of integer is not isomorphic to any of its proper subring.
\end{ex}
\begin{ex}
Show that any homomorphism from a field to any ring either monomorphism or zero homomorphism.
\end{ex}
\begin{ex}
Let $R$ be a commutative ring with identity and $S=R\times R.$ We define two binary operation `$+$' and `$\cdot$' on $S$ by $(a,b)+(c,d)=(a+c,b+d)$ and $(a,b)\cdot (c,d)=(ac+bd,bc+ad)$ for all $(a,b),(c,d)\in S.$ Show that $S$ is a commutative ring with identity having zero divisors. Prove that if $R$ is an integral domain then then the zero divisors of $S$ are of the form $(a,a)$ and $(a,-a).$
\end{ex}
\begin{ex}
Let $S$ be any arbitrary set and $\mathcal{P}(S)$ be its power set. Define `$+$' and `$\cdot$' by $X+Y=X\Delta Y=(X\setminus Y)\cup(Y\setminus X)$ and $X\cdot Y=X\cap Y$ for all $X,Y\in \mathcal{P}(S)$. Show that $(\mathcal{P}(S), \Delta, \cap)$ is a commutative ring with identity having zero divisors such that each element is idempotent.
\end{ex}
\begin{ex}
Let $p$ be a prime number. Show that there are two non-isomorphic rings with $p$ elements.
\end{ex}
\begin{ex}
How many non-isomorphic rings are there of order 105? Justify.
\end{ex}
\begin{ex}
Let $x$ be an nilpotent element and $y$ be an unit of a commutative ring with identity $R.$ Show that $x+y$ is also a unit in $R.$ For $y=1$ is commutativity necessary? Justify. If the element $x,y$ do not commute then show that the above property does not hold.
\end{ex}
\begin{ex}
Show that the ring ${\ZZ}/n{\ZZ}$ has non-zero nilpotent elements if and only if $n$ is not square free.
\end{ex}
\begin{ex}
In a ring $R$ with identity such that $x^6=x$ for all $x\in R.$ Show that $x^2=x$ for all $x\in R.$
\end{ex}
\begin{ex}
Let $R$ be a ring with identity and $e$ be an idempotent element in $R$ such that $e\neq 0,1.$ Show that $e$ is a zero divisor.
\end{ex}
\begin{ex}
Let $R$ be a ring, $a\in R$ and $b(\neq 0)\in R$ such that $aba=0$. Show that $a$ is a left or right zero divisor in $R.$
\end{ex}
\begin{ex}
Give an example of ring homomorphism $f:R\to R'$ and an ideal $I$ of $R$ such that $f(I)$ is not an ideal of $R'.$
\end{ex}
\begin{ex}
Let $R$ be a ring with identity. If $ab+ba=1$ and $a^3=a$ then show that $a^2=1.$ 
\end{ex}
\begin{ex}
If $ab=a$ and $ba=b$ in a ring then show that $a^2=a$ and $b^2=b$.
\end{ex}
\begin{ex}
Let $R$ be a commutative ring with identity. If $I$ and $J$ are comaximal ideal of $R$ then show that $I^2$ and $J^2$ are comaximal ideal.
\end{ex}
\begin{ex}
Let $R$ be a commutative ring with identity. Show that $R$ is a field if and only if $R$ has no nonzero proper ideals.
\end{ex}
\begin{ex}
Show that ${\ZZ}/2{\ZZ}\cong 5{\ZZ}/10{\ZZ}$ as ring.
\end{ex}
\begin{ex}
Let $R$ be the set of all rationals of the form $a/b$ (in lowest terms) such that $b$ is not a multiple of $3.$ Show that $R$ is a subring of ${\QQ}.$\\
Let $I$ be a subset of $R$ consisting of those elements whose numerators are divisible by $3.$ Prove that $I$ is an ideal of $R.$ Also show that $R/I$ is a field.
\end{ex}
\begin{ex}
Let $R$ be a ring and $I$ be a left ideal, $J$ be a right ideal of $R.$ Show that $R$ is a regular ring if and only if $IJ=I\cap J.$
\end{ex}
\begin{ex}
Show that the center of a regular ring is regular.
\end{ex}
\begin{ex}
Let $R$ be a regular ring. If $I$ is an ideal of $R$ and $J$ is an ideal of $J$ then show that $J$ is an ideal of $R.$
\end{ex}
\begin{ex}
Let $R$ be an integral domain and let $F$ be a field such that $F\subset R$. If $R$ is a finite dimensional vector space over $F$ then show that $R$ is a field.
\end{ex}
\begin{ex}
Let $I=\gen{x^2+x+1}.$ Is ${\ZZ}_3[x]/I$ an integral domain? Justify.
\end{ex}
\begin{ex}
Show that $\gen{x^2+1}$ is not an prime ideal of ${\ZZ}_2[x].$
\end{ex}
\begin{ex}
Show that any non zero ideal of a PID if a unique product of prime ideals.
\end{ex}
\begin{ex}
Find $\gcd(2,x)$ in ${\ZZ}[x]$ and show that it can not be put in the form $2r(x)+xs(x)$ for some $r(x),s(x)\in {\ZZ}[x].$
\end{ex}
\begin{ex}
Let $R$ be an integral domain the n prove that $R$ and $R[x]$ have same characteristic.
\end{ex}
\begin{ex}
Let $R={\ZZ}\times {\ZZ}$. Show that the polynomial $(1,0)x$ has infinitely many roots in $R.$
\end{ex}
\begin{ex}
Let $R$ be an integral domain then show that units of $R[x]$ is contained in $R.$
\end{ex}
\begin{ex}
Let $f:R\to S$ be a epimorphism of rings. If $R$ is an PIR then show that $S$ is also a PIR. Does the result hold for PID Justify.
\end{ex}
\begin{ex}
Prove that the ring ${\ZZ}/n{\ZZ}$ is a PIR for all n.
\end{ex}
\begin{ex}
In ${\ZZ}[i]$ find the $\gcd (2-7i,2+11i).$ Also find $x,y$ such that $\gcd (2-7i,2+11i)=x(2-7i)+y(2+11i)$.
\end{ex}
\begin{ex}
Let $I$ be set of all non- units of ${\ZZ}[i].$ Is $I$ is an ideal? Show that for any non-trivial ideal $P$ of ${\ZZ}[i]$, the quotient ring ${\ZZ}[i]/P$ is finite ring.
\end{ex}
\begin{ex}
In ${\ZZ}[i]$, show that 3 is a prime element but 5 is not a prime element.
\end{ex}
\begin{ex}
Show that the polynomial ring ${\QQ}[x,y]$ is UFD but not PID.
\end{ex}
\begin{ex}
If $f(x)$ is an irreducible polynomial over ${\RR}$ then show that either $f(x)$ is linear or $f(x)$ is quadratic. The irreducible polynomial in ${\CC}[x]$ is exactly the linear polynomials.
\end{ex}
\begin{ex}
Show that there only three irreducible monic polynomials over ${\ZZ}/3{\ZZ}.$ In general, for a prime $p$ find the number of irreducible monic polynomial over ${\ZZ}/p{\ZZ}.$
\end{ex}
\begin{ex}
Let $R$ be a commutative ring with identity such that $R[x]$ is PID then show that $R[x]$ is ED.
\end{ex}
\begin{ex}
Is the quotient ring ${\CC}[x]/\gen{x^2+1}$ an integral domain?
\end{ex}
\begin{ex}
Show that $(i)$ ${\RR}[x]/\gen{x^2+1}\cong {\CC}.$\\
$(ii)$ ${\ZZ}[x]/\gen{n,x}\cong {\ZZ}/n{\ZZ}.$
\end{ex}
\begin{ex}
Show that $x^3+mx+n$ is irreducible in ${\ZZ}[x]$ whenever $m$ and $n$ are odd number.
\end{ex}
\begin{ex}
Show that the polynomial $f(x)=x^{2222}+2x^{2220}+4x^{2218}+\cdots +2220x^2+2222$ is irreducible in ${\ZZ}[x].$
\end{ex}
\begin{ex}
Let $F$ be a field. Prove that $\displaystyle\sum_{i\geq 0} a_ix_i \in F[x]$ is a unit iff $a_0\neq 0.$
\end{ex}
\begin{ex}
Let $\alpha=\dfrac{a}{2^b}$ where $a$ is an odd integer and $b>0$. Verify that ${\ZZ}[\alpha]={\ZZ}\left[\frac{1}{2}\right]$ and that any element of this ring can be uniquely written as $\dfrac{m}{2^n}$ where $m$ is odd and $n\geq 0$. Find units of this rings.
\end{ex}
\begin{ex}
Let $S$ be the set of odd prime numbers. Let $R={\ZZ}\left[\left\lbrace\frac{1}{p}\right\rbrace_{p\in S}\right].$ Prove that every element of $R$ can be written as $\dfrac{a}{b}2^n$ where $a,b$ are odd and $n>\geq 0.$ Find the units in $R.$ Deduce that every non-zero ideal in $R$ is of the form $2^nR$ for $n\geq 0.$
\end{ex}
\begin{ex}
Let $R$ be a ring. Let $a_1,\cdots ,a_n\in R.$ Prove that the map $R[x_1,\cdots ,x_n]\stackrel{\phi}{\longrightarrow}R[x_1,\cdots ,x_n]$ sending $x_i\mapsto x_i-a_i$ is an $R-$linear isomorphism. Deduce that every $f\in R[x_1,\cdots ,x_n]$ admits a unique expansion $f=\sum c_{j_1\cdots j_n}(x_1-a_1)^{j_1}\cdots (x_n-a_n)^{j_n}.$
\end{ex}
\begin{ex}
Prove that there is a unique $R-$linear isomorphism $$R[x_1,\cdots ,x_m,y_1,\cdots ,y_n]\rightarrow R[\tilde{x_1},\cdots ,\tilde{x_m}][\tilde{y_1},\cdots ,\tilde{y_n}]$$ sending $x_i\mapsto \tilde{x_i}$ and $y_j\mapsto \tilde{y_j}.$
\end{ex}
\begin{ex}
Find the $\gcd (2x^4+3x^3+6x^2+3x+2,2x^4+2x^3+5x^2+x+2)$ in ${\QQ}[x].$ Express the gcd as a linear combination of these two polynomial.
\end{ex}
\begin{ex}
Fix $n>0$ in ${\ZZ}$. Set $R:={\ZZ}[\frac{1}{n}]$. Prove that for any ideal $I\subseteq R, I\cap {\ZZ}\cdot R=I.$ Deduce that $R$ is a PID. Find all prime elements(upto associates) in $R.$
\end{ex}
\begin{ex}
Let $R=F[x,y]$ where $F$ is a field. Let $m=\gen{a,b}$. For each $k>1$, find a minimal generating set for $m^k.$
\end{ex}
\begin{ex}
Find elements that are irreducible but not primes in $(1)~R={\QQ}[t^2,t^3],(2)~{\ZZ}[2i].$
\end{ex}
\begin{ex}
Let $R={\ZZ}[\sqrt{-5}].$ Let $I_1=\gen{1+\sqrt{-5},2},I_2=\gen{1-\sqrt{-5},2},J_1=\gen{1+\sqrt{-5},3},J_2=\gen{1-\sqrt{-5},3}.$ Verify the followings:\\
$(1)~ I_1=I_2,(2)~I_1I_2=\gen{2},(3)~J_1J_2=\gen{3},(4)~ I_1\cdot J_1=\gen{1+\sqrt{-5}},(5)~ I_2J_2=\gen{1-\sqrt{-5}},(6)~ R=I_1+J_1=I_1+J_2=J_1+J_2.$
\end{ex}
\begin{ex}
Let $R$ be a ring. Let $a_1,\cdots,a_n\in R.$ Consider the evaluation map $R[x_1,\cdots,x_n]\stackrel{\phi}{\longrightarrow} R$ sending $f(x_1,\cdots,x_n)$ to $f(a_1,\cdots,a_n).$ Prove that $\ker \phi=\gen{x_1-a_1,\cdots ,x_n-a_n}.$
\end{ex}
\begin{ex}

\end{ex}
\begin{ex}

\end{ex}
%\begin{ex}

%\end{ex}
%\begin{ex}

%\end{ex}
%\begin{ex}

%\end{ex}
%\begin{ex}

%\end{ex}









\newpage
%\chapter{Appendix II}
\section{Appendix II}

\begin{center}
\textbf{Cartesian product, Partially ordered set and Zorn's lemma}
\end{center}
\subsection{Cartesian Product}
\begin{defn}
Let $A,B$ be two set, we define Cartesian product of two set as $$A\times B:=\{(a,b):a\in A,b\in B\}.$$
\end{defn}
Now, we give an alternative definition of Cartesian product.
\begin{defn}
Let $I$ be an indexing set and let $\{A_i\}_{i\in I}$ be collection of sets. A choice function $f$ is any function $$f:I\to \displaystyle\bigcup_{i\in I} A_i$$ such that $f(i)=A_i$ for all $i.$
\end{defn}
\begin{definition}
We define $\displaystyle\prod_{i\in I}A_i=\left\lbrace f:I\to \displaystyle\bigcup_{i\in I}A_i:f(i)\in A_i,\forall i\in I\right\rbrace  $ i.e., set of all choice function from $I$ to $\displaystyle\bigcup_{i\in I}A_i$ and is denoted by $\displaystyle\prod_{i\in I} A_i$ (where if any $A_i$ is empty or $I$ is empty then Cartesian product is empty). The elements of Cartesian product is denoted by $\displaystyle\prod_{i\in I} a_i$ where this denote the choice function $f(i)=a_i$ for all $i\in I.$
\end{definition}
If $I=\{1,2,\cdots ,n\}$ for some $n\in {\NN}$ and if $f$ is a choice function from $I$ to $A_1\cup A_2\cup \cdots \cup A_n$ where each $A_i$ is non empty, we can associates to $f$ a unique ordered n-tuple: $$f\to (f(a_1),f(a_2),\cdots ,f(a_n))$$
Note that by definition of a choice function $f(i)\in A_i$ for all $i,$ so the above n-tuple has an element of $A_i$ in the $i^{th}$ position for each $i.$\\
Conversely, given an n-tuple $(a_1,\cdots ,a_n)$ where $a_i\in A_i$ for all $i$, there is a unique choice function, $f$ from $I$ to  $\displaystyle\bigcup_{i\in I}A_i$ associated to it, namely $$f(i)=a_i,\quad \text{for all}~i\in I.$$
Thus there is an one-to-one correspondence between ordered  n-tuple and elements of $\displaystyle\prod_{i\in I} A_i.$ Therefore, from now we can write the elements of $\displaystyle\prod_{i\in I} A_i$ as ordered n-tuple. \\
If $I={\NN},$ we shall similarly write $\displaystyle\prod_{i=1}^{\infty} A_i$ or $A_i\times A_2\times \cdots$ for the Cartesian product of $A_i$'s and we shall write the ordered n-tuple as $(a_1,a_2,\cdots)$ i.e., infinite sequences whose $i^{th}$ term is in $A_i.$\\
Note that when $I=\{1,\dots ,n\}$ or ${\NN}$ then we have used natural ordering of $I$ to arrange the elements of our Cartesian products into n-tuple. Any other ordering gives rise to a different representation of the elements of the same Cartesian product.
\begin{eg}
${\RR}^n={\RR}\times {\RR}\times \cdots \times {\RR}$ is the usual n-tuple from real entries.\\
Suppose, $I={\NN}$ then $\displaystyle\prod_{i=1}^{\infty} A_i$ where $A_i=A$ denote the set of infinite sequence. In particularly if $A={\RR}$ then $\displaystyle\prod_{i=1}^{\infty} A_i$ is the set of all real sequence.
\end{eg}
\begin{proposition}
Let $I$ be a non-empty countable set and for each $i\in I$ let $A_i$ be a set. Then $$\bigg|\displaystyle\prod_{i\in I} A_i \bigg|=\displaystyle\prod_{i\in I} |A_i|$$
\end{proposition}
\proof 



\newpage
\subsection{Partially ordered set and Zorn's lemma}
\begin{defn}
A partial order on a non-empty set $A$ is a relation $\leq$ on $A$ satisfying\\
$(1)$ $x\leq x$ for all $x\in A$ i.e., $\leq$ is reflexive,\\
$(2)$ if $x\leq y$ and $y\leq x$ then $x=y$ for all $x,y\in A$(anti-symmetric),\\
$(3)$ if $x\leq y$ and $y\leq z$ then $x\leq z$ for all $x,y,z\in A$ (transitive).
\end{defn}
\begin{defn}
Let $(A,\leq )$ be a non-empty partially ordered set\\
$(1)$ A subset $B$ of $A$ is called chain if for all $x,y\in B$ either $x\leq y$ or $y\leq x.$\\
$(2)$ An upper bound for a subset $B$ of an element $u\in A$ such that $b\leq u$ for all $b\in B.$\\
$(3)$ A maximal element of $A$ is an element $m\in A$ such that if $m\leq x$ for any $x\in A,$ then $m=x.$ 
\end{defn}
A chain is also called a tower or called a totally ordered or linearly ordered subset.
\begin{eg}

\end{eg}
\begin{ex}
Show that in a partially ordered set maximal element may not be unique.
\end{ex}
\begin{ex}
When maximal element in a partially ordered set is unique?
\end{ex}
\begin{defn}
Let $(X,\leq)$ be an partial ordered set. $X$ is said to be linearly ordered if any two elements of $X$ are comparable.
\end{defn}
\textbf{Zorn's lemma:} Let $(A,\leq )$ be a non-empty partially ordered set such that every chain has an upper bound, then $A$ has a maximal element.\\
\textbf{The Axiom of Choice:} Cartesian product of non-empty set indexed by non-empty set is non-empty.\\
\textbf{Hausdroff's maximality principle:} Every partial ordered set has a maximal linearly ordered subset.
\begin{defn}
Let $A$ be a non-empty set. A well ordering on $A$ is a total ordering on $A$ such that every non-empty subset of $A$ has a minimum element, i.e., for each non-empty $B\subseteq A$ there is some $s\in B$ such that $s\leq b,$ for all $b\in B.$
\end{defn}
\textbf{Well-ordering Principle:} Every non-empty set can be well ordered.
\begin{theorem}
Assuming the usual(Zermelo-Frankel)axioms of set theory, the following are equivalent:
\begin{enumerate}
\item Well ordering principle
\item Zorn's lemma
\item Axiom of choice
\item Hausdroff's maximality principle
\end{enumerate}
\end{theorem}
\proof See Topology-Dugunji. 











It is noted that ${\RR}$ is  a vector space over ${\QQ}$ where $\dim_{\QQ} {\RR}$ is infinite (prove!). Now $\{1,\sqrt{2}\}$ is linearly independent set over ${\QQ}$ (why)? so we can extend this set of a basis of ${\RR}$. Let $\beta$ be that basis. Then any $c\in {\RR}$ we can write $$c=c_1\sqrt{2}+\text{~finite sum}$$ Now we define a function $f:{\RR}\to {\RR}$ as follow \begin{align*}
f(x)=\begin{cases}
c_1,\quad&\text{if}~c_1\neq 0\\
0,\quad&\text{if}~c_1=0
\end{cases}
\end{align*} Check that this function is well defined and prove the following: $f(x+y)=f(x)+f(y)$ and $f$ is discontinuous at 0 not only that $f$ is continuous no where on ${\RR}.$















$\xrightarrow{vvsxk;sklsklsgvklsdklgdskldsklvdklsvdlsbsdlbdsilofbss}jhhj$



\begin{defn}
jjj
\end{defn}




\begin{qns}
fgfg
\end{qns}


\begin{qns*}

\end{qns*}

\begin*{qns*}

\end*{qns*}


$\mathbb{ABCDEFGHIJKLMNOPQRSTUVWXYZ}$

$\textbf{GFGFG}$ $\mathcal{ABCDEFGHIJKLMNOPQRSTUVWXYZ}$


\begin{theorem}[First isomorphism theorem]
Let $\phi:R\to S$ be a ring morphism and $I\subseteq \Ker f$ is an ideal of $R$. Then there is an unique ring morphism $g:R/I\to S$ such that the diagram commutes i.e.,
\begin{center}
\begin{tikzcd}
R \arrow[rr, "f"] \arrow[dd, "\pi"] &  & S \\
                                    &  &   \\
R/I \arrow[rruu, "g"]               &  &  
\end{tikzcd}
\end{center}
Moreover, \begin{enumerate}
\item If $I=\Ker f$ then $g$ is injective, therefore, $R/\Ker f\cong \im f.$
\item $\im f=\im g.$
\item If $f$ is onto, then $R/\Ker f\cong S.$
\end{enumerate}
\end{theorem}

\proof 



$\displaystyle\bigcup_{i=1}^n S_i$



$\displaystyle\lim_{n\to \infty} x_n=s$

$\displaystyle\int_a^b f(x)dx$

$\frac{a}{b}$ $\dfrac{a}{B}$ 

$S={f}=\{d\}=\{f:G\to G:f \text{ is a\quad group\qquad morphim}\}$



$\binom{a}{a} \sqrt{7}$







































































































































































































































































































































































































































































































































































































































































































\end{document}



